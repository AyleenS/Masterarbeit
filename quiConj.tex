\section{Konjunktion}

Die Relation \QRel{} verfeinert die Relation \ERel{} nur um die zusätzlich
Inklusion $\QET _P \subseteq \QET _Q$. Es kann also die Konjunktion und die
dafür notwendige Normal Form nur mit kleinen Abwandlungen aus dem
Fehler"=Kapitel übernommen werden.\\
Als erstes sollte überprüft werden, ob die Konstruktion der Normal Form etwas
an den Traces aus der Menge \QET{} verändert. Da die Normalisierung
in~\ref{normalFormDef} die Menge der Fehler"=Traces in \ET{} unverändert lässt,
reicht es aus die Traces aus $\StQT (P)\backslash\ET (P)$ zu betrachten.
Schritt (i) lässt alle Zustände, die nicht in $E_P$ oder $\overline{E_P}$
enthalten sind unverändert, vor allem verändert dieser Schritt noch keine
Transitionen. Alle stillen Zustände sind also auch noch nach dem Schritt (i)
still. Im ersten Fall des Schrittes (ii) ist die Menge $\StQT (P)\backslash \ET
(P)$ leer und ebenso ist es die Menge $\StQT (\NF (P))\backslash\ET (\NF (P))$.
Der zweite Fall von Schritt (ii) verändert nur Transitionen, zu Zuständen
führen, die mit Traces aus $\ET (P)$ erreichbar sind. Es bleiben also auch die
durch $\StQT (P)\backslash \ET (P)$ erreichbaren Zustände im zweiten Schritt
der Konstruktion in~\ref{normalFormDef} still. Es gilt also auch für die Trace
Menge \QET{} $\QET(P) = \QET (\NF (P))$ und somit ist $\NF (P)$ ebenfalls
äquivalent zu $P$ bezüglich der Relation \QRel{}.\\
Die Normal Form hat das Problem mit dem Fehler"=Zuständen gelöst. In der
Definition~\ref{ConjDef} gab es eine spezielle Behandlung der
Input"=Transitionen um in der Konjunktion die Input"=kritischen Traces
entsprechend zu berücksichtigen. Auch für die in diesem Kapitel neu eingeführte
Trace Menge \QET{} kann es zu Problemen kommen, die durch die Normal Form und
Konjunktion im Fehler"=Kapitel noch nicht gelöst wurden. Es soll wiederum $\QET
_{P\land Q} = \QET _P \cap \QET _Q$ gelten. Für die \ET{} Mengen gilt bereits
eine analoge Aussage. Es ist also nur noch notwendig sich mit Traces
auseinander zu setzten, die in mindestens einer Komponente der Konjunktion
einen stillen Zustand erreichen. Falls $P$ und $Q$ beide einen stillen Zustand
erreichen, kann es in der Konjunktion aus den beiden stillen Zuständen auch
keine ausgehenden lokalen Aktionen geben, da in $P\land Q$ nur Transitionen
möglich sind, die mindestens an einem der beteiligten Zustände möglich ist. Es
sind also alle im Schritt der fehler"=gefluteten Stille"=Traces enthaltenen
Elemente auch in der Menge $\QET _{P\land Q}$ enthalten. Jedoch kann die Menge
$\QET _{P\land Q}$ echt größer sein, wenn man die Definitionen aus dem letzten
Kapitel unverändert anwendet. Falls $P$ eine must"=Output"=Transition besitzt
und $Q$ keine passende Output"=Transition enthält. Hat der zusammengesetzt
keine ausgehende Transition für den Output. Somit kann es dazu kommen, dass ein
Zustand, der aus zwei Zuständen zusammen gesetzt ist, von denen mindestens
einer nicht still ist, in der Konjunktion trotzdem still ist. An den Zuständen,
an denen das beschriebene Problem auftritt kann diese durch neu hinzugefügte
$\tau$-Schlingen gelöst werden.

\begin{Def}[Konjunktion]
  Es werden die Voraussetzungen und die Konstruktion der Konjunktion
  aus~\ref{ConjDef} in diese Definition übernommen. Man erhält $P_1\land P_2$
  in diesem Kapitel aus der konstruierten Konjunktion in dem man eine
  $\tau$-Schleife an jeden Zustand $(p_1,p_2)\in Qui$ anfügt, für den keiner
  der $p_j$ ein Fehler-, Input"=kritischer oder stiller Zustand ist.
\end{Def}

Für dies abgewandelte Konjunktions"=Definition kann nun auch für die Relation
\QRel{} gezeigt werden, dass $\land$ die gewünscht größte untere Schranke ist.

\begin{Satz}[Konjunktion]
  Für drei \MEIO{}s $S$, $P_1$ und $P_2$ mit der gleichen Signatur gilt $S\QRel
  P_1\land P_2$ genau dann, wenn $S\QRel P_1$ und $S\QRel P_2$ erfüllt ist.
\end{Satz}

\begin{proof}
  \TODO{zu beweisen}
\end{proof}
