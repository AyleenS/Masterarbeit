\chapter{allgemeine Folgerungen}

\begin{Prop}[Sprache und Implementierung]
  Die \emph{(maximale) Sprache} eines \MEIO{}s $P$ ist $L(P) = \left\{w\in
  \Sigma ^* \mid \exists P'\in\asimp (P) : p'_0\weakmust[w]
  \right\}$.
\end{Prop}

\begin{proof}
  Für ein $w\in L(P)$ gilt nach Definition~\ref{LDef} und den
  Defintionen der Trantistions Notatin $\exists p_1,p_2,\dots,p_{n-1},p' \exists
  w'\in\Sigma _{\tau} ^* : \hat{w'} = w \land w'=\alpha _1\alpha _2\dots\alpha
  _n \land p\may[\alpha _1]p_1\may[\alpha _2] \dots p_{n-1}\may[\alpha _n] p'$.
  Aufgrund von Defintion~\ref{SimDef} 2. kann jedes Element aus $\asRel (P)$
  nur die bereits in $P$ vorhandenen may-Trantitionen implementieren. Somit
  gibt es für jedes $w$, dass in der Sprache einer as"=Implementierung von $P$
  enthalten ist auch ein entsprechendes $w$ mit einem Trace wie oben. Da in
  $\left\{w\in \Sigma ^* \mid \exists P'\in\asimp (P) : p'_0\weakmust[w]
  \right\}$ alle Wörter enthalten sind, für dies es eine as"=Implementierung
  gibt, die dieses Wort ausführen kann, werden somit auch alle möglichen
  as"=Implementierungen betrachtet. Deshalb sind auch alle Wörter, die in
  $L(P)$ enthalten sind in der Menge der Wörter alle Implementierungen
  enthalten.\\
  Da für Implementierungen die must-Transitons-Relations-Menge die gleiche ist,
  wie die Menge der may-Transitions-Relationen könnte man auch für die
  as"=Implementierungen die Defintion~\ref{LDef} anwenden um die jeweilige
  Sprache zu bestimmen. Die Sprache eines \MEIO{} entspricht dann der
  Vereinigung der Sprachen seiner as"=Implementierungen.
\end{proof}

\begin{Prop}[Sprache der Parallelkomposition]
  Für zwei komponierbare \MEIO{}s $P_1$ und $P_2$ gilt: $L_{12} := L(P_{12}) =
  L_1\|L_2$.
\end{Prop}

\begin{proof}
  \TODO{beweisen oder entscheiden, dass kein Beweis notwendig ist}
\end{proof}

\begin{Lem}[as-Implementierungen und Parallelkomposition]\mbox{}
  \label{impParallelLem}
  \begin{enumerate}
    \item $P'_1\in\asimp (P_1) \land P'_2 \in\asimp (P_2) \Rightarrow
      (P'_1\|P'_2) \in\asimp (P_1\|P_2)$.

      \TODO{ausführlicher beweisen (MIA3 als mögliche Vorlage für Konstruktion)}

    \item $P' \in\asimp (P_1\|P_2) \Rightarrow \exists P'_1,P'_2 : P'_1
      \in\asimp (P_1) \land P'_2 \in\asimp (P_2) \land P'_1\|P'_2=P'$.

      \TODO{STIMMT NICHT: Gegenbeispiel einbauen und alternative überlegen}
  \end{enumerate}
\end{Lem}

\begin{proof}\mbox{}
  \begin{enumerate}
    \item Es gelte $i\in\{1,2\}$. Jede Transition ($\may '_i =\must '_i$) in
      $P'_i$ hat eine entsprechende may-Transition in $P_i$, da
      Definition~\ref{SimDef} 2.\ gilt. Aufgrund der Simulations
      Definition~\ref{SimDef}, haben $P_i$ und $P'_i$ die gleichen Signaturen
      und somit gilt $\Synch (P_1,P_2) =\Synch (P'_1,P'_2)$.\\
      Daraus folgt unter Verwendung, dass die Definitionen der must- und
      may"=Transitionen der Parallelkomposition in~\ref{ParallelDef} analog
      formuliert sind, dass alle Transitionen, die in $P'_1\|P'_2$ enthalten
      sind auch in $P_1\|P_2$ als may"=Transitionen vorhanden sind. Es können
      bei der Parallelkomposition von \MEIO{}s, die die gleichen must- und
      may-Transitionsmengen haben (Implementierungen), keine may"=Transitionen
      entstehen, zu denen es keine passende must-Transition gibt (Definition
      von $\must _{12}$ und $\may _{12}$ in~\ref{ParallelDef}). $P'_1\|P'_2$
      ist also eine Implementierung.\\
      $P_1\|P_2$ enthält nur must"=Transitionen, wenn auch $P_1$ bzw.\ $P_2$
      diese enthalten haben. $P'_1$ und $P'_2$ müssen diese must"=Transitionen
      aufgrund von Definition~\ref{SimDef} 1.\ implementieren und somit enthält auch
      $P'_1\|P'_2$ die entsprechenden durch~\ref{SimDef} 1.\ geforderten
      Transitionen, um eine as"=Implementierung von $P_1\|P_2$ sein zu können.\\
      Die Kommunikationsfehler-Zustände, die $P'_1$ und $P'_2$ enthalten sind,
      müssten wegen Definition~\ref{SimDef} 3.\ auch in $P_1$ und $P_2$
      enthalten sein. Somit enthält $P_1\|P_2$ auch die entsprechend geerbten
      Kommunikationsfehler, die dann durch $P'_1\|P'_2$ implementiert werden.
      In $P'_1\|P'_2$ kann es nur zu neuen Kommunikationsfehlern kommen, wenn
      dies auch in $P_1\|P_2$ möglich war. Also ist auch~\ref{SimDef} 3.\ für
      $P'_1\|P'_2$ erfüllt.\\
      $\Rightarrow (P'_1\|P'_2)\in\asimp (P_1\|P_2)$.
    \item Wie im Beweis zu 1.\ muss jede Transition aus $P'$ in $P_1\|P_2$ als
      may-Transition auftauchen (Definition~\ref{SimDef} 2.). Die gleichen
      Signaturen von $P_i$ und $P'_i$ führen ebenso wieder zu den gleichen
      Synchronisationsmengen.\\
      Falls in $P'$ eine Transition mit einer Aktion aus $\Synch$ beschrifte
      ist muss diese auch als may-Transition in $P_1$ und $P_2$ vorhanden sein
      (Argumentation von oben und Definition~\ref{ParallelDef}). Bei Aktionen,
      die nicht in $\Synch$ enthalten sind, muss nur das jeweilige \MEIO{}
      $P_1$ bzw. $P_2$ die Transition als may-Transition enthalten.\\
      $\Rightarrow$ Es kann $P'_1$ und $P'_2$ geben, die alle nötigen
      Transitionen implementierten, so dass $P'=P'_1\|P'_2$ gilt.\\
      Alle Transitionen, die in $P'$ aufgrund von Definition~\ref{SimDef} 1.\
      implementiert werden müssen, mussten auch bereits in $P_1$ bzw.\ $P_2$
      als must"=Transitionen vorhanden sein. Da $P'_1$ und $P'_2$
      as"=Implementierungen von $P_1$ und $P_2$ sind, müssen diese auch die
      Simulations Definition (\ref{SimDef}) erfüllen und somit die
      must"=Transitionen aus $P_1$ bzw.\ $P_2$ implementieren.\\
      Es kann nicht passieren, dass $P_1$ und $P_2$ must"=Transitionen
      enthalten, die $P'_1$ und $P'_2$ implementieren müssen und dann in
      $P'_1\|P'_2$ zu einer Transition führen, die $P'$ auf Basis der
      Definition~\ref{SimDef} 1.\ nicht ebenfalls implementieren muss ($\must
      _{12}$ und $\may _{12}$ Definitionen in~\ref{ParallelDef}).\\
      $P'$ kann nur Kommunikationsfehler-Zustände implementieren, die bereits
      in $P_1\|P_2$ möglich waren. Falls diese geerbt sind, sind die
      entsprechenden Fehler-Zustände auch schon in $P_1$ und $P_2$ möglich und
      können somit durch $P'_1$ und $P'_2$ implementiert werden. Die neuen
      Kommunikationsfehler, die von $P'$ implementiert werden, können in $P'_1$
      und $P'_2$ durch entsprechende nicht Implementierung von
      Input-Transitionen umgesetzt werden. Die Input-Transitionen können
      maximal als may"=Transitionen in $P_1$ bzw. $P_2$ vorkommen, da es sonst
      in $P_1\|P_2$ den neuen Kommunikationsfehler nicht geben würde.\\
      $\Rightarrow$ es gibt passende $P'_1$ und $P'_2$ mit $P'_1\in\asimp (P_1)
      \land P'_2\in\asimp (P_2) \land P'=P'_1\|P'_2$.
  \end{enumerate}
\end{proof}

\begin{Lem}[w-as-Implementierungen und Parallelkomposition]\mbox{}

  \TODO{entsprechend anpassen wie 1.2 oder löschen}
  \begin{enumerate}
    \item $P'_1\in\wasimp (P_1) \land P'_2 \in\wasimp (P_2) \Rightarrow
      (P'_1\|P'_2) \in\wasimp (P_1\|P_2)$.
    \item $P' \in\wasimp (P_1\|P_2) \Rightarrow \exists P'_1,P'_2 : P'_1
      \in\wasimp (P_1) \land P'_2 \in\wasimp (P_2) \land P'_1\|P'_2=P'$.
  \end{enumerate}
\end{Lem}

\begin{proof}\mbox{}
  \begin{enumerate}
    \item Es gelte $i\in\{1,2\}$. Jede Transition ($\may '_i =\must '_i$) einer
      sichtbaren Aktion in $P'_i$ hat eine entsprechende schwache
      may-Transition in $P_i$, da Definition~\ref{wSimDef} 3./4.\ gilt. Aufgrund
      der Simulations Definition~\ref{wSimDef}, haben $P_i$ und $P'_i$ die
      gleichen Signaturen (bezieht sich nur auf sichtbare Aktionen) und somit
      gilt $\Synch (P_1,P_2) = \Synch (P'_1,P'_2)$.\\
      Daraus folgt unter Verwendung, dass die Definitionen der must- und
      may"=Transitionen der Parallelkomposition in~\ref{ParallelDef} analog
      formuliert sind, dass alle sichtbaren Transitionen, die in $P'_1\|P'_2$
      enthalten sind auch in $P_1\|P_2$ als schwache may"=Transitionen
      vorhanden sind. Hierzu ist noch anzumerken, dass die interne Aktion nie
      in der Parallelkomposition synchronisiert wird und somit die \MEIO{}s
      diese in der Komposition jeweils für ihre Komponente alleine ausführen.\\
      $P'_1\|P'_2$ ist mit der gleichen Begründung wie im Beweis von
      Satz~\ref{impParallelLem} 1.\ eine Implementierung.\\
      $P_1\|P_2$ enthält nur must"=Transitionen, wenn auch $P_1$ bzw.\ $P_2$
      diese enthalten haben. $P'_1$ und $P'_2$ müssen diese must"=Transitionen
      aufgrund von Definition~\ref{wSimDef} 1./2.\ schwach implementieren, d.h.\
      bei Inputs können danach noch interne Aktionen möglich sein und bei
      Outputs davor und danach und bei einen $\tau$ können beliebig viele
      interne Aktionen implementiert werden (auch keine). Die durch $P_1\|P_2$
      und~\ref{wSimDef} 1./2.\ geforderten Implementierungen von must"=Transitionen
      in $P'_1\|P'_2$ basieren auf den must"=Transitionen der einzelnen
      \MEIO{}s, die schwach in deren w"=as"=Implementierungen implementiert
      wurden. Die zusätzlichen $\tau$s, die dadurch entstehen, hindern
      $P'_1\|P'_2$ in der Parallelkomposition nicht daran die
      must"=Transitionen auch entsprechend schwach zu implementieren.\\
      Die Argumentation, wieso Definition~\ref{wSimDef} 5.\ hier gilt, kann
      analog zu der Argumentation für Definition~\ref{SimDef} 3.\ aus dem
      Beweis zu Punkt 1.\ aus dem Lemma~\ref{impParallelLem} übernommen werden.
      Die zusätzlichen $\tau$ Transitionen können nichts an geerbten und neuen
      Kommunikationsfehlern verändern, was nicht auch in der
      Parallelkomposition zu analogen Veränderungen führt. Die zu möglichen
      neuen Kommunikationsfehlern führende Inputs müssen entweder sofort
      implementiert werden oder gar nicht.\\
      $\Rightarrow (P'_1\|P'_2)\in\wasimp (P_1\|P_2)$.
    \item Wie im Beweis zu 1.\ muss jede sichtbare Transition aus $P'$ in
      $P_1\|P_2$ als schwache may-Transition auftauchen
      (Definition~\ref{wSimDef} 3./4.). Bei der Projektion auf die einzelnen
      Transitionssysteme muss die schwache Transition im jeweiligen Automaten
      erhalten bleiben. Wobei sich jedoch die Anzahl der $\tau$s auf die beiden
      Systeme aufteilt. Die gleichen Signaturen von $P_i$ und $P'_i$ führen
      ebenso wieder zu den gleichen Synchronisationsmengen.\\
      $\Rightarrow$ Es kann $P'_1$ und $P'_2$ geben, die alle nötigen
      Transitionen implementierten, so dass $P'=P'_1\|P'_2$ gilt.\\
      Alle Transitionen, die in $P'$ aufgrund von Definition~\ref{wSimDef}
      1./2.\ schwach implementiert werden müssen, mussten auch bereits in $P_1$ bzw.\
      $P_2$ als must"=Transitionen vorhanden sein. Da $P'_1$ und $P'_2$
      w"=as"=Implementierungen von $P_1$ und $P_2$ sind, müssen diese auch die
      Simulations Definition (\ref{wSimDef}) erfüllen und somit die
      must"=Transitionen aus $P_1$ bzw.\ $P_2$ schwach implementieren. Hierbei
      muss auf die korrekte Anzahl an $\tau$s geachtet werden, die man dann in
      der Parallelkomposition erhält.\\
      Es kann nicht passieren, dass $P_1$ und $P_2$ must"=Transitionen
      enthalten, die $P'_1$ und $P'_2$ schwach implementieren müssen und dann
      in $P'_1\|P'_2$ zu einer mit einer sichtbaren Aktion beschriften
      Transition führen, die $P'$ auf Basis der Definition~\ref{wSimDef} 1./2.\
      nicht ebenfalls implementieren muss ($\must _{12}$ und $\may _{12}$
      Definitionen in~\ref{ParallelDef}). Natürlich kann es durch die internen
      Aktionen zu Fehler kommen, da diese aber jedoch immer nur schwach
      implementiert werden müssen, kann man dort die Anzahl entsprechend
      regulieren um das gewünschte Ergebnis zu erhalten.\\
      Mit der Argumentation aus Punkte 1.\ dieses Beweises und den
      Feststellungen aus dem Beweis von Punkte 2.\ des
      Lemmas~\ref{impParallelLem} bezüglich der Erfüllung von
      Definition~\ref{SimDef} 3., kann hier die Erfüllung von
      Definition~\ref{wSimDef} 5.\ begründet werden.\\
      $\Rightarrow$ es gibt passende $P'_1$ und $P'_2$ mit $P'_1\in\wasimp (P_1)
      \land P'_2\in\wasimp (P_2) \land P'=P'_1\|P'_2$.
  \end{enumerate}
\end{proof}

Ein neuer Kommunikationsfehler in einer Parallelkomposition muss in einer
Implementierung (as oder w-as) nicht auftauchen, auch nicht in der
Parallelkomposition von Implementierungen der einzelnen Komponenten. Dies liegt
daran, dass für den Input nur gesagt wird, dass keine must"=Transition für die
Synchronisation der Aktion vorhanden ist. Es kann trotzdem eine
may"=Transition für den Input geben, die auch implementiert werden kann.
Falls es aber in der Parallelkomposition zweier \MEIO{} zu einem neuen
Kommunikationsfehler kommt, dann gibt es auch immer mindestens eine
Implementierung, die diesen Kommunikationsfehler enthält und es gibt auch immer
mindestens ein Implementierungs-Paar der Komponenten, in deren
Parallelkomposition sich dieser Kommunikationsfehler ebenfalls zeigt.
