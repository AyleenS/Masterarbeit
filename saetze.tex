\chapter{allgemeine Folgerungen}

\begin{Lem}[as-Implementierungen und Parallelkomposition]\mbox{}
  \label{impParallelLem}
  \begin{enumerate}
    \item $P'_1\in\asimp (P_1) \land P'_2 \in\asimp (P_2) \Rightarrow
      (P'_1\|P'_2) \in\asimp (P_1\|P_2)$.
    \item $P' \in\asimp (P_1\|P_2) \Rightarrow \exists P'_1,P'_2 : P'_1
      \in\asimp (P_1) \land P'_2 \in\asimp (P_2) \land P'_1\|P'_2=P'$.
  \end{enumerate}
\end{Lem}

\begin{proof}\mbox{}
  \begin{enumerate}
    \item Es gelte $i\in\{1,2\}$. Jede Transition ($\may '_i =\must '_i$) in
      $P'_i$ hat eine entsprechende may-Transition in $P_i$, da
      Definition~\ref{SimDef} 2.\ gilt. Aufgrund der Simulations
      Definition~\ref{SimDef}, haben $P_i$ und $P'_i$ die gleichen Signaturen
      und somit gilt $\Synch (P_1,P_2) =\Synch (P'_1,P'_2)$.\\
      Daraus folgt, unter Verwendung das die Definitionen der must- und
      may-Transitionen der Parallelkomposition in~\ref{ParallelDef} analog
      formuliert sind, dass alle Transitionen, die in $P'_1\|P'_2$ enthalten
      sind auch in $P_1\|P_2$ als may-Transitionen vorhanden sind. Es können
      bei der Parallelkomposition von \MEIO{}s (Implementierungen), die die
      gleichen must- und may-Transitionsmengen haben, keine may-Transitionen
      entstehen, zu denen es keine passende must-Transition gibt (Definition
      von $\must _{12}$ und $\may _{12}$ in~\ref{ParallelDef}). $P'_1\|P'_2$
      ist also eine Implementierung.\\
      $P_1\|P_2$ enthält nur must-Transitionen, wenn auch $P_1$ bzw.\ $P_2$
      diese enthalten haben. $P'_1$ und $P'_2$ müssen diese must-Transitionen
      aufgrund von Definition~\ref{SimDef} 1.\ implementieren und somit enthält auch
      $P'_1\|P'_2$ die entsprechenden durch~\ref{SimDef} 1.\ geforderten
      Transitionen, um eine as"=Implementierung von $P_1\|P_2$ sein zu können.\\
      $\Rightarrow (P'_1\|P'_2)\in\asimp (P_1\|P_2)$.
    \item Wie im Beweis zu 1.\ muss jede Transition aus $P'$ in $P_1\|P_2$ als
      may-Transition auftauchen (Definition~\ref{SimDef} 2.). Die gleichen
      Signaturen von $P_i$ und $P'_i$ führen ebenso wieder zu den gleichen
      Synchronisationsmengen.\\
      Falls in $P'$ eine Transition mit einer Aktion aus $\Synch$ beschrifte
      ist muss diese auch als may-Transition in $P_1$ und $P_2$ vorhanden sein
      (Argumentation von oben und Definition~\ref{ParallelDef}). Bei Aktionen,
      die nicht in $\Synch$ enthalten sind, muss nur der jeweilige \MEIO{}
      $P_1$ bzw. $P_2$ die Transition als may-Transition enthalten.\\
      $\Rightarrow$ Es kann $P'_1$ und $P'_2$ geben, die alle nötigen
      Transitionen implementierten, so dass $P'=P'_1\|P'_2$ gilt.\\
      Alle Transitionen, die in $P'$ aufgrund von Definition~\ref{SimDef} 1.\
      implementiert werden müssen, mussten auch bereits in $P_1$ bzw.\ $P_2$
      als must-Transitionen vorhanden sein. Da $P'_1$ und $P'_2$
      as"=Implementierungen von $P_1$ und $P_2$ sind müssen diese auch die
      Simulations Definition (\ref{SimDef}) erfüllen und somit die
      must-Transitionen aus $P_1$ bzw.\ $P_2$ implementieren.\\
      Es kann nicht passieren, dass $P_1$ und $P_2$ must-Transitionen
      enthalten, die $P'_1$ und $P'_2$ implementieren müssen und dann in
      $P'_1\|P'_2$ zu einer Transition führen, die $P'$ auf Basis der
      Definition~\ref{SimDef} 1.\ nicht ebenfalls implementieren muss ($\must
      _{12}$ und $\may _{12}$ Definition in~\ref{ParallelDef}).\\
      $\Rightarrow$ es gibt passende $P'_1$ und $P'_2$ mit $P'_1\in\asimp (P_1)
      \land P'_2\in\asimp (P_2) \land P'=P'_1\|P'_2$.
  \end{enumerate}
\end{proof}
