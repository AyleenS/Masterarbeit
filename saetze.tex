\chapter{allgemeine Folgerungen}

\begin{Prop}[Sprache und Implementierung]
  \label{LImpProp}
  Die Sprache eines \MEIO{}s $P$ entspricht $L(P) = \left\{w\in
  \Sigma ^* \mid \exists P'\in\asimp (P) : p'_0\weakmust[w]_{P'} \right\}$.
\end{Prop}
\begin{proof}
  Für ein $w\in L(P)$ gilt nach Definition~\ref{LDef} und den Definitionen der
  Transitions-Notation $\exists p_1,p_2,\dots,p_{n-1},p' \exists w'\in\Sigma
  _{\tau} ^* : \hat{w'} = w \land w'=\alpha _1\alpha _2\dots\alpha _n \land
  p_0\may[\alpha _1]_P p_1\may[\alpha _2]_P \dots p_{n-1}\may[\alpha _n]_P
  p'$. Für ein $w$ aus $\left\{w\in \Sigma ^* \mid \exists P'\in\asimp (P) :
  p'_0\weakmust[w]_{P'} \right\}$ gilt, für ein $P'\in\asimp (P)$ das analoge
  nur mit must- anstatt may"=Transitionen.\\
  Aufgrund von Definition~\ref{SimDef} 2.\ kann jedes Element aus $\asimp (P)$
  nur die bereits in $P$ vorhandenen may-Transitionen implementieren. Somit
  gibt es für jedes $w$, dass in der Sprache einer as"=Implementierung von $P$
  enthalten ist auch ein entsprechendes $w\in L(P)$ mit einem Trace wie oben.\\
  Da in $\left\{w\in \Sigma ^* \mid \exists P'\in\asimp (P) :
  p'_0\weakmust[w]_P' \right\}$ alle Wörter enthalten sind, für dies es eine
  as"=Implementierung gibt, die dieses Wort ausführen kann, werden somit
  alle möglichen as"=Implementierungen betrachtet. Jede may"=Transition aus $P$
  wird von mindestens einem $P'\in\asimp (P)$ als must"=Transition
  implementiert. Deshalb sind auch alle Wörter, die in $L(P)$ enthalten sind in
  der Menge der Wörter alle as"=Implementierungen von $P$ enthalten.\\
  Da für Implementierungen die must-Transitions-Relations-Menge die gleiche ist,
  wie die Menge der may-Transitions-Relationen könnte man auch für die
  as"=Implementierungen die Definition~\ref{LDef} anwenden um die jeweilige
  Sprache zu bestimmen. Die Sprache eines \MEIO{} entspricht dann der
  Vereinigung der Sprachen seiner as"=Implementierungen.
\end{proof}

\begin{Prop}[Sprache der Parallelkomposition]
  \label{LParallelProp}
  Für zwei komponierbare \MEIO{}s $P_1$ und $P_2$ gilt: $L_{12} := L(P_{12}) =
  L_1\|L_2$.
\end{Prop}
\begin{proof}
  Jedes Wort, dass in $L_{12}$ enthalten ist, kann auf $P_1$ und $P_2$
  projiziert werden und die Projektionen sind dann in $L_1$ und $L_2$
  enthalten. In einer Parallelkomposition werden die Wörter der beiden \MEIO{}s
  gemeinsam ausgeführt, falls es sich um synchronisierte Aktionen handelt, und
  verschränkt sequenziell, wenn es sich um unsynchronisierte Aktionen handelt.
  Somit sind alle Wörter aus $L_1\|L_2$ auch Wörter der Parallelkomposition
  $L(P_{12})$.
\end{proof}

\begin{Lem}[as-Implementierungen und Parallelkomposition]\mbox{}\\
  \label{impParallelLem}
  $P'_1\in\asimp (P_1) \land P'_2 \in\asimp (P_2) \Rightarrow (P'_1\|P'_2)
  \in\asimp (P_1\|P_2)$.
\end{Lem}
\begin{proof}
  Es gelte $j\in\{1,2\}$. Da $P'_j\in\asimp (P_j)$ gilt, gibt es nach
  Definition~\ref{SimDef} eine as-Verfeinerungs"=Relation $\mathcal{R}_j$, die
  beschreibt, wie $P'_j$ $P_j$ verfeinert. Die Parallelkomposition werden
  auf Basis von Definition~\ref{ParallelDef} gebildet. Die Zustände sind also
  Tupel der Zustände der Komponenten. In dem man aus den Zuständen, die die
  $\mathcal{R}_j$ in Relation setzt auch solche Tupel zusammensetzt, kann man
  auch eine neue as-Verfeinerungs-Relation für die Verfeinerung von $P_1\|P_2$
  durch $P'_1\|P'_2$ erstellen. Die neue as-Verfeinerungs-Relation soll
  $\mathcal{R}_{12}$ heißen und wie folgt definiert sein:
  $\forall p'_1,p'_2,p_1,p_2: ((p'_1,p'_2),(p_1,p_2))\in\mathcal{R}_{12}
  \Leftrightarrow (p'_1,p_1)\in\mathcal{R}_1 \land (p'_2,p_2)\in\mathcal{R}_2$.
  Es bleibt nun zu zeigen, dass $\mathcal{R}_{12}$ eine zulässige
  Verfeinerungs"=Relation nach Definition~\ref{SimDef} ist, da die
  Parallelkomposition von zwei Implementierungen auch immer eine
  Implementierung ist (\ref{ParallelDef}). Für die folgenden Fälle soll
  $(p_1,p_2)\mathcal{R}_{12}(p'_1,p'_2)$ vorausgesetzt sein.
  \begin{enumerate}
    \item Für diesen Punkt der Simulations-Definition~\ref{SimDef} ist
      folgendes zu zeigen: $(p_1,p_2)\must[\alpha]_{12}(q_1,q_2)$ impliziert
      $(p'_1,p'_2)\must[\alpha]_{P'_1\|P'_2}(q'_1,q'_2)$ für ein $(q'_1,q'_2)$
      mit $((q_1,q_2),(q'_1,q'_2))\in\mathcal{R}_{12}$.\\
      $(p_1,p_2)\must[\alpha]_{12}(q_1,q_2)$ kann in $P_1\|P_2$ für ein
      $\alpha$ aus $\Synch (P_1,P_2)$ nur gelten, wenn in $P_1$ die
      $\alpha$-Transition zwischen $p_1$ und $q_1$ auch bereits eine
      must"=Transition war und analog für $p_2$ und $q_2$ in $P_2$. Somit
      erzwingen die $\mathcal{R}_j$ für die Komponenten bereits die
      Implementierung der must"=Transitionen, so dass es dann entsprechende
      $(q_j,q'_j)\in\mathcal{R}_j$ gibt. Die Parallelkomposition der
      implementierten must"=Transitionen aus $P'_1$ und $P'_2$ führt in
      $P'_1\|P'_2$ zu der geforderten Transition. Falls $\alpha$ keine
      synchronisiert Aktion ist, enthält die Parallelkomposition die Transition
      nur, da eine Komponente diese Transition alleine ausführen kann (ersten
      beiden Zeilen der $\must _{12}$ Definition in~\ref{ParallelDef}). \OBdA{}
      $p_1\must[\alpha]_1q_1$ und somit gilt $p_2=q_2$. Da $\mathcal{R}_1$ eine
      as-Verfeinerungs-Relation ist, gibt es in $P'_1$ zwischen $p'_1$ und
      $q'_1$ eine must"=Transition und es gilt $(q_1,q'_1)\in\mathcal{R}_1$.
      $\alpha$ ist auch in der Parallelkomposition der as"=Implementierungen
      eine unsynchronisierte Aktion und somit entsteht die Transition dort auch
      nur aus der Transition von $P'_1$ und es gilt $p'_2=q'_2$.\\
      Da in beiden Fällen $(q_j,q'_j)\in \mathcal{R}_j$ für beide Werte von $i$
      gilt, gilt nach der Definition von $\mathcal{R}_{12}$ auch
      $((q_1,q_2),(q'_1,q'_2))\in \mathcal{R}_{12}$.
      $p_2\mathcal{R}_2p'_2$ musst nach Voraussetzung gelten und somit gilt
      wegen der Gleichheiten der Zustände auch $q_2\mathcal{R}_2q'_2$.
    \item Es ist $(p'_1,p'_2)\may[\alpha]_{P'_1\|P'_2}(q'_1,q'_2)$ impliziert
      $(p_1,p_2)\may[\alpha]_{12}(q_1,q_2)$ für ein $(q_1,q_2)$ mit
      $((q_1,q_2),(q'_1,q'_2))\in\mathcal{R}_{12}$ für diesen Punkt zu
      zeigen.\\
      Die Argumentation könnte wie in Punkt 1.\ erneut in unsynchronisierte und
      synchronisierte Aktionen gesplittet werden. Jedoch würde man dadurch nur
      auf das Ergebnis kommen, dass die eine Komponente oder beide die
      entsprechende may"=Transition ausführen können müssen, damit die
      Parallelkomposition dies auch kann. Es gilt also mindestens für eine der
      Komponenten $p'_j\may[\alpha]_{P'_j}q'_j$ und durch die
      Definition~\ref{SimDef}, die für die entsprechende Relation
      $\mathcal{R}_j$ gilt, muss es in $P_j$ die Transition
      $p_j\may[\alpha]_jq_j$ geben, so dass dann $(q_j,q'_j)\in\mathcal{R}_j$
      gilt. Durch die Definition von $\may$ in~\ref{ParallelDef} werden die
      may"=Transitionen der Komponenten entsprechend in die Parallelkomposition
      $P_1\|P_2$ aufgenommen und die Relation $\mathcal{R}_{12}$ gilt, mit den
      analogen Begründungen wie in 1., auch für die entsprechenden
      Zustands-Tupel.
    \item Hierfür muss gezeigt werden, wenn $(p'_1,p'_2)\in E_{P'_1\|P'_2}$
      gilt, dann ist auch $(p_1,p_2)$ in $P_1\|P_2$ ein Zustand, der in der
      Menge der Fehler-Zustande $E_{12}$ enthalten ist.\\
      Falls $(p'_1,p'_2)$ ein geerbter Fehler ist, dann ist \oBdA{} $p'_1\in
      E_{P'_1}$ und $p'_2\in P'_2$. Aufgrund von $\mathcal{R}_1$ und
      Definition~\ref{SimDef} 3.\ muss dann auch $p_1\in E_1$ gelten. Für $p_2$
      gilt durch die Signatur von $\mathcal{R}_2$ $p_2\in P_2$. Zusammen in
      $P_1\|P_2$ ergibt das wieder  einen geerbten Fehler, also $(p_1,p_2)\in
      E_{12}$. $(p'_1,p'_2)$ kann jedoch auch ein neuer Kommunikationsfehler
      sein, dann gilt \oBdA{} $p'_1\nmust[a]_{P'_1}$ und $p'_2\may[a]_{P'_2}$
      für ein $a$ aus $I_1\cap O_2$. Aufgrund von Definition~\ref{SimDef} 2.\
      muss dann auch $p_2\may[a]_2$ gelten. Für $p_1$ kann nicht
      $p_1\must[a]_1$ gelten, da sonst die Simulations Relation $\mathcal{R}_1$
      die Implementierung dieser Transition in $P'_1$ fordern würde
      (\ref{SimDef} 1.). Es gilt also $p_1\nmust[a]_1$ und in der
      Parallelkomposition $P_1\|P_2$ ergibt sich daraus ebenfalls ein neuer
      Kommunikationsfehler mit $(p_1,p_2)\in E_{12}$.
  \end{enumerate}
  $\Rightarrow (P'_1\|P'_2)\in\asimp (P_1\|P_2)$.
\end{proof}

Die entgegengesetzte Richtung von Lemma~\ref{impParallelLem} gilt im
allgemeinen nicht, d.h.\ es muss zu einer as"=Implementierung einer
Parallelkomposition $P'\in\asimp (P_1\|P_2)$ keine as"=Implementierungen $P'_1$
bzw.\ $P'_2$ der einzelnen Komponenten $P_1$ bzw. $P_2$ geben, deren
Parallelkomposition $P'_1\|P'_2$ der as"=Implementierung der
Parallelkomposition $P$ entsprechen. Die Problematik wird
in Abbildung~\ref{impParallelFig} an einem Beispiel dargestellt. In der
Parallelkomposition wird die may"=Transition von $P_2$ zu zwei
may"=Transitionen, für die in einer as"=Implementierung unabhängig
entschieden werden kann, ob sie implementiert werden oder nicht. Somit kommt es
in $P'$ zu dem Problem, dass keine as"=Implementierung von $P_2$ (entweder
keine Transition implementiert oder die $o'$ Transition ist implementiert) in
Parallelkomposition mit der Implementierung $P_1$ $P'$ ergeben würde.

\begin{figure}[htbp]
  \begin{center}
    \begin{tikzpicture}[shorten >=1pt,auto,node distance=2.5cm]
      \node [initial,initial text=$P_1$:] (p01) at (0,0) {$p_{01}$};
      \node (p1) [right of=p01] {$p_1$};

      \path[->]
      (p01) edge node{$o!$} (p1)
      ;

      \node [initial,initial text=$P_2$:] (p02) at (7,0) {$p_{02}$};
      \node (p2) [right of=p02] {$p_2$};

      \path[->]
      (p02) edge[dashed] node{$o'!$} (p2)
      ;

      \node [initial,initial text=$P_1\|P_2$:] (p0) at (0,-2)
      {$p_{01}\|p_{02}$};
      \node (p102) [right of=p0] {$p_1\|p_{02}$};
      \node (p012) [below of=p0] {$p_{01}\|p_2$};
      \node (p12) [below of=p102] {$p_1\|p_2$};

      \path[->]
      (p0) edge node{$o!$} (p102)
      (p012) edge node{$o!$} (p12)
      (p0) edge[dashed] node{$o'!$} (p012)
      (p102) edge[dashed] node{$o'!$} (p12)
      ;

      \node [initial,initial text=$P'$:] (p'0) at (7,-2)
      {$p'_{01}\|p'_{02}$};
      \node (p'102) [right of=p'0] {$p'_1\|p'_{02}$};
      \node (p'012) [below of=p'0] {$p'_{01}\|p'_2$};
      \node (p'12) [below of=p'102] {$p'_1\|p'_2$};

      \path[->]
      (p'0) edge node{$o!$} (p'102)
      (p'012) edge node{$o!$} (p'12)
      (p'0) edge node{$o'!$} (p'012)
      ;
    \end{tikzpicture}
    \caption{Gegenbeispiel für Umkehrung von Lemma~\ref{impParallelLem}}
    \label{impParallelFig}
  \end{center}
\end{figure}

\begin{Lem}[w-as-Implementierungen und Parallelkomposition]\mbox{}\\
  $P'_1\in\wasimp (P_1) \land P'_2 \in\wasimp (P_2) \Rightarrow (P'_1\|P'_2)
  \in\wasimp (P_1\|P_2)$.
\end{Lem}
\begin{proof}\mbox{}
  Es gelte $j\in\{1,2\}$. Da $P'_j\in\wasimp (P_j)$ gilt, gibt es nach
  Definition~\ref{wSimDef} eine schwache as-Verfeinerungs"=Relation
  $\mathcal{R}_j$, die beschreibt, wie $P'_j$ $P_j$ verfeinert. Analog zum
  Beweis von Lemma~\ref{impParallelLem} kann auch hier die neue schwache
  as"=Verfeinerungs"=Relation der Parallelkompositionen $\mathcal{R}_{12}$
  auf Basis der $\mathcal{R}_j$ definiert werden: $\forall p'_1,p'_2,p_1,p_2:
  ((p'_1,p'_2),(p_1,p_2))\in\mathcal{R}_{12} \Leftrightarrow
  (p'_1,p_1)\in\mathcal{R}_1 \land (p'_2,p_2)\in\mathcal{R}_2$. Es bleibt nun
  zu zeigen, dass $\mathcal{R}_{12}$ eine zulässige schwache
  Verfeinerungs"=Relation nach Definition~\ref{wSimDef} ist, da die
  Parallelkomposition von zwei Implementierungen auch immer eine
  Implementierung ist (\ref{ParallelDef}). Für alle folgenden Fälle wird
  $((p_1,p_2),(p'_1,p'_2))\in\mathcal{R}_{12}$ vorausgesetzt.
  \begin{enumerate}
    \item Aus der schwache Simulations-Definition~\ref{wSimDef} folgt, dass für
      den ersten Punkt folgendes zu zeigen ist:
      $(p_1,p_2)\must[i]_{12}(q_1,q_2)$ impliziert $(p'_1,p'_2)
      \must[i]_{P'_1\|P'_2} \weakmust[\varepsilon]_{P'_1\|P'_2} (q'_1,q'_2)$
      für ein $(q'_1,q'_2)$ mit $((q_1,q_2),(q'_1,q'_2))\in\mathcal{R}_{12}$.\\
      Falls die Transition $(p_1,p_2)\must[i]_{12}(q_1,q_2)$ in der
      Parallelkomposition $P_1\|P_2$ für ein $i\in\Synch(P_1,P_2)$ möglich ist,
      dann gab es diese Transition auch bereits in $P_1$ und $P_2$ als
      must"=Transition. Somit verlangen bereits die schwachen
      as"=Verfeinerungs"=Relation $\mathcal{R}_1$ und $\mathcal{R}_2$, dass
      $p'_j\must[i]_{P'_j}\weakmust[\varepsilon]_{P'_j}q'_j$ für $j=1$ und
      $j=2$ gilt. Wenn auf die \MEIO{}s mit diesen Transitionen die
      Parallelkomposition angewendet wird, entstehen der Ablauf
      $(p'_1,p'_2) \must[i]_{P'_1\|P'_2} \weakmust[\varepsilon]_{12}
      (q'_1,q'_2)$. Die $q'_j$ müssen in der  Relation $\mathcal{R}_j$ mit dem
      jeweiligen $q_j$ ein Tupel bilden. Somit gilt auch
      $((q_1,q_2),(q'_1,q'_2))\in\mathcal{R}_{12}$. Falls $i\notin\Synch
      (P_1,P_2)$, ist die $i$ Transition in $P_1\|P_2$ nur aufgrund einer
      entsprechenden Transition in einer Komponente möglich. \OBdA{} gilt $p_1
      \must[i]_1 q_1$ und mit Definition~\ref{wSimDef} 1.\ gilt dann $p'_1
      \must[i]_{P'_1} q'_1$ und $(q_1,q'_1)\in\mathcal{R}_1$. Da es sich um
      eine unsynchronisierte Aktion von $P_1$ bzw. $P'_1$ handelt, muss für die
      zweite Komponente $(p_2,q_2)=(p'_2,q'_2)\in\mathcal{R}_2$ gelten. Die
      interne Aktion ist immer unsynchronisiert und $i$ in diesem Fall auch,
      deshalb ist $(p'_1,p'_2)\must[i]_{P'_1\|P'_2}
      \weakmust[\varepsilon]_{P'_1\|P'_2} (q'_1,q'_2)$ die in der
      Parallelkomposition entstehende Transitionsfolge.
      $(q_1,q_2)\mathcal{R}_{12}(q'_1,q'_2)$ gilt für den Fall des
      unsynchronisierten $i$'s ebenfalls.
    \item Analog zu 1.\ kann für diesen Punkt
      $(p_1,p_2)\must[\omega]_{12}(q_1,q_2)$ impliziert $(p'_1,p'_2)
      \weakmust[\hat{\omega}]_{P'_1\|P'_2} (q'_1,q'_2)$ für ein $(q'_1,q'_2)$
      mit $((q_1,q_2),(q'_1,q'_2))\in\mathcal{R}_{12}$ gezeigt werden.\\
      Die $\omega$ Transition in $P_1\|P_2$ ist entweder aus einem
      synchronisierten oder aus einem unsynchronisierten $\omega$ entstanden.
      Entsprechend ist dann in einem oder beiden $P_j$ die Transition möglich
      und durch die Relationen $\mathcal{R}_j$ folgen die Transitionen $p'_j
      \weakmust[\omega]_{P'_j} q'_j$ mit $(q_j,q'_j)\in\mathcal{R}_j$. Durch
      die Parallelkomposition von $P'_1$ mit $P'_2$ folgt dann das zu zeigende.
    \item $(p'_1,p'_2)\may[i]_{P'_1\|P'_2}(q'_1,q'_2)$ impliziert $(p_1,p_2)
      \may[i]_{12} \weakmay[\varepsilon]_{12} (q_1,q_2)$ für ein $(q_1,q_2)$
      mit $((q_1,q_2),(q'_1,q'_2))\in\mathcal{R}_{12}$ ist die Voraussetzung
      des 3.\ Punktes, um zu beweisen, dass $\mathcal{R}_{12}$ eine schwache
      as"=Verfeinerungs"=Relation ist. Die Transition $i$ kann wiederum durch
      Synchronisation von zwei Transitionen entstanden sein oder durch eine
      Transition aus einer Komponenten und $i\notin\Synch (P'_1\|P'_2)$. In
      beiden Fällen gilt für ein $j$ oder beide $p'_j\may[i]_{P'_j}q'_j$.
      Daraus folgt mit $\mathcal{R}_j$ $p_j \may[i]_j \weakmay[\varepsilon]_j
      q_j$ mit $(q_j,q'_j)\in\mathcal{R}_j$. Durch Synchronisation oder durch übernehmen der
      Transitionen folgt dann $(p_1,p_2) \may[i]_{12} \weakmay[\varepsilon]_{12}
      (q_1,q_2)$ und mit Hilfe der Definition von $\mathcal{R}_{12}$ auch
      $((q_1,q_2),(q'_1,q'_2))\in\mathcal{R}_{12}$.
    \item Analog zu 3.\ kann für diesen Punkt
      $(p'_1,p'_2)\may[\omega]_{P'_1\|P'_2}(q'_1,q'_2)$ impliziert $(p_1,p_2)
      \weakmay[\hat{\omega}]_{12} (q_1,q_2)$ für ein $(q_1,q_2)$ mit
      $((q_1,q_2),(q'_1,q'_2))\in\mathcal{R}_{12}$ gezeigt werden.\\
      Die $\omega$ Transition in $P'_1\|P'_2$ ist entweder aus einem
      synchronisierten oder aus einem unsynchronisierten $\omega$ entstanden.
      Entsprechend ist dann in einem oder beiden $P'_j$ die Transitionen
      möglich und durch die Relationen $\mathcal{R}_j$ folgen die Transitionen
      $p_j \weakmay[\omega]_j q_j$ mit $(q_j,q'_j)\in\mathcal{R}_j$. Durch
      die Parallelkomposition von $P_1$ mit $P_2$ folgt dann das zu zeigende.
    \item Für diesen Punkt ist zu zeigen, dass aus $(p'_1,p'_2)\in
      E_{P'_1\|P'_2}$ $(p_1,p_2)\in E_{12}$ folgt.\\
      Falls $(p'_1,p'_2)\in E_{P'_1\|P'_2}$ ein geerbter Fehler ist, gilt
      \oBdA{} $p'_1\in E_{P'_1}$ und $p'_2\in P'_2$. Mit
      Definition~\ref{wSimDef} 5., die für $\mathcal{R}_1$ gilt, folgt $p_1\in
      E_1$. $p_2$ steht mit $p'_2$ in der Relation $\mathcal{R}_2$. Durch die
      Signatur folgt $p_2\in P_2$. Es gilt also in der Parallelkomposition
      $P_1\|P_2$ $(p_1,p_2)\in E_{12}$ ist ein von $P_1$ geerbter Fehler.
      $(p'_1,p'_2)$ kann jedoch in $P'_1\|P'_2$ auch ein neuer
      Kommunikationsfehler sein. Dann gilt \oBdA{} $p'_1\nmust[a]_{P'_1}$ und
      $p'_2\may[a]_{P'_2}$ für ein $a\in I_0\cap O_2$. Mit $\mathcal{R}_2$
      und~\ref{wSimDef} 4.\ folgt die Gültigkeit von $p_2\weakmay[a]_2$ in
      $P_2$. Da es für $p'_1$ keine ausgehende $a$ must"=Transition geben darf,
      darf es auch in $P_1$ keine $a$ must"=Transition von $p_1$ ausgehend geben,
      ansonsten wäre~\ref{wSimDef} 1.\ für $\mathcal{R}_1$ verletzt.
      $(q_1,q_2)$ ist also auch ein neuer Kommunikationsfehler und somit in
      $E_{12}$ enthalten.
  \end{enumerate}
  $\Rightarrow (P'_1\|P'_2)\in\wasimp (P_1\|P_2)$.
\end{proof}

Ein neuer Kommunikationsfehler in einer Parallelkomposition muss in einer
Implementierung (as oder w-as) nicht auftauchen, auch nicht in der
Parallelkomposition von Implementierungen der einzelnen Komponenten. Dies liegt
daran, dass für den Input nur gesagt wird, dass keine must"=Transition für die
Synchronisation der Aktion vorhanden ist. Es kann trotzdem eine
may"=Transition für den Input geben, die auch implementiert werden kann.
Falls es aber in der Parallelkomposition zweier \MEIO{} zu einem neuen
Kommunikationsfehler kommt, dann gibt es auch immer mindestens eine
Implementierung, die diesen Kommunikationsfehler enthält und es gibt auch immer
mindestens ein Implementierungs-Paar der Komponenten, in deren
Parallelkomposition sich dieser Kommunikationsfehler ebenfalls zeigt.
