\chapter{allgemeine Folgerungen}

\begin{Lem}[as-Implementierungen und Parallelkomposition]\mbox{}
  \label{impParallelLem}
  \begin{enumerate}
    \item $P'_1\in\asimp (P_1) \land P'_2 \in\asimp (P_2) \Rightarrow
      (P'_1\|P'_2) \in\asimp (P_1\|P_2)$.
    \item $P' \in\asimp (P_1\|P_2) \Rightarrow \exists P'_1,P'_2 : P'_1
      \in\asimp (P_1) \land P'_2 \in\asimp (P_2) \land P'_1\|P'_2=P'$.
  \end{enumerate}
\end{Lem}

\begin{proof}\mbox{}
  \begin{enumerate}
    \item Es gelte $i\in\{1,2\}$. Jede Transition ($\may '_i =\must '_i$) in
      $P'_i$ hat eine entsprechende may-Transition in $P_i$, da
      Definition~\ref{SimDef} 2.\ gilt. Aufgrund der Simulations
      Definition~\ref{SimDef}, haben $P_i$ und $P'_i$ die gleichen Signaturen
      und somit gilt $\Synch (P_1,P_2) =\Synch (P'_1,P'_2)$.\\
      Daraus folgt unter Verwendung, dass die Definitionen der must- und
      may"=Transitionen der Parallelkomposition in~\ref{ParallelDef} analog
      formuliert sind, dass alle Transitionen, die in $P'_1\|P'_2$ enthalten
      sind auch in $P_1\|P_2$ als may"=Transitionen vorhanden sind. Es können
      bei der Parallelkomposition von \MEIO{}s, die die gleichen must- und
      may-Transitionsmengen haben (Implementierungen), keine may"=Transitionen
      entstehen, zu denen es keine passende must-Transition gibt (Definition
      von $\must _{12}$ und $\may _{12}$ in~\ref{ParallelDef}). $P'_1\|P'_2$
      ist also eine Implementierung.\\
      $P_1\|P_2$ enthält nur must"=Transitionen, wenn auch $P_1$ bzw.\ $P_2$
      diese enthalten haben. $P'_1$ und $P'_2$ müssen diese must"=Transitionen
      aufgrund von Definition~\ref{SimDef} 1.\ implementieren und somit enthält auch
      $P'_1\|P'_2$ die entsprechenden durch~\ref{SimDef} 1.\ geforderten
      Transitionen, um eine as"=Implementierung von $P_1\|P_2$ sein zu können.\\
      $\Rightarrow (P'_1\|P'_2)\in\asimp (P_1\|P_2)$.
    \item Wie im Beweis zu 1.\ muss jede Transition aus $P'$ in $P_1\|P_2$ als
      may-Transition auftauchen (Definition~\ref{SimDef} 2.). Die gleichen
      Signaturen von $P_i$ und $P'_i$ führen ebenso wieder zu den gleichen
      Synchronisationsmengen.\\
      Falls in $P'$ eine Transition mit einer Aktion aus $\Synch$ beschrifte
      ist muss diese auch als may-Transition in $P_1$ und $P_2$ vorhanden sein
      (Argumentation von oben und Definition~\ref{ParallelDef}). Bei Aktionen,
      die nicht in $\Synch$ enthalten sind, muss nur der jeweilige \MEIO{}
      $P_1$ bzw. $P_2$ die Transition als may-Transition enthalten.\\
      $\Rightarrow$ Es kann $P'_1$ und $P'_2$ geben, die alle nötigen
      Transitionen implementierten, so dass $P'=P'_1\|P'_2$ gilt.\\
      Alle Transitionen, die in $P'$ aufgrund von Definition~\ref{SimDef} 1.\
      implementiert werden müssen, mussten auch bereits in $P_1$ bzw.\ $P_2$
      als must"=Transitionen vorhanden sein. Da $P'_1$ und $P'_2$
      as"=Implementierungen von $P_1$ und $P_2$ sind, müssen diese auch die
      Simulations Definition (\ref{SimDef}) erfüllen und somit die
      must"=Transitionen aus $P_1$ bzw.\ $P_2$ implementieren.\\
      Es kann nicht passieren, dass $P_1$ und $P_2$ must"=Transitionen
      enthalten, die $P'_1$ und $P'_2$ implementieren müssen und dann in
      $P'_1\|P'_2$ zu einer Transition führen, die $P'$ auf Basis der
      Definition~\ref{SimDef} 1.\ nicht ebenfalls implementieren muss ($\must
      _{12}$ und $\may _{12}$ Definitionen in~\ref{ParallelDef}).\\
      $\Rightarrow$ es gibt passende $P'_1$ und $P'_2$ mit $P'_1\in\asimp (P_1)
      \land P'_2\in\asimp (P_2) \land P'=P'_1\|P'_2$.
  \end{enumerate}
\end{proof}

\begin{Lem}[w-as-Implementierungen und Parallelkomposition]\mbox{}
  \begin{enumerate}
    \item $P'_1\in\wasimp (P_1) \land P'_2 \in\wasimp (P_2) \Rightarrow
      (P'_1\|P'_2) \in\wasimp (P_1\|P_2)$.
    \item $P' \in\wasimp (P_1\|P_2) \Rightarrow \exists P'_1,P'_2 : P'_1
      \in\wasimp (P_1) \land P'_2 \in\wasimp (P_2) \land P'_1\|P'_2=P'$.
  \end{enumerate}
\end{Lem}

\begin{proof}\mbox{}
  \begin{enumerate}
    \item Es gelte $i\in\{1,2\}$. Jede Transition ($\may '_i =\must '_i$) einer
      sichtbaren Aktion in $P'_i$ hat eine entsprechende schwache
      may-Transition in $P_i$, da Definition~\ref{wSimDef} 2.\ gilt. Aufgrund
      der Simulations Definition~\ref{wSimDef}, haben $P_i$ und $P'_i$ die
      gleichen Signaturen (bezieht sich nur auf sichtbare Aktionen) und somit
      gilt $\Synch (P_1,P_2) = \Synch (P'_1,P'_2)$.\\
      Daraus folgt unter Verwendung, dass die Definitionen der must- und
      may"=Transitionen der Parallelkomposition in~\ref{ParallelDef} analog
      formuliert sind, dass alle sichtbaren Transitionen, die in $P'_1\|P'_2$
      enthalten sind auch in $P_1\|P_2$ als schwache may"=Transitionen
      vorhanden sind. Hierzu ist noch anzumerken, dass die interne Aktion nie
      in der Parallelkomposition synchronisiert wird und somit die \MEIO{}s
      diese in der Komposition jeweils für ihre Komponente alleine ausführen.\\
      $P'_1\|P'_2$ ist mit der gleichen Begründung wie im Beweis von
      Satz~\ref{impParallelLem} 1.\ eine Implementierung.\\
      $P_1\|P_2$ enthält nur must"=Transitionen, wenn auch $P_1$ bzw.\ $P_2$
      diese enthalten haben. $P'_1$ und $P'_2$ müssen diese must"=Transitionen
      aufgrund von Definition~\ref{SimDef} 1.\ schwach implementieren, d.h.\
      bei Inputs können danach noch interne Aktionen möglich sein und bei
      Outputs davor und danach und bei einen $\tau$ können beliebig viele
      interne Aktionen implementiert werden (auch keine). Die durch $P_1\|P_2$
      und~\ref{SimDef} 1.\ geforderten Implementierungen von must"=Transitionen
      in $P'_1\|P'_2$ basieren auf den must"=Transitionen der einzelnen
      \MEIO{}s, die schwach in deren w"=as"=Implementierungen implementiert
      wurden. Die zusätzlichen $\tau$s, die dadurch entstehen, hindern
      $P'_1\|P'_2$ in der Parallelkomposition nicht daran die
      must"=Transitionen auch entsprechend schwach zu implementieren.\\
      $\Rightarrow (P'_1\|P'_2)\in\wasimp (P_1\|P_2)$.
    \item Wie im Beweis zu 1.\ muss jede sichtbare Transition aus $P'$ in
      $P_1\|P_2$ als schwache may-Transition auftauchen
      (Definition~\ref{wSimDef} 2.). Bei der Projektion auf die einzelnen
      Transitionssysteme muss die schwache Transition im jeweiligen Automaten
      erhalten bleiben. Wobei sich jedoch die Anzahl der $\tau$s auf die beiden
      Systeme aufteilt. Die gleichen Signaturen von $P_i$ und $P'_i$ führen
      ebenso wieder zu den gleichen Synchronisationsmengen.\\
      $\Rightarrow$ Es kann $P'_1$ und $P'_2$ geben, die alle nötigen
      Transitionen implementierten, so dass $P'=P'_1\|P'_2$ gilt.\\
      Alle Transitionen, die in $P'$ aufgrund von Definition~\ref{SimDef} 1.\
      schwach implementiert werden müssen, mussten auch bereits in $P_1$ bzw.\
      $P_2$ als must"=Transitionen vorhanden sein. Da $P'_1$ und $P'_2$
      w"=as"=Implementierungen von $P_1$ und $P_2$ sind, müssen diese auch die
      Simulations Definition (\ref{SimDef}) erfüllen und somit die
      must"=Transitionen aus $P_1$ bzw.\ $P_2$ schwach implementieren. Hierbei
      muss auf die korrekte Anzahl an $\tau$s geachtet werden, die man dann in
      der Parallelkomposition erhält.\\
      Es kann nicht passieren, dass $P_1$ und $P_2$ must"=Transitionen
      enthalten, die $P'_1$ und $P'_2$ schwach implementieren müssen und dann
      in $P'_1\|P'_2$ zu einer mit einer sichtbaren Aktion beschriften
      Transition führen, die $P'$ auf Basis der Definition~\ref{SimDef} 1.\
      nicht ebenfalls implementieren muss ($\must _{12}$ und $\may _{12}$
      Definitionen in~\ref{ParallelDef}). Natürlich kann es durch die internen
      Aktionen zu Fehler kommen, da diese aber jedoch immer nur schwach
      implementiert werden müssen, kann man dort die Anzahl entsprechend
      regulieren um das gewünschte Ergebnis zu erhalten.\\
      $\Rightarrow$ es gibt passende $P'_1$ und $P'_2$ mit $P'_1\in\wasimp (P_1)
      \land P'_2\in\wasimp (P_2) \land P'=P'_1\|P'_2$.
  \end{enumerate}
\end{proof}

Ein neuer Kommunikationsfehler in einer Parallelkomposition muss in einer
Implementierung (as oder w-as) nicht auftauchen, auch nicht in der
Parallelkomposition von Implementierungen der einzelnen Komponenten. Dies liegt
daran, dass für den Input nur gesagt wird, dass keine must"=Transition für die
Synchronisation der Aktion vorhanden ist. Es kann trotzdem eine
may"=Transition für den Input geben, die auch implementiert werden kann.
Falls es aber in der Parallelkomposition zweier \MEIO{} zu einem neuen
Kommunikationsfehler kommt, dann gibt es auch immer mindestens eine
Implementierung, die diesen Kommunikationsfehler enthält und es gibt auch immer
mindestens ein Implementierungs-Paar der Komponenten, in deren
Parallelkomposition sich dieser Kommunikationsfehler ebenfalls zeigt.
