\chapter{allgemeine Folgerungen}

\begin{Prop}[Sprache und Implementierungen]
  \label{LImpProp}
  Für die Sprache eines \MEIO{}s $P$ gilt $L(P) \subseteq \left\{w\in
  \Sigma ^* \mid \exists P'\in\asimp (P) : p'_0\weakmust[w]_{P'} \right\} =
  \underset{P'\in\asimp (P)}{\bigcup} L(P')$.
\end{Prop}
\begin{proof}\mbox{}\\
  Sei $P'$ die as-Implementierung von $P$, die alle may-
  und must-Transitionen von $P$ implementiert. Die entsprechende starke
  as"=Verfeinerungs"=Relation $\mathcal{R}$, die zwischen $P'$ und $P$ gilt,
  ist die Identitäts-Relation zwischen den Zuständen der Transitionssysteme. Die
  Definition von $P'$ lautet dann:
  \begin{itemize}
    \item $P'=P$,
    \item $p'_0=p_0$,
    \item $I_{P'}=I_P$ und $O_{P'}=O_P$,
    \item $\must _{P'} =\may _{P'} = \may _P$,
    \item $E_P'=\emptyset$.
  \end{itemize}
  Für alle $p\in P$, die auch von nicht Fehler-Zuständen aus erreichbar sind,
  muss $p\mathcal{R} p'$ wegen~\ref{SimDef}~3.\ für entsprechenden $p'$ in
  $P'$, die durch die analogen Transitionen erreichbar sind. Diese Tupel sind
  bereits in der Identitäts-Relation, die als as"=Verfeinerungs"=Relation
  vorausgesetzt wurde, enthalten. Da $P'$ alle Transitionen von $P$
  implementiert, gilt~\ref{SimDef}~2.\ bereits durch die Tupel, die durch die
  Identitäts-Relation in $\mathcal{R}$ enthalten sein müssen. Der erste Punkt
  von~\ref{SimDef} gilt, da $E_P'$ leer ist. Für alle $w\in L(P) = \left\{w\in
  \Sigma ^* \mid p_0\weakmay[w]_P\right\}$ folgt nun $w\in L(P') = \left\{w\in
  \Sigma ^* \mid p'_0\weakmust[w]_{P'}\right\}$, da alle Transitionen von $P$
  in $P'$ implementiert werden.
\end{proof}

\begin{Prop}[Sprache der Parallelkomposition]
  \label{LParallelProp}
  Für zwei komponierbare \MEIO{}s $P_1$ und $P_2$ gilt: $L_{12} := L(P_{12}) =
  L_1\|L_2$.
\end{Prop}
\begin{proof}
  Jedes Wort, dass in $L_{12}$ enthalten ist, hat einen einen entsprechenden
  Ablauf, der in $P_{12}$ ausführbar ist. Dieser Ablauf kann auf Abläufe von
  $P_1$ und $P_2$ projiziert werden und die Projektionen sind dann in $L_1$ und
  $L_2$ enthalten.\\
  In einer Parallelkomposition werden die Wörter der beiden \MEIO{}s gemeinsam
  ausgeführt, falls es sich um synchronisierte Aktionen handelt, und
  verschränkt sequenziell, wenn es sich um unsynchronisierte Aktionen handelt.
  Somit sind alle Wörter aus $L_1\|L_2$ auch Wörter der Parallelkomposition
  $L(P_{12})$.
\end{proof}

\begin{Lem}[w-as-Verfeinerung und Parallelkomposition]
  \label{schwVerfParallelLem}
  Für zwei komponierbar \MEIO{}s $P_1$ und $P_2$ gilt, falls $P'_1$ und $P'_2$
  schwache as"=Verfeinerungen von $P_1$ bzw. $P_2$ sind, dann muss $P'_1\|P'_2$
  eine keine schwache as"=Verfeinerung von $P_1\|P_2$ sein. Jedoch für eine
  Relation $\mathcal{R}_{12}$ für die $((p'_1,p'_2),(p_1,p_2)) \in
  \mathcal{R}_{12}$ genau dann gilt, wenn $(p'_1,p_1)\in\mathcal{R}_1$ und
  $(p'_2,p_2)\in\mathcal{R}_2$ für die schwachen as"=Verfeinerungs"=Relationen
  von $P'_j$ auf $P_j$ für $j\in\{1,2\}$, dann sind für $\mathcal{R}_{12}$ die
  Punkte 2.\ bis 5.\ aus der Definition~\ref{wSimDef} erfüllt.
\end{Lem}
\begin{proof}
  Für alle folgenden Fälle wird $((p'_1,p'_2),(p_1,p_2))\in\mathcal{R}_{12}$
  mit $(p_1,p_2)\notin E_{12}$ vorausgesetzt.
  \begin{enumerate}
    \item Es gilt für diesen Punkt nicht unbedingt, dass $(p'_1,p'_2)$ kein
      Element von $E_{P'_1\|P'_2}$ ist. Deshalb ist dieser Punkt der Definition
      nicht erfüllt und $P'_1\|P'_2$ muss keine schwache as"=Verfeinerung von
      $P_1\|P_2$ sein.\\
      Dies folgt direkt aus der Voraussetzung, dass $(p_1,p_2)\notin
      E_{P_1\|P_2}$ für das Tupel $((p'_1,p'_2),(p_1,p_2))$ aus
      $\mathcal{R}_{12}$ gilt. In dem man auf das $\mathcal{R}_{12}$ die
      Definition von oben anwendet, erhält man $(p'_j,p_j)\in\mathcal{R}_j$ für
      beide $j$ Werte. Die $p_j$ dürfen beide keine Fehler-Zustände sein, da
      sonst auch $(p_1,p_2)$ ein solcher wäre. Somit folgt mit
      Definition~\ref{wSimDef}~1.\ $p'_j\notin E_j$ für beide $j$ Werte. Die
      beiden gestrichenen Zustände in Parallelkomposition können also keinen
      geerbten Fehler produzieren. Jedoch könnte $(p'_1,p'_2)$ aufgrund eines
      nicht erzwungenen Inputs ein neuer Fehler-Zustand sein. Dafür müsste
      \oBdA{} $p'_1\nmust[a]_{P'_1}$ und $p'_2\may[a]_{P'_2}$ für ein $a$ aus
      $I_1\cap O_2$ gelten. $\mathcal{R}_2$ erzwingt mit~\ref{wSimDef}~5.\ die
      schwache Ausführbarkeit des Outputs $a$ in $P_2$, d.h.\ $p_2
      \weakmay[a]_2$. Da dieser Output nur schwach ausführbar sein muss, kann
      es in der Parallelkomposition von $P'_1\|P'_2$ zu einem neuen
      Kommunikationsfehler kommen, der in $P_1\|P_2$ keiner ist.\\
      Ein Gegenbeispiel dafür ist in Abbildung~\ref{bsp1wSim} dargestellt.
      Hierfür soll $a$ in Schnitt der Inputs $I_1$ von $P_1$ bzw.\ $P'_1$ und
      der Outputs $O_2$ von $P_2$ bzw.\ $P'_2$ enthalten sein. Die Relation
      $\mathcal{R}_1$ enthält das Zustands-Tupel $(p'_{01},p_{01})$ und die
      $\mathcal{R}_2$ die Tupel $(p'_{02},p_{02})$ und $(p'_2,p_2)$. Somit ist
      $((p'_{01},p'_{02}),(p_{01},p_{02}))$ in $\mathcal{R}_{12}$ enthalten und
      es gilt $(p_{01},p_{02})\notin E_{12}$. Jedoch ist $(p'_{01},p'_{02})$
      trotzdem ein Fehler-Zustand in der Parallelkomposition von $P'_1$ und
      $P'_2$.

    \begin{figure}[htbp]
      \begin{center}
        \begin{tikzpicture}[shorten >=1pt,auto,node distance=2.5cm]
          \node [initial,initial text=$P'_1$:] (p'01) at (0,0) {$p'_{01}$};

          \node [initial,initial text=$P'_2$:] (p'02) at (7.8,0) {$p'_{02}$};
          \node (p'2) [right of=p'02] {$p'_2$};

          \path[->]
          (p'02) edge node{$a!$} (p'2)
          ;

          \node [initial, rectangle, draw, initial text=$P'_1\|P'_2$:] (p'0102)
          at (4,-1) {$p'_{01}\|p'_{02} \in E_{P'_1\|P'_2}$};

          \node [initial,initial text=$P_1$:] (p01) at (0,-2.5) {$p_{01}$};

          \node [initial,initial text=$P_2$:] (p02) at (7.8,-2.5) {$p_{02}$};
          \node (p) [right of=p02] {$p$};
          \node (p2) [right of=p] {$p_2$};

          \path[->]
          (p02) edge[dashed] node{$\tau$} (p)
          (p) edge[dashed] node{$a!$} (p2)
          ;

          \node [initial,initial text=$P_1\|P_2$:] (p0102) at (2.6,-3.5)
          {$p_{01}\|p_{02}$};
          \node (p01p) [rectangle, draw, right of=p0102] {$p_{01}\|p \in
          E_{12}$};

          \path[->]
          (p0102) edge[dashed] node{$\tau$} (p01p)
          ;
        \end{tikzpicture}
        \caption{Gegenbeispiel für 1.\ von $\mathcal{R}_{12}$ bzgl.\
        Definition~\ref{SimDef}}
        \label{bsp1wSim}
      \end{center}
    \end{figure}

    \item Aus der Definition der schwachen alternierenden
      Simulation in~\ref{wSimDef} folgt, dass für diesen Punkt zu zeigen ist:
      $(p_1,p_2)\must[i]_{12}(q_1,q_2)$ impliziert $(p'_1,p'_2)
      \must[i]_{P'_1\|P'_2} \weakmust[\varepsilon]_{P'_1\|P'_2} (q'_1,q'_2)$
      für ein $(q'_1,q'_2)$ mit $((q'_1,q'_2),(q_1,q_2)) \in
      \mathcal{R}_{12}$.\\
      Die $i$-must"=Transition in $P_1\|P_2$ kann entweder aus der
      Synchronisation von zwei must"=Inputs entstanden sein oder als
      unsynchronisierte Aktion aus einem $P_1$ übernommen worden sein.
      \begin{itemize}
        \item Fall 1 ($i\notin\Synch (P_1\|P_2)$): \OBdA{} ist $i$ in $I_1$
          enthalten. Es muss also in $P_1$ die $i$-Transition als
          must"=Transition von $p_1$ ausgehen, es gilt $p_1\must[i]_1 q_1$. Mit
          der Relation $\mathcal{R}_1$ und~\ref{wSimDef}~2.\ folgt, dass in
          $P'_1$ $i$ als schwache Transition in der Form $p'_1\must[i]_{P'_1}
          \weakmust[\varepsilon]_{P'_1}q'_1$ ausführbar sein musst und $q'_1
          \mathcal{R}_1q_1$ gelten muss. $p_2=q_2$ muss gelten, da $i$ nicht in
          $\Sigma _2$ enthalten ist. Aus der Voraussetzung folgt $(p'_2,p_2) =
          (q'_2,q_2) \in \mathcal{R}_2$, da $i$ wenn es kein Element der
          Aktionen von $P_2$ ist auch keine Aktion der schwachen
          as"=Verfeinerung $P'_2$ sein kann. Mit der Definition von
          $\mathcal{R}_{12}$ kann dann daraus $((q'_1,q'_2),(q_1,q_2)) \in
          \mathcal{R}_{12}$ gefolgert werden. In der Parallelkomposition von
          $P'_1$ und $P'_2$ entsteht die Transitionsfolge
          $(p'_1,p'_2)\must[i]_{P'_1\|P'_2} \weakmust[\varepsilon]_{P'_1\|P'_2}
          (q'_1,q'_2)$.
        \item Fall 2 ($i\in\Synch (P_1\|P_2)$): Damit $i$ auch in $P_1\|P_2$
          ein Input ist, muss $i\in I_1\cap I_2$ gelten. Um die Transition
          $(p_1,p_2)\must[i]_{12}(q_1,q_2)$ in der Komposition möglich zu
          machen, muss in beiden $P_j$'s $p_j\must[i]_j q_j$ gelten. Durch
          $\mathcal{R}_j$ und die Definition~\ref{wSimDef}~2.\ folgt für beide
          $j$ Werte $p'_j\must[i]_{P'_j} \weakmust[\varepsilon]_{P'_j}q'_j$ mit
          $(q'_j,q_j)\in\mathcal{R}_j$. Daraus ergibt sich
          $((q'_1,q'_2),(q_1,q_2)) \in \mathcal{R}_{12}$ mit der Definition von
          $\mathcal{R}_{12}$. Durch die Synchronisation der $i$-Inputs in der
          Komposition von $P'_1$ und $P'_2$ gilt $(p'_1,p'_2)
          \must[i]_{P'_1\|P'_2} \weakmust[\varepsilon]_{P'_1\|P'_2}
          (q'_1,q'_2)$.
      \end{itemize}
    \item Analog zu 2.\ kann für diesen Punkt $(p_1,p_2) \must[\omega]_{12}
      (q_1,q_2)$ impliziert $(p'_1,p'_2) \weakmust[\hat{\omega}]_{P'_1\|P'_2}
      (q'_1,q'_2)$ für ein $(q'_1,q'_2)$ mit $((q'_1,q'_2),(q_1,q_2)) \in
      \mathcal{R}_{12}$ gezeigt werden.\\
      Die $\omega$ Transition in $P_1\|P_2$ ist entweder aus einem
      synchronisierten oder aus einem unsynchronisierten $\omega$ entstanden.
      \begin{itemize}
        \item Fall 1 ($\omega\notin\Synch (P_1\|P_2)$): \OBdA{} ist $\omega$ in
          $O_1\cup\{\tau\}$ enthalten. Um in der Komposition $P_1\|P_2$ die
          must"=Transition zu erhalten muss bereits für die Transition in $P_1$
          $p_1 \must[\omega]_1 q_1$ gelten. Mit~\ref{wSimDef}~3.\ kann für
          $\mathcal{R}_1$ gefolgert werden, dass $p'_1
          \weakmust[\hat{\omega}]_{P'_1} q'_1$ mit $(q'_1,q_1)\in\mathcal{R}_1$
          gilt. In der Komposition folgt dann $(p'_1,p'_2)
          \weakmust[\hat{\omega}]_{P'_1\|P'_2} (q'_1,q'_2)$, da $\omega\notin
          \Sigma _2$ ist und somit $(p'_2,p_2)=(q'_2,q_2)\in\mathcal{R}_2$
          gilt. Es folgt insgesamt auch noch die Zugehörigkeit des
          Zustands-Tupels $((q'_1,q'_2),(q_1,q_2))$ zur Relation
          $\mathcal{R}_{12}$.
        \item Fall 2 ($\omega\in\Synch (P_1\|P_2)$): Da in der Menge $\Synch
          (P_1\|P_2)$ nur Inputs und Outputs enthalten sein können, muss in
          diesem Fall $\omega\neq\tau$ gelten. Um einen Output $\omega$ in der
          Parallelkomposition von $P_1$ und $P_2$ zu erhalten, muss \oBdA{}
          $\omega\in I_1\cap O_2$ gelten. Es folgt also $p_1\must[\omega]_1
          q_1$ mit $\omega\in I_1$ und $p_2\must[\omega]_2 q_2$ mit $\omega\in
          O_2$ für die einzeln Transitionssysteme. Mit $\mathcal{R}_1$
          und~\ref{wSimDef}~2.\ folgt $p'_1\must[\omega]_{P'_1}
          \weakmust[\varepsilon]_{P'_1} q'_1$ und $q'_1\mathcal{R}_1 q_1$. Wenn
          man $\mathcal{R}_2$ mit~\ref{wSimDef}~3.\ angewendet erhält man $p'_2
          \weakmust[\omega]_{P'_2} q'_2$ mit $q'_2\mathcal{R}_2 q_2$. Da
          $\omega$ in $P'_2$ ein Output ist, gilt $\omega =\hat{\omega}$. In
          der Parallelkomposition von $P'_1$ und $P'_2$ werden zuerst die
          internen Aktionen von $P'_2$ ausgeführt, bis dort der Output erreicht
          ist, dann wird $\omega$ synchronisiert und danach werden die internen
          Aktionen beider Komponenten ausgeführt, bis man bei den Zuständen
          $q'_1$ und $q'_2$ angekommen ist. Es folgt also die Transitionsfolge
          $(p'_1,p'_2) \weakmust[\hat{\omega}]_{P'_1\|P'_2} (q'_1,q'_2)$ und
          das Tupel $((q'_1,q'_2),(q_1,q_2))$ in der Relation
          $\mathcal{R}_{12}$.
      \end{itemize}
    \item $(p'_1,p'_2)\may[i]_{P'_1\|P'_2}(q'_1,q'_2)$ impliziert $(p_1,p_2)
      \may[i]_{12} \weakmay[\varepsilon]_{12} (q_1,q_2)$ für ein $(q_1,q_2)$
      mit $((q'_1,q'_2),(q_1,q_2))\in\mathcal{R}_{12}$ ist die Voraussetzung
      des 4.\ Punktes, um zu beweisen, dass $\mathcal{R}_{12}$ eine schwache
      as"=Verfeinerungs"=Relation, bis auf die Erfüllung von 1.\ aus der
      Definition~\ref{wSimDef}, ist.\\
      Die Transition $i$ kann wiederum durch Synchronisation von zwei
      Transitionen entstanden sein oder durch eine Transition aus einer der
      beiden Komponenten mit der Voraussetzung $i\notin\Synch (P'_1\|P'_2)$.
      \begin{itemize}
        \item Fall 1 ($i\notin\Synch (P'_1\|P'_2)$): \OBdA{} ist $i$ in $I_1$
          enthalten. Es muss also in $P'_1$ eine ausgehende $i$-Transition von
          Zustand $p'_1$ geben, so dass $p'_1\may[i]_1 q'_1$ gilt. Mit der
          Relation $\mathcal{R}_1$ und~\ref{wSimDef}~4.\ folgt, dass in $P_1$
          $i$ als schwache Transition in der Form $p_1\may[i]_1
          \weakmay[\varepsilon]_1 q_1$ ausführbar sein musst und $q'_1
          \mathcal{R}_1 q_1$ gelten muss. $p'_2=q'_2$ muss gelten, da $i$ nicht
          in $\Sigma _2$ enthalten ist. Aus der Voraussetzung folgt $(p'_2,p_2)
          =(q'_2,q_2) \in \mathcal{R}_2$, da $i$ wenn es kein Element der
          Aktionen von $P'_2$ ist auch keine Aktion der Spezifikation $P_2$
          sein kann. Mit der Definition von $\mathcal{R}_{12}$ kann dann daraus
          $((q'_1,q'_2),(q_1,q_2)) \in \mathcal{R}_{12}$ gefolgert werden. In
          der Parallelkomposition von $P_1$ und $P_2$ entsteht die
          Transitionsfolge $(p_1,p_2)\may[i]_{12} \weakmay[\varepsilon]_{12}
          (q_1,q_2)$.
        \item Fall 2 ($i\in\Synch (P'_1\|P'_2)$): Damit $i$ auch in
          $P'_1\|P'_2$ ein Input ist, muss $i\in I_1\cap I_2$ gelten. Um die
          Transition $(p'_1,p'_2)\may[i]_{P'_1\|P'_2}(q'_1,q'_2)$ in der
          Komposition möglich zu machen, muss in beiden Transitionssystemen
          $P'_j$ $p_j \may[i]_{P'_j} q'_j$ gelten. Durch $\mathcal{R}_j$ und
          die Definition~\ref{wSimDef}~4., die für diese Relationen gilt, folgt
          für beide $j$ Werte $p_j\may[i]_j \weakmay[\varepsilon]_j q_j$ mit
          $(q'_j,q_j)\in\mathcal{R}_j$. Es folgt $((q'_1,q'_2),(q_1,q_2)) \in
          \mathcal{R}_{12}$ mit der Definition von $\mathcal{R}_{12}$. Durch
          die Synchronisation des $i$'s in der Komposition von $P_1$ und $P_2$
          gilt $(p_1,p_2) \may[i]_{12} \weakmay[\varepsilon]_{12} (q_1,q_2)$.
      \end{itemize}
    \item Analog zu 3.\ und 4.\ kann für diesen Punkt $(p'_1,p'_2)
      \may[\omega]_{P'_1\|P'_2} (q'_1,q'_2)$ impliziert $(p_1,p_2)
      \weakmay[\hat{\omega}]_{12} (q_1,q_2)$ für ein $(q_1,q_2)$ mit
      $((q'_1,q'_2),(q_1,q_2))\in\mathcal{R}_{12}$ gezeigt werden.\\
      Die $\omega$ Transition in $P'_1\|P'_2$ ist entweder aus einem
      synchronisierten oder aus einem unsynchronisierten $\omega$ entstanden.
      \begin{itemize}
        \item Fall 1 ($\omega\notin\Synch (P'_1\|P'_2)$): \OBdA{} ist $\omega$
          in $O_1\cup\{\tau\}$ enthalten. Um in $P'_1\|P'_2$ die
          may"=Transition zu erhalten muss bereits in $P'_1$ die Transition
          $p'_1 \may[\omega]_{P'_1} q'_1$ möglich gewesen sein.
          Mit~\ref{wSimDef}~5.\ kann für $\mathcal{R}_1$ gefolgert werden, dass
          $p_1 \weakmay[\hat{\omega}]_1 q_1$ mit $(q'_1,q_1)\in\mathcal{R}_1$
          gilt. In der Komposition folgt dann $(p_1,p_2)
          \weakmay[\hat{\omega}]_{12} (q_1,q_2)$, da $\omega\notin \Sigma _2$
          ist und somit $(p'_2,p_2)=(q'_2,q_2)\in\mathcal{R}_2$ gilt. Es folgt
          insgesamt auch noch die Zugehörigkeit des Zustands-Tupels
          $((q'_1,q'_2),(q_1,q_2))$ zur Relation $\mathcal{R}_{12}$.
        \item Fall 2 ($\omega\in\Synch (P'_1\|P'_2)$): Es muss $\omega\neq\tau$
          gelten und somit muss \oBdA{} $\omega\in I_1\cap O_2$ gelten. Es
          folgt also $p'_1\may[\omega]_{P'_1} q'_1$ mit $\omega\in I_1$ und
          $p'_2\may[\omega]_{P'_2} q'_2$ mit $\omega\in O_2$ für die einzeln
          Transitionssysteme. Mit $\mathcal{R}_1$ und~\ref{wSimDef}~4.\ folgt
          $p_1\may[\omega]_1 \weakmay[\varepsilon]_1 q_1$ und $q'_1
          \mathcal{R}_1 q_1$. Wenn man $\mathcal{R}_2$ mit~\ref{wSimDef}~5.\
          angewendet erhält man $p_2 \weakmay[\omega]_2 q_2$ mit
          $q'_2\mathcal{R}_2 q_2$. Da $\omega$ in $P_2$ ein Output ist, gilt
          $\omega =\hat{\omega}$. In der Parallelkomposition von $P_1$ und
          $P_2$ werden zuerst die internen Aktionen von $P_2$ ausgeführt, bis
          dort der Output erreicht ist, dann wird $\omega$ synchronisiert und
          danach werden die internen Aktionen beider Komponenten ausgeführt,
          bis man bei den Zuständen $q_1$ und $q_2$ angekommen ist. Es folgt
          also die Transitionsfolge $(p_1,p_2) \weakmay[\hat{\omega}]_{12}
          (q_1,q_2)$ und das Tupel $((q'_1,q'_2),(q_1,q_2))$ in der Relation
          $\mathcal{R}_{12}$.
      \end{itemize}
  \end{enumerate}
\end{proof}

\begin{Kor}[as-Verfeinerungen und Parallelkomposition]
  \label{verfParallelKor}
  Für zwei komponierbar \MEIO{}s $P_1$ und $P_2$ gilt, falls $P'_1$ und $P'_2$
  as"=Verfeinerungen von $P_1$ bzw. $P_2$ sind, dann ist auch $P'_1\|P'_2$ eine
  as"=Verfeinerung von $P_1\|P_2$.
\end{Kor}
\begin{proof}
  Falls die Relationen $\mathcal{R}_1$ und $\mathcal{R}_2$ aus dem
  Lemma~\ref{schwVerfParallelLem} keine schwachen as"=Verfeinerungs"=Relationen
  sondern starke as"=Verfeinerungs"=Relation sind, ist auch $\mathcal{R}_{12}$
  eine starke as"=Verfeinerungs"=Relation zwischen $P'_1\|P'_2$ und $P_1\|P_2$.
  Es ist also nur zu zeigen, wie aus den einzelnen Beweispunkten des Beweises
  von~\ref{schwVerfParallelLem} folgt, dass $\mathcal{R}_{12}$ eine starke
  as"=Verfeinerungs"=Relation ist und zusätzlich, dass hier der erste Punkt
  gilt. Es wird hier ebenso für alle Punkte vorausgesetzt, dass
  $((p'_1,p'_2),(p_1,p_2))\in\mathcal{R}_{12}$ mit $(p_1,p_2)\notin E_{12}$
  gilt.
  \begin{enumerate}
    \item Dieser Punkt kann kann im Gegensatz zu 1.\ aus dem Beweis von
      Lemma~\ref{schwVerfParallelLem} bewiesen werden. Dies ist möglich, da für
      $p_2$ der Output $a$ nicht nur schwach sondern direkt ausführbar ist. Es
      ist also zu zeigen, dass $(p'_1,p'_2)$ kein Element von $E_{P'_1\|P'_2}$
      ist.\\
      Dies folgt direkt aus der Voraussetzung, dass $(p_1,p_2)\notin
      E_{P_1\|P_2}$ für das Tupel $((p'_1,p'_2),(p_1,p_2))$ aus
      $\mathcal{R}_{12}$ gilt. In dem man auf das $\mathcal{R}_{12}$ die
      Definition von oben anwendet, erhält man $(p'_j,p_j)\in\mathcal{R}_j$ für
      beide $j$ Werte. Die $p_j$ dürfen beide keine Fehler-Zustände sein, da
      sonst auch $(p_1,p_2)$ ein solcher wäre. Somit folgt mit
      Definition~\ref{SimDef}~1.\ $p'_j\notin E_j$ für beide $j$ Werte. Die
      beiden gestrichenen Zustände in Parallelkomposition können also keinen
      geerbten Fehler produzieren. Jedoch könnte $(p'_1,p'_2)$ aufgrund eines
      nicht sichergestellten Inputs ein neuer Fehler-Zustand sein. Dafür müsste
      \oBdA{} $p'_1\nmust[a]_{P'_1}$ und $p'_2\may[a]_{P'_2}$ für ein $a$ aus
      $I_1\cap O_2$ gelten. $\mathcal{R}_2$ erzwingt mit~\ref{SimDef}~3.\ die
      Ausführbarkeit des Outputs $a$ in $P_2$, d.h.\ $p_2 \may[a]_2$.
      Mit~\ref{SimDef}~2.\ von $\mathcal{R}_1$ folgt $p_1 \nmust[a]_1$. Somit
      müsste auch $(p_1,p_2)\in E_{12}$ gelten, was ein Widerspruch zur
      Voraussetzung wäre. $(p'_1,p'_2)$ kann also weder ein geerbter noch ein
      neuer Fehler-Zustand in $P'_1\|P'_2$ sein und deshalb gilt
      $(p'_1,p'_2)\notin E_{P'_1\|P'_2}$.
    \item $\alpha$ kann sowohl Input, Output wie auch interen Aktion sein. Um
      diesen Punkt zu beweisen muss man 2.\ und 3.\ aus dem Beweis von
      Lemma~\ref{schwVerfParallelLem} kombinieren. Da die $\mathcal{R}_j$ für
      $j\in\{1,2\}$ jedoch die Transition in den $P'_j$ ohne zusätzliche
      $\tau$-Transitionen fordern, entstehen in den einzelnen Komponenten keine
      schwachen Transitionen für die $\alpha$s und somit ist $\alpha$ auch in
      der Parallelkomposition $P'_1\|P'_2$ eine direkte Transition ohne
      zusätzliche $\tau$s. Es folgt also das zu zeigende für diesen die strake
      as"=Verfeinerungs"=Relation für diesen Punkt.
    \item Hierfür werden die Punkte 3.\ und 4.\ aus dem Beweis des
      Lemmas~\ref{schwVerfParallelLem} kombiniert. Analog wie bei 2.\ diese
      Beweises fallen die zusätzlichen $\tau$-Transitionen durch die stärkere
      Forderung an $\mathcal{R}_1$ und $\mathcal{R}_2$ weg. Dieser Punkt gilt
      also ebenfalls.
  \end{enumerate}
\end{proof}

\begin{Kor}[as-Implementierungen und Parallelkomposition]
  Für zwei komponierbare \MEIO{}s $P_1$ und $P_2$ gilt:
  $P'_1\in\asimp (P_1) \land P'_2 \in\asimp (P_2) \Rightarrow (P'_1\|P'_2)
  \in\asimp (P_1\|P_2)$.
\end{Kor}
\begin{proof}
  $P'_1$ und $P'_2$ sind Aufgrund der Definition~\ref{SimDef} auch starke
  as"=Verfeinerungen von $P_1$ bzw.\ $P_2$. Somit ist die Parallelkomposition
  $P'_1\|P'_2$ auch eine starke as"=Verfeinerung von $P_1\|P_2$, wegen
  Korollar~\ref{verfParallelKor}. Für Implementierungen gilt $\must =\may$.
  Durch die Definition der Parallelkomposition in~\ref{ParallelDef} können aus
  aus zwei komponierbaren Implementierungen in der Komposition keine
  may"=Transitionen ohne zugehörige must"=Transitionen entstehen. Es gilt also
  auch $\must _{P'_1\|P'_2} =\may _{P'_1\|P'_2}$ und somit ist $P'_1\|P'_2$
  eine Implementierung und eine as"=Verfeinerung von $P_1\|P_2$. Dies
  entspricht der Definition der starken as"=Implementierung, sodass
  $(P'_1\|P'_2)\in\asimp (P_1\|P_2)$ gilt.
\end{proof}

Die entgegengesetzte Richtung von Korollar~\ref{verfParallelKor} gilt im
allgemeinen nicht, d.h.\ es muss zu einer as"=Verfeinerung $P'$ einer
Parallelkomposition $P_1\|P_2$ keine as"=Verfeinerungen $P'_1$ bzw.\
$P'_2$ der einzelnen Komponenten $P_1$ bzw. $P_2$ geben, deren
Parallelkomposition $P'_1\|P'_2$ der as"=Verfeinerung der Parallelkomposition
$P'$ entsprechen. Die Problematik wird in Abbildung~\ref{impParallelFig} an
einem Beispiel dargestellt. In der Parallelkomposition wird die may"=Transition
von $P_2$ zu zwei may"=Transitionen, für die in einer as"=Verfeinerung
unabhängig entschieden werden kann, ob sie übernommen, implementiert oder
weggelassen werden. Somit kommt es in $P'$ zu dem Problem, dass keine
as"=Verfeinerung von $P_2$ (entweder keine Transition oder die $o'$ Transition
wird als may- oder must-Transition ausgeführt) in Parallelkomposition mit der
Implementierung $P_1$ $P'$ ergeben würde.\\
Da jede as"=Verfeinerungs"=Relation auch eine schwache
as"=Verfeinerungs"=Relation ist, folgt draus auch, dass die entgegengesetzt
Richtung von Lemma~\ref{schwVerfParallelLem} ebenfalls nicht gelten kann. Auch
im Spezialfall von as"=Implementierungen bzw. w-as"=Implementierungen kann das
Gegenbeispiel angewendet werden, da $P'$ auch eine Implementierung von
$P_1\|P_2$ ist und es auch keine passende as"=Implementierung bzw.
w-as"=Implementierung von $P_2$ geben kann, wenn es schon keine passende
Verfeinerung gibt.

\begin{figure}[htbp]
  \begin{center}
    \begin{tikzpicture}[shorten >=1pt,auto,node distance=2.5cm]
      \node [initial,initial text=$P_1$:] (p01) at (0,0) {$p_{01}$};
      \node (p1) [right of=p01] {$p_1$};

      \path[->]
      (p01) edge node{$o!$} (p1)
      ;

      \node [initial,initial text=$P_2$:] (p02) at (7,0) {$p_{02}$};
      \node (p2) [right of=p02] {$p_2$};

      \path[->]
      (p02) edge[dashed] node{$o'!$} (p2)
      ;

      \node [initial,initial text=$P_1\|P_2$:] (p0) at (0,-2)
      {$p_{01}\|p_{02}$};
      \node (p102) [right of=p0] {$p_1\|p_{02}$};
      \node (p012) [below of=p0] {$p_{01}\|p_2$};
      \node (p12) [below of=p102] {$p_1\|p_2$};

      \path[->]
      (p0) edge node{$o!$} (p102)
      (p012) edge node{$o!$} (p12)
      (p0) edge[dashed] node{$o'!$} (p012)
      (p102) edge[dashed] node{$o'!$} (p12)
      ;

      \node [initial,initial text=$P'$:] (p'0) at (7,-2)
      {$p'_{01}\|p'_{02}$};
      \node (p'102) [right of=p'0] {$p'_1\|p'_{02}$};
      \node (p'012) [below of=p'0] {$p'_{01}\|p'_2$};
      \node (p'12) [below of=p'102] {$p'_1\|p'_2$};

      \path[->]
      (p'0) edge node{$o!$} (p'102)
      (p'012) edge node{$o!$} (p'12)
      (p'0) edge node{$o'!$} (p'012)
      ;
    \end{tikzpicture}
    \caption{Gegenbeispiel für Umkehrung von Lemma~\ref{verfParallelKor}}
    \label{impParallelFig}
  \end{center}
\end{figure}

Ein neuer Fehler in einer Parallelkomposition zweier \MEIO{}s
muss in einer Implementierung (as oder w-as) dieser Parallelkomposition nicht
auftauchen, auch nicht in der Parallelkomposition von Implementierungen der
einzelnen Komponenten. Dies liegt daran, dass für den Input nur vorausgesetzt
wird, dass keine must"=Transition für die Synchronisation der Aktion vorhanden
ist. Es kann trotzdem eine may"=Transition für den Input geben, die auch
implementiert werden kann. Falls es aber in der Parallelkomposition zweier
\MEIO{} zu einem neuen Fehler kommt, dann gibt es auch immer mindestens eine
mögliche Implementierung, die diesen Fehler enthält und es gibt auch immer
mindestens ein Implementierungs-Paar der Komponenten, in deren
Parallelkomposition sich dieser Fehler ebenfalls zeigt.
