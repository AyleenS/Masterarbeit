\section{Allgemeine Folgerungen}
\label{saetzeKapitel}

\begin{Prop}[Sprache und Implementierungen]
  \label{LImpProp}
  Für die Sprache eines \MEIO{}s $P$ gilt $L(P) \subseteq \underset{P'\in\asimp
  (P)}{\bigcup} L(P')$.
\end{Prop}
\begin{proof}\mbox{}\\
  Sei $P'$ die as-Implementierung von $P$, die alle may- und must-Transitionen
  von $P$ implementiert. Die Definition von $P'$ lautet also:
  \begin{itemize}
    \item $P'=P$,
    \item $p'_0=p_0$,
    \item $I_{P'}=I_P$ und $O_{P'}=O_P$,
    \item $\must _{P'} =\may _{P'} = \may _P$,
    \item $E_{P'}=\emptyset$.
  \end{itemize}
  Die entsprechende starke as"=Verfeinerungs"=Relation $\mathcal{R}$, die
  zwischen $P'$ und $P$ gilt, ist die Identitäts-Relation zwischen den
  Zuständen der Transitionssysteme. Für~\ref{SimDef}.3 betrachten wir $(p,p)
  \in\mathcal{R}$ mit $p\may[\alpha]_{P'} p'$ in $P'$. Diese Transition muss
  auch in $P$ existieren und da $\mathcal{R}$ die Identitäts-Relation ist auch
  $(p',p')\in\mathcal{R}$ gelten. Alle must"=Transitionen in $P$ haben zugrunde
  liegende may"=Transitionen. Es gibt also zu jeder must"=Transition $p
  \must[\alpha]_P p'$ in $P$ auch die Transition $p\must[\alpha]_{P'} p'$. Es
  folgt dann auch $(p',p') \in \mathcal{R}$ und dadurch ist auch~\ref{SimDef}.2
  erfüllt. Der erste Punkt von~\ref{SimDef} gilt, da $E_{P'}$ leer ist.\\
  Für alle $w\in L(P) = \left\{w\in \Sigma ^* \mid p_0\weakmay[w]_P\right\}$
  gilt $w\in L(P') = \left\{w\in \Sigma ^* \mid p'_0 \weakmust[w]_{P'}
  \right\}$, da alle Transitionen von $P$ in $P'$ implementiert werden.
\end{proof}

\begin{Prop}[Sprache der Parallelkomposition]
  \label{LParallelProp}
  Für zwei komponierbare \MEIO{}s $P_1$ und $P_2$ gilt: $L_{12} := L(P_{12}) =
  L_1\|L_2$.
\end{Prop}
\begin{proof}
  Jedes Wort, dass in $L_{12}$ enthalten ist, hat einen einen entsprechenden
  Ablauf, der in $P_{12}$ ausführbar ist. Dieser Ablauf kann auf Abläufe von
  $P_1$ und $P_2$ projiziert werden und die Projektionen sind dann in $L_1$ und
  $L_2$ enthalten.\\
  In einer Parallelkomposition werden die Wörter der beiden \MEIO{}s gemeinsam
  ausgeführt, falls es sich um synchronisierte Aktionen handelt, und
  verschränkt sequenziell, wenn es sich um unsynchronisierte Aktionen handelt.
  Somit sind alle Wörter aus $L_1\|L_2$ auch Wörter der Parallelkomposition
  $L(P_{12})$.
\end{proof}

\begin{Lem}[w-as-Verfeinerung und Parallelkomposition]
  \label{schwVerfParallelLem}
  Seien $P_1$ und $P_2$ komponierbar \MEIO{}s und $P'_1$ eine schwache
  as"=Verfeinerung von $P_1$ aufgrund der schwachen as"=Verfeinerungs"=Relation
  $\mathcal{R}_1$. Dann gelten für die Relation $\mathcal{R}_{12} =
  \{((p'_1,p_2),(p_1,p_2))\mid (p'_1,p_1)\in\mathcal{R}_1, p_2\in P_2\}$ die
  Aussagen 2.\ bis 5.\ aus der Definition~\ref{wSimDef}.
\end{Lem}
\begin{proof}
  Um zu beweisen, dass $\mathcal{R}_{12}$ die Aussagen 2.\ bis 5.\ der
  Definition~\ref{wSimDef} erfüllt, müssen diese geprüft werden.\\
  Für alle folgenden Fälle wird $((p'_1,p_2),(p_1,p_2))$ immer aus
  $\mathcal{R}_{12}$ mit $(p_1,p_2)\notin E_{12}$ gewählt.
  \begin{enumerate}
    \setcounter{enumi}{1}
    \item Aus der Definition der schwachen alternierenden
      Simulation in~\ref{wSimDef} folgt, dass für diesen Punkt zu zeigen ist:
      $(p_1,p_2)\must[i]_{12}(q_1,q_2)$ impliziert $(p'_1,p_2)
      \must[i]_{P'_1\|P_2} \weakmust[\varepsilon]_{P'_1\|P_2} (q'_1,q_2)$
      für ein $(q'_1,q_2)$ mit $((q'_1,q_2),(q_1,q_2)) \in \mathcal{R}_{12}$.\\
      Die $i$-must"=Transition in $P_1\|P_2$ kann entweder aus der
      Synchronisation von zwei must"=Inputs entstanden sein oder als
      unsynchronisierte Aktion aus einem der komponierten \MEIO{}s übernommen
      worden sein.
      \begin{itemize}
        \item Fall 1 ($i\notin\Synch (P_1,P_2)$): Für den Fall $i$ in $I_2$ ist
          der Input in $P_2$ via must"=Transition ausführbar und somit sowohl
          in der Parallelkomposition $P'_1\|P_2$ wie auch in $P_1\|P_2$ als
          must"=Transition vorhanden. Es gilt dann auch $p_1=q_1$ und
          $p'_1=q'_1$, da $i$ kein Input von $P_1$ ist. Es gilt also die
          geforderte Transitionsfolge und das geforderte Element der Relation
          $\mathcal{R}_{12}$ direkt durch die Voraussetzungen.\\
          Für den Rest dieses Punktes wird davon ausgegangen, dass $i$ in $I_1$
          enthalten ist. Es muss also in $P_1$ die $i$-Transition als
          must"=Transition von $p_1$ ausgehen ($p_1\must[i]_1 q_1$). Mit
          der Relation $\mathcal{R}_1$ und~\ref{wSimDef}.2 folgt, dass in
          $P'_1$ $i$ als schwache Transition in der Form $p'_1\must[i]_{P'_1}
          \weakmust[\varepsilon]_{P'_1}q'_1$ ausführbar ist und $q'_1
          \mathcal{R}_1q_1$ gelten muss. $p_2=q_2 \in P_2$ muss erfüllt sein,
          da $i$ nicht in $\Sigma _2$ enthalten ist und $p_2$ nach
          Voraussetzung ein Zustand von $P_2$ sein muss. In der Komposition von
          $P'_1$ mit $P_2$ entsteht die Transitionsfolge $(p'_1,p_2)
          \must[i]_{P'_1\|P_2} \weakmust[\varepsilon]_{P'_1\|P_2} (q'_1,q_2)$.
          Mit der Definition von $\mathcal{R}_{12}$ kann daraus
          $((q'_1,q_2),(q_1,q_2)) \in \mathcal{R}_{12}$ gefolgert werden.
        \item Fall 2 ($i\in\Synch (P_1,P_2)$): Damit $i$ auch in $P_1\|P_2$
          ein Input ist, muss $i\in I_1\cap I_2$ gelten. Um die Transition
          $(p_1,p_2)\must[i]_{12}(q_1,q_2)$ in der Komposition möglich zu
          machen, muss in beiden Transitionssystemen $P_j$ $p_j\must[i]_j q_j$
          gelten. Durch $\mathcal{R}_1$ und die Definition~\ref{wSimDef}.2
          folgt $p'_1\must[i]_{P'_1} \weakmust[\varepsilon]_{P'_1}q'_1$ mit
          $(q'_1,q_1)\in\mathcal{R}_1$. Daraus ergibt sich
          $((q'_1,q_2),(q_1,q_2)) \in \mathcal{R}_{12}$ mit der Definition von
          $\mathcal{R}_{12}$. Durch die Synchronisation der $i$-Inputs von
          $P'_1$ und $P_2$ gilt $(p'_1,p_2) \must[i]_{P'_1\|P_2}
          \weakmust[\varepsilon]_{P'_1\|P_2} (q'_1,q_2)$.
      \end{itemize}
    \item Analog zu 2.\ kann für diesen Punkt $(p_1,p_2) \must[\omega]_{12}
      (q_1,q_2)$ impliziert $(p'_1,p_2) \weakmust[\hat{\omega}]_{P'_1\|P_2}
      (q'_1,q_2)$ für ein $(q'_1,q_2)$ mit $((q'_1,q_2),(q_1,q_2)) \in
      \mathcal{R}_{12}$ gezeigt werden.\\
      Die $\omega$-Transition in $P_1\|P_2$ ist entweder aus einem
      synchronisierten oder aus einem unsynchronisierten $\omega$ entstanden.
      \begin{itemize}
        \item Fall 1 ($\omega\notin\Synch (P_1,P_2)$): Der Fall $\omega$ ist
          eine Aktion, die in $P_2$ ausgeführt wird, verläuft analog zum Fall
          $i$ in $I_2$ im Fall 1 des zweiten Punkt dieses Beweises. Es sei also
          im folgenden $\omega$ ein Output oder eine interen Aktion von $P_1$.
          Um in der Komposition $P_1\|P_2$ die must"=Transition zu erhalten
          muss bereits für die Transition in $P_1$ $p_1 \must[\omega]_1 q_1$
          gelten sowie $p_2 = q_2$ in $P_2$. Mit~\ref{wSimDef}.3 kann für
          $\mathcal{R}_1$ gefolgert werden, dass $p'_1
          \weakmust[\hat{\omega}]_{P'_1} q'_1$ mit $(q'_1,q_1)\in\mathcal{R}_1$
          gilt. In der Komposition folgt dann $(p'_1,p_2)
          \weakmust[\hat{\omega}]_{P'_1\|P_2} (q'_1,q_2)$. Zusätzlich gilt auch
          die Zugehörigkeit des Zustands-Tupels $((q'_1,q_2),(q_1,q_2))$ zur
          Relation $\mathcal{R}_{12}$.
        \item Fall 2 ($\omega\in\Synch (P_1,P_2)$): Da in der Menge $\Synch
          (P_1,P_2)$ nur Inputs und Outputs enthalten sein können, muss in
          diesem Fall $\omega\neq\tau$ gelten. Um einen Output $\omega$ in der
          Parallelkomposition von $P_1$ und $P_2$ zu erhalten, muss entweder
          $\omega\in I_1\cap O_2$ oder $\omega\in O_1\cap I_2$ gelten. Für
          beide Fälle müssen die Transitionen $p_1\must[\omega]_1 q_1$ und
          $p_2\must[\omega]_2 q_2$ in den einzelnen Komponenten enthalten sein.
          Mit $\mathcal{R}_1$ und~\ref{wSimDef}.2 folgt im Fall $\omega\in
          I_1$ $p'_1\must[\omega]_{P'_1} \weakmust[\varepsilon]_{P'_1} q'_1$
          und $q'_1\mathcal{R}_1 q_1$. Im Fall $\omega\in O_1$ erhält man durch
          $\mathcal{R}_1$ und~\ref{wSimDef}.3 die Transition $p'_1
          \weakmust[\omega]_{P'_1} q'_1$ mit $q'_1\mathcal{R}_1 q_1$. Da
          $\omega$ in beiden Fällen keine interne Aktion ist, gilt $\omega
          =\hat{\omega}$. In der Parallelkomposition von $P'_1$ und $P_2$
          werden zuerst die internen Aktionen von $P'_1$ ausgeführt, falls
          diese existieren, bis dort die Aktion $\omega$ erreicht ist, dann
          wird $\omega$ synchronisiert und danach werden die restlichen
          internen Aktionen ausgeführt, bis man beim Zuständen $q_1$ angekommen
          ist. Es ergibt also die Transitionsfolge $(p'_1,p_2)
          \weakmust[\hat{\omega}]_{P'_1\|P_2} (q'_1,q_2)$ und das Tupel
          $((q'_1,q_2),(q_1,q_2))$ in der Relation $\mathcal{R}_{12}$.
      \end{itemize}
    \item $(p'_1,p_2)\may[i]_{P'_1\|P_2}(q'_1,q_2)$ impliziert $(p_1,p_2)
      \may[i]_{12} \weakmay[\varepsilon]_{12} (q_1,q_2)$ für ein $(q_1,q_2)$
      mit $((q'_1,q_2),(q_1,q_2))\in\mathcal{R}_{12}$ ist die Voraussetzung
      des 4.\ Punktes, um zu beweisen, dass $\mathcal{R}_{12}$ eine schwache
      as"=Verfeinerungs"=Relation, bis auf die Erfüllung von 1.\ aus der
      Definition~\ref{wSimDef}, ist.\\
      Die Transition $i$ kann durch Synchronisation von zwei
      Transitionen entstanden sein oder durch eine Transition aus einer der
      beiden Komponenten mit der Voraussetzung $i\notin\Synch (P'_1,P_2)$.
      \begin{itemize}
        \item Fall 1 ($i\notin\Synch (P'_1,P_2)$): Der Fall $i$ in $I_2$
          verläuft analog zum selben Fall im Fall 1 des Beweis des zweiten
          Punktes. Es muss nur must durch may ersetzt werden. Es kann also für
          den Rest diese Punktes davon ausgegangen werden, dass $i$ in $I_1$
          enthalten ist. Es muss in $P'_1$ eine ausgehende $i$-Transition von
          Zustand $p'_1$ geben, so dass $p'_1\may[i]_1 q'_1$ gilt. Mit der
          Relation $\mathcal{R}_1$ und~\ref{wSimDef}.4 folgt, dass in $P_1$
          $i$ als schwache Transition der Form $p_1\may[i]_1
          \weakmay[\varepsilon]_1 q_1$ ausführbar sein und $q'_1 \mathcal{R}_1
          q_1$ gelten muss. Mit der Definition von $\mathcal{R}_{12}$ kann dann
          $((q'_1,q_2),(q_1,q_2)) \in \mathcal{R}_{12}$ gefolgert werden. In
          der Parallelkomposition von $P_1$ und $P_2$ entsteht die
          Transitionsfolge $(p_1,p_2)\may[i]_{12} \weakmay[\varepsilon]_{12}
          (q_1,q_2)$ mit $p_2=q_2$.
        \item Fall 2 ($i\in\Synch (P'_1,P_2)$): Damit $i$ auch in
          $P'_1\|P_2$ ein Input ist, muss $i\in I_1\cap I_2$ gelten. Um die
          Transition $(p'_1,p_2)\may[i]_{P'_1\|P_2}(q'_1,q_2)$ in der
          Komposition möglich zu machen, muss in den Transitionssystemen
          $P'_1$ und $P_2$ $p'_1 \may[i]_{P'_1} q'_1$ bzw.\ $p_2\may[i]_2 q_2$
          gelten. Durch $\mathcal{R}_1$ und die Definition~\ref{wSimDef}.4,
          die für diese Relation gilt, folgt $p_1\may[i]_1
          \weakmay[\varepsilon]_1 q_1$ mit $(q'_1,q_1)\in\mathcal{R}_1$. Es
          gilt also $((q'_1,q_2),(q_1,q_2)) \in \mathcal{R}_{12}$. Durch die
          Synchronisation des Inputs $i$ in der Komposition von $P_1$ und $P_2$
          ergibt sich $(p_1,p_2) \may[i]_{12} \weakmay[\varepsilon]_{12}
          (q_1,q_2)$.
      \end{itemize}
    \item Analog zu 3.\ und 4.\ kann für diesen Punkt $(p'_1,p_2)
      \may[\omega]_{P'_1\|P_2} (q'_1,q_2)$ impliziert $(p_1,p_2)
      \weakmay[\hat{\omega}]_{12} (q_1,q_2)$ für ein $(q_1,q_2)$ mit
      $((q'_1,q_2),(q_1,q_2))\in\mathcal{R}_{12}$ gezeigt werden.\\
      Die $\omega$ Transition in $P'_1\|P_2$ ist entweder aus einem
      synchronisierten oder aus einem unsynchronisierten $\omega$ entstanden.
      \begin{itemize}
        \item Fall 1 ($\omega\notin\Synch (P'_1,P_2)$): Im Fall $\omega$ ist
          eine Aktion von $P_2$ folgt das zu zeigende direkt aus den
          Voraussetzungen, ebenso wie in allen vorangegangenen Punkten. Somit
          wird im folgenden davon ausgegangen, dass $\omega$ in $O_1$ enthalten
          oder eine interne Aktion ist, die von $P'_1$ geerbt wurde. Um in
          $P'_1\|P_2$ die may"=Transition zu erhalten muss also bereits in
          $P'_1$ die Transition $p'_1 \may[\omega]_{P'_1} q'_1$ möglich gewesen
          sein. Mit~\ref{wSimDef}.5 kann für $\mathcal{R}_1$ gefolgert werden,
          dass $p_1 \weakmay[\hat{\omega}]_1 q_1$ mit $(q'_1,q_1) \in
          \mathcal{R}_1$ gilt. Für die Komposition folgt daraus $(p_1,p_2)
          \weakmay[\hat{\omega}]_{12} (q_1,q_2)$ mit $p_2=q_2$. Es gilt auch
          die Zugehörigkeit des Zustands-Tupels $((q'_1,q_2),(q_1,q_2))$ zur
          Relation $\mathcal{R}_{12}$.
        \item Fall 2 ($\omega\in\Synch (P'_1,P_2)$): Es muss $\omega\neq\tau$
          gelten und somit können die Fälle $\omega\in I_1\cap O_2$ und
          $\omega\in O_1\cap I_1$ unterschieden werden. Es folgt in beiden
          Fällen $p'_1\may[\omega]_{P'_1} q'_1$ und $p_2 \may[\omega]_2
          q_2$. Mit $\mathcal{R}_1$ und~\ref{wSimDef}.4 folgt im Fall
          $\omega\in I_1$ $p_1\may[\omega]_1 \weakmay[\varepsilon]_1 q_1$ und
          $q'_1 \mathcal{R}_1 q_1$. Im Fall $\omega\in O_1$ wendet man
          $\mathcal{R}_1$ mit~\ref{wSimDef}.5 an und erhält $p_1
          \weakmay[\omega]_1 q_1$ und $q'_1\mathcal{R}_1 q_1$. Da $\omega$
          eine sichtbare Aktion ist, gilt $\omega =\hat{\omega}$. In der
          Parallelkomposition von $P_1$ und $P_2$ werden zuerst mögliche
          internen Aktionen von $P_1$ ausgeführt, bis dort die sichtbare Aktion
          erreicht ist, dann wird $\omega$ synchronisiert und danach werden die
          restlichen internen Aktionen ausgeführt, bis man beim Zustand
          $q_1$ angekommen ist. Es ergibt sich also die Transitionsfolge
          $(p_1,p_2) \weakmay[\hat{\omega}]_{12} (q_1,q_2)$ und das Tupel
          $((q'_1,q_2),(q_1,q_2))$ in der Relation $\mathcal{R}_{12}$.
      \end{itemize}
  \end{enumerate}
\end{proof}

\begin{Prop}[w-as-Verfeinerung und Parallelkomposition]
  \label{schwVerfParallelProp}
  Für zwei komponierbare \MEIO{}s $P_1$ und $P_2$ und eine schwache
  as"=Verfeinerung $P'_1$ von $P_1$ muss $P'_1\|P_2$ keine schwache
  as"=Verfeinerung von $P_1\|P_2$ sein. Spezielle erfüllt die Relation
  $\mathcal{R}_{12}$ aus dem Lemma~\ref{schwVerfParallelLem} den ersten Punkt
  der Definition~\ref{wSimDef} im allgemeinen nicht.
\end{Prop}
\begin{proof}
  Die Aussage 1.\ der Definition~\ref{wSimDef} ist im Allgemeinen für die
  Relation $\mathcal{R}_{12}$ aus~\ref{schwVerfParallelLem} nicht erfüllt. Es
  gilt also für ein $((p'_1,p_2),(p_1,p_2))$ aus $\mathcal{R}_{12}$ mit
  $(p_1,p_2)\notin E_{12}$ nicht unbedingt, dass $(p'_1,p_2)$ kein Element der
  Menge $E_{P'_1\|P_2}$ ist. $P'_1\|P_2$ muss also keine schwache
  as"=Verfeinerung von $P_1\|P_2$ sein.\\
  Die Voraussetzung besagt, dass $(p_1,p_2)\notin E_{12}$ für das
  Zustands-Tupel $((p'_1,p_2),(p_1,p_2))$ aus $\mathcal{R}_{12}$ gilt. Nach
  Definition von $\mathcal{R}_{12}$ erhält man $(p'_1,p_1)\in\mathcal{R}_1$ und
  $p_2\in P_2$. Die $p_j$ ($j\in\{1,2\}$) dürfen keine Fehler-Zustände sein, da
  sonst auch $(p_1,p_2)$ ein solcher wäre. Somit folgt mit
  Definition~\ref{wSimDef}.1 auch $p'_1\notin E_1$. Die Zustände $p'_1$ und
  $p_2$ vererben also keinen Fehler. Jedoch könnte $(p'_1,p_2)$ aufgrund eines
  nicht erzwungenen Inputs ein neuer Fehler-Zustand sein. Der nicht
  sichergestellte Input kann in beiden Systemen auftreten. Für den Fall, dass
  $P'_1$ einem vom Zustand $p'_1$ ausgehenden Output hat, für den $P_2$ im
  Zustand $p_2$ nicht den passenden Input sicherstellt gilt
  $p'_1\may[a]_{P'_1}$ und $p_2\nmust[a]_2$ für ein $a$ aus $O_1\cap I_2$.
  $\mathcal{R}_1$ erzwingt nach Definition nur die schwache Ausführbarkeit des
  Outputs $a$ in $P_1$ vom Zustand $p_1$ ausgehend, d.h.\ $p_1 \weakmay[a]_1$.
  Dadurch kann es in der Parallelkomposition von $P'_1\|P_2$ zu einem neuen
  Kommunikationsfehler kommen, der in $P_1\|P_2$ keiner ist.\\
  Um zu zeigen, dass dieser Fall wirklich auftreten kann, ist ein Beispiel mit
  diesem Verhalten in Abbildung~\ref{bsp1wSim} dargestellt. Dabei soll $a$ im
  Schnitt der Outputs $O_1$ von $P_1$ bzw.\ $P'_1$ und der Inputs $I_2$ von
  $P_2$ enthalten sein. Die Relation $\mathcal{R}_1$ enthält die Zustands-Tupel
  $(p'_{01},p_{01})$ und $(p'_1,p_1)$. Somit ist
  $((p'_{01},p_{02}),(p_{01},p_{02}))$ in $\mathcal{R}_{12}$ enthalten und es
  gilt $(p_{01},p_{02})\notin E_{12}$. Jedoch ist $(p'_{01},p_{02})$ trotzdem
  ein Fehler-Zustand in der Parallelkomposition von $P'_1$ und $P_2$.\\
  Es kann auch keine andere schwache as"=Verfeinerungs"=Relation $\mathcal{R}$
  für $P'_1\|P_2$ und $P_1\|P_2$ geben, da $(P'_1\|P_2) \mathcal{R} (P_1\|P_2)$
  nur gilt, wenn die Startzustände der Transitionssysteme in der Relation
  $\mathcal{R}$ stehen. Das Tupel $((p'_{01},p_{02}),(p_{01},p_{02}))$ muss
  also in $\mathcal{R}$ enthalten sein. Für diese Tupel ist jedoch der erste
  Punkt der Definition~\ref{wSimDef} nicht erfüllt mit der analogen Begründung
  wie für die Relation $\mathcal{R}_{12}$.

  \begin{figure}[h!tbp]
    \begin{center}
      \begin{tikzpicture}[->, >=latex', auto,node distance=2.5cm, semithick]
        \node [initial,initial text=$P'_1$:] (p'01) at (0,0) {$p'_{01}$};
        \node (p'1) [right of=p'01] {$p'_1$};

        \path
        (p'01) edge node{$a!$} (p'1)
        ;

        \node [initial,initial text=$P_1$:] (p01) at (7.6,0) {$p_{01}$};
        \node (p) [right of=p01] {$p$};
        \node (p1) [right of=p] {$p_1$};

        \path
        (p01) edge[dashed] node{$\tau$} (p)
        (p) edge[dashed] node{$a!$} (p1)
        ;

        \node [initial,initial text=$P_2$:] (p02) at (7.6,-1) {$p_{02}$};

        \node [initial, rectangle, draw, initial text=$P'_1\|P_2$:] (p'0102)
        at (1.3,-2) {$p'_{01}\|p_{02} \in E_{P'_1\|P_2}$};

        \node [initial,initial text=$P_1\|P_2$:] (p0102) at (8,-2)
        {$p_{01}\|p_{02}$};
        \node (p01p) [rectangle, draw, right of=p0102] {$p_{01}\|p \in
        E_{12}$};

        \path
        (p0102) edge[dashed] node{$\tau$} (p01p)
        ;
      \end{tikzpicture}
      \caption{Gegenbeispiel für 1.\ von $\mathcal{R}_{12}$ bzgl.\
      Definition~\ref{wSimDef} mit $a\in O_1\cap I_2$}
      \label{bsp1wSim}
    \end{center}
  \end{figure}

  Es wäre auch möglich, dass $P_1$ ebenfalls eine Implementierung ist. Die
  must"=$\tau$-Transition würde dann eine schwache Implementierung in $P'_1$
  fordern, jedoch muss es dafür keine echte Transition in $P'_1$ geben, da
  $\tau$ die interne Aktion ist. Die Implementierung von $a$ würde ebenfalls
  schwach gefordert werden, ist jedoch bereits stark vorhanden. $P_1\|P_2$
  würde mit der must"=$\tau$"=Transition dann ebenfalls zu einer
  Implementierung werden. Die must"=$a$"=Transition in $P'_1$ könnte auch eine
  may"=Transition sein, solange die $a$-Transition in $P_1$ keine
  must"=Transition ist.\\
  Um dieses Problem zu lösen könnte man die Definition der Parallelkomposition
  verändern. Es wäre denkbar, dass alle Zustände, die lokal Fehler-Zustände
  erreichen können ebenfalls bereits als Fehler angesehen werden. Jedoch wird
  erwartet dass die Definition der Parallelkomposition dann eine stärkere
  Forderung an die Transitionssysteme stellt. Für Implementierungen wäre die
  Forderung sogar stärkere, wie die der \EIO{}s in z.B.~\cite{Schinko2016BA}.
  Dieser Ansatz käme jedoch dem Vorgehen des Abschneidens der Fehler-Zustände
  mit ihren lokalen Vorgängern in~\cite{Vogler2016MIA3} näher.
\end{proof}


\begin{Kor}[as-Verfeinerungen und Parallelkomposition]
  \label{verfParallelKor}
  Für zwei komponierbar \MEIO{}s $P_1$ und $P_2$ gilt, falls $P'_1$ eine
  as"=Verfeinerung von $P_1$ ist, dann ist auch $P'_1\|P_2$ eine
  as"=Verfeinerung von $P_1\|P_2$.
\end{Kor}
\begin{proof}
  Falls die Relation $\mathcal{R}_1$ aus dem Lemma~\ref{schwVerfParallelLem}
  keine schwachen as"=Verfeinerungs"=Relation sondern eine starke
  as"=Verfeinerungs"=Relation ist, ist auch $\mathcal{R}_{12}$ eine
  as"=Verfeinerungs"=Relation zwischen $P'_1\|P_2$ und $P_1\|P_2$. Dazu ist
  also nur zu zeigen, wie aus den einzelnen Beweispunkten des Beweises
  von~\ref{schwVerfParallelLem} folgt, dass $\mathcal{R}_{12}$ eine starke
  as"=Verfeinerungs"=Relation ist und dass hier zusätzlich der erste Punkt
  erfüllt ist. Es wird hier ebenso für alle Punkte jeweils ein
  $((p'_1,p_2),(p_1,p_2))$ aus $\mathcal{R}_{12}$ mit $(p_1,p_2)\notin E_{12}$
  gewählt.
  \begin{enumerate}
    \item Dieser Punkt kann im Gegensatz zum ersten Punkt der
      Definition~\ref{wSimDef} für die schwache as"=Verfeinerungs"=Relation
      $\mathcal{R}_{12}$ aus Lemma~\ref{schwVerfParallelLem} für die starke
      as"=Verfeinerungs"=Relation bewiesen werden. Dies ist möglich, da für
      $p_1$ im Falle eines Outputs $a$ dieser nicht nur schwach sondern direkt
      ausführbar ist. Es ist also zu zeigen, dass $(p'_1,p_2)$ kein Element von
      $E_{P'_1\|P_2}$ ist.\\
      In dem man auf $\mathcal{R}_{12}$ die Definition anwendet, erhält man
      $(p'_1,p_1)\in\mathcal{R}_1$ und $p_2\in P_2$. Die $p_j$ dürfen beide
      keine Fehler-Zustände sein, da sonst auch $(p_1,p_2)$ ein solcher wäre.
      Somit folgt mit Definition~\ref{SimDef}.1 $p'_1\notin E_{P'_1}$. Die
      Zustände $p'_1$ und $p_2$ in Parallelkomposition können also keinen
      geerbten Fehler produzieren. Jedoch könnte $(p'_1,p_2)$ aufgrund eines
      nicht sichergestellten Inputs ein neuer Fehler-Zustand sein. Dafür müsste
      entweder $p'_1 \nmust[a]_{P'_1}$ und $p_2\may[a]_2$ für ein $a$ aus
      $I_1\cap O_2$ oder $p'_1\may[a]_{P'_1}$ und $p_2\nmust[a]_2$ für ein $a$
      aus $O_1\cap I_2$ gelten. Mit~\ref{SimDef}.2 und $\mathcal{R}_1$ folgt im
      Fall $a\in I_1$ $p_1 \nmust[a]_1$. $\mathcal{R}_1$ erzwingt
      mit~\ref{SimDef}.3 die direkte Ausführbarkeit des Outputs $a$ in $P_1$ im
      Fall $a\in O_1$, d.h.\ $p_1 \may[a]_1$. Somit müsste auch $(p_1,p_2)\in
      E_{12}$ in beiden Fällen gelten, was ein Widerspruch zur Voraussetzung
      wäre. $(p'_1,p'_2)$ kann also weder ein geerbter noch ein neuer Fehler in
      $P'_1\|P_2$ sein und deshalb gilt $(p'_1,p_2) \notin E_{P'_1\|P_2}$.
    \item $\alpha$ kann sowohl Input, Output wie auch interen Aktion sein. Um
      diesen Punkt zu beweisen muss man 2.\ und 3.\ aus dem Beweis von
      Lemma~\ref{schwVerfParallelLem} kombinieren. Da $\mathcal{R}_1$ die
      Transition in $P'_1$ ohne zusätzliche $\tau$-Transitionen fordern,
      entstehen keine schwachen Transitionen für die $\alpha$s und somit ist
      $\alpha$ auch in der Parallelkomposition $P'_1\|P_2$ eine direkte
      Transition ohne zusätzliche $\tau$s. Es folgt das $\mathcal{R}_{12}$ die
      Forderungen für die starke as"=Verfeinerungs"=Relation dieses Punktes
      erfüllt.
    \item Hierfür werden die Punkte 3.\ und 4.\ aus dem Beweis des
      Lemmas~\ref{schwVerfParallelLem} kombiniert. Analog wie bei 2.\ diese
      Beweises fallen die zusätzlichen $\tau$-Transitionen durch die stärkere
      Forderung an $\mathcal{R}_1$ weg. Dieser Punkt gilt also ebenfalls.
  \end{enumerate}
\end{proof}

Die drei vorangegangenen Ergebnisse fordern nur die Verfeinerung der ersten
Komponente. Die Parallelkomposition wurde so definiert, dass sie kommutativ
ist. Somit ist ebenso die Verfeinerung der zweiten Komponente möglich. Da man
beide Komponenten nach einander verfeinern kann und jede Verfeinerung einer
Verfeinerung auch eine Verfeinerung das ursprünglichen Systems ist, kann man
auch beide Komponenten gleichzeitig verfeinern und erhält in der
Parallelkomposition die gleiche Verfeinerung.

\begin{Kor}[as-Implementierungen und Parallelkomposition]
  \label{ImpParallelKor}
  Für zwei komponierbare \MEIO{}s $P_1$ und $P_2$ gilt:
  $P'_1\in\asimp (P_1) \land P'_2 \in\asimp (P_2) \Rightarrow (P'_1\|P'_2)
  \in\asimp (P_1\|P_2)$.
\end{Kor}
\begin{proof}
  $P'_1$ und $P'_2$ sind aufgrund der Definition~\ref{SimDef} auch starke
  as"=Verfeinerungen von $P_1$ bzw.\ $P_2$. Somit ist die Parallelkomposition
  $P'_1\|P'_2$ auch eine starke as"=Verfeinerung von $P_1\|P_2$, nach
  Korollar~\ref{verfParallelKor}. Für Implementierungen gilt $\must =\may$.
  Durch die Definition der Parallelkomposition in~\ref{ParallelDef} können
  aus zwei komponierbaren Implementierungen in der Komposition keine
  may"=Transitionen ohne zugehörige must"=Transitionen entstehen. Es gilt also
  auch $\must _{P'_1\|P'_2} =\may _{P'_1\|P'_2}$ und somit ist $P'_1\|P'_2$
  eine Implementierung und eine as"=Verfeinerung von $P_1\|P_2$. Dies
  entspricht der Definition der starken as"=Implementierung, sodass
  $(P'_1\|P'_2)\in\asimp (P_1\|P_2)$ gilt.
\end{proof}

Für schwache as"=Implementierungen kann es kein analoges Korollar
zu~\ref{ImpParallelKor} gegen, da die Verfeinerung im allgemeinen bereits
scheitert (Proposition~\ref{schwVerfParallelProp}); man beachte auch die
Diskussion des Beispiels~\ref{bsp1wSim}. Die Parallelkomposition von
Implementierungen ist jedoch immer eine Implementierung. Somit würde in den
Fällen, in denen auch~\ref{wSimDef}.1 erfüllt ist für die Parallelkomposition
schwacher as"=Implementierungen, eine analoge Aussage gelten.\\
Die umgekehrte Richtung von Korollar~\ref{verfParallelKor} gilt im
allgemeinen nicht, d.h.\ es muss zu einer as"=Verfeinerung $P'$ einer
Parallelkomposition $P_1\|P_2$ keine as"=Verfeinerungen $P'_1$ und $P'_2$ der
einzelnen Komponenten geben, deren Parallelkomposition $P'_1\|P'_2$ der
as"=Verfeinerung der Parallelkomposition $P'$ entsprechen. Die Problematik wird
in Abbildung~\ref{impParallelFig} an einem Beispiel dargestellt. In der
Parallelkomposition wird die may"=Transition von $P_2$ zu zwei
may"=Transitionen, für die in einer as"=Verfeinerung unabhängig entschieden
werden kann, ob sie übernommen, implementiert oder weggelassen werden. Für eine
as"=Verfeinerung von $P_2$ ist es nur möglich, dass keine Transition umgesetzt
wird oder die $o'$ Transition entweder als may- oder must"=Transition in die
Verfeinerung übernommen wird. Somit kommt es in $P'$ zu dem Problem, dass keine
as"=Verfeinerung von $P_2$ in Parallelkomposition mit der Implementierung
$P_1$ den geforderten \MEIO{} $P'$ ergeben würde.\\
Auch im Spezialfall von as"=Implementierungen kann dieses Gegenbeispiel
angewendet werden, da $P'$ auch eine Implementierung von $P_1\|P_2$ ist und es
auch keine passende as"=Implementierung von $P_2$ geben kann, wenn es schon
keine passende Verfeinerung gibt.

\begin{figure}[htbp]
  \begin{center}
    \begin{tikzpicture}[->, >=latex', auto,node distance=2.5cm, semithick]
      \node [initial,initial text=$P_1$:] (p01) at (0,0) {$p_{01}$};
      \node (p1) [right of=p01] {$p_1$};

      \path
      (p01) edge node{$o!$} (p1)
      ;

      \node [initial,initial text=$P_2$:] (p02) at (7,0) {$p_{02}$};
      \node (p2) [right of=p02] {$p_2$};

      \path
      (p02) edge[dashed] node{$o'!$} (p2)
      ;

      \node [initial,initial text=$P'$:] (p'0) at (0,-2)
      {$p'_{01}\|p'_{02}$};
      \node (p'102) [right of=p'0] {$p'_1\|p'_{02}$};
      \node (p'012) [below of=p'0] {$p'_{01}\|p'_2$};
      \node (p'12) [below of=p'102] {$p'_1\|p'_2$};

      \path
      (p'0) edge node{$o!$} (p'102)
      (p'012) edge node{$o!$} (p'12)
      (p'0) edge node{$o'!$} (p'012)
      ;

      \node [initial,initial text=$P_1\|P_2$:] (p0) at (7,-2)
      {$p_{01}\|p_{02}$};
      \node (p102) [right of=p0] {$p_1\|p_{02}$};
      \node (p012) [below of=p0] {$p_{01}\|p_2$};
      \node (p12) [below of=p102] {$p_1\|p_2$};

      \path
      (p0) edge node{$o!$} (p102)
      (p012) edge node{$o!$} (p12)
      (p0) edge[dashed] node{$o'!$} (p012)
      (p102) edge[dashed] node{$o'!$} (p12)
      ;
    \end{tikzpicture}
    \caption{Gegenbeispiel für Umkehrung von Lemma~\ref{verfParallelKor}}
    \label{impParallelFig}
  \end{center}
\end{figure}

Ein neuer Fehler in einer Parallelkomposition zweier \MEIO{}s
muss in einer Implementierung (as oder w-as) dieser Parallelkomposition nicht
auftauchen, auch nicht in der Parallelkomposition von Implementierungen der
einzelnen Komponenten. Dies liegt daran, dass für den Input nur vorausgesetzt
wird, dass keine must"=Transition für die Synchronisation der Aktion vorhanden
ist. Es kann trotzdem eine may"=Transition für den Input geben, die auch
implementiert werden kann. Falls es aber in der Parallelkomposition zweier
\MEIO{} zu einem neuen Fehler kommt, dann gibt es auch immer mindestens eine
mögliche Implementierung, die diesen Fehler enthält und es gibt auch immer
mindestens ein Implementierungs-Paar der Komponenten, in deren
Parallelkomposition sich dieser Fehler ebenfalls zeigt.

Lemma~\ref{schwVerfParallelLem} und Korollar~\ref{verfParallelKor} lassen schon
die Vermutung zu, dass es einen Zusammenhang zwischen den beiden
Verfeinerungs"=Relation \wasRel{} und \asRel{} gibt. Durch
Proposition~\ref{schwVerfParallelProp} fällt jedoch schon auf, dass es Stellen
gibt, an denen sich die Relationen unterscheiden. Durch die Definitionen wird
klar, das der Unterschied der beiden Relationen in den $\tau$-Transitionen
liegt. Die nächste Aussage, dass jede starke as"=Verfeinerungs"=Relation auch
eine schwache ist, jedoch die Umkehrung nicht gilt sollte somit nicht
überraschen.

\begin{Lem}[Zusammenhang der Verfeinerungs"=Relationen]
  \label{ZusammenhWasAsLem}
  Jede starke as"=Verfeinerungs"=Relation ist auch eine schwache
  as"=Verfeinerungs"=Relation. Jedoch ist eine schwache
  as"=Verfeinerungs"=Relation im allgemeinen keine starke
  as"=Verfeinerungs"=Relation. Es gilt also: $P \asRel Q \Rightarrow P \wasRel
  Q$ und im allgemeinen $P \asRel Q \hspace{0.1cm}\not\hspace{-0.1cm}\Leftarrow
  P \wasRel Q$.
\end{Lem}

\begin{proof}\mbox{}\\
  \glqq $\Rightarrow$\grqq{}:\\
  Um diese Implikation zu zeigen, muss man nachweisen, dass jede starke
  as"=Verfeinerungs"=Relation auch die Definition~\ref{wSimDef} der schwachen
  as"=Verfeinerungs"=Relation erfüllt. In beiden Simulations-Definitionen
  (\ref{SimDef} und~\ref{wSimDef}) müssen die Punkte für alle $(p,q) \in
  \mathcal{R}$ mit $q\notin E_Q$ gelten. Sei $\mathcal{R}$ nun eine
  as"=Verfeinerungs"=Relation. Es gilt also mit~\ref{SimDef}.1, dass $p$ kein
  Fehler-Zustand von $P$ ist. Somit ist auch 1.\ von~\ref{wSimDef} erfüllt. Für
  alle $\alpha\in\Sigma _{\tau}$ impliziert~\ref{SimDef}.2 $q \must[\alpha]_Q
  q'$ $p\must[\alpha]_P p'$ für ein $p'$ mit $p'\mathcal{R} q'$. Da $\Sigma
  _{\tau} = I \cup O \cup \{\tau\}$ gilt, sind dadurch 2.\ und 3.\ der
  Definition~\ref{wSimDef} erfüllt. Die schwache $\varepsilon$-Transition aus
  2.\ führt keine echten Transitionen aus, sondern bleibt beim Zustand $p'$.
  Die schwache $\hat{\omega}$-Transition aus 3.\ entspricht in $\mathcal{R}$
  nur einer einzigen Transition für $\omega$. Die Punkte 4.\ und 5.\ aus
  Definition~\ref{wSimDef} werden durch~\ref{SimDef}.3 erfüllt. Es gilt für
  $\mathcal{R}$ $p\may[\alpha]_P p'$ impliziert $q \may[\alpha]_Q q'$ für ein
  $q'$ mit $p'\mathcal{R}q'$. Die in~\ref{wSimDef} geforderten schwachen
  may"=Transitionen werden hier jeweils stark durch eine einzige Transition
  umgesetzt. $\mathcal{R}$ ist also auch eine schwache
  as"=Verfeinerungs"=Relation.

  \glqq $\hspace{0.1cm}\not\hspace{-0.1cm}\Leftarrow$\grqq{}:\\
  Im Abbildung~\ref{asWasGegenBsp} wird ein Gegenbeispiel dargestellt mit einem
  \MEIO{} $Q$ und einer schwachen as"=Verfeinerung $P$ von $Q$, die jedoch
  keine starke as"=Verfeinerung von $Q$ ist. Die schwache
  as"=Verfeinerungs"=Relationen $\mathcal{R}$ zwischen $P$ und $Q$ enthält die
  Tupel $(p_0,q_0)$ und $(p_1,q_{12})$. Damit $\mathcal{R}$ eine schwache
  Simulations"=Relation zwischen $P$ und $Q$ sein kann müssen die Startzustände
  in Relation stehen. Dies ist durch $(p_0,q_0)\in \mathcal{R}$ erfüllt.
  Es sind keine Fehler-Zustände in $Q$ und $P$ enthalten, somit ist 1.\ der
  Definition~\ref{wSimDef} bereits für beide Zustands-Tupel erfüllt. Für das
  Tupel $(p_1,q_{12})$ sind auch 2.-5.\ von~\ref{wSimDef} erfüllt, da weder
  $p_1$ noch $q_{12}$ ausgehende Transitionen besitzen. Für $(p_0,q_0)\in
  \mathcal{R}$ gibt es keine ausgehende must"=Transitionen. Also ist 2.\ und
  3.\ von~\ref{wSimDef} bereits erfüllt. Falls $\alpha$ ein Input ist,
  fordert~\ref{wSimDef}.4, dass die Transition $p_0 \may[\alpha]_P p_1$ in $Q$
  schwach ausführbar ist in der Form $q_0 \may[\alpha]_Q
  \weakmay[\varepsilon]_Q q$. Ein entsprechendes $q$ ist in diesem Fall
  $q_{12}$ und es gilt $p_1 \mathcal{R} q_{12}$. Falls $\alpha$ eine lokale
  Aktion ist, lautet die Forderung $q_0 \weakmay[\hat{\alpha}]_Q q$ für $Q$ und
  $q_{12}$ ist wieder der passende Zustand für $q$, der mit $p_1$ in Relation
  stehen. $\mathcal{R}$ ist also eine schwache as"=Verfeinerungs"=Relation
  zwischen $P$ und $Q$.\\
  Angenommen es gibt auch eine starke as"=Verfeinerungs"=Relation
  $\mathcal{R}'$ zwischen $P$ und $Q$, dann muss $p_0 \mathcal{R}' q_0$
  gelten. Mit~\ref{SimDef}.3 wird gefordert, dass die Transition $p_0
  \may[\alpha]_P p_1$ durch eine Transition der Form $q_0 \may[\alpha]_Q q$
  in $Q$ gematched werden muss. Für den Zustand $q$ kommt dieses mal nur
  $q_{11}$ in Frage. Es muss also $(p_1,q_{11})\in \mathcal{R}$ gelten. Der
  zweite Punkt der Definition~\ref{SimDef} fordert, dass die
  $\tau$-must"=Transition aus $Q$ auch in $P$ auftauchen muss. Es müsste also
  ein $p$ geben, für dass $p_1 \must[\tau]_P p$ gilt und das Tupel $(p,q_{12})$
  müsste in $\mathcal{R}$ enthalten sein. Da es keine solche Transition gibt,
  tritt ein Widerspruch zur Annahme auf. Es kann also keine starke
  as"=Verfeinerungs"=Relation zwischen $P$ und $Q$ geben.

  \begin{figure}[htbp]
    \begin{center}
      \begin{tikzpicture}[->, >=latex', auto,node distance=2.5cm, semithick]
        \node [initial,initial text=$P$:] (p0) at (0,0) {$p_0$};
        \node (p1) [right of=p0] {$p_1$};

        \path
        (p0) edge[dashed] node{$\alpha$} (p1)
        ;

        \node [initial,initial text=$Q$:] (q0) at (7,0) {$q_0$};
        \node (q11) [right of=q0] {$q_{11}$};
        \node (q12) [right of=q11] {$q_{12}$};

        \path
        (q0) edge[dashed] node{$\alpha$} (q11)
        (q11) edge node{$\tau$} (q12)
        ;
      \end{tikzpicture}
      \caption{Gegenbeispiel zu $\asRel \Leftarrow \wasRel$}
      \label{asWasGegenBsp}
    \end{center}
  \end{figure}
\end{proof}

