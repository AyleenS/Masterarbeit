\chapter{allgemeine Folgerungen}

\TODO{Kommentare nach Gespräch löschen}

% \begin{Prop}[Sprache und Implementierung]
%   \label{LImpProp}
%   Die Sprache eines \MEIO{}s $P$ entspricht $L(P) = \left\{w\in
%   \Sigma ^* \mid \exists P'\in\asimp (P) : p'_0\weakmust[w]_{P'} \right\}$.
% \end{Prop}
% \begin{proof}\mbox{}\\
%   \glqq$\supset$\grqq{}: Für ein $w\in \left\{w\in \Sigma ^* \mid \exists
%   P'\in\asimp (P) : p'_0\weakmust[w]_{P'} \right\}$ gibt es ein
%   as"=Implementierung $P'$ von $P$, für die $\exists
%   p'_1,p'_2,\dots,p'_{n-1},p'_n \exists w'\in\Sigma _{\tau} ^* : \hat{w'} = w
%   \land w'=\alpha _1\alpha _2\dots\alpha _n \land p'_0\must[\alpha _1]_{P'}
%   p'_1\must[\alpha _2]_{P'} \dots p'_{n-1}\must[\alpha _n]_{P'} p'_n$ nach
%   der Definition der Transitions-Notation gilt. $P'$ muss als
%   as"=Implementierung von $P$ in der straken Verfeinerungsrelation \asRel{} mit
%   $P$ stehen. Es gilt also $p'_0\asRel p_0$. Somit folgt für alle $0\leq j <
%   n$, falls $p_j$ nicht in $E_P$ enthalten ist, mit Definition~\ref{SimDef} 3.
%   $p'_{j+1}\asRel p_{j+1}$ mit $p_j \may[\alpha_{j+1}]_P p_{j+1}$. Falls $p_j$
%   bis $j = n-1$ nicht in $E_P$ enthalten war, folgt dadurch, dass eine
%   Transitionsfolge in $P$ ausführbar ist, die Aktionen ausführt, die sich zu
%   dem sichtbaren Wort $w$ zusammenfügen. Es ist jedoch möglicherweise nur ein
%   Präfix von $w$ als Trace in $P$ enthalten, falls bereits für ein $j < n-1$
%   $p_j\in E_P$ gilt. Jedoch gilt dann für das Präfix $\hat{v}$ von $w$
%   $\hat{v}\in \StET (P)$ mit $v=\alpha _1\alpha _2\dots\alpha _j$. $w$ ist dann
%   in $\cont (\StET (P))$ enthalten. In beiden Fällen gilt $w\in \left\{w\in
%   \Sigma ^* \mid p_0\weakmay[w]_P p\notin E\right\} \cup \cont (\StET (P)) =
%   L(P)$.

%   \glqq$\subset$\grqq{}: Sei $P'$ die as-Implementierung von $P$, die alle may-
%   und must-Transitionen von $P$ implementiert und die zusätzlich an alle
%   Zustände aus $E_P$ eine Schleife für alle Aktionen aus $\Sigma _P$ anfügt.
%   Die Definition von $P'$ lautet dann:
%   \begin{itemize}
%     \item $P'=P$ für die Menge der Zustände,
%     \item $p'_0=p_0$,
%     \item $I_{P'}=I_P$ und $O_{P'}=O_P$,
%     \item $\must _{P'} =\may _{P'} =\may _P \cup \{(e,a,e)\mid e\in E_P,
%       a\in\Sigma _P\}$,
%     \item $E_P'=\emptyset$.
%   \end{itemize}
%   Für alle $p\in P$ gilt $p\asRel p'$ wegen~\ref{SimDef} 3.\ für entsprechenden
%   $p'$ in $P'$, die durch die analogen Transitionen erreichbar sind. Da $P'$
%   alle Transitionen von $P$ implementiert, gilt~\ref{SimDef} 2.\ bereits durch
%   die Tupel, die durch~\ref{SimDef} 3.\ in $\asRel$ enthalten sein müssen. Der
%   erste Punkt von~\ref{SimDef} gilt, da $E_P'$ leer ist. Für alle $w\in L(P) =
%   \left\{w\in \Sigma ^* \mid p_0\weakmay[w]_P p\notin E\right\} \cup \cont
%   (\StET (P))$ folgt nun $w\in L(P') = \left\{w\in \Sigma ^* \mid
%   p'_0\weakmust[w]_{P'}\right\}$. Für Wörter $w$ aus $\left\{w\in \Sigma ^*
%   \mid\right.$ $\left.p_0\weakmay[w]_P p\notin E\right\}$ folgt dies direkt aus
%   der Implementierung aller may"=Transitionen von $P$ in $P'$. Falls jedoch ein
%   Präfix von $w$ in $\StET (P)$ enthalten ist, ist dies Präfix aufgrund der
%   Transitions"=Implementierungen in $P'$ ausführbar. Jede beliebige Verlängerung
%   dies Präfixes ist in $P'$ ausführbar, da an jeden Fehler-Zustand von $P$ in
%   $P'$ eine Schleife für alle Aktionen der Signatur von $P$ angefügt wurde.
%   Somit ist also ein $w$ aus $\cont (\StET (P))$ in $P'$ als Trace ausführbar.
%   \TODO{erzwungenen Zeilenumbruch kontrollieren}
% \end{proof}

\begin{Prop}[Sprache der Parallelkomposition]
  \label{LParallelProp}
  Für zwei komponierbare \MEIO{}s $P_1$ und $P_2$ gilt: $L_{12} := L(P_{12}) =
  L_1\|L_2$.
\end{Prop}
\begin{proof}
  Jedes Wort, dass in $L_{12}$ enthalten ist, kann auf $P_1$ und $P_2$
  projiziert werden und die Projektionen sind dann in $L_1$ und $L_2$
  enthalten. In einer Parallelkomposition werden die Wörter der beiden \MEIO{}s
  gemeinsam ausgeführt, falls es sich um synchronisierte Aktionen handelt, und
  verschränkt sequenziell, wenn es sich um unsynchronisierte Aktionen handelt.
  Somit sind alle Wörter aus $L_1\|L_2$ auch Wörter der Parallelkomposition
  $L(P_{12})$.
\end{proof}

\begin{Lem}[w-as-Verfeinerung und Parallelkomposition]
  \label{schwVerfParallelLem}
  Für zwei komponierbar \MEIO{}s $P_1$ und $P_2$ gilt, falls $P'_1$ und $P'_2$
  schwache as"=Verfeinerungen von $P_1$ bzw. $P_2$ sind, dann ist auch
  $P'_1\|P'_2$ eine schwache as"=Verfeinerung von $P_1\|P_2$.
\end{Lem}
\begin{proof}
  Es gelte $j\in\{1,2\}$. Da $P'_j$ eine schwache as"=Verfeinerung von $P_j$
  ist, gibt es nach Definition~\ref{wSimDef} eine schwache
  as-Verfeinerungs"=Relation $\mathcal{R}_j$, die beschreibt, wie $P'_j$ $P_j$
  verfeinert. Die Parallelkomposition werden auf Basis von
  Definition~\ref{ParallelDef} gebildet. Die Zustände sind also Tupel der
  Zustände der Komponenten. In dem man aus den Zuständen, die die
  $\mathcal{R}_j$ in Relation setzt auch solche Tupel zusammensetzt, kann man
  eine neue schwache as-Verfeinerungs-Relation für die Verfeinerung von
  $P_1\|P_2$ durch $P'_1\|P'_2$ erstellen. Die neue schwache
  as"=Verfeinerungs"=Relation soll $\mathcal{R}_{12}$ heißen und wie folgt
  definiert sein: $\forall p'_1,p'_2,p_1,p_2:
  ((p'_1,p'_2),(p_1,p_2))\in\mathcal{R}_{12} \Leftrightarrow
  (p'_1,p_1)\in\mathcal{R}_1 \land (p'_2,p_2)\in\mathcal{R}_2$. Es bleibt nun
  zu zeigen, dass $\mathcal{R}_{12}$ eine zulässige schwache
  as"=Verfeinerungs"=Relation nach Definition~\ref{wSimDef} ist. Für alle
  folgenden Fälle wird $((p'_1,p'_2),(p_1,p_2))\in\mathcal{R}_{12}$ mit
  $(p_1,p_2)\notin E_{12}$ vorausgesetzt.
  \begin{enumerate}
    \item Für den ersten Punkt ist zu zeigen, dass $(p'_1,p'_2)$ kein Element
      von $E_{P'_1\|P'_2}$ ist.\\
      Dies folgt direkt aus der Voraussetzung, dass $(p_1,p_2)\notin
      E_{P_1\|P_2}$ für das Tupel $((p'_1,p'_2),(p_1,p_2))$ aus
      $\mathcal{R}_{12}$ gilt. In dem man auf das $\mathcal{R}_{12}$ die
      Definition anwendet, erhält man $(p'_j,p_j)\in\mathcal{R}_j$ für beide
      $j$ Werte. Die $p_j$ dürfen beide keine Fehler-Zustände sein, da sonst
      auch $(p_1,p_2)$ ein solcher wäre. Somit folgt mit
      Definition~\ref{wSimDef}~1.\ $p'_j\notin E_j$ für beide $j$ Werte. Die
      beiden gestrichenen Zustände in Parallelkomposition können also keinen
      geerbten Fehler produzieren. Jedoch könnte $(p'_1,p'_2)$ aufgrund eines
      nicht erzwungenen Inputs ein neuer Fehler-Zustand sein. Dafür müsste
      \oBdA{} $p'_1\nmust[a]_{P'_1}$ und $p'_2\may[a]_{P'_2}$ für ein $a$ aus
      $I_1\cap O_2$ gelten. $\mathcal{R}_2$ erzwingt mit~\ref{wSimDef}~5.\ die
      schwache Ausführbarkeit des Outputs $a$ in $P_2$, d.h.\ $p_2
      \weakmay[a]_2$. Mit~\ref{wSimDef}~2.\ von $\mathcal{R}_1$ folgt $p_1
      \nmust[a]_1$. Somit müsste auch $(p_1,p_2)\in E_{12}$ gelten, was ein
      Widerspruch zur Voraussetzung wäre. $(p'_1,p'_2)$ kann also weder ein
      geerbter noch ein neuer Fehler-Zustand in $P'_1\|P'_2$ sein und deshalb
      gilt $(p'_1,p'_2)\notin E_{P'_1\|P'_2}$.
    \item Aus der schwache Simulations-Definition~\ref{wSimDef} folgt, dass für
      diesen Punkt folgendes zu zeigen ist: $(p_1,p_2)\must[i]_{12}(q_1,q_2)$
      impliziert $(p'_1,p'_2) \must[i]_{P'_1\|P'_2}
      \weakmust[\varepsilon]_{P'_1\|P'_2} (q'_1,q'_2)$ für ein $(q'_1,q'_2)$
      mit $((q'_1,q'_2),(q_1,q_2))\in\mathcal{R}_{12}$.\\
      Die $i$-must"=Transition in $P_1\|P_2$ kann entweder aus der
      Synchronisation von zwei must"=Inputs entstanden sein oder als
      unsynchronisierte Aktion aus einem $P_1$ übernommen worden sein.
      \begin{itemize}
        \item Fall 1 ($i\notin\Synch (P_1\|P_2)$): \OBdA{} ist $i$ in $I_1$
          enthalten. Es must also in $P_1$ die $i$-Transition als
          must"=Transition von $p_1$ ausgehen, es gilt $p_1\must[i]_1 q_1$.
          $p_2=q_2$ muss gelten, da $i$ nicht in $\Sigma _2$ enthalten ist. Mit
          der Relation $\mathcal{R}_1$ und~\ref{wSimDef}~2.\ folgt, dass in
          $P'_1$ $i$ als schwache Transition in der Form $p'_1\must[i]_{P'_1}
          \weakmay[\varepsilon]_{P'_1}q'_1$ ausführbar sein musst und $q'_1
          \mathcal{R}_1q_1$ gelten muss. Aus der Voraussetzung folgt
          $(p'_2,p_2)=(q'_2,q_2)\in\mathcal{R}_2$, da $i$ wenn es kein Element
          der Aktionen von $P_2$ ist auch keine Aktion der schwachen
          as"=Verfeinerung $P'_2$. Mit der Definition von $\mathcal{R}_{12}$
          kann dann daraus $((q'_1,q'_2),(q_1,q_2)) \in \mathcal{R}_{12}$
          gefolgert werden. In der Parallelkomposition von $P'_1$ und $P'_2$
          entsteht die Transitionsfolge $(p'_1,p'_2)\must[i]_{P'_1\|P'_2}
          \weakmust[\varepsilon]_{P'_1\|P'_2} (q'_1,q'_2)$.
        \item Fall 2 ($i\in\Synch (P_1\|P_2)$): Damit $i$ auch in $P_1\|P_2$
          ein Input ist, muss $i\in I_1\cap I_2$ gelten. Damit die
          $(p_1,p_2)\must[i]_{12}(q_1,q_2)$ in der Komposition möglich ist,
          muss in beiden $P_j$'s $p_j\must[i]_j q_j$ gelten. Durch
          $\mathcal{R}_j$ und die Definition~\ref{wSimDef}~2., die für diese
          Relationen gilt, folgt für beide $j$ Werte $p'_j\must[i]_{P'_j}
          \weakmust[\varepsilon]_{P'_j}q'_j$ mit $(q'_j,q_j)\in\mathcal{R}_j$.
          Es folgt $((q'_1,q'_2),(q_1,q_2)) \in \mathcal{R}_{12}$ mit der
          Definition von $\mathcal{R}_{12}$. Durch die Synchronisation des
          $i$'s in der Komposition von $P'_1$ und $P'_2$ gilt $(p'_1,p'_2)
          \must[i]_{P'_1\|P'_2} \weakmust[\varepsilon]_{P'_1\|P'_2}
          (q'_1,q'_2)$.
      \end{itemize}

  \TODO{Beweis anpassen und konkreter machen analog zu MIA3 Paper}

    \item Analog zu 2.\ kann für diesen Punkt $(p_1,p_2) \must[\omega]_{12}
      (q_1,q_2)$ impliziert $(p'_1,p'_2) \weakmust[\hat{\omega}]_{P'_1\|P'_2}
      (q'_1,q'_2)$ für ein $(q'_1,q'_2)$ mit $((q'_1,q'_2),(q_1,q_2)) \in
      \mathcal{R}_{12}$ gezeigt werden.\\
      Die $\omega$ Transition in $P_1\|P_2$ ist entweder aus einem
      synchronisierten oder aus einem unsynchronisierten $\omega$ entstanden.
      Entsprechend ist dann in einem oder beiden $P_j$ die Transition möglich
      und durch die Relationen $\mathcal{R}_j$ folgen die Transitionen $p'_j
      \weakmust[\omega]_{P'_j} q'_j$ mit $(q_j,q'_j)\in\mathcal{R}_j$. Durch
      die Parallelkomposition von $P'_1$ mit $P'_2$ folgt dann das zu zeigende.
    \item $(p'_1,p'_2)\may[i]_{P'_1\|P'_2}(q'_1,q'_2)$ impliziert $(p_1,p_2)
      \may[i]_{12} \weakmay[\varepsilon]_{12} (q_1,q_2)$ für ein $(q_1,q_2)$
      mit $((q'_1,q'_2),(q_1,q_2))\in\mathcal{R}_{12}$ ist die Voraussetzung
      des 4.\ Punktes, um zu beweisen, dass $\mathcal{R}_{12}$ eine schwache
      as"=Verfeinerungs"=Relation ist.\\
      Die Transition $i$ kann wiederum durch
      Synchronisation von zwei Transitionen entstanden sein oder durch eine
      Transition aus einer Komponenten und $i\notin\Synch (P'_1\|P'_2)$. In
      beiden Fällen gilt für ein $j$ oder beide $p'_j\may[i]_{P'_j}q'_j$.
      Daraus folgt mit $\mathcal{R}_j$ $p_j \may[i]_j \weakmay[\varepsilon]_j
      q_j$ mit $(q_j,q'_j)\in\mathcal{R}_j$. Durch Synchronisation oder durch übernehmen der
      Transitionen folgt dann $(p_1,p_2) \may[i]_{12} \weakmay[\varepsilon]_{12}
      (q_1,q_2)$ und mit Hilfe der Definition von $\mathcal{R}_{12}$ auch
      $((q_1,q_2),(q'_1,q'_2))\in\mathcal{R}_{12}$.
    \item Analog zu 4.\ kann für diesen Punkt $(p'_1,p'_2)
      \may[\omega]_{P'_1\|P'_2} (q'_1,q'_2)$ impliziert $(p_1,p_2)
      \weakmay[\hat{\omega}]_{12} (q_1,q_2)$ für ein $(q_1,q_2)$ mit
      $((q'_1,q'_2),(q_1,q_2))\in\mathcal{R}_{12}$ gezeigt werden.\\
      Die $\omega$ Transition in $P'_1\|P'_2$ ist entweder aus einem
      synchronisierten oder aus einem unsynchronisierten $\omega$ entstanden.
      Entsprechend ist dann in einem oder beiden $P'_j$ die Transitionen
      möglich und durch die Relationen $\mathcal{R}_j$ folgen die Transitionen
      $p_j \weakmay[\omega]_j q_j$ mit $(q_j,q'_j)\in\mathcal{R}_j$. Durch
      die Parallelkomposition von $P_1$ mit $P_2$ folgt dann das zu zeigende.
  \end{enumerate}
  $\Rightarrow P'_1\|P'_2$ ist eine schwache as"=Verfeinerung von $P_1\|P_2$.
\end{proof}

\begin{Kor}[w-as-Implementierungen und Parallelkomposition]
  \label{schwImpParallelKor}
  Für zwei komponierbare \MEIO{}s $P_1$ und $P_2$ gilt:
  $P'_1\in\wasimp (P_1) \land P'_2 \in\wasimp (P_2) \Rightarrow (P'_1\|P'_2)
  \in\wasimp (P_1\|P_2)$.
\end{Kor}
\begin{proof}
  $P'_1$ und $P'_2$ sind Aufgrund der Definition~\ref{wSimDef} auch schwache
  as"=Verfeinerungen von $P_1$ bzw.\ $P_2$. Somit ist die Parallelkomposition
  $P'_1\|P'_2$ auch eine schwache as"=Verfeinerung von $P_1\|P_2$, wegen
  Lemma~\ref{schwVerfParallelLem}. Für Implementierungen gilt $\must =\may$.
  Durch die Definition der Parallelkomposition in~\ref{ParallelDef} können aus
  aus zwei komponierbaren Implementierungen in der Komposition keine
  may"=Transitionen ohne zugehörige must"=Transitionen entstehen. Es gilt also
  auch $\must _{P'_1\|P'_2} =\may _{P'_1\|P'_2}$ und somit ist $P'_1\|P'_2$
  eine Implementierung und eine as"=Verfeinerung von $P_1\|P_2$. Dies
  entspricht der Definition der schwachen as"=Implementierung, sodass
  $(P'_1\|P'_2)\in\wasimp (P_1\|P_2)$ gilt.
\end{proof}

\begin{Kor}[as-Verfeinerungen und Parallelkomposition]
  \label{verfParallelKor}
  Für zwei komponierbar \MEIO{}s $P_1$ und $P_2$ gilt, falls $P'_1$ und $P'_2$
  as"=Verfeinerungen von $P_1$ bzw. $P_2$ sind, dann ist auch $P'_1\|P'_2$ eine
  as"=Verfeinerung von $P_1\|P_2$.
\end{Kor}
\begin{proof}
  \TODO{Beweis gestürzt auf Lemma~\ref{schwVerfParallelLem} anpassen}
%   Es gelte $j\in\{1,2\}$. Da $P'_j\in\asimp (P_j)$ gilt, gibt es nach
%   Definition~\ref{SimDef} eine as-Verfeinerungs"=Relation $\mathcal{R}_j$, die
%   beschreibt, wie $P'_j$ $P_j$ verfeinert. Die Parallelkomposition werden
%   auf Basis von Definition~\ref{ParallelDef} gebildet. Die Zustände sind also
%   Tupel der Zustände der Komponenten. In dem man aus den Zuständen, die die
%   $\mathcal{R}_j$ in Relation setzt auch solche Tupel zusammensetzt, kann man
%   auch eine neue as-Verfeinerungs-Relation für die Verfeinerung von $P_1\|P_2$
%   durch $P'_1\|P'_2$ erstellen. Die neue as-Verfeinerungs-Relation soll
%   $\mathcal{R}_{12}$ heißen und wie folgt definiert sein:
%   $\forall p'_1,p'_2,p_1,p_2: ((p'_1,p'_2),(p_1,p_2))\in\mathcal{R}_{12}
%   \Leftrightarrow (p'_1,p_1)\in\mathcal{R}_1 \land (p'_2,p_2)\in\mathcal{R}_2$.
%   Es bleibt nun zu zeigen, dass $\mathcal{R}_{12}$ eine zulässige
%   Verfeinerungs"=Relation nach Definition~\ref{SimDef} ist, da die
%   Parallelkomposition von zwei Implementierungen auch immer eine
%   Implementierung ist (\ref{ParallelDef}). Für die folgenden Fälle soll
%   $(p_1,p_2)\mathcal{R}_{12}(p'_1,p'_2)$ vorausgesetzt sein.
%   \begin{enumerate}
%     \item Für diesen Punkt der Simulations-Definition~\ref{SimDef} ist
%       folgendes zu zeigen: $(p_1,p_2)\must[\alpha]_{12}(q_1,q_2)$ impliziert
%       $(p'_1,p'_2)\must[\alpha]_{P'_1\|P'_2}(q'_1,q'_2)$ für ein $(q'_1,q'_2)$
%       mit $((q_1,q_2),(q'_1,q'_2))\in\mathcal{R}_{12}$.\\
%       $(p_1,p_2)\must[\alpha]_{12}(q_1,q_2)$ kann in $P_1\|P_2$ für ein
%       $\alpha$ aus $\Synch (P_1,P_2)$ nur gelten, wenn in $P_1$ die
%       $\alpha$-Transition zwischen $p_1$ und $q_1$ auch bereits eine
%       must"=Transition war und analog für $p_2$ und $q_2$ in $P_2$. Somit
%       erzwingen die $\mathcal{R}_j$ für die Komponenten bereits die
%       Implementierung der must"=Transitionen, so dass es dann entsprechende
%       $(q_j,q'_j)\in\mathcal{R}_j$ gibt. Die Parallelkomposition der
%       implementierten must"=Transitionen aus $P'_1$ und $P'_2$ führt in
%       $P'_1\|P'_2$ zu der geforderten Transition. Falls $\alpha$ keine
%       synchronisiert Aktion ist, enthält die Parallelkomposition die Transition
%       nur, da eine Komponente diese Transition alleine ausführen kann (ersten
%       beiden Zeilen der $\must _{12}$ Definition in~\ref{ParallelDef}). \OBdA{}
%       $p_1\must[\alpha]_1q_1$ und somit gilt $p_2=q_2$. Da $\mathcal{R}_1$ eine
%       as-Verfeinerungs-Relation ist, gibt es in $P'_1$ zwischen $p'_1$ und
%       $q'_1$ eine must"=Transition und es gilt $(q_1,q'_1)\in\mathcal{R}_1$.
%       $\alpha$ ist auch in der Parallelkomposition der as"=Implementierungen
%       eine unsynchronisierte Aktion und somit entsteht die Transition dort auch
%       nur aus der Transition von $P'_1$ und es gilt $p'_2=q'_2$.\\
%       Da in beiden Fällen $(q_j,q'_j)\in \mathcal{R}_j$ für beide Werte von $i$
%       gilt, gilt nach der Definition von $\mathcal{R}_{12}$ auch
%       $((q_1,q_2),(q'_1,q'_2))\in \mathcal{R}_{12}$.
%       $p_2\mathcal{R}_2p'_2$ musst nach Voraussetzung gelten und somit gilt
%       wegen der Gleichheiten der Zustände auch $q_2\mathcal{R}_2q'_2$.
%     \item Es ist $(p'_1,p'_2)\may[\alpha]_{P'_1\|P'_2}(q'_1,q'_2)$ impliziert
%       $(p_1,p_2)\may[\alpha]_{12}(q_1,q_2)$ für ein $(q_1,q_2)$ mit
%       $((q_1,q_2),(q'_1,q'_2))\in\mathcal{R}_{12}$ für diesen Punkt zu
%       zeigen.\\
%       Die Argumentation könnte wie in Punkt 1.\ erneut in unsynchronisierte und
%       synchronisierte Aktionen gesplittet werden. Jedoch würde man dadurch nur
%       auf das Ergebnis kommen, dass die eine Komponente oder beide die
%       entsprechende may"=Transition ausführen können müssen, damit die
%       Parallelkomposition dies auch kann. Es gilt also mindestens für eine der
%       Komponenten $p'_j\may[\alpha]_{P'_j}q'_j$ und durch die
%       Definition~\ref{SimDef}, die für die entsprechende Relation
%       $\mathcal{R}_j$ gilt, muss es in $P_j$ die Transition
%       $p_j\may[\alpha]_jq_j$ geben, so dass dann $(q_j,q'_j)\in\mathcal{R}_j$
%       gilt. Durch die Definition von $\may$ in~\ref{ParallelDef} werden die
%       may"=Transitionen der Komponenten entsprechend in die Parallelkomposition
%       $P_1\|P_2$ aufgenommen und die Relation $\mathcal{R}_{12}$ gilt, mit den
%       analogen Begründungen wie in 1., auch für die entsprechenden
%       Zustands-Tupel.
%     \item Hierfür muss gezeigt werden, wenn $(p'_1,p'_2)\in E_{P'_1\|P'_2}$
%       gilt, dann ist auch $(p_1,p_2)$ in $P_1\|P_2$ ein Zustand, der in der
%       Menge der Fehler-Zustande $E_{12}$ enthalten ist.\\
%       Falls $(p'_1,p'_2)$ ein geerbter Fehler ist, dann ist \oBdA{} $p'_1\in
%       E_{P'_1}$ und $p'_2\in P'_2$. Aufgrund von $\mathcal{R}_1$ und
%       Definition~\ref{SimDef} 3.\ muss dann auch $p_1\in E_1$ gelten. Für $p_2$
%       gilt durch die Signatur von $\mathcal{R}_2$ $p_2\in P_2$. Zusammen in
%       $P_1\|P_2$ ergibt das wieder  einen geerbten Fehler, also $(p_1,p_2)\in
%       E_{12}$. $(p'_1,p'_2)$ kann jedoch auch ein neuer Kommunikationsfehler
%       sein, dann gilt \oBdA{} $p'_1\nmust[a]_{P'_1}$ und $p'_2\may[a]_{P'_2}$
%       für ein $a$ aus $I_1\cap O_2$. Aufgrund von Definition~\ref{SimDef} 2.\
%       muss dann auch $p_2\may[a]_2$ gelten. Für $p_1$ kann nicht
%       $p_1\must[a]_1$ gelten, da sonst die Simulations Relation $\mathcal{R}_1$
%       die Implementierung dieser Transition in $P'_1$ fordern würde
%       (\ref{SimDef} 1.). Es gilt also $p_1\nmust[a]_1$ und in der
%       Parallelkomposition $P_1\|P_2$ ergibt sich daraus ebenfalls ein neuer
%       Kommunikationsfehler mit $(p_1,p_2)\in E_{12}$.
%   \end{enumerate}
%   $\Rightarrow (P'_1\|P'_2)\in\asimp (P_1\|P_2)$.
\end{proof}

\begin{Kor}[as-Implementierungen und Parallelkomposition]
  Für zwei komponierbare \MEIO{}s $P_1$ und $P_2$ gilt:
  $P'_1\in\asimp (P_1) \land P'_2 \in\asimp (P_2) \Rightarrow (P'_1\|P'_2)
  \in\asimp (P_1\|P_2)$.
\end{Kor}
\begin{proof}
  Analog zum Beweis des Korollars~\ref{schwImpParallelKor} kann hier auch
  begründet werden, dass $P'_1\|P'_2$ eine Implementierung ist. Zusätzlich
  ist $P'_1\|P'_2$ eine as"=Verfeinerung , wegen der Voraussetzungen diese
  Korollars in Kombination mit der Aussage des Korollars~\ref{verfParallelKor}.
  Es gilt also $(P'_1\|P'_2)\in\asimp (P_1\|P_2)$.
\end{proof}

Die entgegengesetzte Richtung von Korollar~\ref{verfParallelKor} gilt im
allgemeinen nicht, d.h.\ es muss zu einer as"=Verfeinerung einer
Parallelkomposition $P'$ von $P_1\|P_2$ keine as"=Verfeinerungen $P'_1$ bzw.\
$P'_2$ der einzelnen Komponenten $P_1$ bzw. $P_2$ geben, deren
Parallelkomposition $P'_1\|P'_2$ der as"=Verfeinerung der Parallelkomposition
$P'$ entsprechen. Die Problematik wird in Abbildung~\ref{impParallelFig} an
einem Beispiel dargestellt. In der Parallelkomposition wird die may"=Transition
von $P_2$ zu zwei may"=Transitionen, für die in einer as"=Verfeinerung
unabhängig entschieden werden kann, ob sie übernommen, implementiert oder
weggelassen werden. Somit kommt es in $P'$ zu dem Problem, dass keine
as"=Verfeinerung von $P_2$ (entweder keine Transition oder die $o'$ Transition
wird als may- oder must-Transition ausgeführt) in Parallelkomposition mit der
Implementierung $P_1$ $P'$ ergeben würde.\\
Da jede as"=Verfeinerungs"=Relation auch eine schwache
as"=Verfeinerungs"=Relation ist, folgt draus auch, dass die entgegengesetzt
Richtung von Lemma~\ref{schwVerfParallelLem} ebenfalls nicht gelten kann. Auch
im Spezialfall von as"=Implementierungen bzw. w-as"=Implementierungen kann das
Gegenbeispiel angewendet werden, da $P'$ auch eine Implementierung von
$P_1\|P_2$ ist und es auch keine passende as"=Implementierung bzw.
w-as"=Implementierung von $P_2$ geben kann, wenn es schon keine passende
Verfeinerung gibt.

\begin{figure}[htbp]
  \begin{center}
    \begin{tikzpicture}[shorten >=1pt,auto,node distance=2.5cm]
      \node [initial,initial text=$P_1$:] (p01) at (0,0) {$p_{01}$};
      \node (p1) [right of=p01] {$p_1$};

      \path[->]
      (p01) edge node{$o!$} (p1)
      ;

      \node [initial,initial text=$P_2$:] (p02) at (7,0) {$p_{02}$};
      \node (p2) [right of=p02] {$p_2$};

      \path[->]
      (p02) edge[dashed] node{$o'!$} (p2)
      ;

      \node [initial,initial text=$P_1\|P_2$:] (p0) at (0,-2)
      {$p_{01}\|p_{02}$};
      \node (p102) [right of=p0] {$p_1\|p_{02}$};
      \node (p012) [below of=p0] {$p_{01}\|p_2$};
      \node (p12) [below of=p102] {$p_1\|p_2$};

      \path[->]
      (p0) edge node{$o!$} (p102)
      (p012) edge node{$o!$} (p12)
      (p0) edge[dashed] node{$o'!$} (p012)
      (p102) edge[dashed] node{$o'!$} (p12)
      ;

      \node [initial,initial text=$P'$:] (p'0) at (7,-2)
      {$p'_{01}\|p'_{02}$};
      \node (p'102) [right of=p'0] {$p'_1\|p'_{02}$};
      \node (p'012) [below of=p'0] {$p'_{01}\|p'_2$};
      \node (p'12) [below of=p'102] {$p'_1\|p'_2$};

      \path[->]
      (p'0) edge node{$o!$} (p'102)
      (p'012) edge node{$o!$} (p'12)
      (p'0) edge node{$o'!$} (p'012)
      ;
    \end{tikzpicture}
    \caption{Gegenbeispiel für Umkehrung von Lemma~\ref{verfParallelKor}}
    \label{impParallelFig}
  \end{center}
\end{figure}

Ein neuer Kommunikationsfehler in einer Parallelkomposition muss in einer
Implementierung (as oder w-as) nicht auftauchen, auch nicht in der
Parallelkomposition von Implementierungen der einzelnen Komponenten. Dies liegt
daran, dass für den Input nur gesagt wird, dass keine must"=Transition für die
Synchronisation der Aktion vorhanden ist. Es kann trotzdem eine
may"=Transition für den Input geben, die auch implementiert werden kann.
Falls es aber in der Parallelkomposition zweier \MEIO{} zu einem neuen
Kommunikationsfehler kommt, dann gibt es auch immer mindestens eine
Implementierung, die diesen Kommunikationsfehler enthält und es gibt auch immer
mindestens ein Implementierungs-Paar der Komponenten, in deren
Parallelkomposition sich dieser Kommunikationsfehler ebenfalls zeigt.
