\chapter{allgemeine Folgerungen}

\begin{Prop}[Sprache und Implementierung]
  \label{LImpProp}
  Die \emph{(maximale) Sprache} eines \MEIO{}s $P$ ist $L(P) = \left\{w\in
  \Sigma ^* \mid \exists P'\in\asimp (P) : p'_0\weakmust[w]
  \right\}$.
\end{Prop}
\begin{proof}
  Für ein $w\in L(P)$ gilt nach Definition~\ref{LDef} und den
  Definitionen der Transitions-Notation $\exists p_1,p_2,\dots,p_{n-1},p' \exists
  w'\in\Sigma _{\tau} ^* : \hat{w'} = w \land w'=\alpha _1\alpha _2\dots\alpha
  _n \land p_0\may[\alpha _1]p_1\may[\alpha _2] \dots p_{n-1}\may[\alpha _n]
  p'$. Für ein $w$ aus $\left\{w\in \Sigma ^* \mid \exists P'\in\asimp (P) :
  p'_0\weakmust[w] \right\}$ gilt, für ein $P'\in\asimp (P)$ das analoge nur
  mit must- anstatt may"=Transitionen.\\
  Aufgrund von Definition~\ref{SimDef} 2.\ kann jedes Element aus $\asimp (P)$
  nur die bereits in $P$ vorhandenen may-Transitionen implementieren. Somit
  gibt es für jedes $w$, dass in der Sprache einer as"=Implementierung von $P$
  enthalten ist auch ein entsprechendes $w\in L(P)$ mit einem Trace wie oben.\\
  Da in $\left\{w\in \Sigma ^* \mid \exists P'\in\asimp (P) : p'_0\weakmust[w]
  \right\}$ alle Wörter enthalten sind, für dies es eine as"=Implementierung
  gibt, die dieses Wort ausführen kann, werden somit auch alle möglichen
  as"=Implementierungen betrachtet. Jede may"=Transition aus $P$ wird von
  mindestens einem $P'\in\asimp (P)$ als must"=Transition implementiert.
  Deshalb sind auch alle Wörter, die in $L(P)$ enthalten sind in der Menge der
  Wörter alle as"=Implementierungen von $P$ enthalten.\\
  Da für Implementierungen die must-Transitions-Relations-Menge die gleiche ist,
  wie die Menge der may-Transitions-Relationen könnte man auch für die
  as"=Implementierungen die Definition~\ref{LDef} anwenden um die jeweilige
  Sprache zu bestimmen. Die Sprache eines \MEIO{} entspricht dann der
  Vereinigung der Sprachen seiner as"=Implementierungen.
\end{proof}

\begin{Prop}[Sprache der Parallelkomposition]
  \label{LParallelProp}
  Für zwei komponierbare \MEIO{}s $P_1$ und $P_2$ gilt: $L_{12} := L(P_{12}) =
  L_1\|L_2$.
\end{Prop}
\begin{proof}
  Jedes Wort, dass in $L_{12}$ enthalten ist, kann auf $P_1$ und $P_2$
  projiziert werden und die Projektionen sind dann in $L_1$ und $L_2$
  enthalten. In einer Parallelkomposition werden die Wörter der beiden \MEIO{}s
  gemeinsam ausgeführt, falls es sich um synchronisierte Aktionen handelt, und
  verschränkt sequenziell, wenn es sich um unsynchronisierte Aktionen handelt.
  Somit sind alle Wörter aus $L_1\|L_2$ auch Wörter der Parallelkomposition
  $L(P_{12})$.
\end{proof}

\begin{Lem}[as-Implementierungen und Parallelkomposition]
  \label{impParallelLem}
  $P'_1\in\asimp (P_1) \land P'_2 \in\asimp (P_2) \Rightarrow (P'_1\|P'_2)
  \in\asimp (P_1\|P_2)$.

      \TODO{ausführlicher beweisen (MIA3 als mögliche Vorlage für Konstruktion)}
\end{Lem}

\begin{proof}
    Es gelte $i\in\{1,2\}$. Jede Transition ($\may '_i =\must '_i$) in
    $P'_i$ hat eine entsprechende may-Transition in $P_i$, da
    Definition~\ref{SimDef} 2.\ gilt. Aufgrund der Simulations
    Definition~\ref{SimDef}, haben $P_i$ und $P'_i$ die gleichen Signaturen
    und somit gilt $\Synch (P_1,P_2) =\Synch (P'_1,P'_2)$.\\
    Daraus folgt unter Verwendung, dass die Definitionen der must- und
    may"=Transitionen der Parallelkomposition in~\ref{ParallelDef} analog
    formuliert sind, dass alle Transitionen, die in $P'_1\|P'_2$ enthalten
    sind auch in $P_1\|P_2$ als may"=Transitionen vorhanden sind. Es können
    bei der Parallelkomposition von \MEIO{}s, die die gleichen must- und
    may-Transitionsmengen haben (Implementierungen), keine may"=Transitionen
    entstehen, zu denen es keine passende must-Transition gibt (Definition
    von $\must _{12}$ und $\may _{12}$ in~\ref{ParallelDef}). $P'_1\|P'_2$
    ist also eine Implementierung.\\
    $P_1\|P_2$ enthält nur must"=Transitionen, wenn auch $P_1$ bzw.\ $P_2$
    diese enthalten haben. $P'_1$ und $P'_2$ müssen diese must"=Transitionen
    aufgrund von Definition~\ref{SimDef} 1.\ implementieren und somit enthält
    auch $P'_1\|P'_2$ die entsprechenden durch~\ref{SimDef} 1.\ geforderten
    Transitionen, um eine as"=Implementierung von $P_1\|P_2$ sein zu können.\\
    Die Fehler-Zustände, die $P'_1$ und $P'_2$ enthalten sind,
    müssten wegen Definition~\ref{SimDef} 3.\ auch in $P_1$ und $P_2$
    enthalten sein. Somit enthält $P_1\|P_2$ auch die entsprechend geerbten
    Fehler, die dann durch $P'_1\|P'_2$ implementiert werden.
    In $P'_1\|P'_2$ kann es nur zu neuen Kommunikationsfehlern kommen, wenn
    dies auch in $P_1\|P_2$ möglich war. Also ist auch~\ref{SimDef} 3.\ für
    $P'_1\|P'_2$ erfüllt.\\
    $\Rightarrow (P'_1\|P'_2)\in\asimp (P_1\|P_2)$.
\end{proof}

Die entgegengesetzte Richtung von Lemma~\ref{impParallelLem} gilt im
allgemeinen nicht, d.h.\ es muss zu einer as"=Implementierung einer
Parallelkomposition $P'\in\asimp (P_1\|P_2)$ keine as"=Implementierungen $P'_1$
bzw.\ $P'_2$ der einzelnen Komponenten $P_1$ bzw. $P_2$ geben, deren
Parallelkomposition $P'_1\|P'_2$ der as"=Implementierung der
Parallelkomposition $P$ entsprechen. Die Problematik wird
in Abbildung~\ref{impParallelFig} an einem Beispiel dargestellt. In der
Parallelkomposition wird die may"=Transition von $P_2$ zu zwei
may"=Transitionen, für die in einer as"=Implementierung unabhängig
entschieden werden kann, ob sie implementiert werden oder nicht. Somit kommt es
in $P'$ zu dem Problem, dass keine as"=Implementierung von $P_2$ (entweder
keine Transition implementiert oder die $o'$ Transition ist implementiert) in
Parallelkomposition mit der Implementierung $P_1$ $P'$ ergeben würde.

\TODO{Alternative überlegen}

\begin{figure}[htbp]
  \begin{center}
    \begin{tikzpicture}[shorten >=1pt,auto,node distance=2.5cm]
      \node [initial,initial text=$P_1$:] (p01) at (0,0) {$p_{01}$};
      \node (p1) [right of=p01] {$p_1$};

      \path[->]
      (p01) edge node{$o!$} (p1)
      ;

      \node [initial,initial text=$P_2$:] (p02) at (7,0) {$p_{02}$};
      \node (p2) [right of=p02] {$p_2$};

      \path[->]
      (p02) edge[dashed] node{$o'!$} (p2)
      ;

      \node [initial,initial text=$P_1\|P_2$:] (p0) at (0,-2)
      {$p_{01}\|p_{02}$};
      \node (p102) [right of=p0] {$p_1\|p_{02}$};
      \node (p012) [below of=p0] {$p_{01}\|p_2$};
      \node (p12) [below of=p102] {$p_1\|p_2$};

      \path[->]
      (p0) edge node{$o!$} (p102)
      (p012) edge node{$o!$} (p12)
      (p0) edge[dashed] node{$o'!$} (p012)
      (p102) edge[dashed] node{$o'!$} (p12)
      ;

      \node [initial,initial text=$P'$:] (p'0) at (7,-2)
      {$p'_{01}\|p'_{02}$};
      \node (p'102) [right of=p'0] {$p'_1\|p'_{02}$};
      \node (p'012) [below of=p'0] {$p'_{01}\|p'_2$};
      \node (p'12) [below of=p'102] {$p'_1\|p'_2$};

      \path[->]
      (p'0) edge node{$o!$} (p'102)
      (p'012) edge node{$o!$} (p'12)
      (p'0) edge node{$o'!$} (p'012)
      ;
    \end{tikzpicture}
    \caption{Gegenbeispiel für Umkehrung von Lemma~\ref{impParallelLem}}
    \label{impParallelFig}
  \end{center}
\end{figure}

\begin{Lem}[w-as-Implementierungen und Parallelkomposition]\mbox{}

  \TODO{entsprechend anpassen wie 1.2 oder löschen}

  $P'_1\in\wasimp (P_1) \land P'_2 \in\wasimp (P_2) \Rightarrow (P'_1\|P'_2)
  \in\wasimp (P_1\|P_2)$.
\end{Lem}

\begin{proof}\mbox{}
    Es gelte $i\in\{1,2\}$. Jede Transition ($\may '_i =\must '_i$) einer
    sichtbaren Aktion in $P'_i$ hat eine entsprechende schwache
    may-Transition in $P_i$, da Definition~\ref{wSimDef} 3./4.\ gilt. Aufgrund
    der Simulations Definition~\ref{wSimDef}, haben $P_i$ und $P'_i$ die
    gleichen Signaturen (bezieht sich nur auf sichtbare Aktionen) und somit
    gilt $\Synch (P_1,P_2) = \Synch (P'_1,P'_2)$.\\
    Daraus folgt unter Verwendung, dass die Definitionen der must- und
    may"=Transitionen der Parallelkomposition in~\ref{ParallelDef} analog
    formuliert sind, dass alle sichtbaren Transitionen, die in $P'_1\|P'_2$
    enthalten sind auch in $P_1\|P_2$ als schwache may"=Transitionen
    vorhanden sind. Hierzu ist noch anzumerken, dass die interne Aktion nie
    in der Parallelkomposition synchronisiert wird und somit die \MEIO{}s
    diese in der Komposition jeweils für ihre Komponente alleine ausführen.\\
    $P'_1\|P'_2$ ist mit der gleichen Begründung wie im Beweis von
    Satz~\ref{impParallelLem} 1.\ eine Implementierung.\\
    $P_1\|P_2$ enthält nur must"=Transitionen, wenn auch $P_1$ bzw.\ $P_2$
    diese enthalten haben. $P'_1$ und $P'_2$ müssen diese must"=Transitionen
    aufgrund von Definition~\ref{wSimDef} 1./2.\ schwach implementieren, d.h.\
    bei Inputs können danach noch interne Aktionen möglich sein und bei
    Outputs davor und danach und bei einen $\tau$ können beliebig viele
    interne Aktionen implementiert werden (auch keine). Die durch $P_1\|P_2$
    und~\ref{wSimDef} 1./2.\ geforderten Implementierungen von must"=Transitionen
    in $P'_1\|P'_2$ basieren auf den must"=Transitionen der einzelnen
    \MEIO{}s, die schwach in deren w"=as"=Implementierungen implementiert
    wurden. Die zusätzlichen $\tau$s, die dadurch entstehen, hindern
    $P'_1\|P'_2$ in der Parallelkomposition nicht daran die
    must"=Transitionen auch entsprechend schwach zu implementieren.\\
    Die Argumentation, wieso Definition~\ref{wSimDef} 5.\ hier gilt, kann
    analog zu der Argumentation für Definition~\ref{SimDef} 3.\ aus dem
    Beweis zu Punkt 1.\ aus dem Lemma~\ref{impParallelLem} übernommen werden.
    Die zusätzlichen $\tau$ Transitionen können nichts an geerbten und neuen
    Fehlern verändern, was nicht auch in der
    Parallelkomposition zu analogen Veränderungen führt. Die zu möglichen
    neuen Kommunikationsfehlern führende Inputs müssen entweder sofort
    implementiert werden oder gar nicht.\\
    $\Rightarrow (P'_1\|P'_2)\in\wasimp (P_1\|P_2)$.
\end{proof}

Ein neuer Kommunikationsfehler in einer Parallelkomposition muss in einer
Implementierung (as oder w-as) nicht auftauchen, auch nicht in der
Parallelkomposition von Implementierungen der einzelnen Komponenten. Dies liegt
daran, dass für den Input nur gesagt wird, dass keine must"=Transition für die
Synchronisation der Aktion vorhanden ist. Es kann trotzdem eine
may"=Transition für den Input geben, die auch implementiert werden kann.
Falls es aber in der Parallelkomposition zweier \MEIO{} zu einem neuen
Kommunikationsfehler kommt, dann gibt es auch immer mindestens eine
Implementierung, die diesen Kommunikationsfehler enthält und es gibt auch immer
mindestens ein Implementierungs-Paar der Komponenten, in deren
Parallelkomposition sich dieser Kommunikationsfehler ebenfalls zeigt.
