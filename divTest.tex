\section{Testing-Ansatz}

Für die \EIO{} aus z.B.~\cite{Schinko2016BA} sind divergente Zustände Zustände,
die eine unendliche Folge an $\tau$s ausführen können. Da die Definition für
Implementierung hier das gleiche liefern soll, müssen alle Zustände, die eine
unendliche Folge an $\tau$ Zuständen sicher Stellen müssen auf jeden Fall
divergent sein. Jedoch lassen bereits unendliche Folgen von
$\tau$-may-Transitionen Divergenz in einer ihrer as"=Implementierungen zu.
Somit erscheint es sinnvoll Zustände bereits als divergent anzusehen, wenn sie
die Möglichkeit für eine unendliche Folge von $\tau$s via may-Transitionen
besitzen.

\begin{Def}[Divergenz]
  Ein \emph{Divergenz-Zustand} ist ein Zustand in einem \MEIO{} $P$, der via
  may-Transitionen eine unendliche Folge an $\tau$s ausführen kann.\\
  Die Menge $Div(P)$ besteht aus all diesen divergenten Zuständen des \MEIO{}s
  $P$.
\end{Def}

Die unendliche Folge an $\tau$s kann durch einen Kreis von Zuständen, die via
$\tau$-Transitionen verbunden sind, von einem durch interne Aktionen
erreichbaren Zustand ausführbar sein oder durch einen unendlichen Weg, der mit
$\tau$s ausführbar ist, der unendliche viele Zustände durchläuft. Es ist jedoch
zu beachten, dass ein Zustand, von dem aus unendlich viele Zustände durch
$\tau$s  erreichbar sind, nicht divergent sein muss. Es ist auch möglich,
dass dieser Zustand eine unendliche Verzweigung hat und somit keine unendlichen
Folgen an $\tau$s ausführen kann. Aus diesem Grund, schränkt sich diese Kapitel
auf \MEIO{}s ein ohne unendlich Verzweigungen, die dann die Definition der
lokalen erfüllen, dass sie lokal endlich sind.\\
Für die Relevant eines Divergenz-Zustandes in einem Transitionssystem wieder
wieder der optimistische Ansatz der lokalen Erreichbarkeit. Auf
Implementierungen, die den \EIO{}s in z.B.~\cite{Schinko2016BA} entsprechen,
ist Divergenz nicht mehr verhinderbar, sobald ein divergenter Zustand lokal
erreichbar ist. Somit wird sich auch hier herausstellen, dass Divergenz als
ähnlich \glqq schlimm\grqq{} zu bewerten wie ein Fehler-Zustand.

\begin{Def}[Test und Verfeinerung für Divergenz]
  \label{DivTestDef}
  Sei $P$ ein \MEIO{}. Ein \emph{Test} $T$ für $P$ ist eine zu $P$
  komponierbare Implementierung. $P$ \emph{as-erfüllt} $T$ als einen
  Divergenz-Test, falls $S\|T$ lokal fehler-, stillstand- und divergenz-frei
  ist für alle $S\in \asimp (P)$. Es wird dann $P\DsatAs T$ geschrieben. Die
  Parallelkomposition $S\|T$ ist \emph{lokal fehler-}, \emph{stillstand-} und
  \emph{divergenz-frei}, wenn kein Fehler-, stiller oder Divergenz-Zustand
  lokal erreichbar ist.\\
  Ein \MEIO{} $P$ \emph{Divergenz-verfeinert} $P'$, falls sie die selbe
  Signatur haben und für alle ihre Tests $T$: $P'\DsatAs T \Rightarrow P\DsatAs
  T$.
\end{Def}

Da nun die grundlegenden Definitionen für Divergenz festgehalten sind, kann man
sich einen Begriff für die Traces zu divergenten Zuständen bilden. Im letzten
Absatz wurde bereits festgestellt, dass Divergenz wohl als ähnlich \glqq
schlimmes\grqq{} Fehlverhalten anzusehen ist wie Fehler. Da das Divergieren
eines Systems nicht mehr verhinderbar ist, sobald ein divergenter Zustand lokal
erreichbar ist, kommt für die Divergenz-Traces wieder die \prune{} und
letztlich auch die \cont{}-Funktion zum Einsatz. Ein System, das unendliche
viele $\tau$s ausführen kann, ist von außen nicht von so einem System zu
unterscheiden, das einen Fehler-Zustand erreicht. Somit wird später semantisch
in den Trace-Mengen auch nicht zwischen Fehler"=Traces und Divergenz"=Traces
explizit unterschieden. Dadurch genügt es nicht mehr nur mit den Fehler"=Traces
die Sprache zu fluten, sondern es muss sowohl mit den Fehler"=Traces wie auch
den Divergenz"=Traces geflutet werden. Ebenso werden die strikten
Stille"=Traces mit diesen beiden Trace-Mengen geflutet.

\begin{Def}[Divergenz-Traces]
  Sei $P$ ein \MEIO{} und definiere:
  \begin{itemize}
    \item \emph{strikte Divergenz-Traces}: $\StDT (P) := \left\{w\in\Sigma
      ^*\mid p_0\weakmay[w]_P p\in Div(P)\right\}$,
    \item \emph{gekürzte Divergenz-Traces}: $\PrDT (P) := \left\{\prune (w)\mid
      w\in\StDT (P)\right\}$.
  \end{itemize}
\end{Def}

Analog zu den Propositionen~\ref{KommTracesProp} und~\ref{StilleTraceProp} gibt
es hier auch eine Proposition, die die Divergenz-Traces eines \MEIO{}s mit den
Divergenz-Traces seiner as"=Implementierungen verbindet. Die Begründung
verläuft analog zu den Propositionen der vorangegangenen Kapitel.

\begin{Prop}[Divergenz-Traces und Implementierungen]
  \label{DivTraceProp}
  Sei $P$ ein \MEIO{}.
  \begin{enumerate}
    \item Für die strikten Divergenz-Traces gilt: $\StDT (P) \subseteq
      \underset{P'\in\asimp (P)}{\bigcup} \StDT (P')$.
    \item Für die gekürzten Divergenz-Traces von $P$ gilt:
      $\PrDT (P) \subseteq \underset{P'\in\asimp (P)}{\bigcup} \PrDT (P')$.
  \end{enumerate}
\end{Prop}
\begin{proof}\mbox{}
  \begin{enumerate}
    \item Um diese Inklusion beweisen zu können wird wieder eine
      as"=Implementierung $P'$ von $P$ und eine passende
      as"=Verfeinerungs"=Relation $\mathcal{R}$ angegeben, so dass alle
      strikten Divergenz-Traces von $P$ auch in $P'$ enthalten sind. In diesem
      Fall funktioniert der Ansatz alle Traces aus $P$ in $P'$ zu
      implementieren und keine Fehler-Zustände zu übernehmen. Die Definition
      von $P'$ lautet also:
      \begin{itemize}
        \item $P'=P$,
        \item $p'_0=p_0$,
        \item $I_{P'}=I_P$ und $O_{P'}=O_P$,
        \item $\must _{P'} =\may _{P'} = \may _P$,
        \item $E_{P'}=\emptyset$.
      \end{itemize}
      Die passende as"=Verfeinerungs"=Relation $\mathcal{R}$ ist die
      Identität-Relation. Wie bereits im Beweis zu Proposition~\ref{LImpProp}
      begründet erfüllt $\mathcal{R}$ alle Punkte der Definition~\ref{SimDef}.
      Es wird ein $w$ aus $\StDT (P)$ betrachtet. Es gibt also einen Trace in
      $P$ auf dem das Wort $w$ ausgeführt wird und der einen divergenten
      Zustand erreicht. Es gilt also $\exists w' \in \Sigma _{\tau}^*, \exists
      \alpha _1, \alpha _2, \dots , \alpha _n, \exists p_1, p_2, \dots , p_n:
      \hat{w'} = w \land w' = \alpha _1\alpha _2\dots\alpha _n \land p_0
      \may[\alpha _1]_P p_1 \may[\alpha _2]_P \dots p_{n-1} \may[\alpha _n]_P
      p_n \in Div _P$. Die Identitäts"=Relation $\mathcal{R}$ setzt die
      Zustände des Traces mit den analogen Zuständen aus $P'$ in Relation.
      Zusätzlich mit der Implementierung aller Transitionen aus $P$ in $P'$
      ergibt sich der selbe Trace in $P'$. Es gilt also $p'_0 \may[\alpha
      _1]_{P'} p'_1 \may[\alpha _2]_{P'} \dots p'_{n-1} \may[\alpha _n]_{P'}
      p'_n$ mit $(p'_j,p_j) \in \mathcal{R}$ für $0\leq j \leq n$. Da $p_n$ in
      $Div _P$ enthalten ist, gibt es von diesem Zustand in $P$ aus die
      Möglichkeit eine unendliche Folge an $\tau$s auszuführen. Die
      Ausführbarkeit muss sich dabei auf mit $\tau$ beschriftete
      may"=Transitionen in $P$ stützen. Da diese Transitionen in $P'$ alle
      übernommen wurden, ist auch für $p'_n$ eine unendliche Folge an $\tau$s
      ausführbar. Es gilt also $p'_n\in Div _{P'}$ und somit $w\in \StDT (P')$.
      Insgesamt folgt also für diese $P'$ $\StDT (P) = \StDT (P')$.
      \TODO{versuchen als ein Gesamtablauf zu formulieren}
    \item Dieser Punkt entspricht 1.\ bis auf die Anwendung der
      \prune{}-Funktion auf beiden Seiten des Inklusions-Symbols. Da $\prune{}$
      monoton ist, folgt dieser Punkt direkt auf dem letzten.
  \end{enumerate}
\end{proof}

Da die Stille"=Traces mit den Fehler- und Divergenz"=Traces geflutet werden
sollen, kann die Stillstands"=Semantik nicht aus dem letzten Kapitel übernommen
werden. Auch die geflutete Sprache aus dem Fehler-Kapitel kann nicht
beibehalten werden. Nur die Fehler"=Traces \ET{} können ohne Veränderung auch
in diesem Kapitel verwendet werden. Jedoch werden diese Traces im weiteren
Verlauf nur innerhalb der größeren Trace-Menge \EDT{} relevant sein.

\begin{Def}[Kommunikationsfehler-, Stillstands- und Divergenz-Semantik]
  \label{DivSemDef}
  Sei $P$ ein \MEIO{}.
  \begin{itemize}
    \item Die Menge der \emph{Divergenz-Traces} von $P$ ist $\DT (P) := \cont
      (\PrDT (P))$.
    \item Die Menge der \emph{Fehler-Divergenz-Traces} von $P$ ist $\EDT (P) :=
      \ET (P)\cup\DT (P)$.
    \item Die Menge der \emph{Fehler-Divergenz-gefluteten
      Stille-Traces} von $P$ ist $\QDT (P) := \StQT (P)\cup\EDT (P)$.
    \item Die Menge der \emph{Fehler-Divergenz-gefluteten
      Sprache} von $P$ ist $\EDL (P) := L(P)\cup\EDT (P)$.
  \end{itemize}
  Für zwei \MEIO{}s $P_1,P_2$ mit der gleichen Signatur schreibt man $P_1\DRel
  P_2$, wenn $\EDT _1\subseteq \EDT _2, \QDT _1\subseteq \QDT _2$ und $\EDL
  _1\subseteq \EDL _2$ gilt.
\end{Def}

\vspace{0.2cm}

\begin{Prop}[Kommunikationsfehler-, Stillstands-, Divergenz-Semantik und
  Implementierungen]
  \label{DivSemProp}
  Sie $P$ ein \MEIO{}.
  \begin{enumerate}
    \item Für die Menge der Divergenz-Traces von $P$ gilt $\DT (P) \subseteq
      \underset{P'\in\asimp (P)}{\bigcup} \DT (P')$.
    \item Für die Menge der Fehler-Divergenz-Traces von $P$ gilt die folgende
      Gleichheit $\EDT (P) = \underset{P'\in\asimp (P)}{\bigcup} \EDT (P')$.
    \item Für die Menge der Fehler-Divergenz-gefluteten Stille-Traces von $P$
      gilt $\QDT (P) = \underset{P'\in\asimp (P)}{\bigcup} \QDT (P')$.
    \item Für die Menge der Fehler-Divergenz-gefluteten Sprache von $P$ gilt
      $\EDL (P) = \underset{P'\in\asimp (P)}{\bigcup} \EDL (P')$.
  \end{enumerate}
\end{Prop}
\begin{proof}\mbox{}\\
  1.:\\
  Es gilt bereits $\PrDT (P) \subseteq \underset{P'\in\asimp (P)}{\bigcup}
  \PrDT (P')$, wegen 2.\ der Proposition~\ref{DivTraceProp}. Aus der Monotonie
  von \cont{} folgt also auch die hier geforderte Inklusion.

  2. \glqq$\subseteq$\grqq{}:
  {\allowdisplaybreaks
  \begin{align*}
    \EDT (P)&\overset{\ref{DivSemDef}}{=} \ET (P)\cup \DT (P)\\
    &\hspace{-0.2cm}\overset{\ref{KommSemProp}.1}{=} \left(\underset{P'\in
    \asimp (P)}{\bigcup} \ET (P')\right)\cup \DT (P)\\
    &\overset{1.}{\subseteq} \left(\underset{P'\in
    \asimp (P)}{\bigcup} \ET (P')\right)\cup \left(\underset{P'\in \asimp
    (P)}{\bigcup} \DT (P')\right)\\
    &= \underset{P'\in \asimp (P)}{\bigcup} \ET (P') \cup \DT (P')\\
    &\overset{\ref{DivSemDef}}{=} \underset{P'\in\asimp (P)}{\bigcup} \EDT
    (P').
  \end{align*}}

  2. \glqq$\supseteq$\grqq{}:\\
  Für ein präfix-minimales $w$ aus der Menge $\EDT (P')$ einer
  as"=Implementierung $P'$ von $P$ wird für diese Inklusion gezeigt, dass auch
  $w\in\EDT (P)$ gilt. Es genügt ein präfix-minimales Element, da die
  \EDT{}-Mengen unter \cont{} abgeschlossen sind. Falls das $w$ in $\ET (P')$
  enthalten ist, folgt $w\in\ET (P) \subseteq\EDT (P)$ aufgrund des ersten
  Punktes der Proposition~\ref{KommSemProp}. Es ist also nur noch der Fall zu
  betrachten, in dem $w\in\DT (P') \backslash \ET (P')$ gilt. Es gibt also
  einen unendlichen Ablauf analog zu dem in Lemma~\ref{AblaefeVerfSpezLem}
  thematisierten für $wv\in\StDT _P'$ mit $v\in O^*$, wobei nur endlich viele
  Transitionen benötigt werden bis die letzte sichtbare Aktion aus $wv$
  ausgeführt wird und danach unendlich viele $\tau$-Transitionen folgen.
  Zwischen $P'$ und $P$ soll $\mathcal{R}$ als as"=Verfeinerungs"=Relationen
  gelten. Falls in dem matchenden Ablauf in $P$ ein Fehler-Zustand erreicht
  wird, muss nach Lemma~\ref{AblaefeVerfSpezLem} ein Präfix von $wv$ in $\StET
  _P$ enthalten sein. Mit $w=\prune (wv)$ folgt somit $w\in \ET _P \subseteq
  \EDT_P$. Falls in $P$ jedoch ein mit $wv$ beschrifteter Trace ausführbar ist
  und kein mit $wv$ beschrifteter Trace oder ein Präfix davon einen
  Fehler"=Zustand erreicht, muss es ein $p_j$ in $P$ geben, dass durch einen
  $wv$-Trace in $P$ erreicht wird und das mit einem $p'_j$ in der Relation
  $\mathcal{R}$ steht, dass in dem unendlichen Ablauf von $P'$ nach der letzten
  sichtbaren Aktion auftritt. $p'_j$ ist also ein divergenter Zustand für $P'$
  für den es einen unendlichen Ablauf gibt, der nur $\tau$-Transitionen enthält
  und es gilt $p'_j\mathcal{R} p_j$. Mit den Anmerkungen zu unendlichen
  Abläufen in Lemma~\ref{AblaefeVerfSpezLem} muss $P$ diesen unendlichen Ablauf
  von $\tau$s von $p_j$ aus matchen können. Es gilt also $p_j\in Div _P$ und
  somit $wv\in\StDT (P)$. Mit $w=\prune (wv)$ folgt $w\in\PrDT (P) \subseteq
  \EDT (P)$.

  3. \glqq$\subseteq$\grqq{}:
  {\allowdisplaybreaks
  \begin{align*}
    \QDT (P)&\overset{\ref{DivSemDef}}{=} \StQT (P)\cup \EDT (P)\\
    &\overset{\ref{StilleTraceProp}}{\subseteq} \left(\underset{P'\in \asimp
    (P)}{\bigcup} \StQT (P')\right)\cup \EDT (P)\\
    &\overset{2.}{=} \left(\underset{P'\in \asimp (P)}{\bigcup} \StQT
    (P')\right)\cup \left(\underset{P'\in \asimp (P)}{\bigcup} \EDT
    (P')\right)\\
    &= \underset{P'\in \asimp (P)}{\bigcup} \StQT (P') \cup \EDT (P')\\
    &\overset{\ref{DivSemDef}}{=} \underset{P'\in\asimp (P)}{\bigcup} \QDT
    (P').
  \end{align*}}

  3. \glqq$\supseteq$\grqq{}:\\
  Dieser Beweis verläuft analog zu dem Beweis von \glqq$\supseteq$\grqq{} der
  Proposition~\ref{StilleSemProp}, man muss nur die \ET{}-Mengen durch
  \EDT{}-Mengen ersetzten und für den Fall $w\in\EDT (P')$ folgt $w\in\EDT (P)$
  wegen des zweiten Punktes dieser Proposition und nicht wegen
  Proposition~\ref{KommSemProp}.

  4. \glqq$\subseteq$\grqq{}:
  {\allowdisplaybreaks
  \begin{align*}
    \EDL (P)&\overset{\ref{DivSemDef}}{=} L (P)\cup \EDT (P)\\
    &\overset{\ref{LImpProp}}{\subseteq} \left(\underset{P'\in \asimp
    (P)}{\bigcup} L (P')\right) \cup \EDT (P)\\
    &\overset{2.}{=} \left(\underset{P'\in \asimp (P)}{\bigcup} L (P')\right)
    \cup \left(\underset{P'\in \asimp (P)}{\bigcup} \EDT (P')\right)\\
    &= \underset{P'\in \asimp (P)}{\bigcup} L (P') \cup \EDT (P')\\
    &\overset{\ref{DivSemDef}}{=} \underset{P'\in\asimp (P)}{\bigcup} \EDL
    (P').
  \end{align*}}

  4. \glqq$\supseteq$\grqq{}:\\
  Für den Beweis dieser Inklusion kann man auf den Beweis
  von~\ref{KommSemProp}.2~\glqq$\supseteq$\grqq{} zurück greifen. Es müssen
  wie bei 3.\ nur die \ET{}-Mengen durch \EDT{}-Mengen ersetzt werden und die
  Einschränkung auf die geflutete Sprache ohne die Menge \EDT{} ist möglich
  wegen des zweiten Punktes der aktuellen Proposition.
\end{proof}

\TODO{Beispiel einfügen, wieso für 1. kein = gilt}

Aus der so eben bewiesenen Proposition über die Gleichheit der betrachteten
Traces, lässt sich wie in den letzten beiden Kapiteln eine Aussage über die
lokale Erreichbarkeit der fehlerhaften Zustände in einer Spezifikation und den
zugehörigen as"=Implementierungen treffen.

\begin{Kor}[lokale Divergenz Erreichbarkeit]\mbox{}
  \label{lokalDivErrKor}
  \begin{enumerate}[(i)]
    \item Falls in einem \MEIO{} $P$ ein Fehler lokal erreichbar ist, dann
      existiert auch eine as"=Implementierung, in der ein Fehler lokal
      erreichbar ist.
    \item Falls in einem \MEIO{} $P$ Divergenz lokal erreichbar ist, dann
      existiert auch eine as"=Implementierung, in der Divergenz lokal
      erreichbar ist.
    \item Falls ein \MEIO{} $P$ einen lokal erreichbaren stillen Zustand
      besitzt, dann existiert auch eine as"=Implementierung, in der ein stiller
      Zustand lokal erreichbar ist.
    \item Falls es eine as"=Implementierung von $P$ gibt, die Fehler, Stille
      oder Divergenz lokal erreicht, dann ist auch Fehler, Stille oder
      Divergenz in $P$ lokal erreichbar.
  \end{enumerate}
\end{Kor}
\begin{proof}\mbox{}
  \begin{enumerate}[(i)]
    \item Dieser Punkt folgt wie in~\ref{lokaleStilleErrKor} direkt aus
      Korollar~\ref{lokalFehlerErrKor}~(i).
    \item Ein divergenter Zustand ist in $P$ lokal erreichbar, wenn
      $\varepsilon \in \DT _P$ gilt. Mit~\ref{DivSemProp}.1 folgt daraus, dass
      es auch mindestens eine as"=Implementierung $P'$ aus $\asimp (P)$ geben
      muss, für die $\varepsilon$ in $\DT (P')$ enthalten ist. Da \DT{} die
      Menge der fortgesetzten um lokale Aktionen gekürzten strikten
      Divergenz-Traces ist, muss es lokale Aktionen in $P'$ geben, die zu einem
      divergenten Zustand führen. Es ist also auch in $P'$ Divergenz lokal
      erreichbar.
    \item Dieser Punkt folgt direkt aus Korollar~\ref{lokaleStilleErrKor}~(ii).
    \item In $P'\in\asimp (P)$ sei ein Fehler-, stillen oder Divergenz-Zustand
      lokal erreichbar. Es gilt dann $w\in\QDT _{P'}$ für $w\in O^*$. Mit
      Proposition~\ref{DivSemProp}.3 gilt auch $w\in\QDT _P$ Die Menge \QDT{}
      setzt sich aus den Mengen \ET{}, \StQT{} und \DT{} zusammen. Es muss also
      in $P$ ein Fehler-, stiller oder Divergenz-Zustand lokal erreichbar sein,
      da ein $w$, bestehend nur aus lokalen Aktionen, in $\QDT _P$ enthalten
      ist.
  \end{enumerate}
\end{proof}

Die Relation \DRel{} ist somit keine Einschränkung von \ERel{} so wie \QRel{}.
Diese Tatsache wird später im Satz~\ref{ZusammenhDivSatz} durch die Beispiele,
wieso diese Relationen unvergleichbar sind noch einmal verdeutlicht.
Es können Systeme mit einem Fehler nicht von Systemen mit Divergenz
unterschieden werden. Da die Divergenz-Test zwischen diesen \glqq
Fehler-Arten\grqq{} auch keine Unterscheidung machen, muss eine sinnvolle
Relation dies Eigenschaft auch übernehmen, so wie \DRel{} dies tut.

\begin{Satz}[Kommunikationsfehler-, Stillstands- und Divergenz-Semantik für
  Parallelkompositionen]
  \label{DivSemSatz}
  Für zwei komponierbare \MEIO{}s $P_1,P_2$ und ihre Komposition $P_{12}$ gilt:
  \begin{enumerate}
    \item $\EDT _{12} =\cont (\prune ((\EDT _1\|\EDL _2)\cup (\EDL _1\|\EDT
      _2)))$,
    \item $\QDT _{12} =(\QDT _1\|\QDT _2)\cup \EDT _{12}$,
    \item $\EDL _{12} =(\EDL _1\|\EDL _2)\cup \EDT _{12}$.
  \end{enumerate}
\end{Satz}

Man könnte diesen Satz analog zu den Sätzen~\ref{KommFehlerSemSatz}
und~\ref{StilleSemSatz} mit durch die Mengen-Inklusionen in beide Richtungen
beweisen. Alternativ kann man jedoch alle beteiligen \MEIO{}s Normieren, so
dass sie divergenz-frei sind, in dem Sinne, dass die \MEIO{}s keinen
divergenten Zustand mehr enthalten, der nicht auf in der Menge der
Fehler"=Zustände enthalten ist. In für einen \MEIO{}, für den alle Zustände aus
der Menge $Div$ in die Menge $E$ eingefügt wurden, gilt $\EDT = \ET$, $\QDT =
\QET$ und $\EDL = \EL$. Es wäre als der Satz~\ref{StilleSemSatz} anwendbar für
die Traces der Parallelkomposition. Es wäre dann nur noch zu zeigen, dass die
Parallelkomposition von zwei normierten \MEIO{}s die Normierung der
Parallelkomposition ihrer unnormierten \MEIO{}-Versionen ist.

\begin{Def}[Normierung zur Divergenz-Freiheit]
  \label{DivNormDef}
  Ein \MEIO{} $P$ \emph{divergenz-frei}, wenn $Div _P \subseteq E_P$ gilt.

  Die divergenz-frei Form von $P$ ist ein \MEIO{} $\DF (P)$, dam man aus $P$
  erhält, in dem man die Menge der Fehler"=Zustände $E_{\DF (P)}$ als die Menge
  $E_P \cup Div _P$ definiert.
\end{Def}

Die Menge $\MIT _P$, $\StQT _P$ und $L_P$ eines \MEIO{} $P$ bleiben durch die
Konstruktion in~\ref{DivNormDef} zur divergenz-freien Form $\DF (P)$
unverändert. Nur die Menge $\cont (\PrET) _P$ wird zur Menge $\cont (\PrET
_{\DF (P)})$ erweitert, in dem eine Vereinigung mit $\DT _P$ vorgenommen wird.
Semantisch ist jedoch nur die Trace-Menge \EDT{} relevant, die bereits die
Vereinigung der Menge \ET{} und $\DT{}$ ist. Die Mengen \QDT{} und \EDL{}
werden für beide \MEIO{}s mit der Trace Menge \EDT{} geflutet, somit gilt für
alle in der Semantik der Relation \DRel{} relevanten Traces die Gleichheit
zwischen den Traces des \MEIO{}s $P$ und seiner divergenz-freien Form $\DF
(P)$.

\begin{Prop}[Divergenz-Freiheit]
  \label{DivNormProp}
  Jedes \MEIO{} $P$ ist äquivalent zu seiner divergenz-freien Form $\DF (P)$
  bezüglich der Relation \DRel{}.
\end{Prop}

\vspace{0.2cm}

\begin{Lem}[Relationen unter Divergenz-Freiheit]
  Für zwei \MEIO{}s $P$ und $Q$ in divergenz-freier Form gilt: $P \DRel Q
  \Leftrightarrow P \QRel Q$.
\end{Lem}
\begin{proof}
  Da für einen \MEIO{} $P$ in divergenz-freier Form $Div _P \subseteq E _P$
  gilt, ist jeder strikte Divergenz-Trace von $P$ auch ein strikter
  Fehler-Trace von $P$. Auf beide Trace-Mengen wird die Kürzungs-Funktion
  \prune{} und danach die Verlängerungs"=Funktion \cont{} angewendet. In $\EDT
  _P$ führt die Erweiterung der Menge $\ET _P$ um die Menge $\DT _P$ somit zu
  keiner Vergrößerung. Es gilt also $\EDT _P = \ET _P$. Die Menge $\StQT _P$
  und $L _P$ werden im Fall \DRel{} mit der Menge $\EDT _P$ und im Fall \QRel{}
  mit der Menge $\ET _P$ geflutet. Da die Mengen mit denen geflutet wird gleich
  sind, gilt auch $\QDT _P = \QET _P$ und $\EDL _P = \EL _P$. Die Mengen
  Gleichheiten gelten analog auch für $Q$. Da die Mengen für die Inklusionen,
  die die Relationen \DRel{} und \QRel{} fordern, gleich sind, sind auch die
  Relationen äquivalent für \MEIO{}s in divergenz-freier Form.
\end{proof}

\begin{Lem}[Divergenz-Freiheit und Parallelkomposition]
  \label{DivFreiParallelLem}
  Die Parallelkomposition $P_{12} = P_1\|P_2$ zweier \MEIO{}s $P_1$ und $P_2$
  in divergenz-freier Form ist ebenfalls in divergenz-freier Form.
\end{Lem}
\begin{proof}
  Es gilt $Div _j \subseteq E_j$ für beide $j$-Werte. In einer
  Parallelkomposition entsteht ein Divergenz"=Zustand nur, wenn mindestens
  eine der beiden Komponenten bereits divergent war für den Zustand, der an der
  Komposition des Divergenz"=Zustandes beteiligt ist. $Div _{12}$ entspricht
  also der Menge $(Div _1 \times P_2) \cup (P_1 \times Div _2)$. Da geerbte
  Fehler"=Zustände in einer Parallelkomposition auch nur von einem der zu
  komponierenden Zustände geerbt werden müssen, gilt analoge für die Menge der
  Fehler"=Zustände. Es folgt also $Div _{12} \subseteq E _{12}$. $P_{12}$ ist
  also auch in divergenz-freier Form.
\end{proof}

\begin{proof}[Beweis zu Satz~\ref{DivSemSatz}]\mbox{}\\
  Durch Proposition~\ref{DivNormProp} kann man jeden \MEIO{} $P$ so normieren,
  dass es divergenz-frei ist, dass Ergebnis der Normierung jedoch noch
  äquivalent zu $P$ ist bezüglich der Relation \DRel{}. Für einen \MEIO{} $P$
  in divergenz-freier Form gilt jedoch $\EDT _P = \ET _P$, $\QDT _P = \QET _P$
  und $\EDL _P = \EL _P$. Der hier zu beweisende Satz, würde als für \MEIO{}s
  in divergenz-freier Form dem Satz~\ref{StilleSemSatz} aus dem letzten Kapitel
  entsprechen. Jedoch soll der Satz in diesem Kapitel nicht auf \MEIO{}s in
  divergenz-freier Form eingeschränkt werden, sondern für alle \MEIO{}s gelten.
  Es ist also noch zu beweisen, dass $\DF (P_{12})$ äquivalent zu $\DF
  (P_1)\|\DF (P_2)$ ist bezüglich der Relation \DRel{} bzw.\ bezüglich der
  Relation \QRel{}, da die Parallelkomposition von zwei \MEIO{}s in
  divergenz-freier Form auch eine divergenz-freie Form hat
  (Lemma~\ref{DivFreiParallelLem}). Die folgenden Punkte müssen dafür
  nachgewiesen werden:
  \begin{enumerate}[(i)]
    \item $\ET (\DF (P_{12})) = \ET (\DF (P_1)\|\DF (P_2))$,
    \item $\QET (\DF (P_{12})) = \QET (\DF (P_1)\|\DF (P_2))$,
    \item $\EL (\DF (P_{12})) = \EL (\DF (P_1)\|\DF (P_2))$.
  \end{enumerate}

  (i) \glqq $\subseteq$\grqq{}:\\
  Da die \ET{}-Mengen beide unter \cont{} abgeschlossen sind, kann man ein
  beliebiges präfix-minimales Element $w\in\ET (\DF (P_{12}))$ betrachten.
  Das $w$ kann ein Input-kritischer Trace sein oder ein gekürzter
  Fehler"=Trace. Im Fall $w\in\PrET (\DF (P_{12}))$ kann $w\in\PrET _{12}$
  gelten oder $w\in\PrDT _{12}$. Input-kritische Traces bleiben von der
  Normierung unberührt. Die Menge $\MIT (\DF (P_{12}))$ und $\MIT _{11}$
  entsprechen sich also.
  \begin{itemize}
    \item Fall 1 ($w\in \ET _{12}$): Es gilt mit Satz~\ref{KommFehlerSemSatz}.1
      $w\in \cont (\prune ((\ET _1\|\EL _2)\cup (\EL _1\|\ET _2)))$. Durch die
      Normierung können sich die Mengen \ET{} und \EL{} nicht verkleiner, sie
      können werden nur um die Divergenz"=Traces, die zu keinen Fehlern führen
      erweitert. Mit $\ET _j \subseteq\ET _{\DF (P_j)}$ und $\EL _j \subseteq
      \EL _{\DF (P_j)}$ für $j\in \{1,2\}$ folgt $w\in \cont (\prune ((\ET
      _{\DF (P_1)}\|\EL _{\DF (P_2)})\cup (\EL _{\DF (P_1)}\|\ET _{\DF
      (P_2)})))$. Wegen Satz~\ref{KommFehlerSemSatz}.1 ist $w$ auch in $\ET
      (\DF (P_1)\|\DF (P_2))$ enthalten.
    \item Fall 2 ($w\in \PrDT _{12}$): Es gibt eine Verländerung $v\in O_{12}$,
      des Wortes $w$, so dass $wv$ in $P_{12}$ zu einem divergenten Zustand
      führt. Damit der erreichte Zustand in der Parallelkomposition divergent
      ist, muss mindestens eines der Teilsysteme Divergenz an dem in der
      Kompositon beteiligten Zustand aufweisen. Das andere System muss nur den
      auf sein Teilsystem projezierten Trace von $wv$ ausführen können. Die
      Ausführbarkeit von Traces kann durch die Normierung auf ein
      divergenz-freies System nicht verändert werden. Der divergenten Zustand
      in dem Teilsystem wird jedoch in die Menge seiner Fehler"=Zustände
      eingeführt durch die Normierung. In $\DF (P_1)\|\DF (P_2)$ werden also
      ein Trace aus der Menge \ET{} des einen und aus der Menge $L$ des anderen
      Systems kompniert. Das Wort $wv$, dass dabei entsteht, ist also in $\ET
      (\DF (P_1)\|\DF (P_2))$ enthalten. Da $v$ nur aus lokalen Aktionen
      besteht, gilt auch $w\in\ET (\DF (P_1)\|\DF (P_2))$.
  \end{itemize}

  (i) \glqq $\supseteq$\grqq{}:\\
  Es genügt auch für diese Inklusion ein präfix-minimales Element zu
  betrachten, da beide \ET{}-Mengen unter \cont{} abgeschlossen sind. Für ein
  präfix-minimales $w$ aus der Menge $\ET (\DF (P_1)\|\DF (P_2))$ gilt mit
  Satz~\ref{KommFehlerSemSatz}.1 $w\in \cont (\prune ((\ET _{\DF (P_1)}\|\EL
  _{\DF (P_2)})\cup (\EL _{\DF (P_1)}\|\ET _{\DF (P_2)})))$. Es gibt also
  eine Projektion des Wortes $w$ auf die einzelnen Systeme, so dass $w\in w_1\|
  w_2$ gilt mit $w_1\in\ET _{\DF (P_1)}$ bzw.\ $w_1\in\EL _{\DT (P_1)}$ und
  $w_2\in\EL _{\DF (P_2)}$ bzw.\ $w_2\in\ET _{\DT (P_2)}$. Da das $w$
  präfix-minimal ist in $\DF (P_1)\|\DF (P_2)$ sind die $w_j\in \EL _{\DF
  (P_j)}$ entweder ausführbar ($w_j\in L _{\DF (P_j)}$) oder in der Menge $\MIT
  _{\DF (P_j)}$ enthalten. Die Normierung auf ein divergenz-freies System kann
  nichts an der Ausführbarkeit von Traces verändern. Falls das $w_j$ für
  $j\in\{1,2\}$ in der Menge $\cont (\MIT _{\DF (P_j)})$ enthalten ist, gilt
  auch $w_j\in\cont (\MIT _j)$, da die Normierung auf divergenz-freie Systeme
  die Input-kritischen Trace unverändert lässt. Für $w_j\in\cont (\PrET _{\DF
  (P_j)})$ können jedoch zwei Fälle unterschieden werden. $w_j$ kann bereits
  von der Normierung ein Trace aus der Menge $\cont (\PrET _j)$ gewesen sein
  oder $w_j$ ist erst durch die Normierung zu einem forgesetzten gekürzten
  Fehler-Trace geworden und es galt davor $\cont (\PrDT _j) = \DT _j$. Für die
  hier betrachten $w_j$ gilt $w_j\in \EL _j$, wenn nach der Normierung
  $w_j\in\EL _{\DF (P_j)}$. Es kommt also nur auf die Wörter $w_j$ in den
  \ET{}-Mengen an.
  \begin{itemize}
    \item Fall 1 ($w_j\in\ET _j$): In diesem Fall kann man
      Satz~\ref{KommFehlerSemSatz}.1 anwenden und es folgt $w\in\ET _{12}$ und
      nach der Anwendung der Konstuktion aus Definition~\ref{DivNormDef} gilt
      immer noch $w\in\ET _{\DF (P_{12})}$.
    \item Fall 2 ($w_j\in\DT _j$): \OBdA{} $j=1$. $w_2$ ist in der Menge $\EL
      _2$ enthalten. Wegen der präfix-Minimalität von $w$ ist die hier
      relevante unterscheidung $w_2\in\MIT _2$ oder $w_2\in L_2$.
      \begin{itemize}
        \item Fall 2a) ($w_2\in\MIT _2$): Es gibt ein Präfix von $w_1$, dass
          zu einem divergenten Zustand führt. Falls es durch dieses Präfix von
          $w_1$ noch nicht zu einem Problem mit dem fehlenden Input aus $w_2$
          in der Komposition kommt, wird der divergente Zustand von $P_1$ an
          die Parallelkomposition $P_1\|P_2$ vererbt und durch die
          Abgeschlossenheit gegenüber \cont{} folgt dann $w\in\DT _{12}
          \subseteq \ET _{\DF (P_{12})}$. Falls es jedoch zu
          Sychnronisations"=Problemen wegen des nicht sichergestellten Input in
          $P_2$ im Wort $w_2$ kommt, entsteht in der Parallelkomposition ein
          Trace aus der Menge $\ET _{12} \subseteq \ET _{\DF (P_{12})}$ durch
          Satz~\ref{KommFehlerSemSatz}.1, da dann das Präfix von $w_1$, dass zu
          den Sychronsiations"=Problemen führt ausführbar ist und somit in der
          Sprache und auch der Fehler"=gefluteten Sprache enthalten ist.
        \item Fall 2b) ($w_2\in L_2$): In diesem Fall ist $w_2$ komplett
          ausführbar in $P_2$. Das Präfix von $w_1$, dass zu einem
          Divergenz"=Zustand führt, kann also mit einem Präfix von $w_2$
          Komponiert werden und ergibt ein Präfix von $w$ in der
          Parallelkomposition $P_1\|P_2$, dass zu einem divergenten Zustand
          führt. Wegen der Abgeschlossenheit bezüglich \cont{} gilt $w\in \DT
          _{12}\subseteq \ET _{\DF (P_{12})}$.
      \end{itemize}
  \end{itemize}

  (ii):
  Man kann diese Gleichheit wieder in zwei Inklusitionen aufteilen, die jedoch
  durch die das bereits in (i) gezeigte eingeschränkt werden können, dabei
  entstehen dann die beiden folgenden Inklusionen: $\StQT (\DF (P_{12}))
  \backslash \ET (\DF (P_{12})) \subseteq\QET (\DF (P_1)\|\DF (P_2))$ und
  $\StQT (\DF (P_1)\|\DF (P_2)) \backslash \ET (\DF (P_1)\|\DF (P_2))
  \subseteq\QET (\DF (P_{12}))$. Da die Konstuktion der Normierung auf
  divergenz-freie Systeme in Definition~\ref{DivNormDef} jedoch keine
  Transitionen verändert, sondern nur die Zustands-Mengen $E$ erweitert, gilt
  sogar die Gleichheit $\StQT (\DF (P_{12})) \backslash \ET (\DF (P_{12})) =
  \StQT (\DF (P_1)\|\DF (P_2)) \backslash \ET (\DF (P_1)\|\DF (P_2))$. Ein Wort
  aus der Menge einer der \StQT{}-Mengen muss ausführbar sein zu einem stillen
  Zustand führen, da auf beiden Seiten ausgeschlossen wurde, dass das jeweilige
  Wort auch in der \ET{}-Menge enthalten ist, ist der erreicht Zustand ein
  Fehler"=Zustand. Es kann also Lemma~\ref{StilleZustLem} angewendet werden und
  es folgt auf beiden Seiten der Gleichung, das die in den projektionen
  erreichten Zustände in $P_1$ und $P_2$ still sein müssen. Aus zwei stillen
  Zuständen in der Komposition entsteht auch immer weider ein stiller Zustand
  mit Lemma~\ref{StilleZustLem}.

  (iii):
  Dieser Punkt könnte analog zum letzten Punkt durch die bereits bewiesenen
  Punkte eingeschränkte Inklusionen gezeigt werden, jedoch gilt hier ebenfalls
  bereits die Geleichheit für die Einschränkung. Es wird also direkt $L (\DF
  (P_{12})) = L (\DF (P_1)\|\DF (P_2))$ gezeigt. Die Normierung auf
  divergenz"=freie Systeme verändert die ausführbaren Traces nicht. Somit bleibt
  die Sprache gleich. Egal ob vor der Parallelkomposition normiert wurde oder
  erst danach, die Traces, die in den Parallelkomposition ausführbar sind,
  stimmen überein.
\end{proof}

Analog wie in den beiden vorangegangenen Kapitel, ergibt sich aus diesem Satz
als direkte Folgerung, dass es sich bei der Relation \DRel{} um eine
Präkongruenz handelt. Der Beweis dafür ist analog zu den Beweisen der
Korollare~\ref{KommPraekonKor} und~\ref{StillePraekonKor} und soll deshalb hier
nicht wiederholt werden.

\begin{Kor}[Divergenz-Präkongruenz]
  Die Relation \DRel{} ist eine Präkongruenz bezüglich $\cdot\|\cdot$.
\end{Kor}

Im nächsten Lemma soll eine Verfeinerung bezüglich guter Kommunikation
betrachtet werden. Die Vorgaben für gute Kommunikation gibt hierbei die
Definition der Tests und die daraus resultierende Verfeinerung
in~\ref{DivTestDef} vor. Es muss in diesem Lemma eine Veränderung zu den
analogen Lemmata aus den vorangegangenen Kapitel vorgenommen werden. Die
Einschränkung der Tests $T$ auf Partner, kann nicht mehr beibehalten werden, da
die Strategie zur Vermeidung von Stille im Beweis aus dem letzten Kapitel hier
zu Divergenz führen würde. Somit werden für die Stillstands-Vermeidung in
diesem Kapitel Aktionen außerhalb der Menge \Synch{} benötigt, die nicht die
interne Aktionen $\tau$ sind. Jedoch müssen trotzdem nicht alle Tests $T$
betrachtet werden. Es kann eine Einschränkung gemacht werden, sodass $T$ fast
ein Partner ist. Zur Vereinfachung von umständlichen Formulierungen im
Folgenden wird hierfür nun ein neuer Begriff definiert. Der jedoch auch bereits
so in z.B.~\cite{Schinko2016BA} analog für \EIO{}s verwendet und definiert
wurde.

\begin{Def}[{\boldmath$\chi$}-Partner]
  Ein \MEIO{} $P_1$ ist ein $\chi$-Partner von einem \MEIO{} $P_2$, wenn
  $I_1=O_2$ und $O_1=I_2\cup\{\chi\}$ mit $\chi\notin I_2\cup O_2$ gilt.
\end{Def}

Ein $\chi$-Partner $P_1$ von $P_2$ unterscheidet sich von einem Partner von
$P_2$ nur um den Output $\chi$, der nicht in der Menge $\Synch (P_1,P_2)$
enthalten ist.

\begin{Lem}[Testing-Verfeinerung mit Divergenz]
  \label{DivTestVerfeinLem}
  Gegeben sind zwei \MEIO{}s $P_1$ und $P_2$ mit der gleichen Signatur. Wenn
  für alle Tests $T$, die $\chi$-Partner von $P_1$ bzw. $P_2$ sind, $P_2
  \DsatAs T\Rightarrow P_1 \DsatAs T$ gilt, dann folgt daraus die Gültigkeit
  von $P_1\DRel P_2$.
\end{Lem}
\begin{proof}
  Da $P_1$ und $P_2$ die gleiche Signatur haben, definiert man $I:=I_1=I_2$ und
  $O:=O_1=O_2$. Für jeden $\chi$-Partner Test $T$ gilt $I_T=O$ und
  $O_T=I\cup\{\chi\}$ mit $\chi\notin I\cup O$.\\
  Um zu zeigen, dass die Relation $P_1\DRel P_2$ gilt, müssen die folgenden
  Punkte nachgewiesen werden:
  \begin{itemize}
    \item $\EDT _1\subseteq \EDT _2$,
    \item $\QDT _1\subseteq \QDT _2$,
    \item $\EDL _1\subseteq \EDL _2$.
  \end{itemize}
  In den Lemmata~\ref{KommTestVerfeinLem} und~\ref{StilleTestVerfeinLem} wurde
  bereits etwas Ähnliches gezeigt. Jedoch kann daraus aufgrund der
  unterschiedlichen Implikationen, die vorausgesetzt werden, nichts über dieses
  Lemma und dessen Gültigkeit ausgesagt werden. Es kann in diesem Lemma, ebenso
  wie im Lemma~\ref{StilleTestVerfeinLem}, aus der lokalen Erreichbarkeit eines
  Fehlers in einer Parallelkomposition einer as"=Implementierung von $P_1$ mit
  einem Test $T$ und der Implikation $P_2 \DsatAs T\Rightarrow P_1 \DsatAs T$
  nur geschlossen werden, dass es in einer Parallelkomposition einer
  as"=Implementierung von $P_2$ mit $T$ auch einen lokal erreichbaren
  fehlerhaften Zustand geben muss, jedoch kann die Fehlerhaftigkeit hier ein
  Fehler, Stille oder Divergenz sein. Analog verhält es sich, wenn in der
  Parallelkomposition einer as"=Implementierung von $P_1$ mit einem Test $T$
  ein Divergenz-Zustand oder stiller Zustand lokal erreichbar ist. Es kann nur
  geschlossen werden, dass $P_2$ den Test $T$ nicht erfüllen darf. Die nicht
  Erfüllung kann jedoch auf einem beliebigen Fehlverhalten des \MEIO{}s
  basieren.

  Als Erstes wird der erste Beweispunkt gezeigt, also die Inklusion $\EDT
  _1\subseteq\EDT _2$.\\
  Es wird für ein präfix-minimales $w$ aus $\EDT _1$ gezeigt, dass dieses $w$
  oder eines seiner Präfixe in $\EDT _2$ enthalten ist. Diese Möglichkeit
  bietet sich, da beide Mengen unter \cont{} abgeschlossen sind. Wegen
  Proposition~\ref{DivSemProp}.2 ist $w$ auch ein präfix-minimales Elemente
  der Menge $\EDT _{P'_1}$ einer as"=Implementierung $P'_1$ von $P_1$.
  \begin{itemize}
    \item Fall 1 ($w=\varepsilon$): Es handelt sich um einen lokal erreichbaren
      Fehler oder um lokale erreichbare Divergenz in $P'_1$. Für $T$ wird ein
      Transitionssysteme verwendet, das nur aus dem Startzustand und einer
      must"=Schleife für alle Inputs $x\in I_T$ und einer must"=Schleife für
      $\chi$ besteht. Somit kann $P'_1$ im Prinzip die gleichen Fehler- und
      Divergenz-Zustände wie $P'_1\|T$ lokal erreichen. $P_1$ erfüllt den Test
      $T$ nicht. $P_2$ darf $T$ somit auch nicht erfüllen. Es muss eine
      as"=Implementierung $P'_2$ von $P_2$ existieren, für die $P'_2\|T$ einen
      fehlerhaften Zustand lokal erreicht. Durch die Struktur von $T$ ist in
      einer Parallelkomposition mit $T$ kein stiller Zustand möglich. Der
      fehlerhafte Zustand, der in $P'_2\|T$ lokal erreichbar ist, muss also ein
      Fehler- oder Divergenz-Zustand sein. Da von $T$ kein Fehler und keine
      Divergenz geerbt werden kann und durch die Input"=Schleife auch kein
      neuer Fehler entstehen kann, muss der fehlerhafte Zustand von $P'_2$
      geerbt sein. Somit muss in $P'_2$ ein Fehler- oder Divergenz-Zustand
      lokal erreichbar sein. Da $\EDT (P) =\ET (P)\cup\DT (P)$ gilt, folgt
      $w\in \EDT _{P'_2}$ und mit~\ref{DivSemProp}.2 $w\in \EDT _2$.
    \item Fall 2 ($w=x_1\dots x_n x_{n+1}\in\Sigma ^+$ mit $n\geq 0$ und
      $x_{n+1}\in I$): Es wird der folgende $\chi$-Partner $T$ betrachtet
      (siehe auch Abbildung~\ref{TohneEmitO}):
      \begin{itemize}
        \item $T=\{p_0,p_1,\dots ,p_{n+1}\}$,
        \item $p_{0T}=p_0$,
        \item $\begin{aligned}[t]
            \may _T = \must _T&=\{(p_j,x_{j+1},p_{j+1})\mid  0\leq j\leq n\}\\
            &\cup\{(p_j,x,p_{n+1})\mid  x\in I_T\backslash\{x_{j+1}\}, 0\leq
            j\leq n\}\\
            &\cup\{(p_{n+1},x,p_{n+1})\mid  x\in I_T\}\\
            &\cup\{(p_j,\chi ,p_{n+1})\mid 0\leq j\leq n+1\},
        \end{aligned}$
        \item $E_T=\emptyset$.
      \end{itemize}
      \begin{figure} [h!tbp]
      \begin{center}
        \begin{tikzpicture}[auto,node distance =3cm, semithick]
          \node (0) {$p_0$};
          \node (1) [right of=0] {$p_1$};
          \node (dots) [right of=1] {$\dots$};
          \node (n) [right of=dots] {$p_n$};
          \node (n1) at ($(1)!0.5!(dots) + (0,-3)$) {$p_{n+1}$};

          \path[line width=1pt] (dots) edge [loosely dotted] (n1);

          \path[->, >=latex'] ($ (0) + (-1,0) $) edge (0)
                (0) edge node {$x_1$} (1)
                    edge [bend right] node [below, sloped] {$x?\neq x_1, \chi
                    !$} (n1)
                (1) edge node {$x_2$} (dots)
                    edge node [below, sloped] {$x?\neq x_2, \chi !$} (n1)
                (dots) edge node {$x_n$} (n)
                (n) edge node [above, sloped] {$x?\in I_T, \chi !$} (n1)
                    edge [bend left] node [sloped] {$x_{n+1}$!} (n1)
                (n1) edge [loop below] node {$x?\in I_T, \chi !$} (n1);
        \end{tikzpicture}
        \caption{$x?\neq x_j$ steht für alle $x\in I_T\backslash\{x_j\}$}
      \label{TohneEmitO}
      \end{center}
      \end{figure}
      Die Mengen der Divergenz- und stillen Zustände des hier betrachteten $T$s
      sind leer. Da im Vergleich zum Transitionssystem in
      Abbildung~\ref{UohneE} nur die $\chi$-Transitionen zu $p_{n+1}$ ergänzt
      und die Mengen unbenannt wurden, ändert sich nichts an dem Fall 2a) im
      ersten Punkt des Beweises von Lemma~\ref{KommTestVerfeinLem}. Im Fall 2b)
      muss die Menge $O^*$ durch $(O\cup\{\chi\})^*$ ersetzt werden. Die
      Begründungen, wieso in den beiden Fällen $\varepsilon\in\PrET (P'_1\|T)$
      für ein $P'_1\in\asimp (P_1)$ gilt, bleibt also analog zum Beweis von
      Lemma~\ref{KommTestVerfeinLem}. Da nun aber auch Divergenz betrachtet
      wird, muss ein weiterer Fall ergänzt werden:
      \begin{itemize}
        \item Fall 2c) ($w\in\PrDT _{P'_1}$): In $P'_1\|T$ erhält man
          $(p_{01},p_0) \weakmay[w] (p'',p_{n+1})\weakmay[u] (p',p_{n+1})$ für
          $u\in (O\cup \{\chi\})^*$ und $p'\in Div_1$. Daraus folgt
          $(p',p_{n+1})\in Div_{P'_1\|T}$ und somit $wu\in\StDT (P'_1\|F)$. Da
          alle Aktionen aus $w$ synchronisiert werden und $I_T\cap
          I=\emptyset$ gilt $x_1,\dots , x_n, x_{n+1}\in O_{P'_1\|T}$. Da
          zusätzlich $u$ in $(O\cup \{\chi\}) ^*$ enthalten ist, folgt $u\in
          O^*_{P'_1\|T}$. Somit ergibt sich $\varepsilon\in\PrDT (P'_1\|T)$.
      \end{itemize}
      Da $\varepsilon$ in $\PrET (P'_1\|T) \cup \PrDT (P'_1\|T)$ enthalten ist,
      ist ein Fehler oder Divergenz lokal erreichbar in $P'_1\|T$. Mit der
      Implikation $P_2 \DsatAs T\Rightarrow P_1 \DsatAs T$ kann geschlossen
      werden, dass in der Parallelkomposition einer as"=Implementierung $P'_2$
      von $P_2$ mit dem Test $T$ ein fehlerhafter Zustand lokal erreichbar sein
      muss. Durch die $\chi$-Transitionen an den Zuständen von $T$ kann es in
      Komposition mit $T$ keine stillen Zustände geben. Die Fehlerhaftigkeit
      muss also ein Fehler oder Divergenz sein.
      \begin{itemize}
        \item Fall 2i) ($\varepsilon\in\ET (P'_1\|T)$ wegen neuem Fehler): Da
          jeder Zustand von $T$ alle Inputs $x\in I_T=O$ zulässt, muss ein
          lokal erreichbarer Fehler-Zustand in diesem Fall der Form sein, dass
          ein Output $a\in O_T\backslash\{\chi\}$ von $T$ möglich ist, der
          nicht mit einem passenden Input aus $P'_2$ synchronisiert werden
          kann. Durch die Konstruktion von $T$ ist in $p_{n+1}$ kein Output
          außer $\chi$ möglich. Ein neuer Fehler muss also die Form
          $(p',p_j)$ haben mit $j\leq n$, $p'\nmust[x_{i+1}]_{P'_2}$ und
          $x_{i+1}\in O_T\backslash\{\chi\}$. Durch Projektion erhält man
          dann $p_{02} \lweakmay[x_1\dots x_i]_{P'_2} p'
          \nmust[x_{i+1}]_{P'_2}$ und damit gilt $x_1\dots x_{i+1}\in\MIT
          _{P'_2}\subseteq \ET _{P'_2}$. Somit ist ein Präfix von $w$ in $\EDT
          _{P'_2}$ enthalten. Wegen des Abschlusses unter \cont{} und wegen
          Proposition~\ref{DivSemProp}.2 gilt $w\in\EDT _2$.
        \item Fall 2ii) ($\varepsilon\in\ET (P'_2\|T)$ wegen geerbtem Fehler):
          $T$ hat $x_1\dots x_iu$ ausgeführt mit $u\in (O\cup\{\chi\})^*$ und
          ebenso hat $P'_2$ den Weg $x_1\dots x_iu|_{\Sigma _2}$ ausgeführt.
          Durch dies hat $P'_2$ einen Zustand aus $E_{P'_2}$ erreicht, da von
          $T$ kein Fehler geerbt werden kann. Es gilt dann $\prune (x_1\dots
          x_iu| _{\Sigma _2})=\prune (x_1\dots x_i)\in\PrET _{P'_2}\subseteq
          \ET _{P'_2}$. Da $x_1\dots x_i$ ein Präfix von $w$ ist, führt in
          diesem Fall eine Verlängerung um lokale Aktionen von einem Präfix von
          $w$ zu einem Fehler-Zustand. Da \ET{} der Menge aller Verlängerungen
          von gekürzten Fehler-Traces entspricht, ist $x_1\dots x_i$ in $\EDT
          _{P'_2}$ enthalten und mit~\ref{DivSemProp}.2 ist ein Präfix von
          $w$ in $\EDT _2$ enthalten.
        \item Fall 2iii) ($\varepsilon\in\DT (P'_2\|T)\backslash\ET
          (P'_2\|T)$): Da $T$ nicht unendlich viele Zustände hat und auch keine
          $\tau$-Schleifen besitzt, kann das Divergenz"=Verhalten nur von
          $P'_2$ geerbt sein. $T$ hat $x_1\dots x_iu$ ausgeführt mit $u\in (O
          \cup \{\chi\})^*$ und ebenso hat $P'_2$ den Weg $x_1\dots
          x_iu|_{\Sigma _2}$ ausgeführt. Durch dies hat $P'_2$ einen Zustand
          aus $Div_{P'_2}$ erreicht. Es gilt dann $\prune (x_1\dots
          x_iu|_{\Sigma _2}) = \prune (x_1\dots x_i)\in\PrDT _{P'_2} \subseteq
          \DT _{P'_2}$, da $u|_{\Sigma _2}$ in $O^*$ enthalten ist. Da
          $x_1\dots x_i$ ein Präfix von $w$ ist, führt in diesem Fall eine
          Verlängerung um lokale Aktionen von einem Präfix von $w$ zu einem
          divergenten Zustand. Da \DT{} die Menge aller Verlängerungen von
          gekürzten Divergenz-Traces ist und $\DT _{P'_2}\subseteq \EDT
          _{P'_2}$ gilt, ist in diesem Fall das Präfix $x_1\dots x_i$ von $w$
          in $\EDT _{P'_2}$ enthalten. Mit~\ref{DivSemProp}.2 folgt daraus,
          dass ein Präfix von $w$ in $\EDT _2$ enthalten ist.
      \end{itemize}
  \end{itemize}

  Als nächstes wird nun der zweite Beweispunkt gezeigt, d.h.\ die Inklusion
  $\QDT _1\subseteq\QDT _2$. Diese Inklusion kann jedoch noch, analog zum
  Beweis der Inklusion der Fehler-gefluteten Sprache aus dem Fehler-Kapitel,
  weiter eingeschränkt werden. Da bereits bekannt ist, das $\EDT _1 \subseteq
  \EDT _2$ gilt, muss nur noch $\StQT _1\backslash \EDT _1\subseteq\QDT _2$
  gezeigt werden.\\
  Es wird ein $w\in\StQT _1\backslash\EDT _1$ gewählt und gezeigt, dass dieses
  auch in $\QDT _2$ enthalten ist. Das $w$ ist aufgrund der
  Propositionen~\ref{StilleTraceProp} und~\ref{DivSemProp}.2 auch für eine
  as"=Implementierung $P'_1$ von $P_1$ in $\StQT _{P'_1}\backslash \EDT
  _{P'_1}$ enthalten.\\
  Durch die Wahl des $w$s wird in $P'_1$ durch das Wort $w$ ein stiller
  Zustand erreicht. Dies hat nur Auswirkungen auf die Parallelkomposition
  $P'_1\|T$, wenn in $T$ ebenfalls ein stiller Zustand durch $w$ erreicht
  wird.\\
  Das betrachtete $w$ hat die Form $w = x_1\dots x_n\in\Sigma ^*$ mit $n\geq
  0$. Es wird der folgende $\chi$-Partner Test $T$ betrachtet (siehe auch
  Abbildung~\ref{TohneEmitIundO}):
  \begin{itemize}
    \item $T=\{p_0,p_1,\dots ,p_n, p\}$,
    \item $p_{0T}=p_0$,
    \item $\begin{aligned}[t]
        \may _T = \must _T &= \{(p_j,x_{j+1},p_{j+1})\mid  0\leq j< n\}\\
        &\cup\{(p_j,x,p)\mid  x\in (I_T\cup \{\chi\})\backslash\{x_{j+1}\},
        0\leq j< n\}\\
        &\cup\{(p_n,x,p)\mid x\in I_T\}\\
        &\cup\{(p,x,p)\mid x\in I_T\cup\{\chi\}\},
    \end{aligned}$
    \item $E_T=\emptyset$.
  \end{itemize}
  \begin{figure} [h!tbp]
  \begin{center}
    \begin{tikzpicture}[auto,node distance =3cm, semithick]
      \node (0) {$p_0$};
      \node (1) [right of=0] {$p_1$};
      \node (dots) [right of=1] {$\dots$};
      \node (n) [right of=dots, rectangle, dotted, draw] {$p_n\in Qui_T$};
      \node (p) at ($(1)!0.5!(dots) + (0,-3)$) {$p$};

      \path[line width=1pt] (dots) edge [loosely dotted] (p);

      \path[->, >=latex'] ($ (0) + (-1,0) $) edge (0)
            (0) edge node {$x_1$} (1)
                edge [bend right] node [below, sloped] {$x?\neq x_1, \chi !$}
                (p)
            (1) edge node {$x_2$} (dots)
                edge [below, sloped] node {$x?\neq x_2, \chi !$} (p)
            (dots) edge node {$x_n$} (n)
            (n) edge [bend left] node [below,sloped] {$x?\in I_T$} (p)
            (q) edge [loop below] node {$x?\in I_T, \chi !$} (p);
    \end{tikzpicture}
    \caption{$x?\neq x_j$ steht für alle $x\in I_T\backslash\{x_j\}$, $p_n$
      ist der einzige stille Zustand}
    \label{TohneEmitIundO}
  \end{center}
  \end{figure}
  Falls das betrachtete $w$ dem leeren Wort $\varepsilon$ entspricht, reduziert
  sich der $\chi$-Partner Test $T$ auf den Zustand $p_n = p_0$ und den
  Zustand $p$. Es ist also in diesem Fall der Startzustand gleich dem stillen
  Zustand.\\
  Allgemein ist der Zustand $p_n$ aus $T$ der einzige stille Zustand in $T$. Es
  gilt wegen des ersten Punkte von Lemma~\ref{StilleZustLem}, dass auch in der
  Parallelkomposition $P'_1\|T$ ein stiller Zustand mit $w$ erreicht wird. Da
  es sich bei allen in $w$ befindlichen Aktionen um synchronisieren Aktionen
  handelt und $I_T\cap I=\emptyset$ gilt, folgt $w\in O_{P'_1\|T}$ und $w\in
  \StQT (P'_1\|T)$. Es kann also in der Parallelkomposition durch $w$ ein
  stiller Zustand lokal erreicht werden. Da $w\notin\EDT _{P'_1}$, kann auf dem
  Weg, der mit $w$ im Transitionssystem $P'_1$ zurück gelegt wird, kein Fehler-
  oder Divergenz-Zustand lokal erreicht werden. Es kann weder von $P'_1$
  noch von $T$ ein Fehler oder Divergenz auf diesem Weg geerbt werden oder neu
  entstehen. Da ein stiller Zustand in $P'_1\|T$ lokal erreichbar ist, erfüllt
  $P_1$ den Test $T$ nicht. Es muss also auch eine as"=Implementierung $P'_2$
  von $P_2$ geben, für die $P'_2\|T$ einen lokal erreichbaren fehlerhaften
  Zustand besitzt. Hier kann jedoch zunächst keine Aussage darüber getroffen
  werden, ob das $w$ in der Parallelkomposition $P'_2\|T$ ausführbar ist und ob
  es sich bei der Fehlerhaftigkeit um Fehler, Stille oder Divergenz handelt.
  \begin{itemize}
    \item Fall a) ($\varepsilon \in \ET (P'_2\|T)$): Der lokal erreichbare
      fehlerhafte Zustand ist ein Fehler. Das $w$ muss somit nicht ausführbar
      sein. Der Fehler kann sowohl von $P'_2$ geerbt sein, wie durch fehlende
      Synchronisations-Sicherstellung als neuer Fehler in der
      Parallelkomposition entstanden sein. Da nur auf dem Trace $w$ in $T$
      Synchronisations"=Probleme auftreten können und wegen den Fällen 2i) und
      2ii) des ersten Punktes dieses Beweises ist ein Präfix von $w$ in $\EDT
      _{P'_2}$ enthalten. Da die Menge \EDT{} unter \cont{} abgeschlossen ist
      und~\ref{DivSemProp}.2 gilt, folgt $w\in\EDT _2\subseteq\QDT _2$.
    \item Fall b) ($\varepsilon \in \DT (P'_2\|T)\backslash \ET(P'_2\|T)$): Es
      handelt sich bei dem lokal erreichbaren fehlerhaften Zustand um
      Divergenz. Die Divergenz muss von $P'_2$ geerbt sein, da $T$ keine
      Möglichkeit für eine unendliche $\tau$-Folge hat. Es gilt also, dass
      bereits in $P'_2$ ein Präfix von $w$ in $\EDT _{P'_2}$ enthalten ist,
      wegen Fall 2iii) des Beweises des ersten Punktes dieses Lemmas. Mit dem
      Abschluss unter \cont{} und~\ref{DivSemProp}.2 folgt, dass auch $w\in
      \EDT _2\subseteq\QDT _2$ gilt.
    \item Fall c) (stiller Zustand lokal erreichbar in $P'_2\|T$ und $\varepsilon
      \notin \EDT (P'_2\|T)$): Da in $T$ nur durch $w$ ein stiller Zustand
      erreicht werden kann, muss es sich bei dem lokal erreichbaren
      stillen Zustand in $P'_2\|T$ um einen handeln, der mit $w$ erreicht werden
      kann. Mit dem zweiten Punkt von Lemma~\ref{StilleZustLem} kann gefolgert
      werden, dass auch in $P'_2$ ein stiller Zustand mit $w$ erreichbar sein
      muss, da $\ET (P'_2\|T)$ eine Teilmenge von $\EDT (P'_2\|T)$ ist. Es gilt
      also $w\in\StQT _{P'_2}\subseteq \QDT _{P'_2}$. Mit dem dritten Punkt von
      Proposition~\ref{DivSemProp} folgt daraus $w\in\QDT _2$.
  \end{itemize}

  Nun wird mit dem letzten Punkt des Beweises begonnen. Analog wie in den
  Beweisen zu den Lemmata~\ref{KommTestVerfeinLem} und~\ref{StilleTestVerfeinLem}
  ist hier aufgrund der bereits geführten Beweisteile nur noch $L_1\backslash
  \EDT _1 \subseteq \EDL _2$ zu zeigen. Es wird also für ein beliebig gewähltes
  $w\in L_1\backslash \EDT _1$ gezeigt, dass es auch in $\EDL _2$ enthalten
  ist. Es gilt auch $w\in L _{P'_1}\backslash \EDL _{P'_1}$ für eine
  as"=Implementierung $P'_1$ von $P_1$, wegen~\ref{LImpProp}
  und~\ref{DivSemProp}.
  \begin{itemize}
    \item Fall 1 ($w = \varepsilon$): Analog zu den
      Lemmata~\ref{KommTestVerfeinLem} und~\ref{StilleTestVerfeinLem} gilt auch
      hier, dass $\varepsilon$ immer in $\EDL _2$ enthalten ist.
    \item Fall 2 ($w = x_1\dots x_n$ mit $n \geq 1$): Die Konstruktion des
      $\chi$-Partner Tests $T$ weicht nur durch die $\chi$-Transitionen vom
      Transitionssystem aus dem Beweis der Inklusion der Fehler-gefluteten
      Sprache \EL{} aus Lemma~\ref{KommTestVerfeinLem} ab. Der Test $T$ ist
      dann wie folgt definiert (siehe dazu auch Abbildung~\ref{TmitEundO}):
      \begin{itemize}
        \item $T=\{p_0,p_1,\dots ,p_n,p\}$,
        \item $p_{0T}=p_0$,
        \item $\begin{aligned}[t]
            \may _T = \must _T &= \{(p_j,x_{j+1},p_{j+1})\mid 0\leq j< n\}\\
            &\cup\{(p_j,x,p)\mid x\in I_T\backslash\{x_{j+1}\},0\leq j < n\}\\
            &\cup\{(p_j,\chi ,p)\mid 0\leq j\leq n\}\\
            &\cup\{(p,x ,p)\mid x\in I_T\cup \{\chi\}\},
        \end{aligned}$
        \item $E_T=\{p_n\}$.
      \end{itemize}
      \begin{figure} [h!tbp]
      \begin{center}
        \begin{tikzpicture}[auto,node distance =3cm, semithick]

          \node (0) {$p_0$};
          \node (1) [right of=0] {$p_1$};
          \node (dots) [right of=1] {$\dots$};
          \node (n1) [right of=dots] {$p_{n-1}$};
          \node (n) [right of=n1, rectangle, draw] {$p_n\in E_T$};
          \node (p) at ($(dots) + (0,-3)$) {$p$};

          \path[line width=1pt] (dots) edge [loosely dotted] (p);

          \path[->, >=latex'] ($ (0) + (-1,0) $) edge (0)
                (0) edge node {$x_1$} (1)
                    edge [bend right] node [below, sloped] {$x?\neq x_1, \chi
                    !$} (p)
                (1) edge node {$x_2$} (dots)
                    edge node [below, sloped] {$x?\neq x_2, \chi !$} (p)
                (dots) edge node {$x_{n-1}$} (n1)
                (n1) edge node {$x_n$} (n)
                     edge node [below, sloped] {$x?\neq x_n, \chi !$} (p)
                (p) edge [loop below] node {$x?\in I_T, \chi !$} (p)
                (n) edge [bend left] node {$\chi !$} (p);
        \end{tikzpicture}
        \caption{$x?\neq x_j$ steht für alle $x\in I_T\backslash\{x_j\}$, $p_n$
          ist der einzige Fehler-Zustand}
      \label{TmitEundO}
      \end{center}
      \end{figure}
      Durch die $\chi$-Transition an den Zuständen wird wie oben vermieden,
      dass es in einer Komposition mit $T$ und auch in $T$ selbst
      stille Zustände geben kann. Da $p_{01} \weakmust[w]_{P'_1} p'_1$ gilt,
      kann man schließen, dass $P'_1\|T$ einen lokal erreichbaren geerbten
      Fehler hat. Es muss also auch eine as"=Implementierung $P'_2$ von $P_2$
      geben, sodass $P'_2\|T$ einen lokal erreichbaren fehlerhaften Zustand
      hat. Wie oben bereits erwähnt, kommt Stille als Fehlerhaftigkeit nicht in
      Frage.
      \begin{itemize}
        \item Fall 2a) \big(neuer Fehler aufgrund von $x_i\in O_T\backslash
          \{\chi\}$ und $p_{02} \lweakmay[x_1\dots x_{j-1}] p'_2
          \nmust[x_j]$\big): Es gilt $x_1\dots x_j\in\MIT _{P'_2}$ und somit
          $w\in\EDL _{P'_2}$ Anzumerken ist, dass nur auf diesem Weg Outputs
          von $T$ aus der Menge $\Synch (P'_2,T)$ möglich sind, deshalb gibt es
          keine anderen Outputs von $T$, die zu einem neuen Fehler führen
          könnten. Wegen~\ref{DivSemProp} gilt $w\in\EDL _2$.
      \end{itemize}
      Die restlichen Fälle sind analog zu Lemma~\ref{KommTestVerfeinLem}
      möglich. Somit gilt für alle Fälle (2a) bis 2d)), dass $w$ in $\EDL _2$
      enthalten ist, da $\EL _2\subseteq \EDL _2$ gilt.
      \begin{itemize}
        \item Fall 2e) (Divergenz und kein neuer Fehler): Da $T$ keine
          Möglichkeit hat zu divergieren, muss diese Möglichkeit von $P'_2$
          geerbt sein. Es gilt dann $p_{02} \lweakmay[x_1\dots x_j u]_{P'_2} p'
          \in Div _{P'_2}$ für $j\geq 0$ und $u\in O^*$. Somit ist $x_1\dots
          x_j u\in\StDT _{P'_2}$ und damit $\prune (x_1\dots x_j u) = \prune
          (x_1\dots x_j)\in\PrDT _{P'_2}\subseteq\EDT _{P'_2}$. Also folgt
          mit~\ref{DivSemProp}.2, dass $w$ in $\EDT _2 \subseteq\EDL _2$
          enthalten ist, da \DT{} unter \cont{} abgeschlossen ist.
      \end{itemize}
  \end{itemize}
\end{proof}

\begin{Satz}
  \label{DivTestVerfSatz}
  Aus $P_1 \DRel P_2$ folgt, dass $P_1$ $P_2$ Divergenz-verfeinert.
\end{Satz}
\begin{proof}
  Ein Fehler-, stille oder Divergez-Zustand ist nach Definition genau dann in
  einem \MEIO{} $P$ lokal erreichbar, wenn $w\in\QDT _P$ für ein $w\in O_P^*$
  gilt.\\
  Um nachzuweisen, dass $P_1$ $P_2$ Divergenz-verfeinert, muss bewiesen werden,
  dass für alle Tests $T$ von $P_1$ bzw.\ $P_2$ $P_2\DsatAs T \Rightarrow P_1
  \DsatAs T$ gilt. Anstatt diese Aussage direkt zu beweisen, wird hier die
  äquivalente Aussage $\neg P_1\DsatAs T\Rightarrow\neg P_2\DsatAs T$ für alle
  Test $T$ von $P_1$ und $P_2$ nachgewiesen. Sei $T$ ein belibiger Test $T$ von
  $P_1$, für den $\neg P_1\DsatAs T$ gilt. Für eine as"=Implementierung $P'_1$
  von $P_1$ wird also ein Fehler-, stille oder Divergez-Zustand in $P'_1\|T$
  lokal erreicht. Es muss ein $w\in O_{P'_1\|T}^*$ geben, das in der Menge
  $\QDT _{P'_1\|T}$ enthalten ist. Ein Element der Menge $\QDT _{P'_1\|T}$ kann
  in dieser enthalten sein, wenn es nach Definition~\ref{DivSemDef} Element
  einer der zwei folgenden Mengen ist: $\EDT _{P'_1\|T}$ oder $\StQT _{P'_1\|T}$.
  \begin{itemize}
    \item Fall 1 ($w\in\EDT _{P'_1\|T}$): Da $w$ nur aus Outputs besteht, muss
      auch $\varepsilon\in\EDT _{P'_1\|T}$ gelten. Mit Satz~\ref{DivSemSatz}
      folgt $\varepsilon\in\cont (\prune ((\EDT _{P'_1}\|\EDL _T)\cup (\EDL
      _{P'_1}\|\EDT _T)))$. Durch Proposition~\ref{DivSemProp} und die
      präfix-Minimalität von $\varepsilon$ folgt $\varepsilon\in\prune ((\EDT
      _1\|\EDL _T)\cup (\EDL _1\|\EDT _T))$. Eine Verlängerung $w'\in
      O_{P_1\|T}$ von $\varepsilon$ um lokale Aktionen muss in der Menge $\EDT
      _1\|\EDL _T$ oder in der Menge $\EDL _1\|\EDT _T$ enthalten sein. Es muss
      also eine Projektion des Wortes $w$ auf die einzelnen Komponenten geben,
      so dass $w\in w_1\|w_T$ mit $w_1\in\EDT _1$ bzw.\ $w_1\in\EDL _1$ und
      $w_T\in\EDL _T$ bzw.\ $w_T\in\EDT _T$ gilt. Mit der Voraussetzung $P_1
      \DRel P_2$ folgt $w_1\in\EDT _2$ bzw.\ $w_1\in\EDL _2$. Wegen
      Proposition~\ref{DivSemProp} muss es ein $P'_2$ aus $\asimp (P_2)$ geben,
      für dass $w_1\in\EDT _{P'_2}$ bzw.\ $w_1\in\EDL _{P'_2}$ gilt. In der
      Parallelkomposition mit $T$ gilt also $w'\in (\EDT _{P'_2}\|\EDL _T)\cup
      (\EDL _{P'_2}\|\EDT _T)$. Da $P_1$ und $P_2$ die gleiche Signatur haben
      gilt $w'\in O _{P'_2\|T}$ und somit $\varepsilon\in\prune ((\EDT
      _{P'_2}\|\EDL _T)\cup (\EDL _{P'_2}\|\EDT _T))\subseteq \EDT _{P'_2\|T}$,
      wegen Satz~\ref{DivSemSatz}. In der Parallelkomposition $P'_2\|T$ ist ein
      Fehler- oder Divergenz-Zustand lokal erreichbar. Für die Spezifikation
      $P_2$ von $P'_2$ folgt somit also $\neg P_2\DsatAs T$.
    \item Fall 2 ($w\in\StQT _{P'_1\|T} \backslash\EDT _{P'_1\|T}$): Das Wort
      $w$ muss in $P'_1\|T$ zu einem Zustand $(p_1,p_T)$ führen, der still und
      keine Fehler ist. Die Zustände $p_1$ und $p_T$, die in Kompositon den
      Zustand in $P_1\|P_T$ ergeben müssen nach Lemma~\ref{StilleZustLem}.2
      ebenfalls sill sein. Es gibt also Wörter $w_1$ und $w_T$, so dass $w_1$
      in $P'_1$ den Zustand $p_1$ und $P_t$ den Zustand $p_T$ erreicht mit
      $w\in w_1\|w_T$. $w_1$ und $w_T$ sind also in der Menge \StQT{} ihres
      Transitionssystems enthalten. Für $w_1$ gilt $w_1\in\StQT _{P'_1}
      \subseteq \QDT _{P'_1}\subseteq\QDT _1$, wegen
      Proposition~\ref{DivSemProp}. $w_1\in\QDT _2$ folgt durch die
      Voraussetzung $P_1\DRel P_2$. Mit Proposition~\ref{DivSemProp} muss es
      eine as"=Implementierung $P'_2$ von $P_2$ geben, für die $w_1$ in der
      Menge $\QDT _{P'_2} = \EDT _{P'_2} \cup \StQT _{P'_2}$ enthalten ist.
      \begin{itemize}
        \item Fall 2a) ($w\in\EDT _{P'_2}$): Für das Wort $w_T$ gilt $w_T\in L
          _T\subseteq\EDL _T$, da es in $T$ ausfühbar ist. Mit
          Satz~\ref{DivSemSatz}.1 folgt $w\in\EDL _{P'_2\|T}$. Da $w$ nur aus
          Outputs besteht, ist ein Fehler- oder Divergenz-Zustand in $P'_2\|T$
          lokal erreichbar wie in Fall 1 dieses Beweises. Es folgt $\neg P_2
          \DsatAs T$.
        \item Fall 2b) ($w_1\in\StQT _{P'_2}$): Durch $w_1$ wird in $P'_2$ ein
          stiller Zustand $p_2$ erreicht. In der Parallelkomposition ist der
          Zustand $(p_2,p_T)$, wegen Lemma~\ref{StilleZustLem}.1, ebenfalls
          still. Der Zustand wird durch $w_1\|w_T$ erreicht. Es gilt also
          $w\in\StQT _{P'_2\|T}$. Das $w$ besteht auch für die
          Parallelkomposition $P'_2\|T$ nur aus Outputs, da $P_1$ und $P_2$
          nach Voraussetzung die selbe Signatur besitzten müssen. Es ist also
          ein stiller Zustand lokal erreichbar in $P'_2\|T$. $\neg P_2\DsatAs
          T$ gilt also auch in diesem Fall.
      \end{itemize}
  \end{itemize}
\end{proof}

Es wurde, wie in den letzten beiden Kapiteln, eine Kette an Folgerungen
gezeigt, die sich zu einem Ring schließt. Jedoch ändert sich an der Begründung
für einen der Folgepfeile etwas, da in Lemma~\ref{DivTestVerfeinLem} $T$ kein
Partner ist, sondern ein $\chi$-Partner. Die Folgerungskette ist in
Abbildung~\ref{FolgerungsketteDiv} dargestellt.

\begin{figure}[h!tbp]
  \begin{center}
    \begin{tikzpicture}[scale = 3]
      \matrix (m) [matrix of math nodes,row sep=2cm,column sep=4cm]{%
        P_1\DRel P_2 & P_1 \text{ verfeinert } P_2 \\
        \substack{\forall \text{ Test } \chi\text{-Partner } T:\\P_2\DsatAs
        T\Rightarrow P_1\DsatAs T} &
    \substack{\forall \text{ Tests } T:\\P_2\DsatAs T\Rightarrow P_1\DsatAs T} \\};
        \draw[-implies, double, double distance=1mm]
          (m-1-1) -- node [above] {Satz~\ref{DivTestVerfSatz}} (m-1-2);
        \draw[-implies, double, double distance=1mm]
          (m-1-2) -- node [right] {Definition~\ref{DivTestDef}} (m-2-2);
        \draw[-implies, double, double distance=1mm]
          (m-2-1) -- node [left]
          {Lemma~\ref{DivTestVerfeinLem}} (m-1-1);
        \draw[-implies, double, double distance=1mm]
          (m-2-2) -- node [below]
          {$\substack{T \text{ Test } \chi\text{-Partner}\\\Downarrow\\ T \text{ Test}}$} (m-2-1);
    \end{tikzpicture}
    \caption{Folgerungskette der Testing-Verfeinerung und Divergenz-Relation}
  \label{FolgerungsketteDiv}
  \end{center}
\end{figure}

Das nächste Korollar macht die Äquivalenz, die durch die
Abbildung~\ref{FolgerungsketteDiv} deutlich wird, explizit.

\begin{Kor}
  Es gilt: $P_1\DRel P_2 \Leftrightarrow P_1$ Divergenz-verfeinert $P_2$.
\end{Kor}
