\section{Hiding}

In diesem Kapitel soll untersucht werden, ob das Verbergen von Outputs
Auswirkungen auf die Verfeinerungs-Relationen \ERel{} hat.\\
Der relationale Zusammenhang der Basisrelation \EBRel{} bleibt unter der
Internalisierung von Outputs erhalten. Es gilt also $P_1\EBRel P_2 \Rightarrow
P_1/X\EBRel P_2/X$ für alle zulässigen Aktionsmengen $X$. Die Begründung ist
analog zu der in~\cite{Schinko2016BA}. Der Operator $\cdot /\cdot$ verändert
nichts an der Fehler-Erreichbarkeit. Der Ablauf mit dem der Fehler erreicht
wird, enthält nur anstatt der Outputs aus $X$ interne Aktionen.\\
Zwischen den Traces von $P$ und den Traces des Systems $P/X$ mit verborgenen
Outputs gibt es einen allgemeinen Zusammenhang.

\begin{Lem}[Traces unter Internalisierung]
  \label{TraceHidingLem}
  Für ein \MEIO{} $P$ und alle zulässigen Aktionsmengen $X$ gilt: $p
  \weakmay[w]_{P/X} p' \Leftrightarrow \exists w'\in \Sigma ^*:
  w'|_{\Sigma\backslash X} = w \land p \weakmay[w']_P p'$.
\end{Lem}
\begin{proof}
  \glqq $\Rightarrow$\grqq{}:\\
  Zwischen $p$ und $p'$ gibt es in $P/X$ einen Trace $w$. Somit folgt die
  Existenz eines $v=\alpha _1\alpha_2\dots\alpha _n\in\Sigma _{\tau}$ mit der
  Eigenschaft $w = \hat{v}$ und einen Ablauf der Form $p \may[\alpha _1]_{P/X}
  p_1 \may[\alpha _2]_{P/X} \dots p_{n-1} \may[\alpha _n]_{P/X} p'$ in $P/X$.
  Dieser Ablauf ist durch das Internalisieren von Outputs aus einem Ablauf aus
  $P$ entstanden, da nach Definition~\ref{HidingDef} die Transitionen aus $P/X$
  sich auf Transitionen aus $P$ zurückführen lassen müssen. In dem man also
  einige der Aktionen $\alpha_j$, die interne Aktionen sind, durch Outputs aus
  $X$ ersetzt erhält man ein $w'$, für das $w'|_{\Sigma\backslash X}=w$ gilt
  und das die sichtbare Transitionsbeschriftung des Ablaufes zwischen $p$ und
  $p'$ in $P$ ist.

  \glqq $\Leftarrow$\grqq{}:\\
  Analog zur anderen Richtung gibt es auch hier einen entsprechenden Ablauf in
  $P$ von $p$ nach $p'$, dessen sichtbare Beschriftung in diesem Fall $w'$ ist.
  Es gibt also ein $v=\alpha _1\alpha_2\dots\alpha _n\in\Sigma _{\tau}$ mit $w'
  = \hat{v}$ und einen Ablauf der Form $p \may[\alpha _1]_P p_1 \may[\alpha
  _2]_P \dots p_{n-1} \may[\alpha _n]_P p'$ in $P$. Durch die Anwendung des
  Hiding"=Operators werden die Aktionen $\alpha _j$, die in $X$ enthalten sind,
  durch $\tau$s ersetzt. Die sichtbare Beschriftung des analogen Ablaufes in
  $P/X$ lautet somit nicht mehr $w'$ sondern nur noch $w =
  w'|_{\Sigma\backslash X}$.
\end{proof}

Dieses Lemma kann nun dazu verwendet werden, zu zeigen, dass die Relation
\ERel{} eine Präkongruenz bezüglich des Hiding-Operators ist.

\begin{Satz}[Fehler-Präkongruenz bzgl.\ Internalisierung]
  \label{FehlerHidingSatz}
  Seien $P_1$ und $P_2$ zwei \MEIO{}s für die $P_1\ERel P_2$ gilt, somit gilt
  auch $P_1/X\ERel P_2/X$ für alle zulässigen Aktionsmengen $X$. Daraus folge
  insbesondere, dass \ERel{} eine Präkongruenz bezüglich $\cdot /\cdot$ ist. Es
  gilt für die Sprachen und Traces:
  \begin{enumerate}[(i)]
    \item $L(P/X) = \left\{w\in (\Sigma\backslash X)^*\mid \exists w'\in L(P):
      w'|_{\Sigma\backslash X} = w\right\}$,
    \item $\ET (P/X) = \left\{w\in (\Sigma\backslash X)^*\mid \exists w'\in
      \ET(P): w'|_{\Sigma\backslash X} = w\right\}$,
    \item $\EL (P/X) = \left\{w\in (\Sigma\backslash X)^*\mid \exists w'\in
      \EL(P): w'|_{\Sigma\backslash X} = w\right\}$.
  \end{enumerate}
\end{Satz}
\begin{proof}
  Die Hauptaussage des Satzes folgt aus den Aussagen (i) bis (iii), somit
  werden zunächst diese Punkte nachgewiesen.
  \begin{enumerate}[(i)]
    \item Für ein Wort $w'$ aus der Sprache $L_P$ gilt nach
      Definition~\ref{LDef} $p_0 \weakmay[w']_P p$ für einen Zustand $p$ aus
      $P$. Es kann also Lemma~\ref{TraceHidingLem} angewendet werden. Daraus
      folgt dann, der Trace $p_0 \weakmay[w]_{P/X} p$ in $P/X$ für
      $w=w'|_{\Sigma\backslash X}$. Das Wort, das in $P/X$ ausgeführt wird,
      lautet also $w=w'|_{\Sigma\backslash X}\in L_{P/X}$.\\
      Für ein Wort $w$ aus der Sprache des Transitionssystems $P/X$ folgt
      ebenfalls mit Lemma~\ref{TraceHidingLem}, dass es ein $w'$ mit
      $w'|_{\Sigma\backslash X} = w$ gibt, das ein Wort aus $L_P$ ist.
    \item Die \ET{}-Mengen sind unter \cont{} abgeschlossen. Es genügt also ein
      präfix-minimales $w'$ aus $\ET _{P}$ zu betrachten. Das präfix-minimale
      $w'$ kann ein Element aus $\PrET _P$ oder $\MIT _P$ sein.
      \begin{itemize}
        \item Fall 1 ($w'\in\PrET _P$): Da $w'$ in $P$ ausführbar ist, gibt es
          einen Trace für $w'$, der durch einen Trace $v \in O_P^*$ ergänzt
          werden kann, so dass durch $w'v$ ein Zustand aus der Menge $E_P$
          erreicht wird. Der Trace hat also die Form $p_0 \weakmay[w'v]_P p\in
          E_P$. Auf diesen Trace kann Lemma~\ref{TraceHidingLem} angewendet
          werden. Es ergibt sich dann der Trace $p_0 \lweakmay[(w'v)|_{\Sigma
          \backslash X}]_{P/X} p$, der in $P/X$ ausführbar ist. Nach
          Definition~\ref{HidingDef} werden die Fehler"=Zustände aus $P$ in
          $P/X$ übernommen. Es gilt also auch $p\in E_{P/X}$. Das Wort
          $(w'v)|_{\Sigma\backslash X}$ ist somit in $\StET (P/X) \subseteq\ET
          (P/X)$ enthalten. Da $v$ nur aus Outputs besteht, gilt auch nach der
          Ersetzung der Transitionsbeschriftung durch das Hiding
          $v|_{\Sigma\backslash X}\in (O_P\backslash X)^*$. Daraus folgt $w =
          w'|_{\Sigma\backslash X} = \prune ((w'v)|_{\Sigma\backslash X})$. Das
          Wort $w$ ist also in der Menge $\ET _{P/X}$ enthalten.
        \item Fall 2 ($w'\in\MIT _P$): In diesem Fall wird durch das Präfix
          $v'$ von $w'$ ein Zustand erreicht, in dem der Input $a$ nicht
          sichergestellt ist, wobei $w'=v'a$ gilt. Es gibt also einen Trace wie
          in Lemma~\ref{TraceHidingLem} für das Präfix $v'$, wobei $p$ $p_0$
          entspricht und $p'\nmust[a]_P$ gilt. In $P/X$ gibt es einen mit
          $v'|_{\Sigma\backslash X}$ beschrifteten Trace von $p = p_0$ nach
          $p'$, wegen Lemma~\ref{TraceHidingLem}. Es gilt mit
          Definition~\ref{HidingDef} auch in $P/X$ $p'\nmust[a]_{P/X}$ mit
          $a\in I_{P/X}$. $w = w'|_{\Sigma\backslash X}$ ist also in der Menge
          $\MIT _{P/X} \subseteq\ET _{P/X}$ enthalten.
      \end{itemize}

      Für ein präfix-minimales $w$ aus $\ET _{P/X}$ kann unterschieden werden,
      ob $w\in\PrET _{P/X}$ oder $w\in\MIT _{P/X}$ gilt.
      \begin{itemize}
        \item Fall I \big($w\in\PrET _{P/X}$\big): Analog zu Fall 1 gibt es
          eine Verlängerung $v\in O_{P/X}^*$, so dass $wv$ in $P/X$ zu einem
          Fehler-Zustand führt. Es gibt also ein $p$, so dass $p_0 \weakmay[wv]
          p\in E_{P/X}$ gilt. Mit Lemma~\ref{TraceHidingLem} kann begründet
          werden, dass es $w'$ und $v'$ gibt mit $w'|_{\Sigma\backslash X} = w$
          und $v'|_{\Sigma\backslash X} = v$, so dass $w'v'$ in $P$ von $p_0$
          nach $p$ führt. Es muss $p\in E_P$ gelten, wegen
          Definition~\ref{HidingDef}. Die Verlängerung $v'$ kann nur aus
          Outputs bestehen, somit gilt wegen der \prune{}-Funktion $w' \in
          \PrET _P\subseteq\ET _P$.
        \item Fall II \big($w\in\MIT _{P/X}$\big): Das $w$ kann in $P/X$ ohne
          den letzten Input $a$ zu einem Zustand ausgeführt werden, in dem der
          Input $a$ nicht sichergestellt wird. Analog zu Fall 2 muss es wegen
          Lemma~\ref{TraceHidingLem} eine Erweiterung $w'$ mit
          $w'|_{\Sigma\backslash X} = w$ des Wortes $w$ auf die Menge $\Sigma$
          geben, für dessen Präfix ohne den letzten Input $a$ in $P$ ebenfalls
          der Zustand erreicht wird, in dem das $a$ nicht sichergestellt ist.
          Somit ist $w'$ in $P$ ebenfalls ein Input-kritischer Trace und es
          gilt $w'\in\ET _P$.
      \end{itemize}
    \item Die Menge \EL{} ist die Vereinigung der Sprache $L$ mit der
      Trace"=Menge \ET{}. Die Aussage dieses Punktes folgt also direkt aus den
      bereits nachgewiesenen Punkten (i) und (ii).
  \end{enumerate}
  $P_1\ERel P_2$ setzt die Inklusionen $\ET _1\subseteq\ET _2$ und $\EL
  _1\subseteq\EL _2$ voraus. Mit (ii) und (iii) folgt draus $\ET _{P_1/X}
  \subseteq\ET _{P_2/X}$ und $\EL _{P_1/X} \subseteq\EL _{P_2/X}$. Die Relation
  \ERel{} bleibt also trotz Hiding erhalten. Somit ist \ERel{} eine
  Präkongruenz bezüglich des Hiding-Operators $\cdot /\cdot$.
\end{proof}

Aus Korollar~\ref{KommPraekonKor} ist bekannt, dass \ERel{} ein Präkongruenz
bezüglich $\cdot \|\cdot$ ist, und aus Satz~\ref{FehlerHidingSatz}, dass
\ERel{} auch eine Präkongruenz bezüglich $\cdot /\cdot$ ist. Die
Parallelkomposition mit Internalisierung wird nach
Definition~\ref{DefParallelkompInternal} aus diesen beiden Operatoren
zusammengesetzt. Somit erhält man das folgende Korollar.

\begin{Kor}[Fehler-Präkongruenz mit Internalisierung]
  Die Relation \ERel{} ist eine Präkongruenz bezüglich $\cdot |\cdot$.
\end{Kor}
