\chapter{Definitionen}
{\large\textbf{Stand: \today{}}}\\

Die folgenden vier Definitionen sind analog aus\cite{Vogler2016MIA3} übernommen
worden.

\begin{Def}[Modal Interface Automat]
  Ein \emph{Modal Interface Automat (\MIA{})} ist ein Tupel
  $(P,I,O,\must,\may,p_0,e)$ mit:
  \begin{itemize}
    \item $P$: Menge der Zustände
    \item $p_0\in P$: Startzustand
    \item $e\in P$: universeller Zustand
    \item $I,O$: disjunkte Input- und Outputaktionen
    \item $A = I\cup O$: Alphabet
    \item $\tau\notin A$: interne Aktion
    \item $\must{} \subseteq P\times (A\cup\{\tau\})\times
      (\mathcal{P}(P)\backslash\emptyset )$\footnote{$\mathcal{P}(P)$
      bezeichnet die Potenzmenge von $P$}: disjunktive
      must-Transitions-Relation
    \item $\may{} \subseteq P\times (A\cup\{\tau\})\times
      P$: may-Transitions-Relation
  \end{itemize}
  Es werden die folgenden Eigenschaften vorausgesetzt:
  \begin{enumerate}
    \item $\forall\alpha\in A\cup\{\tau\}:p\must[\alpha]P
      \Rightarrow\forall p'\in P:p\may[\alpha]p'$
      (syntaktische Konsistenz)
    \item $e$ tritt nur als Zielzustand von Input may-Transitionen auf
      (Senken-Voraussetzung) \TODO{Übersetzung überdenken}
  \end{enumerate}
\end{Def}

Must-Transitionen sind Transitionen, die von einer Verfeinerung implementiert
werden müssen. Die may-Transitionen sind hingegen die zulässigen Transitionen
für eine Verfeinerung.\\
Für beliebige Alphabete $I,O$ ist dann $P=(\{e\},I,O,\emptyset ,\emptyset,
e,e)$ der universelle MIA, da in $e$ als universellen Zustand beliebiges
Verhalten zulässig ist.\\
\MIA{}s  werden in dieser Arbeit durch ihre Zustandsmenge (z.B.\ $P$)
identifiziert und falls notwendig werden damit auch die Komponenten indiziert
(z.B.\ $I_P$ anstatt $I$). Falls der \MIA{} selbst bereits einen Index hat
(z.B.\ $P_1$) kann an der Komponente die Zustandsmenge als Index wegfallen und
nur noch der Index des gesamten Automaten verwendet werden (z.B.\ $I_1$ anstatt
$I_{P_1}$). Zusätzlich stehen $i,o,a,\omega$ und $\alpha$ für Buchstaben aus
den Alphabeten $I,O,A,O\cup\{\tau\}$ und $A\cup\{\tau\}$. Es kann $A=I/O$
geschrieben werden um die Inputs und Outputs eines Alphabets hervorzuheben. Im
Zusammenhang mit schwachen Transitionen wird die Notation $\hat{\alpha}$
verwendet, wobei gilt $\hat{\alpha}=_{\text{df}}a$, falls $\alpha =a\neq\tau$
und $\hat{\alpha}=_{\text{df}}\varepsilon$, falls $\alpha =\tau$. Desweiteren
werden Outputs und die interne Aktion \emph{lokale Aktionen} genannt, da sie
lokal vom ausführenden \MIA{} kontrolliert sind. Um eine Erleichterung der
Notation zu erhalten, soll gelten, dass $p\must[a]p', p\nmust[a]$ und
$p\nmay[a]$ für $p\may[a]\{p'\}, \nexists P': p\may[a]P'$ und $\nexists p':
p\may[a]p'$ stehen soll. In Graphiken wird eine Aktion $a$ als $a?$ notiert,
falls $a\in I$ und $a!$, falls $a\in O$. Must-Transitionen (may-Transitionen)
werden als durchgezogener, möglicherweise aufspaltender, Pfeil gezeichnet
(gestrichelter Pfeil). Entsprechend der syntaktischen Konsistenz repräsentiert
jede gezeichnete must-Transition auch gleichzeitig die zugrundeliegende
may-Transitionen.

\begin{Def}[Parallelprodukt]
  Zwei \MIA{}s $P_1,P_2$ sind \emph{komponierbar}, falls $O_1\cap
  O_2=\emptyset$. Für solche \MIA{}s ist das Produkt $P_1\otimes
  P_2=((P_1\times P_2)\cup \{e_{12}\}, I, O, {\must,} {\may,}$ $(p_{01}, p_{02}),
  e_{12})$ definiert mit: \TODO{erzwungenen Zeilenumbruch kontrollieren}
  \begin{itemize}
    \item $e_{12}$: frischer universeller Zustand
    \item $I=(I_1\cup I_2)\backslash (O_1\cup O_2)$
    \item $O=(O_1\cup O_2)$
    \item $\must, \may$: kleinste Relationen, die die
      folgenden Regeln erfüllen:
    \begin{itemize}
      \item[(PMust1)] $(p_1,p_2)\must[\alpha] P_1'\times \{p_2\}$, falls
        $p_1\must[\alpha] P_1'$ und $\alpha\notin A_2$
      \item[(PMust2)] $(p_1,p_2)\must[\alpha] \{p_1\}\times P_2'$, falls
        $p_2\must[\alpha] P_2'$ und $\alpha\notin A_1$
      \item[(PMust3)] $(p_1,p_2)\must[a] P_1'\times P_2'$, falls
        $p_1\must[a] P_1'$ und $p_2\must[a] P_2'$
      \item[(PMay1)] $(p_1,p_2)\may[\alpha] P_1'\times \{p_2\}$, falls
        $p_1\may[\alpha] P_1'$ und $\alpha\notin A_2$
      \item[(PMay2)] $(p_1,p_2)\may[\alpha] \{p_1\}\times P_2'$, falls
        $p_2\may[\alpha] P_2'$ und $\alpha\notin A_1$
      \item[(PMay3)] $(p_1,p_2)\may[a] P_1'\times P_2'$, falls
        $p_1\may[a] P_1'$ und $p_2\may[a] P_2'$
    \end{itemize}
  \end{itemize}
\end{Def}

\begin{Def}[Parallelkomposition]
  Gegeben ein Parallelprodukt $P_1\otimes P_2$, ein Zustand $(p_1,p_2)$ ist ein
  neuer Kommunikationsfehler, falls es ein $a\in A_1\cap A_2$ gibt, sodass:
  \begin{enumerate}[(a)]
    \item $a\in O_1, p_1\may[a]$ und $p_2\nmust[a]$ oder
    \item $a\in O_2, p_2\may[a]$ und $p_1\nmust[a]$.
  \end{enumerate}
  $(p_1,p_2)$ ist ein geerbter Kommunikationsfehler, falls eine der Komponenten
  ein universeller Zustand ist, d.h. $p_1=e_1\lor p_2=e_2$.\\
  $E\subseteq P_1\times P_2$ ist die Menge der unzulässigen Zustände. Es gilt
  $(p_1,p_2)\in E$, falls:
  \begin{enumerate}[(i)]
    \item $(p_1,p_2)$ ist ein neuer oder geerbter Kommunikationsfehler,
    \item $(p_1,p_2)\may[w] (p_1',p_2')$ und $(p_1',p_2')\in E$.
      \TODO{$w\in O\cup \{\tau\}$ in Notation aufnehmen}
  \end{enumerate}
  Falls der Startzustand ein unzulässiger Zustand ist, dann wird $e_{12}$
  initial und somit der einzig erreichbare Zustand von $P_1||P_2$ ($P_1$ und
  $P_2$ werden dann inkompatibel genannt).\\
  Sonst erhält man $P_1||P_2$ durch das entfernen unzulässiger Zustände aus
  $P_1\otimes P_2$. Falls es einen Zustand $(p_1,p_2)\notin E$ mit
  $(p_1,p_2)\may[i](p_1',p_2')\in E$ für ein $i\in I$ gibt, dann werden alle
  must- und may-Transitionen mit $i$ startend bei $(p_1,p_2)$ entfernt und eine
  einzige Transition $(p_1,p_2)\may[i]e_{12}$ hinzugefügt. Zusätzlich werden
  alle Zustände aus $E$ und alle unerreichbaren Zustände (außer $e_{12}$) und
  alle ihre eingehenden und ausgehenden Transitionen gelöscht. Falls
  $(p_1,p_2)\in P_1||P_2$, schreiben wir $p_1||p_2$ und nennen $p_1$ und $p_2$
  kompatibel.
\end{Def}

\TODO{weak Transitionen definieren}

\begin{Def}[MIA Verfeinerunge]
  Seien $P$ und $Q$ \MIA{}s mit gemeinsamen Input- und Output-Alphabeten. Dann
  ist $\mathcal{R}\subseteq P\times Q$ eine \MIA{}-Verfeinerungs-Relation,
  falls für alle $(p,q)\in\mathcal{R}$ mit $q\neq e_Q$ die folgenden
  Eigenschaften erfüllt sind:
  \begin{enumerate}[(i)]
    \item $p\neq e_P$
    \item $q\must[i]Q'\Rightarrow\exists P': p\must[i]\weakmust[\varepsilon]
      P'$ und $\forall p'\in P'\exists q'\in Q':(p',q')\in\mathcal{R}$
    \item $q\must[\omega]Q'\Rightarrow\exists P': p\weakmust[\hat{\omega}]
      P'$ und $\forall p'\in P'\exists q'\in Q':(p',q')\in\mathcal{R}$
    \item $p\may[i]p'\Rightarrow\exists q': a\may[i]\weakmay[\varepsilon]
      q'$ und $(p',q')\in\mathcal{R}$
    \item $p\may[\omega]p'\Rightarrow\exists q': q\weakmay[\hat{\omega}]
      q'$ und $(p',q')\in\mathcal{R}$
  \end{enumerate}
  $p$ \MIA{}-verfeinert $q$ ($p\sqsubseteq q$), falls eine
  \MIA{}-Verfeinerungs-Relation $\mathcal{R}$ existiert mit
  $(p,q)\in\mathcal{R}$. Falls es auch in die umgekehrte Richtung eine
  Verfeinerungsrelation gibt, sind die beiden Zustände äquivalent, was durch
  $\sqsupseteq\sqsubseteq$ ausgedrückt wird. Für zwei \MIA{}s gilt $P\sqsubseteq Q$, falls
  $p_0\sqsubseteq q_0$.
\end{Def}
