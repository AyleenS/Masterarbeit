\chapter{Definitionen}
{\large\textbf{Stand: \today{}}}\\

Kombination aus~\cite{Vogler2015FailSem} und~\cite{Schinko2016BA} mit
Einflüssen von~\cite{Vogler2016MIA3}:

\begin{Def}[Modal Error-I/O-Transitionssystem]
  Ein \emph{Modal Error-I/O-Transitionssystem (\MEIO{})} ist ein Tupel
  $(P,I,O,\must,\may,p_0,E)$ mit:
  \begin{itemize}
    \item $P$: Menge der \emph{Zustände},
    \item $p_0\in P$: \emph{Startzustand},
    \item $I,O$: disjunkte Mengen der (sichtbaren) \emph{Input-} und
      \emph{Output-Aktionen},
    \item $\must{} \subseteq P\times \Sigma_{\tau}\times P$:
      \emph{must-Transitions-Relation},
    \item $\may{} \subseteq P\times \Sigma_{\tau}\times P$:
      \emph{may-Transitions-Relation},
    \item $E\subseteq P$: Menge der \emph{Fehler-Zustände}.
  \end{itemize}
  Es wird vorausgesetzt, dass $\must\subseteq\may$ (\emph{syntaktische
  Konsistenz}) gilt.\\
  Das \emph{Alphabet} bzw.\ die Aktionsmenge eines \MEIO{} ist $\Sigma = I\cup
  O$. Die \emph{interne Aktion} $\tau$ ist nicht in $\Sigma$ enthalten. Jedoch
  gilt $\Sigma_{\tau} := \Sigma \cup \{\tau\}$. Die \emph{Signatur} eines
  \MEIO{}s entspricht $\Sig (P)=(I,O)$.\\
  Falls $\must =\may$ gilt, wird $P$ auch \emph{Implementierung} genannt.
\end{Def}

Implementierungen entsprechen den in~\cite{Schinko2016BA} behandelten
\EIO{}s.\\
Must-Transitionen sind Transitionen, die von einer Verfeinerung implementiert
werden müssen. Die may-Transitionen sind hingegen die zulässigen Transitionen
für eine Verfeinerung.\\
\MEIO{}s  werden in dieser Arbeit durch ihre Zustandsmenge (z.B.\ $P$)
identifiziert und falls notwendig werden damit auch die Komponenten indiziert
(z.B.\ $I_P$ anstatt $I$). Falls der \MEIO{} selbst bereits einen Index hat
(z.B.\ $P_1$) kann an der Komponente die Zustandsmenge als Index wegfallen und
nur noch der Index des gesamten Automaten verwendet werden (z.B.\ $I_1$ anstatt
$I_{P_1}$). Zusätzlich stehen $i,o,a,\omega$ und $\alpha$ für Buchstaben aus
den Alphabeten $I,O,\Sigma ,O\cup\{\tau\}$ und $\Sigma_\tau$.\\
Es wir die Notation $p\may[\alpha] p'$ für $(p,\alpha,p')\in\may$ und
$p\may[\alpha]$ für $\exists p':(p,\alpha,p')\in\may$ verwendet. Dies kann
entsprechend auf Buchstaben-Sequenzen $w\in\Sigma_{\tau}^*$ erweitert werden zu
$p\may[w]p'$ ($p\may[w]$) steht für die Existenz eines Laufes $p\may[\alpha _1]
p_1\may[\alpha _2]\dots p_{n-1}\may[\alpha _n] p'$ ($p\may[\alpha _1]
p_1\may[\alpha _2]\dots p_{n-1}\may[\alpha _n]$) mit $w=\alpha _1\dots \alpha
_n$.\\
Desweiteren soll $w|_B$ die Aktions-Sequenz bezeichnen, die man erhält, wenn
man aus $w$ alle Aktionen löscht, die nicht in $B\subseteq\Sigma$ enthalten
sind. $\widehat{w}$ steht für $w|_{\Sigma}$. Es wir $p\weakmay[w] p'$
für ein $w\in\Sigma ^*$ geschrieben, falls $\exists
w'\in\Sigma_{\tau}^*:\widehat{w'}=w\land p\may[w']p'$, und $p\weakmay[w]$,
falls $p\weakmay[w] p'$ für ein beliebiges $p'$ gilt. Falls $p_0\weakmay[w] p$
gilt, dann wird $w$ \emph{Trace} genannt und $p$ ist ein \emph{erreichbarer
Zustand}.\\
Analog zu $\may$ und $\weakmay$ werden $\must$ und $\weakmust$ für die
entsprechenden Relationen der must-Transition verwendet.\\
Outputs und die interne Aktion werden \emph{lokale Aktionen} genannt, da sie
lokal vom ausführenden \MEIO{} kontrolliert sind. Um eine Erleichterung der
Notation zu erhalten, soll gelten, dass $p\nmust[a]$ und $p\nmay[a]$ für
$\nexists p': p\must[a]p'$ und $\nexists p': p\may[a]p'$ stehen soll. In
Graphiken wird eine Aktion $a$ als $a?$ notiert, falls $a\in I$ und $a!$, falls
$a\in O$. Must-Transitionen (may-Transitionen) werden als durchgezogener Pfeil
gezeichnet (gestrichelter Pfeil). Entsprechend der syntaktischen Konsistenz
repräsentiert jede gezeichnete must-Transition auch gleichzeitig die
zugrundeliegende may-Transitionen.

\begin{Def}[Parallelkomposition]
  \label{ParallelDef}
  Zwei \MEIO{}s $P_1 = (P_1,I_1,O_1,{\must_1,}{\may_1,}$ $p_{01},E_1)$ und $P_2 =
  (P_2,I_2,O_2,\must_2,\may_2,p_{02},E_2)$ sind \emph{komponierbar}, falls
  $O_1\cap O_2=\emptyset$. Für solche \MEIO{}s ist die
  \emph{Parallelkomposition} $P_{12} := P_1\|P_2=((P_1\times P_2), I, O,
  {\must_{12},}$ ${\may_{12},}$ $(p_{01}, p_{02}), E)$ definiert mit:
  \TODO{erzwungenen Zeilenumbrüche kontrollieren}
  \begin{itemize}
    \item $I=(I_1\cup I_2)\backslash (O_1\cup O_2)$,
    \item $O=(O_1\cup O_2)$,
  \item $\begin{aligned}[t]
      \must_{12} & = \left\{\left((p_1,p_2),\alpha,(p_1',p_2)\right) \mid p_1
      \must[\alpha]_1 p_1', \alpha\in\Sigma_{\tau}\backslash
      \Synch(P_1,P_2)\right\}\\
        &\cup \left\{\left((p_1,p_2),\alpha,(p_1,p_2')\right) \mid p_2
      \must[\alpha]_2 p_2', \alpha\in\Sigma_{\tau}\backslash
      \Synch(P_1,P_2)\right\}\\
        &\cup \left\{\left((p_1,p_2),\alpha,(p_1',p_2')\right) \mid p_1
      \must[\alpha]_1 p_1', p_2 \must[\alpha]_2 p_2', \alpha\in
      \Synch(P_1,P_2)\right\}
    \end{aligned}$
  \item $\begin{aligned}[t]
      \may_{12} & = \left\{\left((p_1,p_2),\alpha,(p_1',p_2)\right) \mid p_1
      \may[\alpha]_1 p_1', \alpha\in\Sigma_{\tau}\backslash
      \Synch(P_1,P_2)\right\}\\
        &\cup \left\{\left((p_1,p_2),\alpha,(p_1,p_2')\right) \mid p_2
      \may[\alpha]_2 p_2', \alpha\in\Sigma_{\tau}\backslash
      \Synch(P_1,P_2)\right\}\\
        &\cup \left\{\left((p_1,p_2),\alpha,(p_1',p_2')\right) \mid p_1
      \may[\alpha]_1 p_1', p_2 \may[\alpha]_2 p_2', \alpha\in
      \Synch(P_1,P_2)\right\}
    \end{aligned}$
  \item $\begin{aligned}[t]
      E & = (P_1\times E_2) \cup (E_1\times P_2) &&\text{geerbte
      Kommunikationsfehler}\\
        &\quad\left. \begin{aligned}
        &\cup \left\{(p_1,p_2) \mid \exists a \in O_1\cap I_2 : p_1
        \may[a]\land p_2\nmust[a]\right\}\\
        &\cup \left\{(p_1,p_2) \mid \exists a\in I_1\cap O_2 :
        p_1\nmust[a]\land p_2\may[a]\right\}
        \end{aligned}\hspace{0.3cm}\right\} &&\text{neue Kommunikationsfehler}
    \end{aligned}$
  \end{itemize}
  Dabei bezeichnet $\Synch(P_1,P_2)=(I_1\cap O_2)\cup (O_1\cap I_2)\cup
  (I_1\cap I_2)$ die Menge der zu \emph{synchronisierenden Aktionen}. Die
  synchronisierten Aktionen werden als Output bzw. Input der Komposition
  beibehalten.
\end{Def}

Ein neuer Kommunikationsfehler entsteht, wenn einer der \MEIO{}s die
Möglichkeit für einen Output hat (may-Transition) und der andere \MEIO{} der
passenden Input nicht erzwingt (keine must-Transition vorhanden). Es muss also
in möglichen Implementierungen nicht wirklich zu diesem Kommunikationsfehler
kommen, da die Output-Transition nicht zwingender maßen implementiert werden
muss.\\
Wie bereits in~\cite{Schinko2016BA} kann es durch die Synchronisation von
Inputs zu keinen neuen Kommunikationsfehler kommen, da dies in beiden Automaten
keine lokal kontrollierte Aktion ist. Falls jedoch nur einer der Automaten die
Möglichkeit für einen Input hat, der synchronisiert wird, besteht diese
Möglichkeit in der Parallelkomposition nicht mehr. Es kann also in der
Kommunikation mit einem weiteren \MEIO{} dort zu einen neuen
Kommunikationsfehler kommen.

\begin{Def}[Simulation]
  \label{SimDef}
  Eine \emph{(starke) alternierende Simulation} ist eine Relation $R\subseteq P
  \times Q$ auf zwei \MEIO{}s $P$ und $Q$, falls für alle $(p,q)\in R$ gilt:
  \begin{enumerate}
    \item $q\must[\alpha] q'$ impliziert $p\must[\alpha] p'$ für ein $q'$ mit
      $pRq'$,
    \item $p\may[\alpha] p'$ impliziert $q\may[\alpha] q'$ für ein $p'$ mit
      $pRq'$,
  \end{enumerate}
  Die Vereinigung \asRel{} aller dieser Relationen wird als \emph{(starke)
  as-Verfeinerung(-s Relation)} (auch modal Verfeinerung) bezeichnet. Es wird
  $P\asRel Q$ geschrieben, falls $p_0\asRel q_0$ gilt, und $P$
  \emph{as-verfeinert} $Q$ \emph{(stark)} oder $P$ ist eine \emph{(starke)
  as-Verfeinerung} von $Q$.\\
  Für einen \MEIO{} $Q$ und eine Implementierung mit $P$ mit $P\asRel{}Q$, ist
  $P$ eine \emph{as"=Implementierung} von $Q$ und es wird $\asimp (Q)$ für die
  Menge aller as"=Implementierungen von $Q$ verwendet.
\end{Def}

Die Funktionen \prune{} und \cont{} zum Abschneiden und Verlängern von Traces
werden hier genauso verwendet, wie sie in~\cite{Schinko2016BA} definiert
wurden. Man muss dazu nur \EIO{} in der Definition durch \MEIO{} ersetzten.
Ebenso wird die Parallelkomposition von Wörtern und Mengen von Wörtern bzw.\
Sprachen analog übernommen.

\begin{Def}[Sprache]
  Die \emph{(maximale) Sprache} eines \MEIO{}s $P$ ist $L(P) := \left\{w\in
  \Sigma ^* \mid \exists P'\in\asimp (P) : p'_0\weakmust[w]
  \right\}$.\\
  Für zwei komponierbare \MEIO{}s $P_1$ und $P_2$ gilt: $L_{12} := L(P_{12}) =
  L_1\|L_2$.
\end{Def}
