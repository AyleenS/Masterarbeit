\chapter{Definitionen}
\begin{Def}[Modal Interface Automat]
  Ein \emph{Modal Interface Automat (MIA)} ist ein Tupel
  $(P,I,O,\longrightarrow,\dashrightarrow,p_0,e)$ mit:
  \begin{itemize}
    \item $P$: Menge der Zustände
    \item $p_0\in P$: Initialzustand/Startzustand \TODO{Bezeichnung überlegen}
    \item $e\in P$: Universal-Zustand(Fehler-Zustand) \TODO{Bezeichnung
      überlegen}
    \item $I,O$: disjunkte Input- und Outputaktionen (-handlungen)
      \TODO{Bezeichnung in BA nachschauen}
    \item $A = I\cup O$: Alphabet
    \item $\tau\notin A$: interne Aktion
    \item $\longrightarrow{} \subseteq P\times (A\cup\{\tau\})\times
      (\mathcal{P}(P)\backslash\emptyset )$\footnote{$\mathcal{P}(P)$
      bezeichnet die Potenzmenge von $P$}: disjunktive
      must-Transitions-Relation
    \item $\dashrightarrow{} \subseteq P\times (A\cup\{\tau\})\times
      P$: may-Transitions-Relation
  \end{itemize}
  Es werden die folgenden Eigenschaften vorausgesetzt:
  \begin{enumerate}
    \item $\forall\alpha\in A\cup\{\tau\}:p\overset{\alpha}{\longrightarrow}P
      \Rightarrow\forall p'\in P:p\overset{\alpha}{\dashrightarrow}p'$
      (syntaktische Konsistenz)
    \item $e$ tritt nur als Zielzustand von Input may-Transitionen auf
      (Senken-Voraussetzung) \TODO{Übersetzung überdenken}
  \end{enumerate}
\end{Def}
