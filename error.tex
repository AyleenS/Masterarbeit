\chapter{Verfeinerungen für Kommunikationsfehler-Freiheit}

\section{Erweiterungs-Ansatz}

Dieses Kapitel versucht die Präkongruenz für Error bei \EIO{}s aus
z.B.~\cite{Schinko2016BA} auf die hier betrachten \MEIO{}s zu erweitern.

\begin{Def}[fehler-freie Kommunikation]
  Ein Fehler"=Zustand ist \emph{lokal erreichbar} in einem \MEIO{} $P$, wenn
  ein $w\in O^*$ existiert mit $p_0 \weakmay[w]_{P} p\in E$.\\
  Zwei \MEIO{}s $P_1$ und $P_2$ \emph{kommunizieren fehler-frei}, wenn keine
  as"=Implementierungen ihrer Parallelkomposition $P_{12}$ einen
  Fehler"=Zustände lokal erreichen kann.
\end{Def}

\vspace{0.2cm}

\begin{Def}[Kommunikationsfehler-Verfeinerungs-Basirelation]
  \label{EBRelDef}
  Für zwei \MEIO{}s $P_1$ und $P_2$ mit der gleichen Signatur wird $P_1\EBRel{}
  P_2$ geschrieben, wenn nur dann ein Fehler"=Zustand in einer
  as"=Implementierung von $P_1$ lokal erreichbar ist, wenn es auch eine
  as"=Implementierung von $P_2$ gibt, in der ein Fehler"=Zustand auch lokal
  erreichbar ist. Die Basisrelation stellt eine \emph{Verfeinerung-Relation}
  bezüglich \emph{Kommunikationsfehler"=Freiheit} dar.\\
  \ECRel{} bezeichnet die \emph{vollständig abstrakte Präkongruenz} von
  \EBRel{} bezüglich $\cdot\|\cdot$, d.h.\ die gröbste Präkongruenz bezüglich
  $\cdot\|\cdot$, die in \EBRel{} enthalten ist.
\end{Def}

Für as"=Implementierungen $P_1$ und $P_2$ entspricht \EBRel{} der Relation
\EBbaRel{} aus~\cite{Schinko2016BA}.

Wie z.B.\ in~\cite{Schinko2016BA} werden die Fehler hier Trace-basiert
betrachtet.

\begin{Def}[Kommunikationsfehler-Traces]
  \label{KommTracesDef}
  Für ein \MEIO{} $P$ wird definiert:
  \begin{itemize}
    \item \emph{strikte Fehler-Traces}: $\StET (P) :=
      \left\{w\in\Sigma ^* \mid p_0 \weakmay[w]_P p \in E\right\}$,
    \item \emph{gekürzte Fehler-Traces}: $\PrET (P) :=
      \left\{\prune (w) \mid w\in\StET (P)\right\}$,
    \item \emph{Input-kritische-Traces}: $\MIT (P) := \left\{wa\in\Sigma ^*
      \mid p_0 \weakmay[w]_P p \land a\in I\land p\nmust[a]_P \right\}$.
  \end{itemize}
\end{Def}

Da die Basisrelation über as"=Implementierungen spricht, ist es wichtig bereits
in den Trace-Mengen eine Beziehung zwischen der allgemeinen Definition für
\MEIO{}s und deren as"=Implementierungen herzustellen. Die nächste Proposition
stellt eine Teilmengen"=Beziehung zwischen den Traces eines \MEIO{}s und den
Traces seiner as"=Implementierungen dar.

\begin{Prop}[Kommunikationsfehler-Traces und Implementierungen]
  \label{KommTracesProp}
  Sei $P$ ein \MEIO{}.
  \begin{enumerate}
    \item Für die strikten Fehler-Traces von $P$ gilt: $\StET (P) \subseteq
      \big\{w\in\Sigma ^* \mid \exists P' \in\asimp (P) :$ $p'_0
      \weakmust[w]_{P'} p' \in E\big\} = \underset{P' \in\asimp (P)}{\bigcup}
      \StET (P')$.
      \TODO{erzwungen Zeilenumbruch kontrollieren}
    \item Für die gekürzten Fehler-Traces von $P$ gilt: $\PrET (P) \subseteq
      \{\prune (w) \mid \exists P' \in$\linebreak $\asimp (P) : w\in\StET (P')
      \} = \underset{P' \in\asimp (P)}{\bigcup} \PrET (P')$,
      \TODO{erzwungen Zeilenumbruch kontrollieren}
    \item Für Input-kritischen-Traces von $P$ gilt: $\MIT (P) \subseteq
      \big\{wa\in\Sigma ^* \mid \exists P' \in\asimp (P) :$ $p'_0
      \weakmust[w]_{P'} p' \land a\in I\land p'\nmust[a]_{P'} \big\} =
      \underset{P' \in\asimp (P)}{\bigcup} \MIT (P')$.
      \TODO{erzwungen Zeilenumbruch kontrollieren}
  \end{enumerate}
\end{Prop}
\begin{proof}\mbox{}
  \begin{enumerate}
    \item Um die Inklusion zu zeigen wird eine Implementierung $P'$ angegeben,
      die die strickten Fehler-Traces von $P$ implementiert und zusätzlich auch
      noch eine passende as"=Verfeinerungs"=Relation $\mathcal{R}$ zwischen den
      beiden Transitionssystemen. $P'$ implementiert wie im Beweis zu
      Proposition~\ref{LImpProp} alle Transitionen von $P$. Das $P'$ wird hier
      jedoch im Gegensatz zu Beweis von~\ref{LImpProp} auch noch alle
      Fehler-Zustände aus $P$ implementieren. Die entsprechende
      as"=Verfeinerungs"=Relation $\mathcal{R}$ ist jedoch hier ebenfalls
      wieder die Identitäts-Relation zwischen den Zuständen der
      Transitionssysteme. Die Definition von $P'$ lautet also:
      \begin{itemize}
        \item $P'=P$,
        \item $p'_0=p_0$,
        \item $I_{P'}=I_P$ und $O_{P'}=O_P$,
        \item $\must _{P'} =\may _{P'} = \may _P$,
        \item $E_P'=E_P$.
      \end{itemize}
      Die Tupel, die von~\ref{SimDef}~2.\ und 3.\ als Elemente der
      as"=Verfeinerungs"=Relation $\mathcal{R}$ gefordert werden, sind bereits
      durch die Identitäts"=Relation garantiert. Wie im Beweis~\ref{LImpProp}.
      Für~\ref{SimDef}~1.\ muss für jedes in der as"=Verfeinerungs"=Relation
      $\mathcal{R}$ enthaltene Zustands-Paar gelten, wenn der Zustand aus $P$
      kein Fehler-Zustand ist, dann ist auch der Zustand aus $P'$ keiner, dies
      folgt aus der Gleichheit der Mengen $E_P$ und $E_P'$. Jeder Trace aus
      $\StET (P)$ ist via may"=Transitionen in $P$ ausführbar und führt dort zu
      einem Fehler-Zustand. Der analoge Trace ist auch in $P'$ möglich, da alle
      may"=Transitionen aus $P$ in $P'$ als must"=Transitionen implementiert
      wurden. Der dabei erreichte Zustand steht mit dem Fehler-Zustand in $P$
      in der Identitäts"=Relation $\mathcal{R}$, die Zustände entsprechen sich
      also. Es gilt mit $E_P'=E_P$, dass auch der in $P'$ erreichte Zustand ein
      Fehler-Zustand ist. Für die as"=Implementierung $P'$ von $P$ und der
      Identitäts"=Relation $\mathcal{R}$ als starke as"=Verfeinerungs"=Relation
      zwischen den Transitionssystemen gilt also $\StET (P) =\StET (P')$.
    \item Da der erste Punkt dieser Proposition bereits beweisen wurde, gilt
      bereits, dass alle strikten Fehler-Traces von $P$ in der Vereinigung
      aller strikten Fehler-Traces der as"=Implementierungen von $P$ enthalten
      sind. Wenn auf alle Wörter in beiden Mengen die \prune{}-Funktion
      angewendet wird, gilt die Inklusion der daraus entstanden Mengen
      weiterhin. Dies entspricht der Behauptung dies Punktes.
    \item Auch für diese Inklusion wird eine starke as"=Verfeinerungs"=Relation
      $\mathcal{R}$ und eine Implementierung $P'$ angegeben. Jedoch werden
      nicht wie bei 1.\ alle Transitionen von $P$ in $P'$ implementiert. Es
      wird auch für jedes $wa$ aus $\MIT (P)$ eine eigene Implementierung $P'$
      geben und nicht eine für alle. Es werden alle must"=Transitionen aus $P$
      in $P'$ implementiert und zusätzlich die may"=Transitionen, die zum
      Ausführen von $w$ benötigt werden, so dass das $a$ danach in $P$ nicht
      gefordert wird. Es kann aufgrund möglicher Schleifen in $P$ auch nicht
      mehr die Identitäts"=Relation als as"=Verfeinerungs"=Relation gewählt
      werden. Der Trace $w$ wird entsprechend seiner Länge abgewickelt, so dass
      sicher gestellt wird, dass der Zustand am Ende dieses Traces wirklich
      einen fehlenden Input aufweist. Für das Abwickeln werden die Zustände
      entsprechend ihrer Position im Ablauf, auf dem $w$ ausgeführt wird,
      durchnummeriert. Für ein $w$, für das $wa\in\MIT (P)$, gilt: $\exists
      w'\in\Sigma _{\tau} ^*, \exists \alpha _1,\alpha _2,\dots ,\alpha _n,
      \exists p_1,p_2,\dots ,p_n: \hat{w'} = w\land w'=\alpha _1\alpha _2\dots
      \alpha _n\land p_0 \may[\alpha _1]_P p_1 \may[\alpha _2]_P \dots
      p_{n-1}\may[\alpha _n]_P p_n \nmust[a]_P$.
      Die starke as"=Verfeinerungs"=Relation $\mathcal{R}$ enthält in diesem
      Fall Tupel $((p,j),p)$ für alle $0\leq j \leq n$. Die entsprechende
      Definition für das $P'$, das $wa$ als Input-kritischen-Trace enthalten
      soll lautet:
      \begin{itemize}
        \item $P'=P\times \{0,1,\dots n\}$,
        \item $p'_0=(p_0,0)$,
        \item $I_P'=I_P$ und $O_P'=O_P$,
        \item $\begin{aligned}[t]
            \must _P'=\may _P'&= \left\{((p,j),\alpha ,(p',j+1))\mid p
            \may[\alpha]_P p', 0\leq j < n\right\}\\
            &\cup\left\{((p,j),\alpha ,(p',j))\mid p\must[\alpha]_P p',
            0\leq j \leq n\right\},
        \end{aligned}$
        \item $E_P'=\emptyset$.
      \end{itemize}
      Der Ablauf von $w$ aus $P$ wird in $P'$ durch $(p_0,0) \must[\alpha
      _1]_{P'} (p_1,1) \must[\alpha _2]_{P'} \dots (p_{n-1},n-1) \must[\alpha
      _n]_{P'} (p_n,n)$ simuliert. $w$ ist also in $P'$ ausführbar. $a$ ist für
      $(p_n,n)$ nicht ausführbar, da in $P$ für $p_n$ $p_n\nmust[a]_P$ gilt und
      für die Zustände mit der Nummer $n$ in $P'$ nur die must"=Transitionen
      implementiert werden. Die Transitionen $p\must[\alpha]_P p'$ aus $P$
      werden in $P'$ durch die Transitionen $(p,j)\must[\alpha]_{P'} (p',j)$
      gematched. Für die may-Transitionen $(p,j)\may[\alpha]_{P'} (p',j+1)$
      (bzw. $(p',j)$) in $P'$ sind die entsprechenden matchenden Transitionen
      in $P$ die Transition $p \may[\alpha]_P p'$. Da die Menge $E_P'$ leer
      ist, gelten alle Bedingungen, damit $\mathcal{R}$ eine
      as"=Verfeinerungs"=Relation zwischen $P'$ und $P$ ist. Mit der Begründung
      von oben folgt auch $wa\in \MIT (P')$. Da für alle $wa$ aus $\MIT (P)$
      eine entsprechende Implementierung mit as"=Verfeinerungs"=Relation
      $\mathcal{R}$ angegeben, werden kann, gilt die Inklusion.
  \end{enumerate}
\end{proof}

\begin{Def}[Kommunikationsfehler-Semantik]
  \label{KommFehlerSemDef}
  Sei $P$ ein \MEIO{}.
  \begin{itemize}
    \item Die Menge der \emph{Fehler-Traces} von $P$ ist $\ET (P)
      := {\cont (\PrET (P)) \cup \cont (\MIT (P))}$.
    \item Die \emph{Fehler-geflutete Sprache} von $P$ ist $\EL
      (P) := L(P) \cup \ET (P)$.
  \end{itemize}
  Für zwei \MEIO{}s $P_1,P_2$ mit der gleichen Signatur wird $P_1\ERel P_2$
  geschrieben, wenn $\ET _1\subseteq \ET _2$ und $\EL _1 \subseteq \EL _2$
  gilt.
\end{Def}

Hierbei ist zu beachten, dass die Mengen \StET{}, \PrET{}, \MIT{}, \ET{} und
\EL{} nur denen aus~\cite{Schinko2016BA} entsprechen, wenn $P$ bereits eine
as"=Implementierung ist.\\
Aus den Propositionen für die Sprache und die Traces konnte für die Vereinigung
der gleichen Mengen über die Implementierungen immer nur eine
Inklusionsrichtung gefolgert werden, da die Definition~\ref{SimDef} nach einen
Fehler-Zustand in $P$ beliebiges Verhalten ins dessen as"=Implementierungen
zulässt. Mit dem Einsatz der \cont{}-Funktion zum beliebigen fortsetzten der
Traces kann dies ausgeglichen werden. Somit gilt wie die nächsten Proposition
behauptet für die Fehler-Traces und die Fehler-geflutete Sprache Gleichheit
und nicht nur die Inklusion, die aus den Propositionen vorher bereits folgt.

\begin{Prop}[Kommunikationsfehler-Semantik und Implementierungen]
  \label{KommSemProp}
  Sie $P$ ein \MEIO{}.
  \begin{enumerate}
    \item Für die Menge der Fehler-Traces von $P$ gilt $\ET (P) =
      \underset{P'\in\asimp (P)}{\bigcup} \ET (P')$.
    \item Für die Fehler-geflutete Sprache von $P$ gilt $\EL (P) =
      \underset{P'\in\asimp (P)}{\bigcup} \EL (P')$.
  \end{enumerate}
\end{Prop}
\begin{proof}\mbox{}\\
  1. \glqq$\subseteq$\grqq{}:
  \begin{align*}
    \ET (P)&\overset{\ref{KommFehlerSemDef}}{=} \cont (\PrET (P))\cup \cont
    (\MIT (P))\\
    &\overset{\ref{KommTracesProp}}{\subseteq} \cont
    \left(\underset{P'\in\asimp (P)}{\bigcup} \PrET (P')\right)\cup \cont
    \left(\underset{P'\in\asimp (P)}{\bigcup} \MIT (P')\right)\\
    &\hspace{-0.4cm}\overset{\cont}{\overset{\text{monoton}}{=}} \underset{P'\in\asimp
    (P)}{\bigcup} \cont (\PrET (P'))\cup \cont (\MIT (P'))\\
    &\overset{\ref{KommFehlerSemDef}}{=} \underset{P'\in\asimp
    (P)}{\bigcup} \ET (P').\\
  \end{align*}

  1. \glqq$\supseteq$\grqq{}:\\
  Da für $P$ $\ET (P) = \cont (\PrET (P)) \cup \cont (\MIT (P))$ gilt und für
  alle as"=Implementierungen $P'$ von $P$ die analogen Gleichungen für $\ET
  (P')$ gelten, genügt es ein präfix-minimales $w$ aus $\ET (P')$ für eine
  as"=Implementierung $P'$ von $P$ zu betrachten. Da $w$ präfix-minimal ist für
  $P'$ ist $w$ entweder vollständig oder bis auf den letzten Buchstaben in $P'$
  ausführbar. Dies hängt davon ab, ob $w\in \PrET (P')$ oder $w\in \MIT (P')$
  gilt. Für $P$ muss jedoch nicht mal das Präfix von $w$ ohne den letzten
  Buchstaben von $w$ ausführbar sein.
  \begin{itemize}
    \item Fall 1 ($w\in\PrET (P')$): In $P'$ existiert eine Verlängerung $v\in
      O^*$ von $w$, so dass $wv\in\StET (P')$ gilt. Es gibt also Zustände
      $p'_1, p'_2,\dots ,p'_n$ in $P'$ und ein Wort $w'$ für das $\hat{w'}=wv$
      und $w'=\alpha _1\alpha _2\dots \alpha _n$ gilt. Aus diesen Bausteinen
      ist der strikte Fehler-Trace $p'_0 \must[\alpha _1]_{P'} p'_1
      \must[\alpha _2]_{P'} \dots p'_{n-1} \must[\alpha _n]_{P'} p'_n \in
      E_{P'}$ aufgebaut. Es kann \oBdA{} davon ausgegangen werden, dass für
      alle Zustände außer $p'_n$ gilt, dass sie keine Fehler-Zustände sind. Da
      $P'$ eine as"=Implementierung von $P$ ist, muss für die Startzustände der
      beiden Transitionssysteme $p'_0\asRel p_0$ gelten. Es wird nun versucht
      via Induktion zu zeigen, dass man den strikten Fehler-Trace in $P$
      nachmachen kann. Es soll dafür beweisen werden, falls $p_j\notin E_P$
      gilt und $p'_j\asRel p_j$ bereits nachgewiesen wurde, dass auch
      $p'_{j+1}\asRel p_{j+1}$ gelten muss für ein $p_{j+1}\in P$ und $0\leq j
      < n$. Für $j=0$ gilt bereits $p'_j\asRel p_j$ wie oben beschrieben wurde.
      Falls ein $p_j$ vor $p_n$ in $E_P$ enthalten ist, dann ist ein Präfix von
      $wv$ ein strikter Fehler-Trace in $P$ und mit $\prune(w)=\prune(wv)$ ist
      somit ein Präfix von $w$ in $\ET (P)$ enthalten. Da \ET{} unter
      Fortsetzung der Traces abgeschlossen ist, gilt auch $w\in \ET (P)$. Es
      wird im Folgenden also davon ausgegangen, dass für alle $p_j$ mit $j < n$
      gilt $p_j\notin E_P$. Es gelten somit alle Voraussetzungen für die
      Induktion und in $P'$ gibt es die Transition $p'_j \may[\alpha _{j}]_{P'}
      p'_{j+1}$. Mit Definition~\ref{SimDef}~3.\ folgt draus, dass es auch eine
      Transition $p_j \may[\alpha _{j}]_P p_{j+1}$ in $P$ geben muss, so dass
      $p'_{j+1}\asRel p_{j+1}$ gilt. Aus dieses Induktion kann nun gefolgert
      werden, dass $p'_n\asRel p_n$ gelten muss. Da $p'_n\in E_{P'}$ gilt,
      folgt draus mit~\ref{SimDef}~1., dass bereits $p_n$ in $E_P$ enthalten
      sein muss. Es gilt also $wv\in\StET (P)$ und mit der Argumentation von
      oben folgt daraus $w\in\ET (P)$.
    \item Fall 2 ($w\in\MIT (P')\backslash\PrET (P')$): Da $w$ in $\MIT (P')$
      enthalten ist, gibt es ein $a\in I$ für das $w=va$ gilt mit $v\in\Sigma
      ^*$. Analog zu Fall 1 kann man auch den Input-kritischen Trace in $P'$
      darstellen als $p'_0 \must[\alpha _1]_{P'_1} p'_1 \must[\alpha _2]_{P'_1}
      \dots p'_{n-1} \nmust[\alpha _n]_{P'_1}$ wobei $w = (\alpha _1\alpha _2
      \dots \alpha _n)|_{\Sigma}$ und $a=\alpha _n$ gilt. Mit einer analogen
      Induktion wie im letzten Fall für $0\leq j < n-1$ kann $p'_{n-1}\asRel
      p_{n-1}$ für einen Zustand $p_{n-1}$ in $P$ gefolgert werden, falls nicht
      davor ein $p_j$ in der Induktion auftaucht, für das $p_j\in E_P$ gilt.
      Falls eines der $p_j$ mit $0\leq j \leq n-1$ ein Fehler-Zustand ist, gilt
      $w\in\ET (P)$ mit der analogen Argumentation wie oben. Es wird im
      folgenden davon ausgegangen, dass $p_{n-1}$ mit $p'_{n-1}$ in Relation
      steht und keine Fehler-Zustände aufgetreten sind. Falls $a$ für $p_{n-1}$
      eine ausgehende must"=Transition wäre, würde~\ref{SimDef}~2.\ auch für
      $p'_{n-1}$ die Implementierung der $a$ Transition fordern und somit würde
      nicht $va\in\MIT (P')$ gelten. Es gilt also $p\nmust[a]$ und $va\in\MIT
      (P)$. Daraus ergibt sich direkt $w\in\ET (P)$.
  \end{itemize}

  2. \glqq$\subseteq$\grqq{}:
  {\allowdisplaybreaks
  \begin{align*}
    \EL (P)&\overset{\ref{KommFehlerSemDef}}{=} L(P)\cup \ET (P)\\
    &\overset{\ref{LImpProp}}{\subseteq} \left(\underset{P'\in\asimp
    (P)}{\bigcup} L (P')\right)\cup \ET (P)\\
    &\overset{1.}{=} \left(\underset{P'\in\asimp (P)}{\bigcup} L
    (P')\right)\cup \left(\underset{P'\in\asimp (P)}{\bigcup} \ET (P')\right)\\
    &= \underset{P'\in\asimp (P)}{\bigcup} L (P')\cup \ET (P')\\
    &\overset{\ref{KommFehlerSemDef}}{=} \underset{P'\in\asimp (P)}{\bigcup}
    \EL (P').\\
  \end{align*}}

  2. \glqq$\supseteq$\grqq{}:\\
  Da der erste Punkt dieser Proposition bereits beweisen ist, reicht es aus für
  diesen Punkt zu zeigen, dass $\underset{P'\in\asimp (P)}{\bigcup} \EL
  (P')\backslash \ET (P')$ eine Teilmenge von $\EL (P)$ ist. Die Menge $\EL
  (P')\backslash \ET (P')$ entspricht $L (P') \backslash \ET (P')$. Es musst
  also ein Wort $w$ aus der Sprache einer as"=Implementierung von $P$
  betrachtet werden, dass nicht in den Fehler-Traces dieser as"=Implementierung
  enthalten ist. Das Wort $w$ ist also in $P'$ ausführbar. Falls das $w$ in $P$
  jedoch nicht ausführbar ist, folgt wie zuvor $w\in\ET (P)\subseteq \EL (P)$,
  da ein Präfix von $w$ in $P$ zu einem Fehler-Zustand führen musst. Falls $w$
  ausführbar ist in $P$ gilt $w\in L(P)\subseteq \EL (P)$.
\end{proof}

\begin{Kor}[lokale Fehler Erreichbarkeit]\mbox{}
  \label{lokalFehlerErrKor}
  \begin{enumerate}[(i)]
    \item Es ist ein Fehler lokal erreichbar in einem \MEIO{} $P$
      $\Leftrightarrow \exists$ as"=Implementierung von $P$, in der ein Fehler
      lokal erreichbar ist.
    \item Falls für zwei \MEIO{}s $P_1\EBRel P_2$ gilt und in $P_1$ ein Fehler
      lokal erreichbar ist, dann ist auch in $P_2$ ein Fehler lokal erreichbar.
  \end{enumerate}
  \TODO{in spätere Argumentation als Verkürzung aufnehmen}
\end{Kor}
\begin{proof}\mbox{}\\
  $(i) \Rightarrow$:\\
  Da ein Fehler in $P$ lokal erreichbar ist, gilt $\varepsilon \in\PrET _P
  \subseteq \ET _P$. Es muss aufgrund von Proposition~\ref{KommSemProp}~1.\
  mindestens ein $P'\in\asimp (P)$ geben, für dass $\varepsilon \in \ET _{P'}$
  gilt. Dies kann auch nur der Fall sein, wenn durch lokale Aktionen in $P'$
  ein Fehler-Zustand erreicht werden kann. Es ist also auch ein der
  as"=Implikation $P'$ ein Fehler lokal erreichbar.

  $(i) \Leftarrow$:\\
  Sie $P'$ die as"=Implementierung von $P$, in der ein Fehler lokal erreichbar
  ist. Es gilt dann $\varepsilon\in \PrET (P') \subseteq \ET _{P'}$. Mit
  Proposition~\ref{KommSemProp}~1. folgt daraus $\varepsilon \in\ET _P$. Es
  muss also auch in $P$ ein Fehler-Zustand lokal erreichbar sein.

  $(ii)$:\\
  Da für $P_1$ ein Fehler lokale erreichbar ist folgt mit 1.\, dass es auch
  eine as"=Implikation $P'_1$ von $P_1$ gibt, für die ein Fehler lokal
  erreichbar ist. Mit $P_1\EBRel P_2$ ergibt sich daraus, dass es auch eine
  as"=Implementierung $P'_2$ von $P_2$ geben muss, die lokal einen
  Fehler-Zustand erreichen kann. $P_2$ muss aufgrund von 1.\ ebenfalls einen
  Fehler lokal erreichen können, da es eine as"=Implementierung gibt, die dies
  kann.
\end{proof}

\begin{Satz}[Kommunikationsfehler-Semantik für Parallelkompositionen]
  \label{KommFehlerSemSatz}
  Für zwei komponierbare \MEIO{}s $P_1,P_2$ und ihre Komposition $P_{12}$ gilt:
  \begin{enumerate}
    \item $\ET _{12} = \cont (\prune ((\ET _1\|\EL _2) \cup (\EL _1\|\ET
      _2)))$,
    \item $\EL _{12} = (\EL _1\|\EL _2) \cup \ET _{12}$.
  \end{enumerate}
\end{Satz}
\begin{proof}\mbox{}\\
  1. \glqq$\subseteq$\grqq{}:\\
  Da beide Seiten der Gleichung unter der Fortsetzung \cont{} abgeschlossen
  sind, genügt es ein präfix-minimales Element $w$ von $\ET _{12}$ zu
  betrachten. Diese Element ist aufgrund der Definition der Menge der
  Fehler-Traces in $\MIT _{12}$ oder in $\PrET _{12}$ enthalten.
  \begin{itemize}
    \item Fall 1 ($w\in\MIT _{12}$): Aus der Definition von \MIT{} folgt, dass
      es eine Aufteilung $w=xa$ gibt mit $(p_{01},p_{02}) \weakmay[x]_{12}
      (p_1,p_2) \land a \in I_{12} \land (p_1,p_2) \nmust[a]_{12}$. Da $I_{12}
      = (I_1\cup I_2) \backslash (O_1\cup O_2)$ ist, folgt $a\in (I_1\cup I_2)$
      und $a\notin (O_1\cup O_2)$. Es wird unterschieden, ob $a\in (I_1\cap
      I_2)$ oder $a\in (I_1\cup I_2) \backslash (I_1\cap I_2)$ ist.
    \begin{itemize}
      \item Fall 1a) ($a\in (I_1\cap I_2)$): Durch Projektion des Ablaufes auf
        die einzelnen Transitionssysteme erhält man \oBdA{}
        $p_{01}\weakmay[x_1]_1 p_1\nmust[a]_1$ und $p_{02}\weakmay[x_2]_2
        p_2\nmust[a]_2$ oder $p_{02}\weakmay[x_2]_2 p_2\must[a]_2$ mit $x\in
        x_1\|x_2$. Daraus kann $x_1a\in \cont (\MIT _1) \subseteq \ET _1$ und
        $x_2a\in \EL _2$ ($x_2a\in \MIT _2$ oder $x_2a \in L_2$) gefolgert
        werden. Damit folgt $w\in (x_1\|x_2) \cdot \{a\} =
        (x_1a)\|(x_2a)\subseteq \ET _1\|\EL _2$, und somit ist $w$ in der
        rechten Seite der Gleichung enthalten.
      \item Fall 1b) ($a\in (I_1\cup I_2)\backslash (I_1\cap I_2)$): \OBdA{}
        gilt $a\in I_1$. Durch die Projektion auf die einzelnen Komponenten
        erhält man: $p_{01}\weakmay[x_1]_1 p_1\nmust[a]_1$ und
        $p_{02}\weakmay[x_2]_2 p_2$ mit $x\in x_1\|x_2$. Daraus folgt $x_1a\in
        \MIT _1 \subseteq \ET _1$ und $x_2\in L_2\subseteq \EL _2$. Somit gilt
        $w\in (x_1\|x_2) \cdot \{a\} \subseteq (x_1a)\|x_2\subseteq \ET _1\|\EL
        _2$. Dies ist eine Teilmenge der rechten Seite der Gleichung.
    \end{itemize}
  \item Fall 2 ($w\in\PrET _{12}$): Aus der Definition von \PrET{} und \prune{}
    folgt, dass ein $v\in O_{12}^*$ gibt, so dass $(p_{01},p_{02})
      \weakmay[w]_{12} (p_1,p_2) \weakmay[v]_{12} (p'_1,p'_2)$ gilt mit
      $(p'_1,p'_2)\in E_{12}$ und $w=\prune (wv)$. Durch Projektion auf die
      Komponenten erhält man $p_{01} \weakmay[w_1]_1 p_1 \weakmay[v_1]_1 p'_1$
      und $p_{02} \weakmay[w_2]_2 p_2 \weakmay[v_2]_2 p'_2$ mit $w\in w_1\|w_2$
      und $v\in v_1\|v_2$. Aus $(p'_1,p'_2)\in\ET _{12}$ folgt, dass es sich
      entweder um einen geerbten oder einen neuen Fehler handelt. Bei einem
      geerbten wäre bereits einer der beiden Zustände $p'_1$ bzw.\ $p'_2$ ein
      Fehler"=Zustand gewesen. Ein neuer Fehler hingegen wäre
      durch das fehlen fehlende Sicherstellen der Synchronisation (fehlende
      must"=Transition) in einer der Komponenten entstanden.
    \begin{itemize}
      \item Fall 2a) (geerbter Fehler): \OBdA{} gilt $p'_1\in E_1$. Daraus
        folgt, $w_1v_1\in StET_1 \subseteq \cont (\PrET _1) \subseteq \ET _1$.
        Da $p_{02} \weakmay[w_2v_2]_2$ gilt, erhält man $w_2v_2\in L_2\subseteq
        \EL _2$. Dadurch ergibt sich $wv\in \ET _1\|\EL _2$ mit $w=\prune (wv)$
        und somit ist $w$ in der rechten Seite der Gleichung enthalten.
      \item Fall 2b) (neuer Fehler): \OBdA{} gilt $a\in I_1\cap O_2$ mit
        $p'_1\nmust[a]_1$ und $p'_2\may[a]_2$. Daraus folgt $w_1v_1a\in \MIT _1
        \subseteq \ET _1$ und $w_2v_2a\in L_2 \subseteq \EL _2$. Damit ergibt
        sich $wva\in \ET _1\|\EL_2$, da $a\in O_1\subseteq O_{12}$ gilt
        $w=\prune (wva)$ und somit ist $w$ in der rechten Seite der Gleichung
        enthalten.
    \end{itemize}
  \end{itemize}

  1. \glqq$\supseteq$\grqq{}:\\
  Wegen der Abgeschlossenheit beider Seiten der Gleichung gegenüber \cont{}
  wird auch in diesem Fall nur ein präfix-minimales Element $x\in\prune ((\ET
  _1\|\EL _2)\cup (\EL _1\| \ET _2))$ betrachtet. Da $x$ durch die Anwendung
  der \prune{}-Funktion entstanden ist, existiert ein $y\in O_{12} ^*$ mit
  $xy\in (\ET _1\|\EL _2)\cup (\EL _1\| \ET _2)$. \OBdA{} wird davon
  ausgegangen, dass $xy\in \ET _1\| \EL _2$ gilt, d.h.\ es gibt $w_1\in \ET _1$
  und $w_2\in \EL _2$ mit $xy \in w_1\| w_2$.\\
  Im Folgenden wird für alle Fälle von $xy$ gezeigt, dass es ein $v\in \PrET
  _{12} \cup \MIT _{12}$ gibt, das ein Präfix von $xy$ ist. Also $v$ entweder
  auf einen Input $I_{12}$ endet oder $v=\varepsilon$. Damit muss $v$ ein
  Präfix von $x$ sein, denn $\varepsilon$ ist Präfix von jedem Wort und sobald
  $v$ mindestens einen Buchstaben enthält, muss das Ende von $v$ vor dem Anfang
  von $y\in O_{12}^*$ liegen. Dadurch ist ein Präfix von $x$ in $\PrET
  _{12}\cup \MIT _{12}$ enthalten und somit gilt $x\in \ET _{12}$, da \ET{} die
  Fortsetzung der Mengenvereinigung aus \PrET{} und \MIT{} ist.\\
  Sei $v_1$ das kürzeste Präfix von $w_1$ in $\PrET _1\cup \MIT _1$. Falls $w_2
  \in L_2$, so sei $v_2=w_2$, sonst soll $v_2$ das kürzeste Präfix von $w_2$ in
  $\PrET _2\cup \MIT _2$ sein. Jede Aktion in $v_1$ und $v_2$ hängt mit einer
  aus $xy$ zusammen. Es kann nun davon ausgegangen werden, dass entweder $v_2 =
  w_2\in L_2$ gilt oder die letzte Aktion von $v_1$ vor oder gleichzeitig mit
  der letzten Aktion von $v_2$ statt findet. Ansonsten endet $v_2 \in \PrET
  _2\cup \MIT _2$ vor $v_1$. Es gilt dann $w_2\in\ET _2$ und somit ist dieser
  Fall analog zu $v_1$ endet vor $v_2$.
  \begin{itemize}
    \item Fall 1 ($v_1=\varepsilon$): Da $\varepsilon\in\PrET _1\cup \MIT _1$,
      ist bereits in $P_1$ ein Fehler"=Zustand lokal erreichbar. $\varepsilon
      \in \MIT _1$ ist nicht möglich, da jedes Element aus \MIT{} nach
      Definition mindestens die Länge $1$ haben muss. Mit der Wahl
      $v'_2=v'=\varepsilon$ ist $v'_2$ ein Präfix von $v_2$.
    \item Fall 2 ($v_1\neq\varepsilon$): Aufgrund der Definitionen von \PrET{}
      und \MIT{} endet $v_1$ auf ein $a\in I_1$, d.h.\ $v_1=v'_1a$. $v'$ sei
      das Präfix von $xy$, das mit der letzten Aktion von $v_1$ endet, d.h.\
      mit $a$ und $v'_2=v'|_{\Sigma _2}$. Falls $v_2=w_2\in L_2$, dann ist
      $v'_2$ ein Präfix von $v_2$. Falls $v_2\in \PrET _2\cup \MIT _2$ gilt,
      dann ist durch die Annahme, dass $v_2$ nicht vor $v_1$ endet, $v'_2$ ein
      Präfix von $v_2$. Im Fall $v_2 \in \MIT _2$ weiß man zusätzlich, dass
      $v_2$ auf $b\in I_2$ endet. Es kann jedoch $a=b$ gelten.
  \end{itemize}
  In den beiden vorangegangen Fällen erhält man $v'_2=v'|_{\Sigma _2}$ ist ein
  Präfix von $v_2$ und $v'\in v_1\|v'_2$ ist ein Präfix von $xy$. Es kann nur
  für die Fälle $a\notin I_2$ gefolgert werden, dass $p_{02} \weakmay[v'_2]_2$
  gilt. Falls $p_{02} \may[v_2]$ nicht gilt, ist $v'_2=v_2\in\MIT _2$ und
  $v'_2$ endet auf $a\in I_2$.
  \begin{itemize}
    \item Fall I ($v_1\in\MIT _1$ und $v_1\neq \varepsilon$): Es gibt einen
      Ablauf der Form $p_{01}\weakmay[v'_1]_1p_1\nmust[a]_1$ und es gilt
      $v'=v''a$.
      \begin{itemize}
        \item Fall Ia) ($a\notin\Sigma _2$): Es gilt $p_{02}\weakmay[v'_2]_2
          p_2$ mit $v''\in v'_1\|v'_2$. Dadurch erhält man $(p_{01},p_{02})
          \weakmay[v'']_{12} (p_1,p_2) \nmust[a]_{12}$ mit $a\in I_{12}$. Somit
          wird $v := v''a=v'\in\MIT _{12}$ gewählt.
        \item Fall Ib) ($a\in I_2$ und $v'_2\in\MIT _2$): Es gilt $v'_2=v''_2a$
          mit $p_{02}\weakmay[v''_2]_2 p_2 \nmust[a]_2$ und $v''\in
          v'_1\|v''_2$. $a$ ist für $P_2$, ebenso wie für $P_1$, ein nicht
          sichergestellter Input. Daraus folgt, dass $(p_1,p_2)\nmust[a]_{12}$
          gilt. Es wird ebenfalls $v := v''a=v'\in\MIT _{12}$ gewählt.
        \item Fall Ic) ($a\in I_2$ und $v'_2\notin \backslash\MIT _2$): Es gilt
          $p_{02}\weakmay[v''_2]_2p_2\must[a]_2$ mit $v'_2=v''_2a \in L_2$. Da
          die gemeinsamen Inputs synchronisiert werden, folgt $(p_1,p_2)
          \nmust[a]_{12}$ bereits aus $q_1\nmust[a]_1$. Somit kann hier
          nochmals $v:=v''a=v'\in \MIT _{12}$ gewählt werden.
        \item Fall Id) ($a\in O_2$): Es gilt $v'_2=v''_2a$ und $p_{02}
          \weakmay[v'_2]_2$. Man erhält also $p_{02}\weakmay[v''_2]_2 p_2 \may[a]_2$
          mit $v''\in v'_1\|v''_2$. Daraus ergibt sich $(p_{01},p_{02})
          \weakmay[v'']_{12} (p_1,p_2)$ mit $p_2\may[a]_1$, $p_1\nmust[a]_1,
          a\in I_1$ und $a\in O_2$, somit gilt $(p_1,p_2)\in E_{12}$.
          Es wird $v:= \prune (v'')\in\PrET _{12}$ gewählt.
      \end{itemize}
    \item Fall II ($v_1\in\PrET _1$): $\exists u_1\in O_1^*: p_{01}
      \weakmay[v_1]_1 p_1 \weakmay[u_1]_1 p'_1$ mit $p'_1\in E_1$. Im Fall $v_1
      \neq\varepsilon$ kann das $a$, auf das $v_1$ endet, ebenfalls der letzte
      Buchstabe von $v_2$ sein. Im Fall von $v_2\in\MIT _2$ kann somit $a=b$
      gelten, wodurch $v_2=v'_2$ gilt. Dieser Fall verläuft jedoch analog zu
      Fall Ic) und wird hier nicht weiter betrachtet. Es gilt für alle anderen
      Fälle $p_{02}\weakmay[v'_2]_2p_2$ mit $(p_{01},p_{02}) \weakmay[v']_{12}
      (p_1,p_2)$.
      \begin{itemize}
        \item Fall IIa) \big($u_2\in (O_1\cap I_2)^*,c\in (O_1\cap I_2)$,
          sodass $u_2c$ Präfix von $u_1|_{I_2}$ mit $p_2 \weakmust[u_2]_2 p'_2
          \nmust[c]_2$\big): Für das Präfix $u'_1c$ von $u_1$ mit
          $(u'_1c)|_{I_2}=u_2c$ weiß man, dass $p_1\weakmay[u'_1]_1 p''_1
          \may[c]_1$. Somit gilt $u'_1\in u'_1\|u_2$ und $(p_1,p_2)
          \weakmay[u'_1]_{12} (p''_1,p'_2)\in E_{12}$, da für $P_2$ der
          entsprechende Input nicht sichergestellt wird, der mit dem $c$ Output
          von $P_1$ zu koppeln wäre. Es handelt sich also um einen neuen
          Fehler. Es wird $v := \prune (v'u'_1)\in \PrET _{12}$
          gewählt, dies ist ein Präfix von $v'$, da $u_1\in O_1^*$.
        \item Fall IIb) \big($p_2\weakmust[u_2]_2p'_2$ mit
          $u_2=u_1|_{I_2}$\big): Es gilt $u_1\in u_1\|u_2$ und $(p_1,p_2)
          \weakmay[u_1]_{12} (p'_1,p'_2)\in E_{12}$, da $p'_1\in E_1$ und somit
          handelt es sich in $P_{12}$ um einen geerbten Fehler. Nun wird
          $v:=\prune (v'u_1)\in\PrET _{12}$ gewählt, das wiederum ein Präfix
          von $v'$ ist.
      \end{itemize}
  \end{itemize}

  2.:\\
  Durch die Definitionen ist klar, dass $L_i\subseteq \EL _i$ und $\ET
  _i\subseteq \EL _i$ gilt. Die Argumentation startet auf den rechten Seite der
  Gleichung:
  \begin{align*}
    (\EL{}_1\| \EL{}_2)\cup
    \ET{}_{12}&\overset{\ref{KommFehlerSemDef}}{=}\left(\left(L_1\cup
    \ET{}_1\right)\|\left(L_2\cup \ET{}_2\right)\right)\cup \ET{}_{12}\\
    &=(L_1\|L_2) \cup \underset{\overset{1.}{\subseteq}
    \ET{}_{12}}{\underset{\subseteq
    (\EL{}_1\|\ET{}_2)}{\underbrace{(L_1\|\ET{}_2)}}} \cup
    \underset{\overset{1.}{\subseteq} \ET{}_{12}}{\underset{\subseteq
    (\ET{}_1\|\EL{}_2)}{\underbrace{(\ET{}_1\|L_2)}}} \cup
    \underset{\overset{1.}{\subseteq} \ET{}_{12}}{\underset{\subseteq
    (\EL{}_1\|\ET{}_2)}{\underbrace{(\ET{}_1\|\ET{}_2)}}} \cup \ET{}_{12}\\
    &=(L_1\|L_2) \cup \ET{}_{12}\\
    &\overset{\ref{LParallelProp}}{=}L_{12}\cup \ET{}_{12}\\
    &\overset{\ref{KommFehlerSemDef}}{=}\EL{}_{12}.
  \end{align*}
\end{proof}

\begin{Kor}[Kommunikationsfehler-Präkongruenz]
  \label{KommPraekonKor}
  Die Relation \ERel{} ist eine Präkongruenz bezüglich $\cdot\|\cdot$.
\end{Kor}
\begin{proof}
  Es muss gezeigt werden: Wenn $P_1\ERel P_2$ gilt, dann für jedes
  komponierbare $P_3$ auch $P_{31}\ERel P_{32}$. D.h.\ es ist zu zeigen,
  dass aus $\ET{}_1\subseteq \ET{}_2$ und $\EL{}_1\subseteq \EL{}_2$,
  $\ET{}_{31}\subseteq \ET{}_{32}$ und $\EL{}_{31}\subseteq
  \EL{}_{32}$ folgt. Dies ergibt sich aus der Monotonie von \cont{},
  \prune{} und $\cdot \|\cdot$ auf Sprachen wie folgt:\\
  \begin{itemize}
    \item $\begin{aligned}[t]
        \ET{}_{31} &\overset{\ref{KommFehlerSemSatz}~1.}{=}
      \cont{}\left(\prune{}\left(\left(\ET{}_3\|\EL{}_1\right)\cup
          \left(\EL{}_3\|\ET{}_1\right)\right)\right)\\
      &\hspace{-0.4cm}\overset{\ET{}_1\subseteq
    \ET{}_2}{\overset{\mathrm{und}}{\overset{\EL{}_1\subseteq \EL{}_2}{\subseteq}}}
    \cont{}\left(\prune{}\left(\left(\ET{}_3\|\EL{}_2\right)\cup
        \left(\EL{}_3\|\ET{}_2\right)\right)\right)\\
    &\overset{\ref{KommFehlerSemSatz}~1.}{=} \ET{}_{32},
    \end{aligned}$
    \item $\begin{aligned}[t]
        \EL{}_{31} &\overset{\ref{KommFehlerSemSatz}~2.}{=}
        (\EL{}_3\|\EL{}_1)\cup E_{31}\\
        &\hspace{-0.5cm}\overset{\EL{}_1\subseteq
      \EL{}_2}{\overset{\mathrm{und}}{\overset{\ET{}_{31}\subseteq
      \ET{}_{32}}{\subseteq}}} (\EL{}_3\|\EL{}_2)\cup \ET{}_{32}\\
      &\overset{\ref{KommFehlerSemSatz}~2.}{=} \EL{}_{32}.
    \end{aligned}$
  \end{itemize}
\end{proof}

\begin{Lem}[Verfeinerung mit Kommunikationsfehlern]
  \label{KommVerfeinLem}
  Gegeben sind zwei \MEIO{}s $P_1$ und $P_2$ mit der gleichen Signatur. Wenn
  $U\|P_1\EBRel{} U\|P_2$ für alle Partner $U$ gilt, dann folgt daraus die
  Gültigkeit von $P_1\ERel{} P_2$.
\end{Lem}
\begin{proof}
  Da $P_1$ und $P_2$ die gleiche Signaturen haben wird $I:=I_1=I_2$ und
  $O:=O_1=O_2$ definiert. Für jeden Partner $U$ gilt $I_U=O$ und $O_U=I$.\\
  Um $P_1\ERel{}P_2$ zu zeigen, wird nachgeprüft, ob folgendes gilt:
  \begin{itemize}
    \item $\ET _1\subseteq \ET _2$,
    \item $\EL _1\subseteq \EL _2$.
  \end{itemize}
  Für ein gewähltes präfix-minimales Element $w\in\ET _1$ wir gezeigt, dass
  dieses $w$ oder eines seiner Präfixe in $\ET _2$ enthalten ist. Dies ist
  möglich, da die beiden Mengen $\ET _1$ und $\ET _2$ durch \cont{}
  abgeschlossen sind.
  \begin{itemize}
    \item Fall 1 ($w=\varepsilon$): Es handelt sich um einen lokal erreichbaren
      Fehler-Zustand in $P_1$. Für $U$ wird ein Transitionssystem verwendet,
      das nur aus dem Startzustand und einer must-Schleife für alle Inputs
      $x\in I_U$ besteht. Somit kann $P_1$ die im Prinzip gleichen
      Fehler-Zustände lokal erreichen wie $U\|P_1$.
      Wegen~\ref{lokalFehlerErrKor}~(i) erreicht auch eine as"=Implementierung
      von $U\|P_1$ lokal einen Fehler-Zustand und daher muss auch mindestens
      eine as"=Implementierung von $U\|P_2$ einen lokal erreichbaren Fehler
      haben. Dies gilt wegen wegen~\ref{lokalFehlerErrKor}~(i) auch für
      $U\|P_2$ und durch die Definition von $U$ kann dieser Fehler nur von
      $P_2$ geerbt sein. Es muss also in $P_2$ ein Fehler-Zustand durch interne
      Aktionen und Outputs erreichbar sein, d.h.\ es gilt $\varepsilon\in
      \PrET{}_2$.
    \item Fall 2 ($w=x_1\dots x_n x_{n+1}\in\Sigma ^+$ mit $n\geq 0$ und
      $x_{n+1}\in I = O_U$): Es wird der folgende Partner $U$ betrachtet (siehe
      auch Abbildung~\ref{UohneE}):
      \begin{itemize}
        \item $U=\{p_0,p_1,\dots ,p_{n+1}\}$,
        \item $p_{0U}=p_0$,
        \item $\begin{aligned}[t]
            \may _U = \must _U&=\{(p_j,x_{j+1},p_{j+1})\mid  0\leq j\leq n\}\\
            &\cup\{(p_j,x,p_{n+1})\mid  x\in I_U\backslash\{x_{j+1}\}, 0\leq
            j\leq n\}\\
            &\cup\{(p_{n+1},x,p_{n+1})\mid  x\in I_U\},
        \end{aligned}$
        \item $E_U=\emptyset$.
      \end{itemize}
      \begin{figure} [h!tbp]
      \begin{center}
        \begin{tikzpicture}[->, >=latex',auto,node distance =3cm, semithick]

          \node (0) {$p_0$};
          \node (1) [right of=0] {$p_1$};
          \node (dots) [right of=1] {$\dots$};
          \node (n) [right of=dots] {$p_n$};
          \node (n1) at ($(1)!0.5!(dots) + (0,-3)$) {$p_{n+1}$};

          \path ($ (0) + (-1,0) $) edge (0)
                (0) edge node {$x_1$} (1)
                    edge [bend right] node [below, sloped] {$x?\neq x_1$} (n1)
                (1) edge node {$x_2$} (dots)
                    edge node [below, sloped] {$x?\neq x_2$} (n1)
                (dots) edge node {$x_n$} (n)
                       edge [dashed] (n1)
                (n) edge node [above, sloped] {$x?\in I_U$} (n1)
                    edge [bend left] node [sloped] {$x_{n+1}$!} (n1)
                (n1) edge [loop below] node {$x?\in I_U$} (n1);
        \end{tikzpicture}
        \caption{$x?\neq x_j$ steht für alle $x\in I_U\backslash\{x_j\}$}
      \label{UohneE}
      \end{center}
      \end{figure}
      Für $w$ können nun zwei Fälle unterschieden werden. Aus beiden wird
      folgen, dass für mindestens eine as"=Implementierung $P'$ von $U\|P_1$
      $\varepsilon\in\PrET(P')$ gilt.
      \begin{itemize}
        \item Fall 2a) ($w\in\MIT _1$): In $U\|P_1$ erhält man $(p_0,p_{01})
          \lweakmay[x_1\dots x_n]_{U\|P_1} (p_n,p')$ mit $p'\nmust[x_{n+1}]_1$
          und $p_n\must[x_{n+1}]_U$. Deshalb gilt $(p_n,p')\in E_{U\|P_1}$. Da
          alle Aktionen aus $w$ bis auf $x_{n+1}$ synchronisiert werden und
          $I\cap I_U=\emptyset$, gilt $x_1,\dots , x_n\in O_{U\|P_1}$. Da
          $(p_n,p')\in E_{U\|P_1}$ in $U\|P_1$ lokal erreichbar ist gibt es
          mindestens ein $P'$ in $\asimp (U\|P_1)$, das ebenfalls einen lokal
          erreichbaren Fehler-Zustand enthält, wegen~\ref{lokalFehlerErrKor}.
          Daraus ergibt sich dann $\varepsilon\in\PrET (P')$.
        \item Fall 2b) ($w\in\PrET _1$): In der Parallelkomposition von $U$ und
          $P_1$ erhält man $(p_0,p_{01}) \weakmay[w]_{U\|P_1} (p_{n+1},p'')
          \weakmay[u]_{U\|P_1} (p_{n+1},p')$ für $u\in O^*$ und $p'\in E_1$.
          Daraus folgt $(p_{n+1},p')\in E_{U\|P_1}$ und somit $wu\in\StET
          (U\|P_1)$. Da alle Aktionen in $w$ synchronisiert werden und $I\cap
          I_U=\emptyset$, gilt $x_1,\dots ,x_n,x_{n+1}\in O_{U\|P_1}$ und, da
          $u\in O^*$, folgt $u\in O_{U\|P_1}^*$. Somit ergibt sich für eine
          as"=Implementierung $P'$ von $U\|P_1$ $\varepsilon\in\PrET (P')$.
      \end{itemize}
      Da $\varepsilon\in\PrET (P')$ für ein $P'$ aus $\asimp (U\|P_1)$ gilt,
      kann durch $U\|P_1\EBRel U\|P_2$ geschlossen werden, dass auch in
      mindestens einer as"=Implementierung von $U\|P_2$ ein Fehler-Zustand
      lokal erreichbar sein muss. Da as"=Implementierungen die
      Definition~\ref{SimDef} erfüllen müssen, muss falls in einer
      as"=Implementierung von $U\|P_2$ ein Fehler lokal erreichbare ist auch in
      $U\|P_2$ ein Fehler-Zustand lokal erreichbar sein.\\
      Der in $U\|P_2$ erreichbare Fehler kann geerbt oder neu sein.
      \begin{itemize}
        \item Fall 2i) (neuer Fehler): Da jeder Zustand von $U$ alle Inputs
          $x\in O=I_U$ durch must"=Transitionen sicherstellt, muss ein lokal
          erreichbarer Fehler-Zustand der Form sein, dass ein Output $a\in O_U$
          von $U$ möglich ist, der nicht mit einem passenden Input aus $P_2$
          synchronisiert werden muss ($P_2$ enthält die entsprechende $a$
          Transitionen nicht als must"=Transition). Durch die Konstruktion von
          $U$ sind in $p_{n+1}$ keine Outputs möglich. Ein neuer
          Fehler muss also die Form $(p_i,p')$ haben mit $i\leq
          n, p'\nmust[x_{i+1}]_2$ und $x_{i+1}\in O_U=I$. Durch Projektion
          erhält man dann $p_{02}\lweakmay[x_1\dots x_i]_2p'\nmust[x_{i+1}]_2$
          und damit gilt $x_1\dots x_{i+1}\in\MIT _2\subseteq \ET _2$. Somit
          ist ein Präfix von $w$ in $\ET _2$ enthalten.
        \item Fall 2ii) (geerbter Fehler): $U$ hat $x_1\dots x_iu$ mit $u\in
          I_U^*=O^*$ ausgeführt und ebenso hat $P_2$ dieses Wort abgearbeitet.
          Durch dies hat $P_2$ einen Zustand  in $E_2$ erreicht, da von $U$
          keine Fehler geerbt werden können. Es gilt dann $\prune (x_1\dots
          x_iu) = \prune (x_1\dots x_i)\in\PrET _2\subseteq \ET _2$. Da
          $x_1\dots x_i$ ein Präfix von $w$ ist, führt in diesem Fall eine
          Verlängerung um lokale Aktionen von einem Präfix von $w$ zu einem
          Fehler-Zustand. Da \ET{} der Menge aller Verlängerungen von gekürzten
          Fehler-Traces entspricht, ist $x_1\dots x_i$ in $\ET
          _2$ enthalten und somit ist ein Präfix von $w$ in $\ET _2$ enthalten.
      \end{itemize}
  \end{itemize}

  Um die andere Inklusion zu beweisen, reicht es aufgrund der ersten
  Inklusion und der Definition von \EL{} aus zu zeigen, dass $L_1\backslash\ET
  _1\subseteq \EL _2$ gilt.\\
  Es wird dafür ein beliebiges $w\in L_1\backslash \ET _1$ gewählt und gezeigt,
  dass es in $\EL _2$ enthalten ist.
  \begin{itemize}
    \item Fall 1 ($w=\varepsilon$): Da $\varepsilon$ immer in $\EL _2$
      enthalten ist, muss hier nichts gezeigt werden.
    \item Fall 2 ($w=x_1\dots x_n$ mit $n\geq 1$): Es wird ein Partner $U$ wie
      folgt konstruiert (siehe dazu auch Abbildung~\ref{UmitE}):
      \begin{itemize}
        \item $U=\{p_0,p_1,\dots ,p_n,p\}$,
        \item $p_{0U}=p_0$,
        \item $\begin{aligned}[t]
            \may _U = \must _U&=\{(p_j,x_{j+1},p_{j+1})\mid 0\leq j< n\}\\
            &\cup\{(p_j,x,p)\mid x\in I_U\backslash\{x_{j+1}\},0\leq j< n\}\\
            &\cup\{(p,x,p)\mid x\in I_U\},
        \end{aligned}$
        \item $E_U=\{p_n\}$.
      \end{itemize}
      \begin{figure} [h!tbp]
      \begin{center}
        \begin{tikzpicture}[->, >=latex',auto,node distance =3cm, semithick]

          \node (0) {$p_0$};
          \node (1) [right of=0] {$p_1$};
          \node (dots) [right of=1] {$\dots$};
          \node (n1) [right of=dots] {$p_{n-1}$};
          \node (n) [right of=n1, rectangle, draw] {$p_n\in E_U$};
          \node (q) at ($(1)!0.5!(dots) + (0,-3)$) {$p$};

          \path ($ (0) + (-1,0) $) edge (0)
                (0) edge node {$x_1$} (1)
                    edge [bend right] node [below, sloped] {$x?\neq x_1$} (q)
                (1) edge node {$x_2$} (dots)
                    edge node [below, sloped] {$x?\neq x_2$} (q)
                (dots) edge node {$x_{n-1}$} (n1)
                       edge [dashed] (q)
                (n1) edge node {$x_n$} (n)
                edge [bend left] node [below, sloped] {$x?\neq x_n$} (q)
                (q) edge [loop below] node {$x?\in I_U$} (q);
        \end{tikzpicture}
        \caption{$x?\neq x_j$ steht für alle $x\in I_U\backslash\{x_j\}$, $p_n$
          ist der einzige Fehler-Zustand}
      \label{UmitE}
      \end{center}
      \end{figure}
      Da $p_{01}\weakmay[w]_1p'$ gilt, kann man schließen, dass $U\|P_1$ einen
      lokal erreichbaren geerbten Fehler hat. Es gibt also auch mindestens eine
      as"=Implementierung in der Menge $\asimp (U\|P_1)$, die diesen lokal
      erreichbaren Fehler implementiert. Somit muss es eine as"=Implementierung
      von $U\|P_2$ geben, die ebenfalls einen lokal erreichbaren Fehler-Zustand
      hat. Aufgrund von Definition~\ref{SimDef}~3.\ muss dann ebenfalls ein
      Fehler-Zustand in $U\|P_2$ lokal erreichbar sein.
      \begin{itemize}
        \item Fall 2a) (neuer Fehler aufgrund von $x_i\in O_U$ und $p_{02}
          \lweakmay[x_1\dots x_{i-1}]_2p''\nmust[x_i]_2$): Es gilt $x_1\dots
          x_i\in \MIT _2$ und somit $w\in \EL_2$. Anzumerken ist, dass es nur
          auf diesem Weg Outputs von $U$ möglich sind, deshalb gibt es keine
          anderen Outputs von $U$, die zu einem neuen Fehler führen können.
        \item Fall 2b) (neuer Fehler aufgrund von $a\in O=I_U$): Der einzige
          Zustand, in dem $U$ nicht alle Inputs erlaubt sind, ist $p_n$, der
          bereits ein Fehler-Zustand ist. Da in diesem Fall dieser Zustand in
          $U\|P_2$ erreichbar ist, besitzt das komponierte \MEIO{} einen
          geerbten Fehler und es gilt $w\in L_2\subseteq \EL _2$, wegen dem
          folgenden Fall 2c).
        \item Fall 2c) (geerbter Fehler von $U$): Da $p_n$ der einzige
          Fehler-Zustand in $U$ ist und alle Aktionen synchronisiert sind, ist
          dies nur möglich, wenn $p_{02} \lweakmay[x_1\dots x_n]_2$ gilt. In
          diesem Fall gilt $w\in L_2\subseteq \EL _2$.
        \item Fall 2d) (geerbter Fehler von $P_2$): Es gilt dann $p_{02}
          \lweakmay[x_1\dots x_iu]_2 p'\in E_2$ für $i\geq 0$ und $u\in O^*$.
          Somit ist $x_1\dots x_iu\in\StET _2$ und damit $\prune (x_1\dots
          x_iu) =\prune (x_1\dots x_i)\in\PrET _2\subseteq \EL _2$. Es gilt
          also $w\in\EL _2$.
      \end{itemize}
  \end{itemize}
\end{proof}

Der folgende Satz sagt aus, dass \ERel{} die gröbste Präkongruenz ist, die
charakterisiert werden soll, also gleich der vollständig abstrakten
Präkongruenz \ECRel{}.

\begin{Satz}[Vollständige Abstraktheit für Kommunikationsfehler-Semantik]
  \label{KommVollAbstraktSatz}
  Für zwei \MEIO{}s $P_1$ und $P_2$ mit derselben Signatur gilt $P_1\ECRel
  P_2\Leftrightarrow P_1\ERel{} P_2$.
\end{Satz}
\begin{proof}\mbox{}\\
  \glqq$\Leftarrow$\grqq: Nach Definition gilt, genau dann wenn
  $\varepsilon\in\ET (P)$, ist ein Fehler-Zustand lokal erreichbar in $P$. $P_1
  \ERel P_2$ impliziert, dass $\varepsilon\in\ET _2$ gilt, wenn
  $\varepsilon\in\ET _1$. Somit ist ein Fehler-Zustand in $P_1$ nur dann lokal
  erreichbar, wenn dieser auch in $P_2$ lokal erreichbar ist. Falls es also
  eine as"=Implementierung von $P_1$ gibt, in der ein Fehler-Zustand lokal
  erreichbar ist, dann gibt es auch mindestens eine as"=Implementierung von
  $P_2$, die einen Fehler-Zustand lokal erreichen kann. Dadurch folgt,
  dass $P_1\EBRel P_2$ gilt, da \EBRel{} in Definition~\ref{EBRelDef} über die
  lokale Erreichbarkeit der Fehler-Zustande in den as"=Implementierungen
  definiert wurde und die \ET{}-Mengen von $P_1$ und $P_2$ auch in der
  Vereinigung der Traces ihrer as"=Implementierungen enthalten sind, wie in
  Proposition~\ref{KommSemProp}, ausgedrückt werden können. Es ist also
  \ERel{} in \EBRel{} enthalten. Wie in Korollar~\ref{KommPraekonKor} gezeigt,
  ist \ERel{} eine Präkongruenz bezüglich $\cdot\|\cdot$. Da \ECRel{} die
  gröbste Präkongruenz bezüglich $\cdot\|\cdot$ ist, die in \EBRel{} enthalten
  ist, muss \ERel{} in \ECRel{} enthalten sein. Es folgt also aus $P_1\ERel{}
  P_2$, dass auch $P_1\ECRel{} P_2$ gilt.

  \glqq$\Rightarrow$\grqq: Durch die Definition von \ECRel{} als Präkongruenz
  in~\ref{EBRelDef} folgt aus $P_1\ECRel{} P_2$, dass $U\|P_1\ECRel U\|P_2$ für
  alle \MEIO{}s $U$ gilt, die mit $P_1$ komponierbar sind. Da \ECRel{} nach
  Definition auch in \EBRel{} enthalten sein soll, folgt aus $U\|P_1\ECRel{}
  U\|P_2$ auch die Gültigkeit von $U\|P_1\EBRel{}U\|P_2$ für alle diese
  \MEIO{}s $U$. Mit Lemma~\ref{KommVerfeinLem} folgt dann $P_1\ERel{} P_2$.
\end{proof}

Es wurde somit eine Kette an Folgerungen gezeigt, die sich zu einem Ring
schließt. Dies ist in Abbildung~\ref{KommFolgerungskette} dargestellt.

\begin{figure}[h!tbp]
  \begin{center}
    \begin{tikzpicture}[scale = 3]
      \matrix (m) [matrix of math nodes,row sep=2cm,column sep=4cm]{%
        P_1\ERel P_2 & P_1\ECRel P_2 \\
        \substack{\forall~\mathrm{Partner}~U:\\U\|P_1\EBRel U\|P_2} &
    \substack{\forall~\mathrm{komponierbaren}~U:\\U\|P_1\EBRel U\|P_2} \\};
        \draw[-implies, double, double distance=1mm]
          (m-1-1) -- node [above] {\glqq{}$\Leftarrow$\grqq{} von
            Satz~\ref{KommVollAbstraktSatz}} (m-1-2);
        \draw[-implies, double, double distance=1mm]
          (m-1-2) -- node [right] {Definition von \ECRel{}
          in~\ref{EBRelDef}} (m-2-2);
        \draw[-implies, double, double distance=1mm]
          (m-2-1) -- node [left]
          {Lemma~\ref{KommVerfeinLem}} (m-1-1);
        \draw[-implies, double, double distance=1mm]
          (m-2-2) -- node [below]
          {$\substack{U~\mathrm{Partner}\\\Downarrow\\
          U~\mathrm{komponierbar}}$} (m-2-1);
    \end{tikzpicture}
    \caption{Folgerungskette der Fehler-Relationen}
  \label{KommFolgerungskette}
  \end{center}
\end{figure}

Angenommen man definiert, dass $P_1$ $P_2$ verfeinern soll, genau dann wenn für
alle Partner \MEIO{}s $U$, für die $P_2$ fehler-frei mit $U$ kommuniziert,
folgt, dass $P_1$ ebenfalls fehler-frei mit $U$ kommuniziert. Dann wird auch
diese Verfeinerung durch \ERel{} charakterisiert.

\begin{Kor}
  Es gilt: $P_1\ERel P_2\Leftrightarrow U\|P_1\EBRel U\|P_2$ für alle Partner
  $U$.
\end{Kor}

\section{Testing-Ansatz}

Der Testing-Ansatz stützt sich auf den Ansatz, der in~\cite{Vogler2015FailSem}
angewendet wurde. Jedoch sind Tests dort Tupel aus einer Implementierung und
einer Menge an Aktionen, über denen synchronisiert werden soll. Die \MEIO{}s,
die hier komponiert werden bringen die Menge $\Synch$ an Aktionen, die
synchronisiert werden bereits mit. Es wird im Gegensatz
zu~\cite{Vogler2015FailSem} mit Inputs und Outputs gearbeitet und dadurch
scheint der Ansatz die gemeinsamen Aktionen zu synchronisieren natürlicher, wie
eine Menge beim Test vorzugeben.\\
Die Definition von lokaler Erreichbarkeit eines Fehler-Zustandes soll aus dem
Erweiterungs"=Ansatz übernommen werden.

\begin{Def}[Test und Verfeinerung für Kommunikationsfehler]
  \label{KommTestDef}
  Ein \emph{Test} $T$ ist eine Implementierung. Ein \MEIO{} $P$
  \emph{as-erfüllt} einen Kommunikationsfehler-Test $T$, falls $S\|T$
  fehler-frei ist für alle $S\in \asimp (P)$. Es wird dann $P \EsatAs T$
  geschrieben. Die Parallelkomposition $S\|T$ ist \emph{fehler-frei}, wenn kein
  Fehler lokal erreichbar ist.\\
  Ein \MEIO{} $P$ \emph{Fehler-verfeinert} $P'$, falls für alle Tests $T$:
  $P'\EsatAs T \Rightarrow P\EsatAs T$.
\end{Def}

Abgesehen von Korollar~\ref{lokalFehlerErrKor}~(ii) erweisen sich
Definition~\ref{KommTracesDef} bis Korollar~\ref{KommPraekonKor} auch hier als
nützlich.\\
Die Basisrelation aus dem Erweiterungs-Ansatz gibt es in diesem Ansatz nicht,
somit kann das Lemma~\ref{KommVerfeinLem} nicht in dieser Art formuliert
werden. Jedoch kann hier jetzt mit Tests gearbeitet werden.

\begin{Lem}[Testing-Verfeinerung mit Kommunikationsfehlern]
  \label{KommTestVerfeinLem}
  Gegeben sind zwei \MEIO{}s $P_1$ und $P_2$ mit der gleichen Signatur. Wenn
  für alle Tests $T$, die Partner von $P_1$ bzw. $P_2$ sind, $P_2\EsatAs T
  \Rightarrow P_1\EsatAs T$ gilt, dann folgt daraus die Gültigkeit von $P_1\ERel
  P_2$.
\end{Lem}
\begin{proof}
  Da $P_1$ und $P_2$ die gleichen Signaturen haben wird $I:=I_1=I_2$ und $O:=
  O_1=O_2$ definiert. Für jeden Test Partner $T$ gilt $I_T=O$ und $O_T=I$.\\
  Um $P_1\ERel P_2$ zu zeigen, wird nachgeprüft, ob folgendes gilt:
  \begin{itemize}
    \item $\ET _1\subseteq \ET _2$,
    \item $\EL _1\subseteq \EL _2$.
  \end{itemize}
  Für ein gewähltes präfix-minimales Element $w\in\ET _1$ wird gezeigt, dass
  dies $w$ oder eines seiner Präfixe in $\ET _2$ enthalten ist. Dies ist
  möglich, da die beiden Mengen $\ET _1$ und $\ET _2$ durch \cont{}
  abgeschlossen sind.\\
  Mit Proposition~\ref{KommSemProp} folgt aus $w\in\ET _1$, dass es auch eine
  as"=Implementierung $P'_1$ von $P_1$ geben muss, für die $w$ ebenfalls in
  $\ET _{P'_1}$ enthalten ist.
  \begin{itemize}
    \item Fall 1 ($w=\varepsilon$): Es ist ein Fehler-Zustand in $P'_1$ lokal
      erreichbar. Für $T$ wird ein Transitionssystem verwendet, das nur aus dem
      Startzustand und einer must"=Schleife für alle Inputs $x\in I_T$ besteht.
      Somit kann $P'_1$ die im Prinzip gleichen Fehler-Zustände lokal
      erreichen wie $P'_1\|T$. $P'_1$ ist in Parallelkomposition mit $T$ nicht
      fehler-frei somit gilt $P_1\EsatAs T$ nicht. Es darf also auch $P_2$ den
      Test $T$ nicht as"=erfüllen, wegen der Implikation $P_2\EsatAs T
      \Rightarrow P_1\EsatAs T$. Damit $P_2$ $T$ nicht as"=erfüllt muss es eine
      as"=Implementierungen $P'_2$ geben, die in Parallelkomposition mit $T$ zu
      einem nicht fehler-freien System führt. Da $T$ ein Partner von $P'_2$
      ist, gibt es in der Parallelkomposition nur lokale Aktionen. Somit muss
      in $P'_2\|T$ ein Fehler lokal erreichbar sein. Durch die Definition von
      $T$ kann dieser Fehler nur von $P'_2$ geerbt sein. In $P'_2$ kann dieser
      Fehler-Zustand nur durch interen Aktionen und Outputs erreichbar sein, da
      $T$ keine Outputs besitzt, die man mit Inputs aus $P'_2$ synchronisieren
      könnte und unsynchronisierte Aktionen sind in einer Parallelkomposition
      von Partner nicht möglich. Somit gilt $\varepsilon\in \PrET
      _{P'_2}\subseteq \ET _{P'_2}$. Mit Proposition~\ref{KommSemProp} folgt
      daraus $\varepsilon\in \ET _2$.
    \item Fall 2 ($w=x_1\dots x_n x_{n+1}\in\Sigma ^+$ mit $n\geq 0$ und
      $x_{n+1}\in I=O_T$): Es wird der folgende Partner $T$ betrachtet (dieser
      entspricht bis auf die Benennung der Mengen dem Transitionssystem $U$
      aus Abbildung~\ref{UohneE}):
      \begin{itemize}
        \item $T=\{p_0,p_1,\dots ,p_{n+1}\}$,
        \item $p_{0T}=p_0$,
        \item $\begin{aligned}[t]
            \may _T = \must _T&=\{(p_j,x_{j+1},p_{j+1})\mid  0\leq j\leq n\}\\
            &\cup\{(p_j,x,p_{n+1})\mid  x\in I_T\backslash\{x_{j+1}\}, 0\leq
            j\leq n\}\\
            &\cup\{(p_{n+1},x,p_{n+1})\mid  x\in I_T\},
        \end{aligned}$
        \item $E_T=\emptyset$.
      \end{itemize}
      Für $w$ können nun zwei Fälle unterschieden werden, für die beide
      $\varepsilon\in\PrET (P'_1\|T)$ folgen wird.
      \begin{itemize}
        \item Fall 2a) ($w\in\MIT _{P'_1}$): In $P'_1\|T$ erhält man
          $(p'_{01},p_0) \lweakmay[x_1\dots x_n]_{P'_1\|T} (p',p_n)$ mit $p'
          \nmust[x_{n+1}]_{P'_1}$ und $p_n\must[x_{n+1}]_T$. Deshalb gilt
          $(p',p_n)\in E_{P'_1\|T}$. Da alle Aktionen aus $w$ bis auf $x_{n+1}$
          synchronisiert werden und $I\cap I_T=\emptyset$, gilt $x_1,\dots ,
          x_n\in O_{P'_1\|T}$. Es gilt also $\varepsilon\in \PrET (P'_1\|T)$.
        \item Fall 2b) ($w\in\PrET _{P'_1}$): In der Parallelkomposition
          $P'_1\|T$ erhält man die Transitionsfolge $(p_{01},p_0)
          \weakmay[w]_{P'_1\|T} (p'',p_{n+1}) \weakmay[u]_{P'_1\|T}
          (p',p_{n+1})$ für $u\in O^*$ und $p'\in E_{P'_1}$. Daraus folgt
          $(p',p_{n+1})\in E_{P'_1\|T}$ und somit $wu\in\StET (P'_1\|T)$. Da
          alle Aktionen in $w$ synchronisiert werden und $I\cap I_T =
          \emptyset$, gilt $x_1,\dots , x_n, x_{n+1}\in O_{P'_1\|T}$ und, da
          $u\in O^*$, folgt $u\in O^*_{P'_1\|T}$. Somit ergibt sich
          $\varepsilon \in \PrET (P'_1\|T)$.
      \end{itemize}
      Da $\varepsilon\in\PrET (P'_1\|T)$ gilt, kann durch $P_2\EsatAs T
      \Rightarrow P_1\EsatAs T$ geschlossen werden, dass auch $P_2\EsatAs T$
      nicht gelten kann. Die Relation gilt für $P_2$ nicht, da es eine
      as"=Implementierung $P'_2$ von $P_2$ gibt, so dass $P'_2\|T$ nicht
      fehler-frei ist. Da es nur lokale Aktionen in $P'_2\|T$ gibt, muss der
      Fehler-Zustand lokal erreichbar sein.\\
      Der Fehler kann geerbt oder neu sein.
      \begin{itemize}
        \item Fall 2i) (neuer Fehler): Da jeder Zustand von $T$ alle Inputs
          $x\in O=I_T$ zulässt, muss ein lokal erreichbarer Fehler-Zustand der
          Form sein, dass ein Outputs $a\in O_T$ von $T$ möglich ist, der nicht
          mit einem passenden Input aus $P'_2$ synchronisiert werden werden
          kann. Durch die Konstruktion von $T$ sind in $p_{n+1}$ keine Outputs
          möglich. Ein neuer Fehler muss also die Form $(p',p_i)$ haben mit
          $i\leq n$, $p'\nmust[x_{i+1}]_{P'_2}$ und $x_{i+1}\in O_T=I$. Durch
          Projektion erhält man dann $p_{02} \lweakmay[x_1\dots x_i]_{P'_2} p'
          \nmust[x_{i+1}]_{P'_2}$ und damit gilt $x_1\dots x_{i+1}\in\MIT
          _{P'_2} \subseteq \ET _{P'_2}$. Es ist also ein Präfix von $w$ in
          $\ET _{P'_2}$ enthalten und mit Proposition~\ref{KommSemProp} auch in
          $\ET _2$.
        \item Fall 2ii) (geerbter Fehler): $T$ hat $x_1\dots x_i u$ mit $u\in
          I^*_T =O^*$ ausgeführt und ebenso hat $P'_2$ dieses Wort
          abgearbeitet. Durch dies hat $P'_2$ einen Zustand in $E _{P'_2}$
          erreicht, da von $T$ keine Fehler geerbt werden können. Es gilt dann
          $\prune (x_1\dots x_iu) = \prune (x_1\dots x_i)\in\PrET _{P'_2}
          \subseteq \ET _{P'_2}$. Da $x_1\dots x_i$ ein Präfix von $w$ ist,
          führt in diesem Fall eine Verlängerung um lokale Aktionen von einem
          Präfix von $w$ zu einem Fehler-Zustand. Da \ET{} der Menge aller
          Verlängerungen von gekürzten Fehler-Traces entspricht, ist $x_1\dots
          x_i$ in $\ET _{P'_2}$ enthalten und somit ist mit
          Proposition~\ref{KommSemProp} ein Präfix von $w$ in $\ET _2$
          enthalten.
      \end{itemize}
  \end{itemize}
  Um die andere Inklusion zu beweisen, reich es aufgrund der ersten Inklusion
  und der Definition von \EL{} aus zu zeigen, dass $L_1\backslash \ET
  _1\subseteq \EL _2$ gilt.\\
  Es wird dafür ein beliebiges $w\in L_1\backslash \ET _1$ gewählt und gezeigt,
  dass es in $\EL _2$ enthalten ist. Das $w$ ist wegen der
  Propositionen~\ref{LImpProp} und~\ref{KommSemProp} auch für eine
  as"=Implementierung $P'_1$ von $P_1$ in $L_{P'_1}\backslash \ET _{P'_1}$
  enthalten.
  \begin{itemize}
    \item Fall 1 ($w=\varepsilon$): Da $\varepsilon$ immer in $\EL _2$ muss
      hier nichts gezeigt werden.
    \item Fall 2 ($w=x_1\dots x_n$ mit $n\geq 1$): Es wird ein Partner $T$ wie
      folgt konstruiert ($T$ entspricht dabei $U$ aus Abbildung~\ref{UmitE} bis
      auf die Benennung der Mengen):
      \begin{itemize}
        \item $T=\{p_0,p_1,\dots ,p_n,p\}$,
        \item $p_{0T}=p_0$,
        \item $\begin{aligned}[t]
            \may _T = \must _T&=\{(p_j,x_{j+1},p_{j+1})\mid 0\leq j< n\}\\
            &\cup\{(p_j,x,p)\mid x\in I_T\backslash\{x_{j+1}\},0\leq j< n\}\\
            &\cup\{(p,x,p)\mid x\in I_T\},
        \end{aligned}$
        \item $E_T=\{p_n\}$.
      \end{itemize}
      Da $p_{01} \weakmay[w]_{P'_1} p'$ gilt, kann man schließen, dass
      $P'_1\|T$ ein lokal erreichbaren geerbten Fehler hat. Es muss also auch
      eine as"=Implementierung $P'_2$ von $P_2$ geben, für die $P'_2\|T$ einen
      lokal erreichbaren Fehler-Zustand hat.
      \begin{itemize}
        \item Fall 2a) (neuer Fehler aufgrund von $x_i\in O_T$ und $p_{02}
          \lweakmay[x_1\dots x_{i-1}]_{P'_2}p''\nmust[x_i]_{P'_2}$): Es gilt
          $x_1\dots x_i\in\MIT _{P'_2}$ und somit $w\in\EL _{P'_2}$. Anzumerken
          ist, dass es nur auf diesem Weg Outputs von $T$ möglich sind, deshalb
          gibt es keine anderen Outputs von $T$, die zu einem neuen Fehler
          führen können. Es gilt $w\in\EL _2$ wegen
          Proposition~\ref{KommSemProp}.
        \item Fall 2b) (neuer Fehler aufgrund von $a\in O=I_T$): Der einzige
          Zustand, in dem $T$ nicht alle Inputs erlaubt sind, ist $p_n$, der
          bereits ein Fehler-Zustand ist. Da dieser Zustand in $P'_2\|T$
          erreichbar ist, besitzt der komponierte \MEIO{} einen geerbten Fehler
          und es gilt $w\in L_{P'_2}\subseteq \EL _2$, wegen dem folgenden Fall
          2c).
        \item Fall 2c) (geerbter Fehler von $T$): Da $p_n$ der einzige
          Fehler-Zustand in $T$ ist und alle Aktionen synchronisiert sind, ist
          dies nur möglich, wenn $p_{02} \lweakmay[x_1\dots x_n]_{P'_2}$ gilt.
          In diesem Fall gilt $w\in L_{P'_2}\subseteq \EL _{P'_2}$. Daraus
          folgt mit Proposition~\ref{KommSemProp} $w\in\EL _2$.
        \item Fall 2d) (geerbter Fehler von $P'_2$): Es gilt dann $p_{02}
          \lweakmay[x_1\dots x_n]_{P'_2} p'\in E_{P'_2}$ für $i\geq 0$ und
          $u\in O^*$. Somit ist $x_1\dots x_i u\in\StET _{P'_2}$ und damit
          $\prune (x_1\dots x_iu) =\prune (x_1\dots x_i)\in\PrET
          _{P'_2}\subseteq \EL _{P'_2}$. Somit gilt mit Hilfe von
          Proposition~\ref{KommSemProp} $w\in\EL _{P'_2}\subseteq \EL _2$.
      \end{itemize}
  \end{itemize}
\end{proof}

\begin{Satz}
  \label{KommTestVerfSatz}
  Falls $P_1\ERel P_2$ gilt folgt draus auch, dass $P_1$ $P_2$
  Fehler-verfeinert.
\end{Satz}
\begin{proof}
  Nach Definition gilt, genau dann wenn $\varepsilon\in \ET (P)$, ist ein
  Fehler-Zustand lokal erreichbar in $P$. $P_1\ERel P_2$ impliziert, dass
  $\varepsilon\in\ET _2$ gilt, wenn $\varepsilon\in\ET _1$. Somit ist ein
  Fehler-Zustand in $P_1$ nur dann lokal erreichbar, wenn dieser auch in $P_2$
  lokal erreichbar ist. Aufgrund von Proposition~\ref{lokalFehlerErrKor}~(i)
  gibt es dann auch entsprechende as"=Implementierungen $P'_1$ und $P'_2$ für
  $P_1$ bzw. $P_2$, die ebenfalls Fehler-Zustände lokal erreichen können, wenn
  dies in $P_1$ bzw. $P_2$ möglich ist. Für alle Tests $T$ gilt dann auch, dass
  $P'_j\|T$ für $j\in\{1,2\}$ einen lokal erreichbaren Fehler hat, wenn $P'_j$
  einen solchen hat. Dies Schlussfolgerung ist möglich, da
  Satz~\ref{KommFehlerSemSatz}~1.\ aussagt, dass die Parallelkomposition von
  einem Fehler-Trace aus $\ET _{P'_j}$ mit einem Wort aus $\EL _T$ in den
  Fehler-Traces von $\ET _{P'_j\|T}$ enthalten ist. Es gilt $\varepsilon\in \ET
  _{P'_j}$, da ein Fehler lokal erreichbar ist und $\varepsilon\in\EL _T$ gilt
  für alle Test $T$ und somit folgt $\varepsilon\in \ET _{P'_j\|T}$. Dies
  impliziert die lokal Fehler Erreichbarkeit in $P'_j\|T$. Falls also $P_1$
  einen lokal erreichbaren Fehler-Zustand enthält, dann zeigt sich dies
  Verhalten auch in einer as"=Implementierungen und auch in der
  Parallelkomposition der as"=Implementierung mit allen Tests $T$. Aus einem
  lokal erreichbaren Fehler in $P_1$ folgt auch ein lokal erreichbarer
  Fehler-Zustand in $P_2$ und somit auch ein lokal erreichbarer Fehler-Zustand
  in der Parallelkomposition einer as"=Implementierung von $P_2$ mit allen
  Tests $T$. Aus $P_1\ERel P_2$ folgt also die Implikation $\neg P_1 \EsatAs T
  \Rightarrow \neg P_2 \EsatAs T$ für alle Test $T$. Wenn man die Negationen
  entfernt, ergibt sich für alle Tests $T$: $P_2\EsatAs T\Rightarrow P_1\EsatAs
  T$. Dies entspricht der Definition von $P_1$ Fehler-verfeinert $P_2$.
\end{proof}

Auch die in diesem Abschnitt gezeigten Folgerungen schließen sich zu einem
Ring. Dies ist in Abbildung~\ref{KommTestFolgerungskette} dargestellt.

\begin{figure}[h!tbp]
  \begin{center}
    \begin{tikzpicture}[scale = 3]
      \matrix (m) [matrix of math nodes,row sep=2cm,column sep=4cm]{%
        P_1\ERel P_2 & P_1 \text{ Fehler-verfeinert } P_2 \\
        \substack{\forall \text{ Test Partner } T:\\P_2\EsatAs T\Rightarrow
        P_1\EsatAs T} &
    \substack{\forall \text{ Tests } T:\\P_2\EsatAs T\Rightarrow P_1\EsatAs T} \\};
        \draw[-implies, double, double distance=1mm]
          (m-1-1) -- node [above] {Satz~\ref{KommTestVerfSatz}} (m-1-2);
        \draw[-implies, double, double distance=1mm]
          (m-1-2) -- node [right] {Definition~\ref{KommTestDef}} (m-2-2);
        \draw[-implies, double, double distance=1mm]
          (m-2-1) -- node [left]
          {Lemma~\ref{KommTestVerfeinLem}} (m-1-1);
        \draw[-implies, double, double distance=1mm]
          (m-2-2) -- node [below]
          {$\substack{T \text{ Test Partner}\\\Downarrow\\ T \text{ Test}}$} (m-2-1);
    \end{tikzpicture}
    \caption{Folgerungskette der Testing-Verfeinerung und Fehler-Relation}
  \label{KommTestFolgerungskette}
  \end{center}
\end{figure}

\TODO{evtl. $\EBRel$-Normalisierung, nur ein $e$ mit may-Schleife für alle
$i$'s und $o$'s}

\section{Zusammenhänge der Relationen}

\begin{Satz}[Zusammenhand der Verfeinerungs-Relationen mit der Fehler-Relation]
  Es gilt $\asRel \Rightarrow \wasRel \Rightarrow \ERel$. Die Implikationen in
  die andere Richtung gelten jedoch nicht.
\end{Satz}
\begin{proof}\mbox{}\\
  $\asRel \Rightarrow \wasRel$:\\
  Um diese Implikation zu zeigen, muss man nachweisen, dass jede starke
  as"=Verfeinerungs"=Relation auch die Definition~\ref{wSimDef} der schwachen
  as"=Verfeinerungs"=Relation erfüllt. In beiden Simulations-Definitionen
  (\ref{SimDef} und~\ref{wSimDef}) müssen die Punkte für alle $(p,q) \in
  \mathcal{R}$ mit $q\notin E_Q$ gelten. Sei $\mathcal{R}$ nur eine
  as"=Verfeinerungs"=Relation. Es gilt also mit~\ref{SimDef}~1.\, dass $p$ kein
  Fehler-Zustand von $P$ ist. Somit ist auch bereits 1.\ von~\ref{wSimDef}
  erfüllt. Für alle $\alpha\in\Sigma _{\tau}$ impliziert $q\must[\alpha]_Q q'$
  $p\must[\alpha]_P p'$ für ein $p'$ mit $p'\mathcal{R} q'$. Da $\Sigma
  _{\tau} = I \cup O \cup \{\tau\}$ gilt, sind dadurch 2.\ und 3.\ der
  Definition~\ref{wSimDef} erfüllt. Die schwache $\varepsilon$-Transition aus
  2.\ führt keine echten Transitionen aus, sondern bleibt beim Zustand $p'$ und
  Die schwache $\hat{\omega}$-Transition aus 3.\ entspricht in $\mathcal{R}$
  nur einer einzigen Transition für $\omega$. Die Punkte 4.\ und 5.\ aus
  Definition~\ref{wSimDef} werden durch~\ref{SimDef}~3.\ erfüllt. Es gilt für
  $\mathcal{R}$ die Implikation $p\may[\alpha]_P p'$ impliziert $q
  \may[\alpha]_Q q'$ für ein $q'$ mit $p'\mathcal{R}q'$. Die in~\ref{wSimDef}
  geforderten schwachen may"=Transitionen werden hier jeweils stark umgesetzt
  durch eine einzige Transition. $\mathcal{R}$ ist also auch eine schwache
  as"=Verfeinerungs"=Relation.

  $\wasRel \Rightarrow \ERel$:\\
  Um diese Implikation zu beweisen wird gezeigt, dass für eine beliebige
  schwache as"=Verfeinerungs"=Relation $\mathcal{R}$ zwischen \MEIO{}s $P$ und
  $Q$ auch die Eigenschaften für $\ERel$ erfüllt sind. Es ist also
  nachzuweisen, dass $\ET _P \subseteq \ET _Q$ und $\EL _P \subseteq \EL _Q$
  gilt, wenn $\mathcal{R}$ als alternierenden Simulation zwischen $P$ und $Q$
  vorausgesetzt ist.\\
  Für den ersten Punkt wird ein beliebiges $w$ aus $\ET _P$ betrachtet und
  gezeigt, dass dieses auch in $\ET _Q$ enthalten ist. Es kann davon
  ausgegangen werden, dass $w$ präfix-minimal ist, da beide \ET{}-Mengen unter
  \cont{} abgeschlossen sind. $w$ kann ein Element aus $\PrET (P)$ sein oder
  ein Element aus $\MIT (P)\backslash \PrET (P)$.
  \begin{itemize}
    \item Fall 1 ($w\in\PrET (P)$): Es existiert ein $v\in O_P$, sodass das
      Wort $wv$ in $P$ einen Fehler-Zustand erreicht. Für die entsprechende
      Transitionsfolge existieren Zustände $p_1, p_2,\dots , p_n$ in $P$ und
      Aktionen $\alpha _1, \alpha _2,\dots , \alpha _n$, die zusammengesetzt
      ein Wort $w'=\alpha _1 \alpha _2 \dots \alpha _n$ bilden, dass ohne die
      internen Aktionen $wv$ entspricht. Die entsprechende Transitionsfolge ist
      dann $p_0 \may[\alpha _1]_P p_1 \may[\alpha _2]_P \dots p_{n-1}
      \may[\alpha _n]_P p_n \in E_P$. Da $w$ präfix-minimal ist, kann $v$ so
      gewählt werden, dass auf den Trace von $w'$ alle Zustände außer $p_n$
      nicht in $E_P$ enthalten sind. Da $\mathcal{R}$ eine
      as"=Verfeinerungs"=Relation zwischen $P$ und $Q$ ist, muss
      $p_0\mathcal{R}q_0$ gelten. Abhängig davon, ob $\alpha _j$ ein Input oder
      eine lokale Aktion ist, kann mit \ref{wSimDef}~4.\ bzw.~5.\ argumentiert
      werden, dass das jeweilige $\alpha _j$ auch in $Q$ schwach ausführbar
      ist, solange der entsprechende Zustand $q_{j-1}$ kein Fehler-Zustand ist
      für $j\in \{1,2,\dots n\}$. Falls ein $q_j$ für $j < n$ in $E_Q$
      enthalten ist, wurde bis dort ein Präfix von $wv$ ausgeführt. Es gilt
      also mit $w=\prune (wv)$ und dem Abschluss unter \cont{} von \ET{}
      $w\in\ET _Q$. Ansonsten gibt es in $Q$ einen Trace $q_0 \weakmay[\alpha
      _1]_Q q_1 \weakmay[\alpha _2]_Q \dots q_{n-1} \weakmay[\alpha _n]_Q q_n$,
      wobei $p_j \mathcal{R} q_j$ für alle $0 \leq j \leq n$ gilt.
      Mit~\ref{wSimDef}~1.\ folgt, dass $q_n\in E_Q$ gelten muss und somit auch
      $w\in\ET _Q$ mit der Begründung von oben.
    \item Fall 2 ($w\in\MIT (P)\backslash \PrET (P)$): $w$ ist in $P$ ein
      Input-kritischer Trace, der jedoch nicht zu einem Fehler-Zustand führen
      kann. Man kann $w$ also aufteilen in $va$ mit $v\in \Sigma ^*$ und $a\in
      I$. Der Trace in $P$ kann wie folgt dargestellt werden: $\exists v'\in
      \Sigma _{\tau}^*, \exists p_1, p_2, \dots , p_n, \exists \alpha _1,
      \alpha _2, \dots , \alpha _n: \hat{v'}=v \land v'=\alpha _1 \alpha _2
      \dots \alpha _n \land p_0 \may[\alpha _1]_P p_1 \may[\alpha _2]_P \dots
      p_{n-1} \may[\alpha _n]_P p_n \nmust[a]_P$. Basierend
      auf~\ref{wSimDef}~4.\ bzw.~5.\ kann daraus Schritt für Schritt gefolgert
      werden, dass die $\alpha _j$ auch schwach ausführbar sind in $Q$, falls
      kein $p_{j-1}$ angetroffen wird, dass in $E_Q$ enthalten inst für $j\in
      \{1,2,\dots n\}$. Falls eines der $p_j$ mit $0 \leq j \leq n$ ein
      Fehler-Zustand ist, dann ist $w$ in $\ET _Q$ enthalten, da ein Präfix von
      $w$ ein strikter Fehler-Trace in $Q$ ist. Falls alle $q_j$ nicht in $E_Q$
      enthalten sind, gilt $q_0 \weakmay[\alpha _1]_Q q_1 \weakmay[\alpha _2]_Q
      \dots q_{n-1} \weakmay[\alpha _n]_Q q_n$ in $Q$ mit $p_j \mathcal{R} q_j$
      für alle $j\in \{0,1, \dots , n\}$. Da $a$ für $p_n$ in $P$ keine
      ausgehende must"=Transition sein kann, gilt mit~\ref{wSimDef}~2.\ auch
      $q_n \nmust[a]$. Es gilt dann also $w\in\MIT _Q \subseteq \ET _Q$.
  \end{itemize}
  Für den zweiten Punkt kann man sich auf die Inklusion $\EL _P\backslash\ET _P
  \subseteq \EL _Q$ einschränken, da der erste Punkt bereits vorausgesetzt
  werden kann. $\EL _P\backslash\ET _P$ entspricht $L _P\backslash\ET _P$.
  Somit ist ein $w\in L_P\backslash \ET _P$ in $P$ ausführbar. Es gibt also
  einen Trace $p_0 \may[\alpha _1]_P p_1 \may[\alpha _2]_P \dots p_{n-1}
  \may[\alpha _n]_P p_n$ in $P$, wobei $w$ $\alpha _1\alpha _2 \dots \alpha _n$
  entspricht bis auf die internen Aktionen. Erneut können
  hier~\ref{wSimDef}~4.\ und~5.\ dazu verwendet werden einen analogen Trace in
  $Q$ zu finden. Falls ein $q_j$ für $0 \leq j \leq n$ in $E_Q$ enthalten ist,
  gilt $w\in\ET _Q \subseteq \EL _Q$. Es wird also für die folgende Begründung
  davon ausgegangen, dass kein $q_j$ ein Fehler-Zustand ist, damit gilt $q_0
  \weakmay[\alpha _1]_Q q_1 \weakmay[\alpha _2]_Q \dots q_{n-1} \weakmay[\alpha
  _n]_Q q_n$ in $Q$ mit $p_j \mathcal{R} q_j$ für $0 \leq j \leq n$ und somit
  $w\in L _Q\subseteq \EL _Q$.

  $\asRel \hspace{0.1cm}\not\hspace{-0.1cm}\Leftarrow \wasRel$:\\
  Im Abbildung~\ref{asWasGegenBsp} wird ein Gegenbeispiel dargestellt mit einem
  \MEIO{} $Q$ und einer schwachen as"=Verfeinerung $P$ von $Q$, die jedoch
  keine starke as"=Verfeinerung von $Q$ ist. Die starke
  as"=Verfeinerungs"=Relationen $\mathcal{R}$ zwischen $P$ und $Q$ enthält die
  Tupel $(p_0,q_0)$ und $(p_1,q_{12})$. Es sind keine Fehler-Zustände in $Q$
  und $P$ enthalten, somit ist 1.\ der Definition~\ref{wSimDef} bereits
  erfüllt. $q_{11} \must[\tau]_Q q_{12}$ ist die einzige must"=Transition in
  $Q$, da jedoch kein Zustand in $P$ existiert, der mit $q_{11}$ in der
  Relation $\mathcal{R}$ steht, fordert~\ref{wSimDef}~3.\ nichts. Der zweite
  Punkt von~\ref{wSimDef} ist erfüllt, da $Q$ keine must"=Input"=Transitionen
  enthält. Damit $\mathcal{R}$ eine schwache Simulations"=Relation zwischen
  $P$ und $Q$ sein kann, die auch die Verfeinerung sicher stellt, müssen die
  Startzustände in Relation stehen. Dies ist durch das $(p_0,q_0)\in
  \mathcal{R}$ erfüllt. Falls $\alpha$ ein Input ist, fordert
  dann~\ref{wSimDef}~4., dass die Transition $p_0 \may[\alpha]_P p_1$ in $Q$
  schwach ausführbar ist in der Form $q_0 \may[\alpha]_Q
  \weakmay[\varepsilon]_Q q$ als entsprechendes $q$ kommen $q_{11}$ und
  $q_{12}$ in Frage. Da $p_1 \mathcal{R} q_{12}$ gilt, ist die Forderung
  von~\ref{wSimDef}~4.\ erfüllt. Falls $\alpha$ eine lokale Aktion ist, würde
  die Forderung $q_0 \weakmay[\hat{\alpha}]_Q q$ für $Q$ lauten und es würden
  wieder $q_{11}$ und $q_{12}$ als mögliche Zustände für $q$ in Frage kommen,
  im Fall von $\alpha = \tau$ wäre auch $q = q_0$ möglich. Die Forderung ist
  jedoch nicht stärker, wie die von~\ref{wSimDef}~4.\ im Falle eines Inputs,
  somit ist auch~\ref{wSimDef}~5.\ im Falle einer lokalen Aktion erfüllt. Falls
  4.\ wegen eines Inputs erfüllt ist, ist 5.\ auf jeden Fall erfüllt, da dann
  $\alpha \notin O \cup \{\tau\}$ gilt und umgekehrt, ist 4.\ für $\alpha \in O
  \cup \{\tau\}$ auf jeden Fall erfüllt. $\mathcal{R}$ ist also eine schwache
  as"=Verfeinerungs"=Relation zwischen $P$ und $Q$.\\
  Angenommen es gibt auch eine starke as"=Verfeinerungs"=Relation
  $\mathcal{R}'$ zwischen $P$ und $Q$, dann muss $p_0 \mathcal{R}' Q_0$
  gelten. Mit~\ref{SimDef}~3.\ wird gefordert, dass die Transition $p_0
  \may[\alpha]_P p_1$ durch eine Transition der Form $q_0 \may[\alpha]_Q q$
  in $Q$ gematched werden muss. Für den Zustand $q$ kommt dieses mal nur
  $q_{11}$ in Frage. Es muss also $(p_1,q_{11})\in \mathcal{R}$ gelten. Der
  zweite Punkt der Definition~\ref{SimDef} fordert, dass die
  $\tau$-must"=Transition aus $Q$ auch in $P$ auftauchen muss. Es müsste also
  ein $p$ geben, für dass $p_1 \must[\tau]_P p$ gilt und das Tupel $(p,q_{12})$
  müsste in $\mathcal{R}$ enthalten sein. Da es diese Transition nicht gibt,
  tritt ein Widerspruch zur Annahme auf. Es kann also keine starke
  as"=Verfeinerungs"=Relation zwischen $P$ und $Q$ geben.

  \begin{figure}[htbp]
    \begin{center}
      \begin{tikzpicture}[shorten >=1pt,auto,node distance=2.5cm]
        \node [initial,initial text=$Q$:] (q0) at (0,0) {$q_0$};
        \node (q11) [right of=q0] {$q_{11}$};
        \node (q12) [right of=q11] {$q_{12}$};

        \path[->]
        (q0) edge[dashed] node{$\alpha$} (q11)
        (q11) edge node{$\tau$} (q12)
        ;

        \node [initial,initial text=$P$:] (p0) at (10,0) {$p_0$};
        \node (p1) [right of=p0] {$p_1$};

        \path[->]
        (p0) edge[dashed] node{$\alpha$} (p1)
        ;
      \end{tikzpicture}
      \caption{Gegenbeispiel zu $\asRel \Leftarrow \wasRel$}
      \label{asWasGegenBsp}
    \end{center}
  \end{figure}

  $\wasRel \hspace{0.1cm}\not\hspace{-0.1cm}\Leftarrow \ERel$:\\
  Die nicht Gültigkeit dieser Implikation beruht darauf, dass Simulationen
  strenger sind wie Sprach Inklusion. Das Gegenbeispiel hier ist also so
  aufgebaut, dass $\ET (P) =\ET (Q) = \emptyset$ und $L(P) \subseteq L(Q)$
  gilt, jedoch keine schwache as"=Verfeinerungs"=Relation zwischen $P$ und $Q$
  existieren kann. $Q$ und $P$ sind in der Abbildung~\ref{WasEGegenBsp}
  dargestellt. Damit $\ET (P) =\ET (Q) = \emptyset$ gilt, dürfen keine der
  Zustände Fehler-Zustände sein und es muss gefordert werden, dass die Menge
  $I$ der Inputs für die \MEIO{} leer ist, ansonsten würde es Input-kritische
  Traces gegen. $P$ kann keine Aktionen ausführen und $Q$ nur die Output Aktion
  $o$ somit gilt für die Sprachen $\{\varepsilon\} = L(P) \subset L(Q) =
  \{\varepsilon , o\}$.\\
  Angenommen es gibt eine schwache as"=Verfeinerungs"=Relation $\mathcal{R}$
  zwischen $P$ und $Q$. Dafür muss $(p_0,q_0)\in \mathcal{R}$ gelten. Die
  Punkte 1., 2., 4.\ und 5.\ stellen keine Forderungen an die Relation
  $\mathcal{R}$. Da jedoch die Transition $q_0 \must[o]_Q q_1$ in $Q$ vorhanden
  ist, wird die Verfeinerung dieser in $P$ gefordert es müsste also auch eine
  must"=Output"=Transition geben. Da diese nicht vorhanden ist, stellt dies
  einen Widerspruch zur Annahme her und deshalb folgt, dass es keine schwache
  as"=Verfeinerungs"=Relation zwischen $P$ und $Q$ geben kann.

  \begin{figure}[htbp]
    \begin{center}
      \begin{tikzpicture}[shorten >=1pt,auto,node distance=2.5cm]
        \node [initial,initial text=$Q$:] (q0) at (0,0) {$q_0$};
        \node (q1) [right of=q0] {$q_1$};

        \path[->]
        (q0) edge node{$o!$} (q1)
        ;

        \node [initial,initial text=$P$:] (p0) at (7,0) {$p_0$};
      \end{tikzpicture}
      \caption{Gegenbeispiel zu $\wasRel \Leftarrow \ERel$}
      \label{WasEGegenBsp}
    \end{center}
  \end{figure}
\end{proof}
