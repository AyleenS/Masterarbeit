\chapter{Verfeinerungen für Kommunikationsfehler-Freiheit}

Dieses Kapitel versucht die Präkongruenz für Error bei \EIO{}s
aus~\cite{Schinko2016BA} auf die hier betrachten \MEIO{}s zu erweitern.

\begin{Def}[fehler-freie Kommunikation]
  Ein Fehler"=Zustand ist \emph{lokal erreichbar} in einer
  as"=Implementierung $P'$ eines \MEIO{} $P$, wenn ein $w\in O^*$ existiert mit
  $p'_0 \weakmust[w]p'\in E'$.\\
  Zwei \MEIO{}s $P_1$ und $P_2$ \emph{kommunizieren fehler-frei}, wenn keine
  as"=Implementierungen ihrer Parallelkomposition $P_{12}$ einen
  Fehler"=Zustände lokal erreichen kann.
\end{Def}

\begin{Def}[Kommunikationsfehler-Verfeinerungs-Basirelation]
  Für zwei \MEIO{}s $P_1$ und $P_2$ mit der gleichen Signatur wird $P_1\EBRel{} P_2$
  geschrieben, wenn ein Fehler"=Zustand in einer as"=Implementierung
  von $P_1$ nur dann lokal erreichbar ist, wenn es auch eine as"=Implementierung
  von $P_2$ gibt, in der dieser Fehler"=Zustand auch lokal
  erreichbar ist. Die Basisrelation stellt eine \emph{Verfeinerung} bezüglich
  \emph{Kommunikationsfehlern} dar.\\
  \ECRel{} bezeichnet die \emph{vollständig abstrakte Präkongruenz} von
  \EBRel{} bezüglich $\cdot\|\cdot$, d.h.\ die gröbste Präkongruenz bezüglich
  $\cdot\|\cdot$, die in \EBRel{} enthalten ist.
\end{Def}

Für as"=Implementierungen $P_1$ und $P_2$ entspricht \EBRel{} der Relation
\EBbaRel{} aus~\cite{Schinko2016BA}.

Wie in~\cite{Schinko2016BA} werden die Fehler hier Trace basiert betrachtet. Da
jedoch die Basisrelation über as"=Implementierungen spricht, sind die Trace Mengen
auch nicht für die \MEIO{}s mit may-Transitionen definiert sondern nur für die
Menge der möglichen as"=Implementierungen eines solchen \MEIO{}s.

\begin{Def}[Kommunikationsfehler-Traces]
  Für ein \MEIO{} $P$ wird definiert:
  \begin{itemize}
    \item \emph{strikte Kommunikationsfehler-Traces}: $\StET (P) :=
      \left\{w\in\Sigma ^* \mid p_0 \weakmay[w] p \in E\right\}$,
    \item \emph{gekürzte Kommunikationsfehler-Traces}: $\PrET (P) :=
      \left\{\prune (w) \mid w\in\StET (P)\right\}$,
    \item \emph{Input-kritische-Traces}: $\MIT (P) := \left\{wa\in\Sigma ^*
      \mid p_0 \weakmay[w] p \land a\in I\land p\nmust[a] \right\}$.
  \end{itemize}
\end{Def}

\begin{Prop}[Kommunikationsfehler-Traces und Implementierung]
  Für ein \MEIO{} $P$ gilt:
  \begin{enumerate}
    \item \emph{strikte Kommunikationsfehler-Traces}: $\StET (P) =
      \left\{w\in\Sigma ^* \mid \exists P' \in\asimp (P) :
      \phantom{\weakmust[w]} \right.$ $\left. p'_0 \weakmust[w] p' \in
      E\right\}$ \TODO{erzwungen Zeilenumbruch kontrollieren}
    \item \emph{Input-kritische-Traces}: $\MIT (P) = \left\{wa\in\Sigma ^* \mid
      \exists P' \in\asimp (P) : p'_0 \weakmust[w] p' \land \right.$\linebreak
      $\left. a\in I\land p'\nmust[a] \right\}$ \TODO{erzwungen Zeilenumbruch kontrollieren}
  \end{enumerate}
\end{Prop}
\begin{proof}\mbox{}
  \begin{enumerate}
    \item Wie schon in Beweis zu~\ref{LImpProp} festgestellt, sind alle
      Abläufe, die in $P$ via may"=Transitionen möglich ist in mindestens einer
      as"=Implementierung von $P$ via must"=Transitionen möglich. Umgekehrt ist
      auch jeder Ablauf, der in einer as"=Implementierung von $P$ möglich ist
      auch in $P$ durch may"=Transitionen möglich.\\
      Aufgrund des 3.\ Punktes der Definition~\ref{SimDef} kann jede
      as"=Implementierung von $P$ nur Fehler-Zustände enthalten,
      die auch $P$ enthält. Da alle möglichen Implementierungen von $P$ in
      $\left\{w\in\Sigma ^* \mid \exists P' \in\asimp (P) : p'_0 \weakmust[w]
      p' \in E\right\}$ betrachtet werden, ist auch jeder in $P$ durch
      may"=Transitionen erreichbare Fehler"=Zustand auch in
      mindestens einer as"=Implementierung ebenfalls erreichbar, jedoch durch
      must"=Transitionen.
    \item Für jedes $w$ in $L(P)$ gibt es mindestens eine as"=Implementierung
      von $P$, die dieses $w$ auch ausführen kann und umgekehrt (Beweis
      von~\ref{LImpProp}). Falls in $\MIT (P)$ mindestens ein Element gibt,
      gibt es in $P$ einen Trace von $w$, nach dem ein Input $a$ nicht
      zwingendermaßen folgen muss. Die $a$ Transition also entweder eine
      may"=Transition ist oder gar nicht existiert in $P$. Es muss also auch
      einen $w$ Trace in einer as"=Implementierung geben, die in einem Zustand
      endet, der mit dem Zustand aus $P$ in Relation steht, in dem das $a$
      nicht erzwungen wird. Falls die Transition in $P$ nicht vorhanden ist,
      muss jede as"=Implementierung, die so einen $w$ Trace enthält auch $wa$
      als Input-kritischen-Trace haben. Andernfalls handelt es sich bei $a$ in
      $P$ um eine may"=Transition, die von mindestens einer
      as"=Implementierung, die den $w$ Trace enthält, nicht implementiert wird
      und somit $wa$ auch als Input-kritischen-Trace enthält.\\
      Für ein $wa\in \left\{wa\in\Sigma ^* \mid \exists P' \in\asimp (P) : p'_0
      \weakmust[w] p' \land a\in I\land p'\nmust[a] \right\}$ muss es eine
      as"=Implementierung von $P$ geben, die diesen Input-kritischen-Trace
      implementiert. $P$ muss also auch das $w$ ausführen können zu einem
      Zustand, in dem $P$ $a$ nicht als must"=Transition enthalten. Falls $P$
      $a$ nach $w$ nur als must"=Transition enthalten würde, würde~\ref{SimDef}
      1.\ die Implementierung von $a$ erzwingen würde und somit könnte für
      keine as"=Implementierung von $P$ $wa$ ein Input-kritischer-Trace sein.
      Es gilt also auch $wa\in\MIT (P)$.
  \end{enumerate}
\end{proof}

\begin{Def}[Kommunikationsfehler-Semantik]
  \label{KommFehlerSemDef}
  Sei $P$ ein \MEIO{}.
  \begin{itemize}
    \item Die Menge der \emph{Kommunikationsfehler-Traces} von $P$ ist $\ET (P)
      := \cont (\PrET (P)) \cup\cont (\MIT (P))$.
    \item Die \emph{Kommunikationsfehler-geflutete Sprache} von $P$ ist $\EL
      (P) := L(P) \cup \ET (P)$.
  \end{itemize}
  Für zwei \MEIO{}s $P_1,P_2$ mit der gleichen Signatur wird $P_1\ERel P_2$
  geschrieben, wenn $\ET _1\subseteq \ET _2$ und $\EL _1 \subseteq \EL _2$
  gilt.
\end{Def}

Hierbei ist zu beachten, dass die Mengen \StET{}, \PrET{}, \MIT{}, \ET{} und
\EL{} nur denen aus~\cite{Schinko2016BA} entsprechen, wenn $P$ bereits eine
as"=Implementierung ist.

\begin{Satz}[Kommunikationsfehler-Semantik für Parallelkompositionen]
  Für zwei komponierbare \MEIO{}s $P_1,P_2$ und ihre Komposition $P_{12}$ gilt:
  \begin{enumerate}
    \item $\ET _{12} = \cont (\prune ((\ET _1\|\EL _2) \cup (\EL _1\|\ET
      _2)))$,
    \item $\EL _{12} = (\EL _1\|\EL _2) \cup \ET _{12}$.
  \end{enumerate}
\end{Satz}
\begin{proof}

  \TODO{neuer Beweis auf Basis der Sprach/Trace-Definitionen ohne
  Implementierungen}

  Da die Trace-Mengen und die Sprache alle nur über möglichen Traces in
  mindestens einer as"=Implementierung der betrachteten \MEIO{}s spricht, muss
  man zu jedem Element auch die passende as"=Implementierung betrachten. Wegen
  Lemma~\ref{impParallelLem} gibt es aber auch immer eine passende
  as"=Implementierung der Parallelkomposition bzw.\ der einzelnen Teile der
  Parallelkomposition.

  % \textbf{TODO: beweisen}

  \TODO{3. Punkt der Simulations-Definition beachten}

  Ansonsten verläuft der Beweis analog zu dem Beweis für
  \EIO{}s aus~\cite{Schinko2016BA}.

  1. \glqq$\subseteq$\grqq{}:\\
  Da beide Seiten der Gleichung unter der Fortsetzung \cont{} abgeschlossen
  sind, genügt es ein präfix-minimales Element $w$ von $\ET _{12}$ zu
  betrachten. Diese Element ist aufgrund der Definition der Menge der
  Errortraces in $\MIT _{12}$ oder in $\PrET _{12}$ enthalten.
  \begin{itemize}
    \item Fall 1 ($w\in\MIT _{12}$): Aus der Definition von \MIT{} folgt, dass
      es eine as"=Implementierung von $P_{12}$ gibt, in der eine Aufteilung
      $w=xa$ existiert mit $(p_{01},p_{02}) \weakmust[x] (p_1,p_2) \land a
      \in I_{12} \land (p_1,p_2) \nmust[a]$. Da $I_{12} = (I_1\cup I_2)
      \backslash (O_1\cup O_2)$ ist, folgt $a\in (I_1\cup I_2)$ und $a\notin
      (O_1\cup O_2)$. Es wird unterschieden, ob $a\in (I_1\cap I_2)$ oder $a\in
      (I_1\cup I_2) \backslash (I_1\cap I_2)$ ist.
    \begin{itemize}
      \item Fall 1a) ($a\in (I_1\cap I_2)$): Der Ablauf der as"=Implementierung
        der Komposition kann und auf as"=Implementierung der einzelnen
        Transitionssysteme projiziert werden (Lemma~\ref{impParallelLem} 2.).
        Man erhält dann \oBdA{} $p_{01}\weakmust[x_1] p_1\nmust[a]$ und
        $p_{02}\weakmust[x_2] p_2\nmust[a]$ oder $p_{02}\weakmust[x_2]
        p_2\must[a]$ mit $x\in x_1\|x_2$. Daraus kann $x_1a\in \cont (\MIT _1)
        \subseteq \ET _1$ und $x_2a\in \EL _2$ ($x_2a\in \MIT _2$ oder $x_2a
        \in L_2$) gefolgert werden. Damit folgt $w\in (x_1\|x_2) \cdot \{a\}
        \subseteq (x_1a)\|(x_2a)\subseteq \ET _1\|\EL _2$, und somit ist $w$ in
        der rechten Seite der Gleichung enthalten.
      \item Fall 1b) ($a\in (I_1\cup I_2)\backslash (I_1\cap I_2)$): \OBdA{}
        gilt $a\in I_1$. In der Projektion auf as"=Implementierungen erhält
        man: $p_{01}\weakmust[x_1] p_1\nmust[a]$ und $p_{02}\weakmust[x_2]
        p_2$ mit $x\in x_1\|x_2$. Daraus folgt $x_1a\in \cont (\MIT
        _1)\subseteq \ET _1$ und $x_2\in L_2\subseteq \EL _2$. Somit gilt $w\in
        (x_1\|x_2) \cdot \{a\} \subseteq (x_1a)\|x_2\subseteq \ET _1\|\EL _2$.
        Dies ist eine Teilmenge der rechten Seite der Gleichung.
    \end{itemize}
  \item \TODO{in diesem Fall as-Implementierung/Implementierung prüfen}

      Fall 2 ($w\in\PrET _{12}$): Aus der Definition von \PrET{} und \prune{}
      folgt, dass es eine as"=Implementierung und ein $v\in O_{12}^*$ gilt, so
      dass $(p_{01},p_{02}) \weakmust{w} (p_1,p_2) \weakmust[v] (p'_1,p'_2)$
      gilt mit $(p'_1,p'_2)\in E_{12}$ und $w=\prune (wv)$. Durch Projektion
      auf entsprechende as"=Implementierung der Komponenten erhält man $p_{01}
      \weakmust[w_1] p_1 \weakmust[v_1] p'_1$ und $p_{02} \weakmust[w_2] p_2
      \weakmust[v_2] p'_2$ mit $w\in w_1\|w_2$ und $v\in v_1\|v_2$. Aus
      $(p'_1,p'_2)\in\ET _{12}$ folgt, dass es sich entweder um einen geerbten
      oder einen neuen Fehler handelt. Bei einem geerbten wäre
      bereits einer der beiden Zustände $p'_1$ bzw.\ $p'_2$ ein
      Fehler"=Zustand gewesen. Ein neuer Kommunikationsfehler
      hingegen wäre durch das fehlen der Synchronisations-Erzwingung (fehlende
      must"=Transition) in der Spezifikation einer der Komponenten entstanden.
      Die Spezifikation würde es also zulassen, dass die Möglichkeit in einer
      ihrer Implementierungen fehlt, eine synchronisierte Aktion auszuführen.
    \begin{itemize}
      \item Fall 2a) (geerbter Fehler): \OBdA{} gilt $p'_1\in
        E_1$. Daraus folgt, $w_1v_1\in StET_1 \subseteq \cont (\PrET _1)
        \subseteq \ET _1$. Da $p_{02} \weakmust{w_2v_2}$ gilt, erhält man
        $w_2v_2\in L_2\subseteq \EL _2$. Dadurch ergibt sich $wv\in \ET _1\|\EL
        _2$ mit $w=\prune (wv)$ und somit ist $w$ in der rechten Seite der
        Gleichung enthalten.
      \item Fall 2b) (neuer Kommunikationsfehler): \OBdA{} gilt $a\in I_1\cap
        O_2$ mit $p'_1\nmust[a]\land p'_2\must[a]$ für zwei
        as"=Implementierungen von $P_1$ und $P_2$, die in $P_{12}$ komponiert
        wurden. Hierbei muss es sich nicht mehr um die passenden
        as"=Implementierungen zu der betrachteten as"=Implementierung der
        Komposition handeln. Da es möglich ist, dass $P_1$ die $a$-Transition
        als may-Transition enthält und die passende Implementierung von $P_1$
        diese implementiert und es somit in der Parallelkomposition der
        Implementierungen gar nicht zu dem neuen Kommunikationsfehler kommen
        muss. Damit es aber in der Parallelkomposition der Spezifikationen zu
        den neuen Kommunikationsfehler kommt, darf die $a$-Transition in $P_1$
        keine must"=Transition sein und somit muss es mindestens eine
        as"=Implementierung von $P_1$ geben, die diese Transition nicht
        implementiert.\\
        Es folgt also $w_1v_1a\in \MIT _1 \subseteq \ET _1$ und $w_2v_2a\in L_2
        \subseteq \EL _2$. Damit ergibt sich $wva\in \ET _1\|\EL_2$, da $a\in
        O_1\subseteq O_{12}$ gilt $w=\prune (wva)$ und somit ist $w$ in der
        rechten Seite der Gleichung enthalten.
    \end{itemize}
  \end{itemize}

  1. \glqq$\supseteq$\grqq{}:\\
  Wegen der Abgeschlossenheit beider Seiten der Gleichung gegenüber \cont{}
  wird auch in diesem Fall nur ein präfix-minimales Element $x\in\prune ((\ET
  _1\|\EL _2)\cup (\EL _1\| \ET _2))$ betrachtet. Da $x$ durch die Anwendung
  der \prune{}-Funktion entstanden ist, existiert ein $y\in O_{12} ^*$ mit
  $xy\in (\ET _1\|\EL _2)\cup (\EL _1\| \ET _2)$. \OBdA{} wird davon
  ausgegangen, dass $xy\in \ET _1\| \EL _2$ gilt, d.h.\ es gibt $w_1\in \ET _1$
  und $w_2\in \EL _2$ mit $xy \in w_1\| w_2$.\\
  Im Folgenden wird für alle Fälle von $xy$ gezeigt, dass es ein $v\in \PrET
  _{12} \cup \MIT _{12}$ gibt, das ein Präfix von $xy$ ist und $v$ entweder auf
  einen Input $I_{12}$ endet oder $v=\varepsilon$. Damit muss $v$ ein Präfix
  von $x$ sein. $\varepsilon$ ist Präfix von jedem Wort und sobald $v$
  mindestens einen Buchstaben enthält, muss das Ende von $v$ vor dem Anfang von
  $y\in O_{12}^*$ liegen. Dadurch ist ein Präfix von $x$ in $\PrET _{12}\cup
  \MIT _{12}$ enthalten und somit gilt $x\in \ET _{12}$, da \ET{} die
  Fortsetzung der Mengenvereinigung aus \PrET{} und \MIT{} ist.\\
  Sei $v_1$ das kürzeste Präfix von $w_1$ in $\PrET _1\cup \MIT _1$. Falls $w_2
  \in L_2$, so sei $v_2=w_2$, sonst soll $v_2$ das kürzeste Präfix von $w_2$ in
  $\PrET _2\cup \MIT _2$ sein. Jede Aktion in $v_1$ und $v_2$ hängt mit einer
  aus $xy$ zusammen. Es kann nun davon ausgegangen werden, dass entweder $v_2 =
  w_2\in L_2$ gilt oder die letzte Aktion von $v_1$ vor oder gleichzeitig mit
  der letzten Aktion von $v_2$ statt findet. Ansonsten endet $v_2 \in \PrET
  _2\cup \MIT _2$ vor $v_1$ und somit ist dieser Fall analog zu $v_1$ endet vor
  $v_2$.
  \begin{itemize}
    \item Fall 1 ($v_1=\varepsilon$): Da $\varepsilon\in\PrET _1\cup \MIT _1$,
      ist bereits in $P_1$ ein Fehler"=Zustand lokal erreichbar
      (möglicherweise auch via may"=Transitionen). $\varepsilon \in \MIT _1$
      ist nicht möglich, da jedes Element aus \MIT{} nach Definition mindestens
      die Länge $1$ haben muss. Mit der Wahl $v'_2=v'=\varepsilon$ ist $v'_2$
      ein Präfix von $v_2$.
    \item Fall 2 ($v_1\neq\varepsilon$): Aufgrund der Definitionen von \PrET{}
      und \MIT{} endet $v_1$ auf ein $a\in I_1$, d.h.\ $v_1=v'_1a$. $v'$ sei
      das Präfix von $xy$, das mit der letzten Aktion von $v_1$ endet, d.h.\
      mit $a$ und $v'_2=v'|_{\Sigma _2}$. Falls $v_2=w_2\in L_2$, dann ist
      $v'_2$ ein Präfix von $v_2$. Falls $v_2\in \PrET _2\cup \MIT _2$ gilt,
      dann ist durch die Annahme, dass $v_2$ nicht vor $v_1$ endet, $v'_2$ ein
      Präfix von $v_2$. Im Fall $v_2 \in \MIT _2$ zwei man zusätzlich, dass
      $v_2$ auf $b\in I_2$ endet. Es kann jedoch $a=b$ gelten.
  \end{itemize}
  In allen Fällen erhält man $v'_2=v'|_{\Sigma _2}$ ist ein Präfix von $v_2$
  und $v'\in v_1\|v'_2$ ist ein Präfix von $xy$. Es kann nur für die Fälle
  $a\notin I_2$ gefolgert werden, dass $p_{02} \weakmust[v'_2]$ gilt.
  \begin{itemize}
    \item Fall I ($v_1\in\MIT _1$ und $v_1\neq \varepsilon$): \TODO{beweisen}
    \item Fall II ($v_1\in\PrET _1$): \TODO{beweisen}
  \end{itemize}

  2.:\\
  Durch die Definitionen ist klar, dass $L_i\subseteq \EL _i$ und $\ET
  _i\subseteq \EL _i$ gilt. Die Argumentation startet auf den rechten Seite der
  Gleichung:
  \begin{align*}
    (\EL{}_1\| \EL{}_2)\cup
    \ET{}_{12}&\overset{\ref{KommFehlerSemDef}}{=}\left(\left(L_1\cup
    \ET{}_1\right)\|\left(L_2\cup \ET{}_2\right)\right)\cup \ET{}_{12}\\
    &=(L_1\|L_2) \cup \underset{\overset{1.}{\subseteq}
    \ET{}_{12}}{\underset{\subseteq
    (\EL{}_1\|\ET{}_2)}{\underbrace{(L_1\|\ET{}_2)}}} \cup
    \underset{\overset{1.}{\subseteq} \ET{}_{12}}{\underset{\subseteq
    (\ET{}_1\|\EL{}_2)}{\underbrace{(\ET{}_1\|L_2)}}} \cup
    \underset{\overset{1.}{\subseteq} \ET{}_{12}}{\underset{\subseteq
    (\EL{}_1\|\ET{}_2)}{\underbrace{(\ET{}_1\|\ET{}_2)}}} \cup \ET{}_{12}\\
    &=(L_1\|L_2) \cup \ET{}_{12}\\
    &\overset{\ref{LDef}}{=}L_{12}\cup \ET{}_{12}\\
    &\overset{\ref{KommFehlerSemDef}}{=}\EL{}_{12}.
  \end{align*}
\end{proof}

\begin{Kor}[Kommunikationsfehler-Präkongruenz]
  Die Relation \ERel{} ist eine Präkongruenz bezüglich $\cdot\|\cdot$.
\end{Kor}
\begin{proof}
  \TODO{beweisen}
\end{proof}

