\chapter{Verfeinerungen für Kommunikationsfehler-Freiheit}

Dieses Kapitel versucht die Präkongruenz für Error bei \EIO{}s
aus~\cite{Schinko2016BA} auf die hier betrachten \MEIO{}s zu erweitern.

\begin{Def}[Fehler-freie Kommunikation]
  Ein Kommunikationsfehler"=Zustand ist \emph{lokal erreichbar} in einer
  as"=Implementierung $P'$ eines \MEIO{} $P$, wenn ein $w\in O^*$ existiert mit
  $p'_0 \weakmust[w]p'\in E'$.\\
  Zwei \MEIO{}s $P_1$ und $P_2$ \emph{kommunizieren Fehler-frei}, wenn alle
  as"=Implementierungen ihrer Parallelkomposition $P_{12}$ keine
  Kommunikationsfehler"=Zustände lokal erreichen können.
\end{Def}

\begin{Def}[Kommunikationsfehler-Verfeinerungs-Basirelation]
  Für zwei \MEIO{}s $P_1$ und $P_2$ mit der gleichen Signatur wird $P_1\EBRel{} P_2$
  geschrieben, wenn ein Kommunikationsfehler"=Zustand in einer as"=Implementierung
  von $P_1$ nur dann lokal erreichbar ist, wenn es auch eine as"=Implementierung
  von $P_2$ gibt, in der dieser Kommunikationsfehler"=Zustand auch lokal
  erreichbar ist. Die Basisrelation stellt eine \emph{Verfeinerung} bezüglich
  \emph{Kommunikationsfehlern} dar.\\
  \ECRel{} bezeichnet die \emph{vollständig abstrakte Präkongruenz} von
  \EBRel{} bezüglich $\cdot\|\cdot$, d.h.\ die gröbste Präkongruenz bezüglich
  $\cdot\|\cdot$, die in \EBRel{} enthalten ist.
\end{Def}

Für Implementierungen $P_1$ und $P_2$ entspricht \EBRel{} der Relation
\EBbaRel{} aus~\cite{Schinko2016BA}.

Wie in~\cite{Schinko2016BA} werden die Fehler hier Trace basiert betrachtet. Da
jedoch die Basisrelation über as"=Implementierungen spricht, sind die Trace Mengen
auch nicht für die \MEIO{}s mit may-Transitionen definiert sondern nur für die
Menge der möglichen as"=Implementierungen eines solchen \MEIO{}s.

\begin{Def}[Kommunikationsfehler-Traces]
  Für einen \MEIO{} $P$ wird definiert:
  \begin{itemize}
    \item \emph{strikte Kommunikationsfehler-Traces}: $\StET (P) :=
      \left\{w\in\Sigma ^* \mid \exists P' \in\asimp (P) :
      \phantom{\weakmust[w]} \right.$ $\left. p'_0 \weakmust[w] p' \in
      E\right\}$ \TODO{erzwungen Zeilenumbruch kontrollieren}
    \item \emph{gekürzte Kommunikationsfehler-Traces}: $\PrET (P) :=
      \left\{\prune (w) \mid w\in\StET (P)\right\}$
    \item \emph{Input-kritische-Traces}: $\MIT (P) := \left\{wa\in\Sigma ^* \mid
      \exists P' \in\asimp (P) : p'_0
      \weakmust[w] p' \land \right.$\linebreak $\left. a\in I\land p'\nmust[a] \right\}$ \TODO{erzwungen
      Zeilenumbruch kontrollieren}
  \end{itemize}
\end{Def}

\begin{Def}[Kommunikationsfehler-Semantik]
  Sie $P$ ein \MEIO{}.
  \begin{itemize}
    \item Die Menge der \emph{Kommunikationsfehler-Traces} von $P$ ist $\ET (P)
      := \cont (\PrET (P)) \cup\cont (\MIT (P))$.
    \item Die \emph{Kommunikationsfehler-geflutete Sprache} von $P$ ist $\EL
      (P) := L(P) \cup \ET (P)$.
  \end{itemize}
  Für zwei \MEIO{}s $P_1,P_2$ mit der gleichen Signatur wird $P_1\ERel P_2$
  geschrieben, wenn $\ET _1\subseteq \ET _2$ und $\EL _1 \subseteq \EL _2$
  gilt.
\end{Def}

Hierbei ist zu beachten, dass die Mengen \StET{}, \PrET{}, \MIT{}, \ET{} und
\EL{} nur denen aus~\cite{Schinko2016BA} entsprechen, wenn $P$ bereits eine
Implementierung ist.

\begin{Satz}[Kommunikationsfehler-Semantik für Parallelkompositionen]
  Für zwei komponierbare \MEIO{}s $P_1,P_2$ und ihre Komposition $P_{12}$ gilt:
  \begin{enumerate}
    \item $\ET _{12} = \cont (\prune ((\ET _1\|\EL _2) \cup (\EL _1\|\ET
      _2)))$,
    \item $\EL _{12} = (\EL _1\|\EL _2) \cup \ET _{12}$.
  \end{enumerate}
\end{Satz}
