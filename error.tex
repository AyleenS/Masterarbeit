\chapter{Verfeinerungen für Kommunikationsfehler-Freiheit}

Dieses Kapitel versucht die Präkongruenz für Error bei \EIO{}s
aus~\cite{Schinko2016BA} auf die hier betrachten \MEIO{}s zu erweitern.

\begin{Def}[fehler-freie Kommunikation]
  Ein Kommunikationsfehler-Zustand ist \emph{lokal erreichbar} in einer
  Implementierung $P'$ eines \MEIO{} $P$, wenn ein $w\in O^*$ exsitiert mit
  $p'_0 \weakmust[w]p'\in E'$.\\
  Zwei \MEIO{}s $P_1$ und $P_2$ \emph{kommunizieren fehler-frei}, wenn alle
  Implementierungen ihrer Parallelkomposition $P_{12}$ keine
  Kommunikationsfehler-Zustände lokal erreichen können.
\end{Def}

\begin{Def}[Kommunikationsfehler-Verfeinerungs-Basirelation]
  Für \MEIO{}s $P_1$ und $P_2$ mit der gleichen Signatur \TODO{Signatur
  definieren} wird $P_1\EBRel{} P_2$ geschrieben, wenn ein
  Kommunikationsfehler-Zustand in einer Implementierung von $P_1$ nur dann
  lokal erreichbar ist, wenn es auch eine Implementierung von $P_2$ gibt, in
  der dieser Kommunikationsfehler-Zustand auch lokal erreichbar ist. Die
  Basisrelation stellt eine \emph{Verfeinerung} bezüglich
  \emph{Kommunikationsfehlern} dar.\\
  \ECRel{} bezeichnet die \emph{vollständig abstrakte Präkongurenz} von
  \EBRel{} bezüglich $\cdot\|\cdot$, d.h.\ die gröbste Präkongurenz bezüglich
  $\cdot\|\cdot$, die in \EBRel{} enthalten ist.
\end{Def}
