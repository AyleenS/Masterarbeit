\section{Hiding}

Es soll nun auch für die Relation \QRel{} die Auswirkungen des Hiding-Operators
untersucht werden. Outputs, die dabei in der Menge $X$ enthalten sind, werden
in interne Aktionen umgewandelt. Die Definition der stillen Zustände verlangt,
dass keine ausgehenden must"=Transitionen für lokale Aktionen an einem Zustand
existieren dürfen, damit er als still gilt. Die Menge $Qui$ bleibt also unter
Hiding erhalten. Da die Trace-Definitionen lokale Erreichbarkeit für die
Relevanz von fehlerhaften Zuständen verwenden, sollte sich unter der
Internalisierung von Aktionen aus der Menge $X$ nicht an der Relevanz von
fehlerhaften Zuständen ändern.

\begin{Satz}[Stillstands-Präkongruenz bzgl.\ Internalisierung]
  Seien $P_1$ und $P_2$ zwei \MEIO{}s für die $P_1\QRel P_2$ gilt, dann folgt
  auch die Gültigkeit von $P_1/X\QRel P_2/X$. Die Relation \QRel{} ist also ein
  Präkongruenz bezüglich $\cdot /\cdot$. Es gilt für die Sprachen und Traces:
  \begin{enumerate}[(i)]
    \item $L(P/X) = \{w\in (\Sigma\backslash X)^*\mid \exists w'\in L(P):
      w'|_{\Sigma\backslash X} = w\}$,
    \item $\ET (P/X) = \{w\in (\Sigma\backslash X)^*\mid \exists w'\in \ET(P):
      w'|_{\Sigma\backslash X} = w\}$,
    \item $\EL (P/X) = \{w\in (\Sigma\backslash X)^*\mid \exists w'\in \EL(P):
      w'|_{\Sigma\backslash X} = w\}$,
    \item $\QET (P/X) = \{w\in (\Sigma\backslash X)^*\mid \exists w'\in
      \QET(P): w'|_{\Sigma\backslash X} = w\}$.
  \end{enumerate}
\end{Satz}
\begin{proof}
  \TODO{zu beweisen}
\end{proof}
