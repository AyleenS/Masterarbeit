\section{Hiding}

Es soll nun auch für die Relation \QRel{} die Auswirkungen des Hiding-Operators
untersucht werden. Outputs, die dabei in der Menge $X$ enthalten sind, werden
in interne Aktionen umgewandelt. Die Definition der stillen Zustände verlangt,
dass keine ausgehenden must"=Transitionen für lokale Aktionen an einem Zustand
existieren dürfen, damit er als still gilt. Die Menge $Qui$ bleibt also unter
Hiding erhalten. Da die Trace-Definitionen lokale Erreichbarkeit für die
Relevanz von fehlerhaften Zuständen verwenden, sollte sich unter der
Internalisierung von Aktionen aus der Menge $X$ nicht an der Relevanz von
fehlerhaften Zuständen ändern.

\begin{Satz}[Stillstands-Präkongruenz bzgl.\ Internalisierung]
  \label{StilleHidingSatz}
  Seien $P_1$ und $P_2$ zwei \MEIO{}s für die $P_1\QRel P_2$ gilt, dann folgt
  auch die Gültigkeit von $P_1/X\QRel P_2/X$. Die Relation \QRel{} ist also ein
  Präkongruenz bezüglich $\cdot /\cdot$. Es gilt für die Sprachen und Traces:
  \begin{enumerate}[(i)]
    \item $L(P/X) = \left\{w\in (\Sigma\backslash X)^*\mid \exists w'\in L(P):
      w'|_{\Sigma\backslash X} = w\right\}$,
    \item $\ET (P/X) = \left\{w\in (\Sigma\backslash X)^*\mid \exists w'\in
      \ET(P): w'|_{\Sigma\backslash X} = w\right\}$,
    \item $\EL (P/X) = \left\{w\in (\Sigma\backslash X)^*\mid \exists w'\in
      \EL(P): w'|_{\Sigma\backslash X} = w\right\}$,
    \item $\QET (P/X) = \left\{w\in (\Sigma\backslash X)^*\mid \exists w'\in
      \QET(P): w'|_{\Sigma\backslash X} = w\right\}$.
  \end{enumerate}
\end{Satz}
\begin{proof}
  Zunächst sollen hier die Aussagen (i) bis (iv) beweisen werden. Die Punkte
  (i) bis (iii) wurden bereits in Satz~\ref{FehlerHidingSatz} entsprechend
  nachgewiesen. Es muss also nur noch (iv) gezeigt werden.
  \begin{itemize}
    \item[(iv)] Dieser Punkt kann auf die Aussagen von
      Satz~\ref{FehlerHidingSatz} zurück geführt werden. Ein $w'$ aus $\QET _P$
      ist nach Definition in $\ET _P$ oder $\StQT _P \subseteq L_P$ enthalten.
      Mit (i) und (ii) folgt die Zugehörigkeit von $w=w'|_{\Sigma\backslash X}$
      zur Menge $\ET _{P/X}$ bzw. $\StQT _{P/X}$. $w\in\StQT _{P/X}$ folgt
      jedoch nicht direkt aus (i), es wird noch eine zusätzliche Feststellung
      benötigt. Für $w'$ in $P$ wurde im Beweis von Satz~\ref{FehlerHidingSatz}
      (i) ein Ablauf zu einem Zustand $p$ angegeben. Durch das Internalisieren
      der Transitionen, die mit Outputs beschriftet werden führt das
      eingeschränkte Wort $w$ weiterhin zum Zustand $p$. $p$ muss für
      $w'\in\StQT _P$ ein stiller Zustand sein, somit ist $w$ nicht nur in $L
      _{P/X}$ sondern auch in $\StQT _{P/X}$ enthalten. Insgesamt folgt also
      $w\in\QET _{P/X}$.\\
      Für die andere Inklusionsrichtung kann man für ein $w$ aus $\QET _{P/X}$
      analog zu den Beweisen von (i) und (ii) aus~\ref{FehlerHidingSatz}
      $\tau$-Transitionsbeschriftungen durch Outputs aus $X$ ersetzten um einen
      Ablauf aus $P$ zu erzeugen. Das Wort $w$ kann also um Aktionen aus $X$
      erweitert werden um ein $w'$ zu erhalten, für das $w'\in\QET _P$ gilt.
  \end{itemize}
  Aus $P_1\ERel P_2$ folgt mit Satz~\ref{FehlerHidingSatz} $P_1/X\ERel P_2/X$.
  Um die analoge Folgerung für \QRel{} zu begründen, muss wegen
  Definition~\ref{StilleSemDef} also nur noch nachgewiesen werden, dass $\QET
  _{P_1/X} \subseteq \QET _{P_2/X}$ unter der Voraussetzung der Inklusion $\QET
  _1 \subseteq \QET _2$ gilt. Dies folgt aus der Aussage des Punktes (iv).\\
  Die Relation \QRel{} bleibt also unter der Anwendung des Hiding-Operators
  erhalten und ist somit bezüglich diesem eine Präkongruenz.
\end{proof}

In Definition~\ref{DefParallelkompInternal} wurde mit Hilfe des
Hiding-Operators aus der Parallelkomposition ohne Verbergen die
Parallelkomposition mit Internalisierung der synchronisierten Aktionen
nachgebildet. Die Präkongruenz-Eigenschaft bezüglich \QRel{} der
Parallelkomposition $\cdot |\cdot$ kann aus den Präkongruenz-Eigenschaften von
$\cdot\|\cdot$ und $\cdot /\cdot$ bezüglich der Relation \QRel{} aus dem
Korollar~\ref{StillePraekonKor} und dem Satz~\ref{StilleHidingSatz} geschlossen
werden.

\begin{Kor}[Stillstands-Präkongruenz mit Internalisierung]
  Die Relation \QRel{} ist eine Präkongruenz bezüglich $\cdot |\cdot$.
\end{Kor}
