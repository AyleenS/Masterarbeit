\section{Testing-Ansatz}

Der gröbste Präkongruenz-Ansatz aus dem letzten Teil stützt sich auf die
Parallelkomposition von \MEIO{}s, die wie bereits in
Kapitel~\ref{saetzeKapitel} erwähnt auch anders gestaltet hätte werden können.
Spezielle bei neuen Fehler ist dort gezeigt worden, dass es zu einem
ungewöhnlichen Verfeinerungs-Verhalten kommen kann, das möglicherweise gar
nicht so gewollt ist. Deshalb ist es sinnvoll sich nur auf die Definition der
Parallelkomposition von \EIO{} zu stützen, die hier für die Implementierungen
gilt. Dies führt zu dem in diesem Teil behandelten Testing-Ansatz. Dieser lässt
jedoch keine Aussage zu gröbste Präkongruenz zu, wie das im letzen Teil möglich
war. Der Test ersetzt die Basisrelation als grundlegende Definition für den
Ansatz. Die Präkongruenz, die sich ergibt, ist jedoch die selbe.\\
Der Testing-Ansatz stützt sich auf das Vorgehen, das
in~\cite{Vogler2015FailSem} angewendet wurde. Jedoch sind Tests dort Tupel aus
einer Implementierung und einer Menge an Aktionen, über denen synchronisiert
werden soll. Die \MEIO{}s, die hier komponiert werden bringen die Menge
$\Synch$ an Aktionen, die synchronisiert werden bereits mit. Es wird im
Gegensatz zu~\cite{Vogler2015FailSem} mit Inputs und Outputs gearbeitet und
dadurch scheint der Ansatz die gemeinsamen Aktionen zu synchronisieren
natürlicher, wie eine Menge an Aktionen beim Test vorzugeben.\\
Die Definition von lokaler Erreichbarkeit eines Fehler-Zustandes soll aus dem
gröbsten Präkongruenz"=Ansatz übernommen werden. Es wird also trotzdem noch
optimistisch davon ausgegangen, dass Fehler erst zu Problemen führen, wenn eine
hilfreiche Umgebung dies nicht mehr verhindern kann.

\begin{Def}[Test und Verfeinerung für Kommunikationsfehler]
  \label{KommTestDef}
  Ein \emph{Test} $T$ ist eine komponierbare Implementierung. Ein \MEIO{} $P$
  \emph{as-erfüllt} einen Kommunikationsfehler"=Test $T$, falls $S\|T$ lokal
  fehler-frei ist für alle $S\in \asimp (P)$. Es wird dann $P \EsatAs T$
  geschrieben.\\
  Ein \MEIO{} $P$ \emph{Fehler-verfeinert} $P'$, falls sie die selbe Signaturen
  haben und für alle Tests $T$: $P'\EsatAs T \Rightarrow P\EsatAs T$.
\end{Def}

Abgesehen von Korollar~\ref{lokalFehlerErrKor}~(ii) erweisen sich
Definition~\ref{KommTracesDef} bis Korollar~\ref{KommPraekonKor} auch hier als
nützlich.\\
Die Basisrelation aus dem gröbsten Präkongruenz-Ansatz gibt es in diesem Teil
nicht, somit kann das Lemma~\ref{KommVerfeinLem} nicht in dieser Art formuliert
werden. Jedoch kann hier mit Tests gearbeitet werden und unter deren
zuhilfenahme ein ähnliches Lemma formuliert werden.

\begin{Lem}[Testing-Verfeinerung mit Kommunikationsfehlern]
  \label{KommTestVerfeinLem}
  Gegeben sind zwei \MEIO{}s $P_1$ und $P_2$ mit der gleichen Signatur. Wenn
  für alle Tests $T$, die Partner von $P_1$ bzw. $P_2$ sind, $P_2\EsatAs T
  \Rightarrow P_1\EsatAs T$ gilt, dann folgt daraus die Gültigkeit von $P_1\ERel
  P_2$.
\end{Lem}
\begin{proof}
  Da $P_1$ und $P_2$ die gleichen Signaturen haben wird $I:=I_1=I_2$ und $O:=
  O_1=O_2$ definiert. Für jeden Test Partner $T$ gilt $I_T=O$ und $O_T=I$.\\
  Um $P_1\ERel P_2$ zu zeigen, wird nachgeprüft, ob folgendes gilt:
  \begin{itemize}
    \item $\ET _1\subseteq \ET _2$,
    \item $\EL _1\subseteq \EL _2$.
  \end{itemize}
  Für ein gewähltes präfix-minimales Element $w\in\ET _1$ wird gezeigt, dass
  dies $w$ oder eines seiner Präfixe in $\ET _2$ enthalten ist. Dies ist
  möglich, da die beiden Mengen $\ET _1$ und $\ET _2$ unter \cont{}
  abgeschlossen sind.\\
  Mit Proposition~\ref{KommSemProp} folgt aus $w\in\ET _1$, dass es auch eine
  as"=Implementierung $P'_1$ von $P_1$ geben muss, für die $w$ ebenfalls in
  $\ET _{P'_1}$ enthalten ist. Da $w$ für $P_1$ präfix-minimal ist, gilt $w\in
  \PrET _1$ oder $w\in\MIT _1$ mit Proposition~\ref{KommTracesProp} kann die
  as"=Implementierung $P'_1$ sogar so gewählt werden, dass $w$ für $P'_1$
  ebenfalls präfix-minimal ist.
  \begin{itemize}
    \item Fall 1 ($w=\varepsilon$): Es ist ein Fehler-Zustand in $P'_1$ lokal
      erreichbar. Für $T$ wird ein Transitionssystem verwendet, das nur aus dem
      Startzustand und einer must"=Schleife für alle Inputs $x\in I_T$ besteht.
      Somit kann $P'_1$ die im Prinzip gleichen Fehler-Zustände lokal
      erreichen wie $P'_1\|T$. $P'_1$ ist in Parallelkomposition mit $T$ nicht
      lokal fehler-frei somit gilt $P_1\EsatAs T$ nicht. Es darf also auch
      $P_2$ den Test $T$ nicht as"=erfüllen, wegen der Implikation $P_2\EsatAs
      T \Rightarrow P_1\EsatAs T$. Damit $P_2$ $T$ nicht as"=erfüllt muss es
      eine as"=Implementierungen $P'_2$ geben, die in Parallelkomposition mit
      $T$ zu einem nicht lokal fehler-freien System führt. Es muss also in
      $P'_2\|T$ ein Fehler lokal erreichbar sein. Durch die Definition von $T$
      kann dieser Fehler nur von $P'_2$ geerbt sein. In $P'_2$ kann dieser
      Fehler-Zustand nur durch interen Aktionen und Outputs erreichbar sein, da
      $T$ keine Outputs besitzt, die man mit Inputs aus $P'_2$ synchronisieren
      könnte und unsynchronisierte Aktionen sind in einer Parallelkomposition
      von Partner nicht möglich. Somit gilt $\varepsilon\in \PrET _{P'_2}
      \subseteq \ET _{P'_2}$. Mit Proposition~\ref{KommSemProp} folgt daraus
      $\varepsilon\in \ET _2$.
    \item Fall 2 ($w=x_1\dots x_n x_{n+1}\in\Sigma ^+$ mit $n\geq 0$ und
      $x_{n+1}\in I=O_T$): Es wird der folgende Partner $T$ betrachtet (dieser
      entspricht bis auf die Benennungen der Mengen dem Transitionssystem $U$
      aus Abbildung~\ref{UohneE}):
      \begin{itemize}
        \item $T=\{p_0,p_1,\dots ,p_{n+1}\}$,
        \item $p_{0T}=p_0$,
        \item $\begin{aligned}[t]
            \may _T = \must _T&=\{(p_j,x_{j+1},p_{j+1})\mid  0\leq j\leq n\}\\
            &\cup\{(p_j,x,p_{n+1})\mid  x\in I_T\backslash\{x_{j+1}\}, 0\leq
            j\leq n\}\\
            &\cup\{(p_{n+1},x,p_{n+1})\mid  x\in I_T\},
        \end{aligned}$
        \item $E_T=\emptyset$.
      \end{itemize}
      Für das präfix-minimale $w$ können nun zwei Fälle unterschieden werden,
      für die beide $\varepsilon\in\PrET (P'_1\|T)$ folgen wird.
      \begin{itemize}
        \item Fall 2a) ($w\in\MIT _{P'_1}$): In $P'_1\|T$ erhält man
          $(p'_{01},p_0) \lweakmay[x_1\dots x_n]_{P'_1\|T} (p',p_n)$ mit $p'
          \nmust[x_{n+1}]_{P'_1}$ und $p_n\must[x_{n+1}]_T$. Deshalb gilt
          $(p',p_n)\in E_{P'_1\|T}$. Da alle Aktionen aus $w$ bis auf $x_{n+1}$
          synchronisiert werden und $I\cap I_T=\emptyset$, gilt $x_1,\dots ,
          x_n\in O_{P'_1\|T}$. Es folgt also $\varepsilon\in \PrET (P'_1\|T)$.
        \item Fall 2b) ($w\in\PrET _{P'_1}$): In der Parallelkomposition
          $P'_1\|T$ erhält man die Transitionsfolge $(p_{01},p_0)
          \weakmay[w]_{P'_1\|T} (p'',p_{n+1}) \weakmay[u]_{P'_1\|T}
          (p',p_{n+1})$ für $u\in O^*$ und $p'\in E_{P'_1}$. Daraus folgt
          $(p',p_{n+1})\in E_{P'_1\|T}$ und somit $wu\in\StET (P'_1\|T)$. Da
          alle Aktionen in $w$ synchronisiert werden und $I\cap I_T =
          \emptyset$, gilt $x_1,\dots , x_n, x_{n+1}\in O_{P'_1\|T}$ und, da
          $u\in O^*$, folgt $u\in O^*_{P'_1\|T}$. Somit ergibt sich
          $\varepsilon \in \PrET (P'_1\|T)$.
      \end{itemize}
      Da $\varepsilon\in\PrET (P'_1\|T)$ gilt, kann durch $P_2\EsatAs T
      \Rightarrow P_1\EsatAs T$ geschlossen werden, dass auch $P_2\EsatAs T$
      nicht gelten kann. Die Relation gilt für $P_2$ nicht, da es eine
      as"=Implementierung $P'_2$ von $P_2$ gibt, so dass $P'_2\|T$ nicht
      lokal fehler-frei ist.\\
      Der lokal erreichbar Fehler in $P'_2\|T$ kann geerbt oder neu sein.
      \begin{itemize}
        \item Fall 2i) (neuer Fehler): Da jeder Zustand von $T$ alle Inputs
          $x\in O=I_T$ zulässt, muss ein lokal erreichbarer Fehler-Zustand der
          Form sein, dass ein Outputs $a\in O_T$ von $T$ möglich ist, der nicht
          mit einem passenden Input aus $P'_2$ synchronisiert werden werden
          kann. Durch die Konstruktion von $T$ sind in $p_{n+1}$ keine Outputs
          möglich. Ein neuer Fehler muss also die Form $(p',p_i)$ haben mit
          $i\leq n$, $p'\nmust[x_{i+1}]_{P'_2}$ und $x_{i+1}\in O_T=I$. Durch
          Projektion erhält man dann $p_{02} \lweakmay[x_1\dots x_i]_{P'_2} p'
          \nmust[x_{i+1}]_{P'_2}$ und damit gilt $x_1\dots x_{i+1}\in\MIT
          _{P'_2} \subseteq \ET _{P'_2}$. Es ist also ein Präfix von $w$ in
          $\ET _{P'_2}$ enthalten und mit Proposition~\ref{KommSemProp} auch in
          $\ET _2$.
        \item Fall 2ii) (geerbter Fehler): $T$ hat $x_1\dots x_i u$ mit $u\in
          I^*_T =O^*$ ausgeführt und ebenso hat $P'_2$ dieses Wort
          abgearbeitet. Durch dies hat $P'_2$ einen Zustand in $E _{P'_2}$
          erreicht, da von $T$ keine Fehler geerbt werden können. Es gilt dann
          $\prune (x_1\dots x_iu) = \prune (x_1\dots x_i)\in\PrET _{P'_2}
          \subseteq \ET _{P'_2}$. Da $x_1\dots x_i$ ein Präfix von $w$ ist,
          führt in diesem Fall eine Verlängerung um lokale Aktionen eines
          Präfix von $w$ zu einem Fehler-Zustand. Da \ET{} der Menge aller
          Verlängerungen von gekürzten Fehler-Traces entspricht, ist $x_1\dots
          x_i$ in $\ET _{P'_2}$ enthalten und somit ist mit
          Proposition~\ref{KommSemProp} ein Präfix von $w$ in $\ET _2$
          enthalten.
      \end{itemize}
  \end{itemize}
  Um die andere Inklusion zu beweisen, reich es aufgrund der ersten Inklusion
  und der Definition von \EL{} aus zu zeigen, dass $L_1\backslash \ET
  _1\subseteq \EL _2$ gilt.\\
  Es wird dafür ein beliebiges $w\in L_1\backslash \ET _1$ gewählt und gezeigt,
  dass es in $\EL _2$ enthalten ist. Das $w$ ist wegen der
  Propositionen~\ref{LImpProp} und~\ref{KommSemProp} auch für eine
  as"=Implementierung $P'_1$ von $P_1$ in $L_{P'_1}\backslash \ET _{P'_1}$
  enthalten.
  \begin{itemize}
    \item Fall 1 ($w=\varepsilon$): Da $\varepsilon$ immer in $\EL _2$
      enthalten ist, muss hier nichts gezeigt werden.
    \item Fall 2 ($w=x_1\dots x_n$ mit $n\geq 1$): Es wird ein Partner $T$ wie
      folgt konstruiert ($T$ entspricht dabei $U$ aus Abbildung~\ref{UmitE} bis
      auf die Benennung der Mengen):
      \begin{itemize}
        \item $T=\{p_0,p_1,\dots ,p_n,p\}$,
        \item $p_{0T}=p_0$,
        \item $\begin{aligned}[t]
            \may _T = \must _T&=\{(p_j,x_{j+1},p_{j+1})\mid 0\leq j< n\}\\
            &\cup\{(p_j,x,p)\mid x\in I_T\backslash\{x_{j+1}\},0\leq j< n\}\\
            &\cup\{(p,x,p)\mid x\in I_T\},
        \end{aligned}$
        \item $E_T=\{p_n\}$.
      \end{itemize}
      Da $p_{01} \weakmay[w]_{P'_1} p'$ gilt, kann man schließen, dass
      $P'_1\|T$ ein lokal erreichbaren geerbten Fehler hat. Es muss also auch
      eine as"=Implementierung $P'_2$ von $P_2$ geben, für die $P'_2\|T$ einen
      lokal erreichbaren Fehler-Zustand hat.
      \begin{itemize}
        \item Fall 2a) (neuer Fehler aufgrund von $x_i\in O_T$ und $p_{02}
          \lweakmay[x_1\dots x_{i-1}]_{P'_2}p''\nmust[x_i]_{P'_2}$): Es gilt
          $x_1\dots x_i\in\MIT _{P'_2}$ und somit $w\in\EL _{P'_2}$. Anzumerken
          ist, dass es nur auf diesem Weg Outputs von $T$ möglich sind, deshalb
          gibt es keine anderen Outputs von $T$, die zu einem neuen Fehler
          führen können. Es gilt $w\in\EL _2$ wegen
          Proposition~\ref{KommSemProp}.
        \item Fall 2b) (neuer Fehler aufgrund von $a\in O=I_T$): Der einzige
          Zustand, in dem $T$ nicht alle Inputs erlaubt sind, ist $p_n$, der
          bereits ein Fehler-Zustand ist. Da in diesem Fall der Fehler"=Zustand
          in $P'_2\|T$ erreichbar ist, besitzt der komponierte \MEIO{} einen
          geerbten Fehler und es gilt $w\in L_{P'_2}\subseteq \EL _2$, wegen
          dem folgenden Fall 2c).
        \item Fall 2c) (geerbter Fehler von $T$): Da $p_n$ der einzige
          Fehler-Zustand in $T$ ist und alle Aktionen synchronisiert sind, ist
          dies nur möglich, wenn $p_{02} \lweakmay[x_1\dots x_n]_{P'_2}$ gilt.
          In diesem Fall ist $w\in L_{P'_2}\subseteq \EL _{P'_2}$. Daraus
          folgt mit Proposition~\ref{KommSemProp} $w\in\EL _2$.
        \item Fall 2d) (geerbter Fehler von $P'_2$): Es gilt $p_{02}
          \lweakmay[x_1\dots x_iu]_{P'_2} p'\in E_{P'_2}$ für ein $i\geq 0$ und
          $u\in O^*$. Somit ist $x_1\dots x_i u\in\StET _{P'_2}$ und damit
          $\prune (x_1\dots x_iu) =\prune (x_1\dots x_i)\in\PrET
          _{P'_2}\subseteq \EL _{P'_2}$. Mit Hilfe von
          Proposition~\ref{KommSemProp} folgt $w\in\EL _{P'_2}\subseteq \EL
          _2$.
      \end{itemize}
  \end{itemize}
\end{proof}

\TODO{ab hier Verbesserungen einarbeiten}

\begin{Satz}
  \label{KommTestVerfSatz}
  Falls $P_1\ERel P_2$ gilt folgt draus, dass $P_1$ $P_2$
  Fehler-verfeinert.
\end{Satz}
\begin{proof}
  Nach Definition gilt genau dann, wenn $\varepsilon\in \ET (P)$, ist ein
  Fehler-Zustand lokal erreichbar in $P$. $P_1\ERel P_2$ impliziert, dass
  $\varepsilon\in\ET _2$ gilt, wenn $\varepsilon\in\ET _1$. Somit ist ein
  Fehler-Zustand in $P_1$ nur dann lokal erreichbar, wenn dieser auch in $P_2$
  lokal erreichbar ist. Aufgrund von Proposition~\ref{lokalFehlerErrKor}~(i)
  gibt es dann auch entsprechende as"=Implementierungen $P'_1$ und $P'_2$ für
  $P_1$ bzw. $P_2$, die ebenfalls Fehler-Zustände lokal erreichen können, wenn
  dies in $P_1$ bzw. $P_2$ möglich ist. Für alle Tests $T$ gilt, dass $P'_j\|T$
  für $j\in\{1,2\}$ einen lokal erreichbaren Fehler hat, wenn $P'_j$ einen
  solchen hat. Dies Schlussfolgerung ist möglich, da
  Satz~\ref{KommFehlerSemSatz}.1 aussagt, dass die Parallelkomposition von
  einem Fehler-Trace aus $\ET _{P'_j}$ mit einem Wort aus $\EL _T$ in den
  Fehler-Traces von $\ET _{P'_j\|T}$ enthalten ist. $\varepsilon$ ist ein
  Element von $\ET _{P'_j}$, da ein Fehler lokal erreichbar ist und
  $\varepsilon \in \EL _T$ gilt  für alle Test $T$ und somit folgt $\varepsilon
  \in \ET _{P'_j\|T}$. Dies impliziert die lokal Fehler Erreichbarkeit in
  $P'_j\|T$. Falls also $P_1$ einen lokal erreichbaren Fehler-Zustand enthält,
  dann zeigt sich dies Verhalten auch in einer as"=Implementierungen und
  ebenfalls in der Parallelkomposition der as"=Implementierung mit allen Tests
  $T$. Aus einem lokal erreichbaren Fehler in $P_1$ folgt auch ein lokal
  erreichbarer Fehler-Zustand in $P_2$ und somit auch ein lokal erreichbarer
  Fehler-Zustand in der Parallelkomposition einer as"=Implementierung von $P_2$
  mit allen Tests $T$. Aus $P_1\ERel P_2$ folgt also die Implikation $\neg P_1
  \EsatAs T \Rightarrow \neg P_2 \EsatAs T$ für alle Test $T$. Wenn man die
  Negationen entfernt, ergibt sich für alle Tests $T$: $P_2\EsatAs T\Rightarrow
  P_1\EsatAs T$. Dies entspricht der Definition von $P_1$ Fehler-verfeinert
  $P_2$.
\end{proof}

Auch die in diesem Abschnitt gezeigten Folgerungen schließen sich zu einem
Ring. Dies ist in Abbildung~\ref{KommTestFolgerungskette} dargestellt.

\begin{figure}[h!tbp]
  \begin{center}
    \begin{tikzpicture}[scale = 3]
      \matrix (m) [matrix of math nodes,row sep=2cm,column sep=4cm]{%
        P_1\ERel P_2 & P_1 \text{ Fehler-verfeinert } P_2 \\
        \substack{\forall \text{ Test Partner } T:\\P_2\EsatAs T\Rightarrow
        P_1\EsatAs T} &
    \substack{\forall \text{ Tests } T:\\P_2\EsatAs T\Rightarrow P_1\EsatAs T} \\};
        \draw[-implies, double, double distance=1mm]
          (m-1-1) -- node [above] {Satz~\ref{KommTestVerfSatz}} (m-1-2);
        \draw[-implies, double, double distance=1mm]
          (m-1-2) -- node [right] {Definition~\ref{KommTestDef}} (m-2-2);
        \draw[-implies, double, double distance=1mm]
          (m-2-1) -- node [left]
          {Lemma~\ref{KommTestVerfeinLem}} (m-1-1);
        \draw[-implies, double, double distance=1mm]
          (m-2-2) -- node [below]
          {$\substack{T \text{ Test Partner}\\\Downarrow\\ T \text{ Test}}$} (m-2-1);
    \end{tikzpicture}
    \caption{Folgerungskette der Testing-Verfeinerung und Fehler-Relation}
  \label{KommTestFolgerungskette}
  \end{center}
\end{figure}
