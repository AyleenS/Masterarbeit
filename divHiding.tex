\section{Hiding}

Der Hiding-Operator wandelt Outputs in $\tau$s um. Somit hat das Hiding auf die
Divergenz-Eigenschaft im Vergleich zu den betrachteten Eigenschaften aus den
beiden vorangegangen Kapiteln deutlich größere Auswirkungen. Die Menge der
divergenten Zustände kann sich durch das Internalisieren vergrößern. Es kann
ein Zustand divergent werden, wenn von diesem bereits lokal ein divergenter
Zustand aus erreichbar war oder wenn er eine unendliche Folge von Aktionen aus
$X\cup \{\tau\}$ ausführen konnte, jedoch nur endlich viele davon $\tau$s
waren. Durch die zusätzlichen Divergenz"=Zustände vergrößern sich alle
Trace-Mengen, die in der Präkongruenz \DRel{} betrachtet werden.\\
Um den Zusätzlichen Aufwand der Untersuchung möglichst gering zu halten, wird
dieses Teilkapitel auf endliche \MEIO{}s beschränkt. Falls man unendlich große
\MEIO{}s zulassen würde, müsste man an anderen Stellen
Endlichkeits"=Voraussetzungen machen.\\
Die Menge $X$ muss Teilmenge der Outputs $O$ sein, für einen endlichen \MEIO{}
kann $X$ ebenfalls nur endlich sein. Um eine endliche Menge von Aktionen zu
internalisieren, kann man jede Aktion einzeln aus dem entsprechend \MEIO{}
entfernen. Der folgende Satz kann also darauf beschränkt werden, dass nur ein
einzelnen Output verborgen werden soll. $P/o$ soll dabei für $P/\{o\}$ stehen.
Die Menge $\EDT _{P/o}$ kann aus $\EDT _P$ konstruiert werden. Im allgemeinen
ist die Menge $\EDT _{P/o}$ jedoch größer wie $\EDT _P$ somit müssen die
anderen Trace-Mengen mit der neuen Menge geflutet werden.


\begin{Satz}[Divergenz-Präkongruenz bzgl.\ Internalisierung]
  \label{DivHidingSatz}
  Seien $P_1$ und $P_2$ zwei endliche \MEIO{}s für die $P_1\DRel P_2$ gilt,
  dann folgt auch die Gültigkeit von $P_1/X\DRel P_2/X$. Die Relation \DRel{}
  ist also ein Präkongruenz bezüglich $\cdot /\cdot$ für endliche \MEIO{}s. Es
  gilt für die Sprachen und Traces:
  \begin{enumerate}[(i)]
    \item $L(P/o) = \{w\in (\Sigma\backslash \{o\})^*\mid \exists w'\in L(P):
      w'|_{\Sigma\backslash \{o\}} = w\}$,
    \item $\EDT (P/o) = \cont (\prune (\{w\in (\Sigma\backslash \{o\})^*\mid
      \exists w':w'|_{\Sigma\backslash \{o\}} = w\land \forall n\geq 0:
      w'o^n\in\EDL (P)\}))$,
    \item $\EDL (P/o) = \{w\in (\Sigma\backslash \{o\})^*\mid \exists w'\in
      \EDL(P): w'|_{\Sigma\backslash \{o\}} = w\} \cup \EDT (P/o)$,
    \item $\QDT (P/o) = \{w\in (\Sigma\backslash \{o\})^*\mid \exists w'\in
      \QDT(P): w'|_{\Sigma\backslash \{o\}} = w\} \cup \EDT (P/o)$.
  \end{enumerate}
\end{Satz}
\begin{proof}
  Die Präkongruenz-Eigenschaft lässt sich wie bei den
  Sätzen~\ref{FehlerHidingSatz} und~\ref{StilleHidingSatz} aus den Aussagen
  über die Sprachen und Traces folgern. Somit sollen nun zunächst (i) bis (iv)
  nachgewiesen werden. Der Punkt (i) folgt aus (i) von
  Satz~\ref{FehlerHidingSatz} bzw.~\ref{StilleHidingSatz}.

  (ii) \glqq $\subseteq$\grqq{}:\\
  Beide Seiten nicht abgeschlossen gegenüber \prune{} und \cont{}. Somit genügt
  es ein Element $w$ aus $\StET (P/o)\cup \StDT (P/o)$ zu betrachten. Es gibt
  also einen Ablauf in $P/o$ für das $w$, der zu einem Zustand $p$ führt, der
  in $E_{P/o}\cup Div_{P/o}$ enthalten ist. Der selbe Zustand $p$ kann analog
  zu den Sätzen~\ref{FehlerHidingSatz} und~\ref{StilleHidingSatz} in $P$ durch
  ein $w'$ erreicht werden mit $w'|_{\Sigma\backslash\{o\}} = w$.\\
  Falls $p$ in $P$ ein Fehler- oder Divergenz"=Zustand lokal erreichen kann,
  gilt $w'\in \EDT (P)\subseteq\EDL (P)$. Da die Menge \EDL{} unter \cont{}
  abgeschlossen ist, gilt auch $w'o^n\in\EDT (P)\subseteq\EDL (P)$ für alle
  $n\geq 0$, falls $p$ lokal Fehler oder Divergenz erreichen kann.\\
  Der Zustand $p$ kann jedoch in $P$ auch ein Zustand sein, der weder einen
  lokale erreichbaren Fehler noch lokal erreichbare Divergenz aufweist. Es muss
  also durch das Internalisieren des Outputs $o$ in $P/o$ ein Divergenz
  entstanden sein. Da nur $o$s in $\tau$s umgewandelt wurden, muss $p$ in $P$
  eine unendliche Folge an $o$s ausführen können, damit durch das Anwenden des
  Hiding"=Operators neue Divergenz entstehen kann. Es muss also $w'o^n$ für
  alle $n\geq 0$ in $P$ ausführbar sein. Somit gilt $w'o^n\in\EDL (P)$ für alle
  $n\geq 0$.

  (ii)  \glqq $\supseteq$\grqq{}:\\
  Für ein beliebiges $w'o^n\in \EDL (P)$ können zwei Fälle unterschieden
  werden.
  \begin{itemize}
    \item Fall 1 ($w'o^n\in\EDT (P)$): Es muss ein Präfix $v'$ von $w'$ geben,
      dass in $\PrET (P)\cup \PrDT (P)$ enthalten ist. \TODO{fertig beweisen}
    \item Fall 2 ($w'o^n\in L (P)$): \TODO{zu beweisen}
  \end{itemize}

  (iii):\\
  \TODO{zu beweisen}

  (iv):\\
  \TODO{zu beweisen}
\end{proof}

\TODO{Begründung für nachfolgendes Korollar}

\begin{Kor}[Divergenz-Präkongruenz mit Internalisierung]
  Die Relation \DRel{} ist eine Präkongruenz bezüglich $\cdot |\cdot$.
\end{Kor}
