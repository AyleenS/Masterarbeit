\section{Zusammenhänge}

\begin{Satz}[Zusammenhang der Verfeinerungs-Relationenen mit der
  Stillstands-Relation]
  \label{ZusammenhStilleSatz}
  Für \MEIO{}s $P$ und $Q$ gilt $P \asRel Q \Rightarrow P \QRel Q \Rightarrow
  P \ERel Q$. Die Implikationen in die entgegengesetzte Richtung gelten jedoch
  nicht. Die Relationen \wasRel{} und \QRel{} sind unvergleichbar.
\end{Satz}
\begin{proof}\mbox{}\\
  $P \asRel Q \Rightarrow P \QRel Q$:\\
  Im Beweis des Satzes~\ref{ZusammenhFehlerSatz} wurde bereits bewiesen, dass
  eine schwache as"=Verfeinerungs"=Relation $\mathcal{R}$ zwischen $P$ und $Q$
  die Eigenschaften der Fehler"=Relation \ERel{} erfüllt. Nach
  Lemma~\ref{ZusammenhWasAsLem} ist jede starke as"=Verfeinerungs"=Relation
  auch eine schwache. Es fehlt also für diese Implikation nur noch der Beweis
  der Inklusion $\QET _P \subseteq \QET _Q$. Da bereits $\ET _P\subseteq \ET
  _Q$ beweisen wurde, reicht es aus zu beweisen, dass $\StQT _P \backslash \ET
  _P\subseteq \QET _Q$ gilt. Die as"=Verfeinerungs"=Relation $\mathcal{R}$
  zwischen $P$ und $Q$ ist in diesem Fall als stark anzunehmen. Für ein Wort
  $w$ aus $\StQT _P \backslash \ET _P$ gibt es einen ausführbaren Trace der
  Form wie in Lemma~\ref{AblaefeVerfSpezLem}, der mit $p_n$ einen stillen
  Zustand erreicht. Es gilt $w\in\cont (\StET _Q) \subseteq\ET _Q\subseteq\QET
  _Q$, falls ein Fehler-Zustand in $Q$ auf einem Präfix-Trace von $w$ auftritt,
  wegen Lemma~\ref{AblaefeVerfSpezLem}. Im Folgenden wird davon ausgegangen,
  das $w$ in $Q$ ohne das Erreichen eines Fehler"=Zustandes ausführbar ist.
  Wegen~\ref{AblaefeVerfSpezLem} gibt es also einen Ablauf für $w$ in $Q$, der
  in einem Zustand $q_n$ mit $p_n \mathcal{R} q_n$ endet. $p_n$ ist für $P$ ein
  stiller Zustand, es gilt also für alle $\omega \in (O\cup \{\tau\})$ $p_n
  \nmust[\omega]_P$. Da $\mathcal{R}$ eine starke as"=Verfeinerungs"=Relation
  ist, muss wegen~\ref{SimDef}.2 auch $q_n \nmust[\omega]_Q$ gelten für alle
  $\omega \in (O\cup \{\tau\})$. $q_n$ ist also auch ein stiller Zustand. Somit
  ist $w$ in $\StQT _Q\subseteq\QET _Q$ enthalten.

  $P \QRel Q \Rightarrow P \ERel Q$:\\
  Diese Implikation folgt direkt aus der Definition von \QRel{}
  in~\ref{StilleSemDef}. Da $P \QRel Q$ dort definiert wurde als Relation, die
  $P \ERel Q$ und $\QET _P \subseteq \QET _Q$ erfüllt. Es gilt also $P \ERel Q$.

  $P \wasRel Q \not\Rightarrow P \QRel Q$:\\
  Diese Implikation scheitert dran, dass für stille Zustände keine
  $\tau$"=must"=Transitionen zulässig sind und \QRel{} keine Divergenz
  betrachtet. Ein entsprechendes Gegenbeispiel ist in
  Abbildung~\ref{QuiWasGegenBsp} dargestellt. Da hier nur die strikten
  Stille"=Trace das Problem erzeugen, soll $I = \emptyset$ gelten, damit die
  \MIT{}-Mengen für beide Systeme leer sind. $\mathcal{R} = \{(p_0,q_0)\}$ ist
  eine schwache as"=Verfeinerungs"=Relation zwischen $P$ und $Q$. Beide
  Transitionssysteme enthalten keine Fehler"=Zustände, $P$ besitzt keine
  Transitionen und $Q$ nur eine Transition für eine interne Aktion, somit sind
  die Punkt 1.,2.,4.\ und 5.\ der Definition~\ref{wSimDef} für $\mathcal{R}$
  sicher erfüllt. Der dritte Punkt von~\ref{wSimDef} fordert, dass die
  Transition $q_0 \must[\tau]_Q q_0$ in $P$ schwach gematched wird. Da
  $\hat{\tau}$ jedoch $\varepsilon$ entspricht, muss es keine echte interne
  Transition in $P$ für $p_0$ geben. Die Definition~\ref{wSimDef}.3 ist also
  ebenfalls durch das bereits enthaltene Tupel $(p_0,q_0)$ erfüllt.\\
  Für $P$ ist $\varepsilon$ ein strikter Stille"=Trace. Die Menge $\StQT (Q) =
  \QET (Q)$ ist jedoch leer. Die Inklusion $\QET (P) \subseteq \QET (Q)$, die
  für $P\QRel Q$ gelten müsste, ist also nicht erfüllt.

  \begin{figure}[htbp]
    \begin{center}
      \begin{tikzpicture}[->, >=latex', auto,node distance=2.5cm, semithick]
        \node [initial,initial text=$P$:, rectangle, draw, dotted] (p0) at
        (0,0) {$p_0\in Qui _P$};

        \node [initial,initial text=$Q$:] (q0) at (7,0) {$q_0$};

        \path
        (q0) edge[loop right] node{$\tau$} (q0)
        ;
      \end{tikzpicture}
      \caption{Gegenbeispiel zu $\wasRel \Rightarrow \QRel$ mit $I_P = I_Q =
      \emptyset$}
      \label{QuiWasGegenBsp}
    \end{center}
  \end{figure}

  $P \wasRel Q \hspace{0.1cm}\not\hspace{-0.1cm}\Leftarrow P \QRel Q$:\\
  Wie im Gegenbeispiel für die analoge Implikation aus der Relation \ERel{}
  beruht der Grund für die nicht Gültigkeit hier auch darauf, dass Simulationen
  strenger sind als Sprach"=Inklusionen. Jedoch funktioniert hier nicht das
  gleiche Gegenbeispiel wie im letzten Kapitel, da es zu Problemen mit den
  Stille-Traces führen würde. Um diese zu vermeiden, wird wieder die Technik
  angewendet an alle Zustände eine $\tau$-Schleife anzufügen. Das daraus
  entstehende Gegenbeispiel ist in Abbildung~\ref{WasQuiGegenBsp} dargestellt.
  Die Menge der Inputs $I$ der beiden \MEIO{}s ist leer. Es gilt also $\ET _P =
  \ET _Q = \QET _P = \QET _Q = \emptyset$ und $\{\varepsilon\} = L(P) \subset
  L(Q) = \{\varepsilon , o\}$.\\
  Angenommen es gib eine schwache as"=Verfeinerungs"=Relation $\mathcal{R}$
  zwischen $P$ und $Q$. Dann stehen die Startzustände in dieser Relation, es
  gilt also $p_0 \mathcal{R} q_0$. Da es keine Fehler"=Zustände in $P$ gibt,
  ist~\ref{wSimDef}.1 erfüllt. Die Transition $q_0 \must[o]_Q q_1$ fordert
  durch~\ref{wSimDef}.3 ihre Verfeinerung in $P$. Da es jedoch keine mit $o$
  beschriftete Transition in $P$ gibt, kann $\mathcal{R}$ keine schwache
  as"=Verfeinerungs"=Relation zwischen $P$ und $Q$ sein.

  \begin{figure}[htbp]
    \begin{center}
      \begin{tikzpicture}[->, >=latex', auto,node distance=2.5cm, semithick]
        \node [initial,initial text=$P$:] (p0) at (0,0) {$p_0$};

        \path
        (p0) edge[loop right] node{$\tau$} (p0)
        ;

        \node [initial,initial text=$Q$:] (q0) at (6,0) {$q_0$};
        \node (q1) [right of=q0] {$q_1$};

        \path
        (q0) edge node{$o!$} (q1)
        (q0) edge[loop above] node{$\tau$} (q0)
        (q1) edge[loop right] node{$\tau$} (q1)
        ;
      \end{tikzpicture}
      \caption{Gegenbeispiel zu $\wasRel \Leftarrow \QRel$}
      \label{WasQuiGegenBsp}
    \end{center}
  \end{figure}

  $P \asRel Q \hspace{0.1cm}\not\hspace{-0.1cm}\Leftarrow P \QRel Q$:\\
  Falls diese Implikation gelten würde, würde jedes Paar von \MEIO{}s, das in
  der Relation \QRel{} steht auch in der Relation \asRel{} stehen. Mit
  Lemma~\ref{ZusammenhWasAsLem} würde draus folgen, dass das \MEIO{} Paar auch
  in der Relation \wasRel{} stehen muss Dies stellt jedoch ein Widerspruch zur
  Unvergleichbarkeit von \QRel{} und \wasRel{} dar.

  $P \QRel Q \hspace{0.1cm}\not\hspace{-0.1cm}\Leftarrow P \ERel Q$:\\
  Die Relation \QRel{} stützt sich auf die Definition der Relation \ERel{}.
  Jedoch erweitert sie die Definition noch um eine weitere Voraussetzung. Es
  muss also in einem entsprechenden Gegenbeispiel die Inklusion $\QET _P
  \subseteq \QET _Q$ verletzt sein. Das Gegenbeispiel ist in
  Abbildung~\ref{QuiEGegenBsp} dargestellt. Es wird $I = \emptyset$
  vorausgesetzt, damit keine Input-kritischen Traces auftreten. Es gilt also
  $\ET _P = \ET _Q = \emptyset$ und $L(P) = L(Q) = \{\varepsilon\}$. Deshalb
  gilt auch $P\ERel Q$.\\
  Jedoch gilt für die strikten Stille-Traces $\StQT _P = \{\varepsilon\}$ und
  $\StQT _Q = \emptyset$. Es folgt also $\QET _P \not\subseteq \QET _Q$ und
  somit ist die Relation \QRel{} zwischen $P$ und $Q$ auch nicht erfüllt.

  \begin{figure}[htbp]
    \begin{center}
      \begin{tikzpicture}[->, >=latex', auto,node distance=2.5cm, semithick]
        \node [initial,initial text=$P$:] (p0) at (0,0) {$p_0$};

        \path
        (p0) edge[loop right] node{$\tau$} (p0)
        ;

        \node [initial,initial text=$Q$:, rectangle, draw, dotted] (q0) at
        (7,0) {$q_0 \in Qui _Q$};
      \end{tikzpicture}
      \caption{Gegenbeispiel zu $\QRel \Leftarrow \ERel$ mit $I_P = I_Q =
      \emptyset$}
      \label{QuiEGegenBsp}
    \end{center}
  \end{figure}
\end{proof}

Alternativ wäre es auch möglich gewesen eine andere Betrachtung zu wählen, die
für stille Zustände zunächst nur fordert, dass keine must"=Outputs möglich sein
dürfen. Falls dieser Zustand die Möglichkeit für eine interne Aktion via einer
must"=Transition hat, darf durch die $\tau$s niemals ein Zustand erreicht
werden, von dem aus ein Output in Implementierungen sicher gestellt wird. Die
Menge der stillen Zustände würde in der alternativen Betrachtung durch
$\left\{p\in P\mid \forall a\in O: p\nweakmust[a]_P \right\}$ beschrieben
werden. Diese Menge ist in der hier betrachteten Menge der stillen Zustände
$Qui (P)$ enthalten. Zusätzlich zu den in dieser Arbeit betrachteten
Verklemmungen der Art Deadlock, lässt die Betrachtungsweise mit den
zugelassenen $\tau$-must"=Transitionen auch Verklemmungen der Art Livelock zu,
da diese Zustände möglicherweise beliebig viele interne Aktionen ausführen
können, jedoch nie aus eigener Kraft einen wirklichen Fortschritt in Form eines
Outputs bewirken können müssen. Somit wären dies alle Zustände, die keine
Möglichkeit haben ohne einen Input von Außen oder eine implementierte
may"=Output"=Transition je wieder einen Output machen zu können. Falls man
diese Definition verwenden würde, müsste man immer alle Zustände betrachten,
die durch $\tau$s erreichbar sind. Es ist unklar zu welchen Konsequenzen dies
führen kann. Vor allem im Bezug auf Hiding ist es schwierig, da
unterschiedliche Transitionen in einem System zu unterschiedlichem Verhalten
führen, obwohl sie von der Transitionsbeschriftung her nicht zu unterscheiden
sind. Hierzu ist auch das Beispiel in Abbildung~\ref{QuiTracesBsp} zu beachten.
Die Sprachen beider \MEIO{}s sind gleich und auch die strikten Stille"=Traces
stimmen überein, da jeweils durch beliebig viele $o$s (mindestens eins), die
von einem $o'$ gefolgt werden eine ruhiger Zustand erreicht wird. Bei $P$ ist
dies immer möglich. Falls aber in $Q$ die erste $o$ Transition nach rechts
ausgeführt wird, kann kein stiller Zustand mehr erreicht werden. Die \MEIO{}s
können also auf Trace-Ebenen nicht von einander unterschieden werden, jedoch
können durch lokal Entscheidungen sehr verschiedenen Ergebnisse entstehen.
Falls man das $o$ durch Hiding in ein $\tau$ umwandelt, wäre $p_0$ und $p_1$
sicher nicht still. $q_{12}$ hingegen würde in der alternativen Betrachtung
still werden. Die davor über die Trace-Mengen nicht unterscheidbaren Systeme
würden durch Hiding dann einen Unterschied in den Trace-Mengen aufweisen.

\begin{figure}[htbp]
  \begin{center}
    \begin{tikzpicture}[->, >=latex', auto,node distance=2.5cm, semithick]
      \node [initial above,initial text=$P$:] (p0) at (0,0) {$p_0$};
      \node (p1) [below of=p0] {$p_1$};
      \node (p2) [below of=p1] {$p_2$};

      \path
      (p0) edge node{$o$} (p1)
      (p1) edge[loop left] node{$o$} (p1)
      (p1) edge node{$o'$} (p2)
      ;

      \node [initial above,initial text=$Q$:] (q0) at (7,0) {$q_0$};
      \node (q11) [below left of=q0] {$q_{11}$};
      \node (q12) [below right of=q0] {$q_{12}$};
      \node (q2) [below of=q11] {$q_2$};

      \path
      (q0) edge node[swap]{$o$} (q11)
      (q0) edge node{$o$} (q12)
      (q11) edge node{$o'$} (q2)
      (q11) edge[loop left] node{$o$} (q11)
      (q12) edge[loop right] node{$o$} (q12)
      ;
    \end{tikzpicture}
    \caption{Beispiel für Probleme bei Trace-Betrachtung}
    \label{QuiTracesBsp}
  \end{center}
\end{figure}

Im nächsten Kapitel, in dem Zustände mit Divergenz betrachtet werden, heben
sich die Unterschiede in den Trace-Mengen, die die beiden unterschiedlichen
Ansätze hier erzeugen würden, auf. Da die Zustände, die eine unendliche Folge
an $\tau$s ausführen können, als divergent betrachtet werden und die Stille als
nicht so \glqq schlimm\grqq{} angesehen wird und auf Trace-Ebene somit geflutet
wird.
