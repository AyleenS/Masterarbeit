\chapter{Einleitung}

Durch modale Spezifikation können gesamte Systeme oder einzelne Komponenten,
die nebenläufig zu einander ausgeführt werden sollen, modelliert werden. Die
Modalitäten geben einem dabei die Freiheit Forderungen an potentielle
Implementierungen zu stellen. Die Forderungen beinhalten die Spezifikation von
Verhalten, dass zwingendermaßen umgesetzt werden muss und zusätzlich noch den
erlaubten Spielraum.\\
Die modalen Spezifikationen werden hier als Transitionssysteme aufgefasst.\\
Durch nebenläufige Kommunikation zwischen unterschiedlichen Komponenten kann es
zu Fehler kommen oder Implementierungen können bereits Fehler enthalten. In der
Praxis ist wäre es wünschenswert nur Implementierungen zu betrachten, die
fehler-frei sind. In dem auch in der Interaktion mit einem beliebigen System
ein Fehler erreicht werden kann. Dies zu erreichen ist jedoch recht schwierig.
Die Betrachtungen, die in dieser Arbeit gemacht werden, sollen jedoch dabei
helfen, die Fehlerquellen möglichst gut einschränken zu können. Dazu müssen
jedoch in machen Untersuchungen auch Implementierungen verwendet werden, die
Fehler enthalten.\\
Es wird in dieser Arbeit auch eher davon ausgegangen, dass die Kommunikation
mit einer hilfreichen Umgebung bzw. User statt findet. Dadurch sind Systemen
mit Fehler nicht automatisch schlecht, sondern man kann optimistisch davon
ausgehen, dass Fehler erst ein Problem sind, wenn sie lokal erreicht werden
können. Sobald ein Fehler jedoch lokal erreichbar ist, kann selbst eine
hilfreiche Umgebung nicht mehr verhindern, dass der Fehler auftritt.\\
In dem hier verwendeten Ansatz wird im Gegensatz zu den \MIA{}s
aus~\cite{Vogler2016MIA3} auf das abscheiden von Wegen die zu Fehlern führen
verzichtet. Dadurch können auch nach einer Kommunikation zwischen zwei Systemen
noch Rückschlüsse auf die Quellen der Fehler gezogen werden. Dies ermöglicht es
Spezifikationen gezielt verbessern zu können, falls dies gewünscht ist.

\TODO{Literaturquellen einfügen und weiter schreiben}
