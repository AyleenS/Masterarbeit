\chapter{Einleitung}

Interfaces werden oft verwendet um Komplexe Systeme zu entwerfen. Damit kann
bereits während des Designs überprüft werden, wie gut die einzelnen Komponenten
zusammenarbeiten. Durch Interfaces können nebenläufige Systeme Komponentenweise
entworfen werden. Einige der Theorien über Interfaces basieren auf den
Interface Automaten aus~\cite{Alfaro2004}. Dort ist eine
Parallelkomposition auf Labelled Transitionssystemen mit Inputs und Outputs der
charakterisierende Punkt. Wenn ein unerwarteter Input empfangen wird, wird
dieser als Fehler aufgefasst, d.h.\ es kommt zu einem Kommunikationsfehler
zwischen den Systemen. In so genannten pessimistischen Ansätzen wie
in~\cite{Bauer2010} ist die Parallelkomposition zweier Systemen nicht
definiert, wenn ein solcher Fehler durch ihre Kommunikation entstehen würde. In
optimistischen Ansätzen, wie z.B.\ in~\cite{Luttgen2013MIA1}
und~\cite{Vogler2016MIA3}, wird ein Kommunikationsfehler so lange als
akzeptabel angesehen, solange die Systemumgebung verhindert, dass er erreicht
wird. In dieser Arbeit soll auch die optimistische Sichtweise angewendet
werden. Jedoch sollen im Gegensatz zu den \MIA{}s aus~\cite{Luttgen2013MIA1}
und~\cite{Vogler2016MIA3} die Zustände von denen aus ein Fehler von einer
hilfreichen Umgebung nicht mehr verhindert werden kann, nicht aus der
Parallelkomposition entfernt werden.\\
Für die Betrachtung der Fehler mit Hilfe von Trace-Mengen orientiert sich diese
Arbeit an~\cite{Vogler2014EIO} und~\cite{Schinko2016BA}.\\
Interface Automaten wurden schon in mehreren Veröffentlichungen mit Modalen
Transitions Systemen (MTS) aus~\cite{Larsen1989} kombiniert. Die Kombination
die hier als Grundlage dient sind die \MIA{}s aus~\cite{Luttgen2013MIA1}
und~\cite{Vogler2016MIA3}. Die \MIA{}s enthalten disjunktive must"=Transitionen
in ihrer Definition. Solche must"=Transitionen sollen in dieser Arbeit nicht
betrachtet werden. Für die must- und may"=Transitionen sollen hier die gleiche
Form besitzen, so dass via einer Transition von einem Zustand immer nur ein
Zustand erreicht wird. Dieser Unterschied bezüglich der must"=Transitionen wird
vor allem im Vergleich der hier verwendeten Transitionssysteme mit den~\MIA{}s
am Ende des Kapitels~\ref{errorZusammenh} relevant.

Durch modale Spezifikation können gesamte Systeme oder einzelne Komponenten,
die nebenläufig zu einander ausgeführt werden sollen, modelliert werden. Die
Modalitäten geben einem dabei die Freiheit Forderungen an potentielle
Implementierungen zu stellen. Die Forderungen beinhalten die Spezifikation von
Verhalten, dass zwingendermaßen umgesetzt werden muss und zusätzlich noch den
erlaubten Spielraum.\\
Die modalen Spezifikationen werden hier als Transitionssysteme aufgefasst.\\
Durch nebenläufige Kommunikation zwischen unterschiedlichen Komponenten kann es
zu Fehler kommen oder Implementierungen können bereits Fehler enthalten. In der
Praxis ist wäre es wünschenswert nur Implementierungen zu betrachten, die
fehler-frei sind. In dem auch in der Interaktion mit einem beliebigen System
ein Fehler erreicht werden kann. Dies zu erreichen ist jedoch recht schwierig.
Die Betrachtungen, die in dieser Arbeit gemacht werden, sollen jedoch dabei
helfen, die Fehlerquellen möglichst gut einschränken zu können. Dazu müssen
jedoch in machen Untersuchungen auch Implementierungen verwendet werden, die
Fehler enthalten.\\
Es wird in dieser Arbeit auch eher davon ausgegangen, dass die Kommunikation
mit einer hilfreichen Umgebung bzw. User statt findet. Dadurch sind Systemen
mit Fehler nicht automatisch schlecht, sondern man kann optimistisch davon
ausgehen, dass Fehler erst ein Problem sind, wenn sie lokal erreicht werden
können. Sobald ein Fehler jedoch lokal erreichbar ist, kann selbst eine
hilfreiche Umgebung nicht mehr verhindern, dass der Fehler auftritt.\\
In dem hier verwendeten Ansatz wird im Gegensatz zu den \MIA{}s
aus~\cite{Vogler2016MIA3} auf das abscheiden von Wegen die zu Fehlern führen
verzichtet. Dadurch können auch nach einer Kommunikation zwischen zwei Systemen
noch Rückschlüsse auf die Quellen der Fehler gezogen werden. Dies ermöglicht es
Spezifikationen gezielt verbessern zu können, falls dies gewünscht ist.
