\chapter{Verfeinerungen für Kommunikationsfehler- und Ruhe-Freiheit}

In diesem Kapitel wird die Menge der betrachteten Zustandsmengen von den
Kommunikations-basierten Fehlern im letzten Kapitel erweitert um
Ruhe-Zustände.\\
Zustände, die keine Outputs ohne einen Input ausführen können in Form einer
must"=Transition, werden als in einer Art Verklemmung angesehen, da sie ohne
Zutun von Außen den Zustand nicht mehr verlassen können, falls möglicherweise
vorhandener may"=Output nicht implementiert wird. So ein Zustand hat also keine
must"=Transitions"=Möglichkeiten für einen Output. Falls dieser Zustand die
Möglichkeit für eine interne Aktion via einer must"=Transition hat, darf durch
die $\tau$s niemals ein Zustand erreicht werden, von dem aus ein Output in
Implementierungen erzwungen wird. Ein Zustand, der keine Outputs und $\tau$s
via must"=Transitionen ausführen kann, ist also ein Deadlock-Zustand, in denen
das System nichts mehr tun können muss ohne einen Input. Wenn man eine
Erweiterung um $\tau$s zu Zuständen ohne must"=Outputs zulässt, hat man
zusätzlich noch Verklemmungen der Art Livelock, da diese Zustände
möglicherweise beliebig viele interne Aktionen ausführen können, jedoch nie aus
eigener Kraft einen wirklichen Fortschritt in Form eines Outputs bewirken
können müssen. Die Menge der Zustände, die sich in einer Verklemmung
befinden, würde also durch $\left\{p\in P\mid \forall a\in O: p\nweakmust[a]
\right\}$ beschrieben werden. Somit wären dies alle Zustände, die keine
Möglichkeit haben ohne einen Input von Außen oder eine implementierte
may"=Output"=Transition je weider einen Output machen zu können. Falls man
diese Definition verwenden würde, müsste man immer alle Zustände betrachten,
die durch $\tau$s erreichbar sind. Dies würde einige Betrachtungen deutlich
aufwendiger machen und soll deshalb hier nicht behandelt werden. Die Definition
für die betrachteten Verklemmungen, hier Ruhe genannt, beschränkt sich auf
Zustände, die keine Outputs und $\tau$s ausführen können.

\begin{Def}[Ruhe]
  Ein \emph{Ruhe-Zustand} ist ein Zustand in einem \MEIO{} $P$, der keine
  Outputs und kein $\tau$ zulässt.\
  Somit ist die Menge der Ruhe-Zustände in einem \MEIO{} $P$ wie folgt formal
  definiert: $Qui(P):=\left\{p\in P\mid \forall\alpha\in (O\cup\{\tau\}):p
  \nmust[\alpha]\right\}$.
\end{Def}

\TODO{Proposition Ruhe-Zustände eines \MEIO{}s via seiner as-Implementierungen}

Für die Erreichbarkeit wird wie im letzten Kapitel ein optimistischer Anzahl
der lokalen Erreichbarkeit für die Fehler-Zustände, der Kommunikationsfehler,
verwendet. Ruhe ist kein unabwendbarer Fehler, sondern kann durch einen Input
repariert werden oder im Fall von vorhandenen may"=Output"=Transitionen, durch
eine Implementierung dieses Outputs. Daraus ergibt sich, dass Ruhe im Vergleich
zu Kommunikationsfehler als weniger \glqq schlimmer Fehler\grqq{} anzusehen
ist. Somit ist ein Ruhe-Zustand ebenso wie ein Fehler-Zustand erreichbar,
sobald er durch Outputs und $\tau$s erreicht werden kann, jedoch ist nicht jede
beliebige Fortsetzung eines Traces, das durch lokale Aktionen zu einem
Ruhe-Zustand führt ein Ruhe"=Trace.

\begin{Def}[fehler- und ruhe-freie Kommunikation]
  Zwei \MEIO{}s $P_1$ und $P_2$ \emph{kommunizieren fehler- und ruhe-frei},
  wenn keine as"=Implementierung ihrer Parallelkomposition $P_{12}$ einen
  Fehler- oder Ruhe-Zustand lokal erreichen kann.
\end{Def}

\begin{Def}[Ruhe-Verfeinerungs-Basisrelation]
  Für \MEIO{}s $P_1$ und $P_2$ mit der gleichen Signatur wird $P_1\QBRel P_2$
  geschrieben, wenn ein Fehler- oder Ruhe-Zustand in einer as"=Implementierung
  von $P_1$ nur dann lokal erreichbar ist, wenn es auch eine
  as"=Implementierung von $P_2$ gibt, in der ein solcher lokal erreichbar ist.
  Diese \emph{Basisrelation} stellt eine \emph{Verfeinerung} bezüglich
  \emph{Kommunikationsfehlern} und \emph{Ruhe} dar.\\
  \QCRel{} bezeichnet die \emph{vollständig abstrakte Präkongruenz} von
  \QBRel{} bezüglich $\cdot\|\cdot$.
\end{Def}

Um eine genauere Auseinandersetzung mit den Präkongruenzen zu ermöglichen,
benötigt man wie im letzten Kapitel die Definition von Traces auf der Struktur.
Dadurch erhält man die Möglichkeit die gröbste Präkongruenz charakterisieren zu
können. Wie bereits oben erwähnt, ist Ruhe ein reparierbarer Fehler im
Gegensatz zu Kommunikationsfehlern. Es genügt deshalb für Ruhe die strikten
Traces ohne Kürzung zu betrachten.

\begin{Def}[Ruhe-Traces]
  Sei $P$ ein \MEIO{} und definiere:
  \begin{itemize}
    \item strikte Ruhe-Traces: $\StQT (P) := \left\{w\in\Sigma ^*\mid p_0
      \weakmay[w] p\in Qui(P)\right\}$.
  \end{itemize}
\end{Def}

\TODO{Proposition Ruhe-Traces via as-Implementierungen}

Für \ET{} und \EL{} gelten die Definitionen aus dem letzten Kapitel. Es wir nur
für Ruhe eine neue Semantik definiert.

\begin{Def}[Ruhe-Semantik]
  Sei $P$ ein \MEIO{}.
  \begin{itemize}
    \item Die Menge der \emph{fehler-gefluteten Ruhe-Traces} von $P$ ist $\QET
      (P):= \StQT (P)\cup\ET (P)$.
  \end{itemize}
  Für zwei \MEIO{}s $P_1,P_2$ mit der gleichen Signatur wird $P_1\QRel{} P_2$
  geschrieben, wenn $P_1\ERel{} P_2$ und $\QET _1\subseteq \QET _2$ gilt.
\end{Def}

\begin{Lem}[Ruhe-Zustände unter Parallelkomposition]
  \begin{enumerate}
    \item Ein Zustand $(p_1,p_2)$ aus der Parallelkomposition $P_{12}$ ist
      ruhig, wenn es auch die Zustände $p_1$ und $p_2$ in $P_1$ bzw. $P_2$
      sind.
    \item Wenn der Zustand $(p_1,p_2)$ ruhig ist und nicht in $E_{12}$
      enthalten ist, dann sind auch die auf die Teilsysteme projizierten
      Zustände $p_1$ und $p_2$ ruhig.
  \end{enumerate}
\end{Lem}

\begin{proof}
  \TODO{zu beweisen}
\end{proof}

\begin{Satz}[Kommunikationsfehler- und Ruhe-Semantik für Parallelkompositionen]
  Für zwei komponierbare \MEIO{}s $P_1,P_2$ und ihre Komposition $P_{12}$ gilt:
  \begin{enumerate}
    \item $\ET _{12}=\cont (\prune ((\ET _1\|\EL _2)\cup (\EL _1\|\ET _2)))$,
    \item $\QET _{12}=(\QET _1\|\QET _2)\cup\ET _{12}$,
    \item $\EL _{12}=(\EL _1\|\EL _2)\cup\ET _{12}$.
  \end{enumerate}
\end{Satz}

\begin{proof}
  Es wird nur der 2. Punkt beweisen.\\
  \TODO{zu beweisen}
\end{proof}
