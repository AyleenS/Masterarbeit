\chapter{Verfeinerungen für Kommunikationsfehler- und Ruhe-Freiheit}

In diesem Kapitel wird die Menge der betrachteten Zustandsmengen von den
Kommunikations-basierten Fehlern im letzten Kapitel erweitert um
Ruhe-Zustände.\\
Zustände, die keine Outputs ohne einen Input ausführen können in Form einer
must"=Transition, werden als in einer Art Verklemmung angesehen, da sie ohne
Zutun von Außen den Zustand nicht mehr verlassen können, falls möglicherweise
vorhandener may"=Output nicht implementiert wird. So ein Zustand hat also keine
must"=Transitions"=Möglichkeiten für einen Output. Falls dieser Zustand die
Möglichkeit für eine interne Aktion via einer must"=Transition hat, darf durch
die $\tau$s niemals ein Zustand erreicht werden, von dem aus ein Output in
Implementierungen erzwungen wird. Ein Zustand, der keine Outputs und $\tau$s
via must"=Transitionen ausführen kann, ist also ein Deadlock-Zustand, in denen
das System nichts mehr tun können muss ohne einen Input. Wenn man eine
Erweiterung um $\tau$s zu Zuständen ohne must"=Outputs zulässt, hat man
zusätzlich noch Verklemmungen der Art Livelock, da diese Zustände
möglicherweise beliebig viele interne Aktionen ausführen können, jedoch nie aus
eigener Kraft einen wirklichen Fortschritt in Form eines Outputs bewirken
können müssen. Die Menge der Zustände, die sich in einer Verklemmung
befinden, würde also durch $\left\{p\in P\mid \forall a\in O: p\nweakmust[a]_P
\right\}$ beschrieben werden. Somit wären dies alle Zustände, die keine
Möglichkeit haben ohne einen Input von Außen oder eine implementierte
may"=Output"=Transition je weider einen Output machen zu können. Falls man
diese Definition verwenden würde, müsste man immer alle Zustände betrachten,
die durch $\tau$s erreichbar sind. Dies würde einige Betrachtungen deutlich
aufwendiger machen und soll deshalb hier nicht behandelt werden. Die Definition
für die betrachteten Verklemmungen, hier Ruhe genannt, beschränkt sich auf
Zustände, die keine Outputs und $\tau$s ausführen können.

\begin{Def}[Ruhe]
  Ein \emph{Ruhe-Zustand} ist ein Zustand in einem \MEIO{} $P$, der keine
  Outputs und kein $\tau$ zulässt via must"=Transitionen.\\
  Somit ist die Menge der Ruhe-Zustände in einem \MEIO{} $P$ wie folgt formal
  definiert: $Qui(P):=\left\{p\in P\mid \forall\alpha\in (O\cup\{\tau\}):p
  \nmust[\alpha]_P\right\}$.
\end{Def}

\begin{Prop}[Ruhe und Implementierung]
  Für ein \MEIO{} $P$ gilt: $Qui(P) = \left\{p'\in P'\mid P'\in\asimp (P)\land
  \forall\alpha\in (O\cup\{\tau\}):p' \nmust[\alpha]_{P'}\right\}$.
\end{Prop}
\begin{proof}
  Da für alle $P'\in\asimp (P)$ $\must_{P'} =\may_{P'}$ gilt, dürfen alle $p'$
  keine ausgehenden Transitionen haben. Nach Definition~\ref{SimDef} dürfen zu
  $p'$ in Relation stehenden Zustände $p$ in $P$ keine ausgehenden
  must"=Transitionen haben, da $p'$ diese sonst implementieren müsste. Somit
  sind alle $p'$, die die Forderung dieser Proposition enthalten auch in $Qui
  (P)$ enthalten. Für alle $p$, die in $Qui (P)$ enthalten sind, gibt es keine
  ausgehenden must"=Transitionen für Outputs oder die interne Aktion $\tau$,
  somit können diese Aktionen entweder keine ausgehende Transition von $p$ sein
  oder nur als may"=Transition von $p$ zu einem möglicherweise anderen Zustand
  führen. Nicht vorhandenen Transitionen in $P$ sind auch in keiner
  as"=Implementierung von $P$ enthalten und es gibt mindestens eine
  Implementierung in $\asimp (P)$, die alle ausgehenden may"=Transitionen der
  lokalen Aktionen von $p$ nicht implementiert, somit sind alle $p$ aus $Qui
  (P)$ auch in $\left\{p'\in P'\mid P'\in\asimp (P)\land \forall\alpha\in
  (O\cup\{\tau\}):p' \nmust[\alpha]_{P'}\right\}$ enthalten.
\end{proof}

Für die Erreichbarkeit wird wie im letzten Kapitel ein optimistischer Anzahl
der lokalen Erreichbarkeit für die Fehler-Zustände, der Kommunikationsfehler,
verwendet. Ruhe ist kein unabwendbarer Fehler, sondern kann durch einen Input
repariert werden oder im Fall von vorhandenen may"=Output"=Transitionen, durch
eine Implementierung dieses Outputs. Daraus ergibt sich, dass Ruhe im Vergleich
zu Kommunikationsfehler als weniger \glqq schlimmer Fehler\grqq{} anzusehen
ist. Somit ist ein Ruhe-Zustand ebenso wie ein Fehler-Zustand erreichbar,
sobald er durch Outputs und $\tau$s erreicht werden kann, jedoch ist nicht jede
beliebige Fortsetzung eines Traces, das durch lokale Aktionen zu einem
Ruhe-Zustand führt ein Ruhe"=Trace.

\begin{Def}[fehler- und ruhe-freie Kommunikation]
  Zwei \MEIO{}s $P_1$ und $P_2$ \emph{kommunizieren fehler- und ruhe-frei},
  wenn keine as"=Implementierung ihrer Parallelkomposition $P_{12}$ einen
  Fehler- oder Ruhe-Zustand lokal erreichen kann.
\end{Def}

\begin{Def}[Ruhe-Verfeinerungs-Basisrelation]
  Für \MEIO{}s $P_1$ und $P_2$ mit der gleichen Signatur wird $P_1\QBRel P_2$
  geschrieben, wenn ein Fehler- oder Ruhe-Zustand in einer as"=Implementierung
  von $P_1$ nur dann lokal erreichbar ist, wenn es auch eine
  as"=Implementierung von $P_2$ gibt, in der ein solcher lokal erreichbar ist.
  Diese \emph{Basisrelation} stellt eine \emph{Verfeinerung} bezüglich
  \emph{Kommunikationsfehlern} und \emph{Ruhe} dar.\\
  \QCRel{} bezeichnet die \emph{vollständig abstrakte Präkongruenz} von
  \QBRel{} bezüglich $\cdot\|\cdot$.
\end{Def}

Um eine genauere Auseinandersetzung mit den Präkongruenzen zu ermöglichen,
benötigt man wie im letzten Kapitel die Definition von Traces auf der Struktur.
Dadurch erhält man die Möglichkeit die gröbste Präkongruenz charakterisieren zu
können. Wie bereits oben erwähnt, ist Ruhe ein reparierbarer Fehler im
Gegensatz zu Kommunikationsfehlern. Es genügt deshalb für Ruhe die strikten
Traces ohne Kürzung zu betrachten.

\begin{Def}[Ruhe-Traces]
  \label{RuheTraceDef}
  Sei $P$ ein \MEIO{} und definiere:
  \begin{itemize}
    \item \emph{strikte Ruhe-Traces}: $\StQT (P) := \left\{w\in\Sigma ^*\mid
      p_0 \weakmay[w]_P p\in Qui(P)\right\}$.
  \end{itemize}
\end{Def}

\begin{Prop}[Ruhe-Traces und Implementierung]
  \label{QuiTraceProp}
  Für ein \MEIO{} $P$ gilt $\StQT (P) = \left\{w\in\Sigma ^*\mid \exists
  P'\in\asimp (P): p'_0 \weakmust[w]_{P'} p'\in Qui(P')\right\}$.
\end{Prop}
\begin{proof}
  Da $P'$ einen mit $w$ beschrifteten must"=Trace enthält und $P'$ eine
  as"=Implementierung von $P$ ist, muss $w$ als may"=Trace bereits in $P$
  möglich gewesen sein (Definition~\ref{SimDef}). Der Zustand $p'$ kann in $P'$
  auch nur ruhig sein, wenn der entsprechende in Relation stehende Zustand aus
  $P$ auch bereits ruhig war. Alle $w$ aus $\big\{w\in\Sigma ^*\mid \exists
  P'\in\asimp (P):$ $p'_0 \weakmust[w]_{P'} p'\in Qui(P')\big\}$ sind somit also
  auch in $\StQT (P)$ enthalten. Jeder may"=Trace aus $P$ kann implementiert
  werden durch eine as"=Implementierung. Die Menge $\asimp (P)$ enthält alle
  as"=Implementierungen von $P$, also auch eine, die den $w$ may"=Trace aus
  $\StQT (P)$ zu einem mit in $p$ Relation stehenden Zustand implementiert. $p$
  enthält keine ausgehenden Transitionen für lokale Aktionen, also gibt es
  unter den as"=Implementierungen, die $w$ entsprechend als must"=Trace
  enthalten auch mindestens eine, die keine ausgehende lokale Aktion an dem zu
  $p$ in Relation stehenden Zustand implementiert. Die Behauptung gilt also.
\end{proof}

Für \ET{} und \EL{} gelten die Definitionen aus dem letzten Kapitel. Es wir nur
für Ruhe eine neue Semantik definiert.

\begin{Def}[Ruhe-Semantik]
  Sei $P$ ein \MEIO{}.
  \begin{itemize}
    \item Die Menge der \emph{fehler-gefluteten Ruhe-Traces} von $P$ ist $\QET
      (P):= \StQT (P)\cup\ET (P)$.
  \end{itemize}
  Für zwei \MEIO{}s $P_1,P_2$ mit der gleichen Signatur wird $P_1\QRel{} P_2$
  geschrieben, wenn $P_1\ERel{} P_2$ und $\QET _1\subseteq \QET _2$ gilt.
\end{Def}

\begin{Lem}[Ruhe-Zustände unter Parallelkomposition]\mbox{}
  \label{RuheZustLem}
  \begin{enumerate}
    \item Ein Zustand $(p_1,p_2)$ aus der Parallelkomposition $P_{12}$ ist
      ruhig, wenn es auch die Zustände $p_1$ und $p_2$ in $P_1$ bzw. $P_2$
      sind.
    \item Wenn der Zustand $(p_1,p_2)$ ruhig ist und nicht in $E_{12}$
      enthalten ist, dann sind auch die auf die Teilsysteme projizierten
      Zustände $p_1$ und $p_2$ ruhig.
  \end{enumerate}
\end{Lem}
\begin{proof}\mbox{}
  \begin{enumerate}
    \item Da $p_1\in Qui_1$ und $p_2\in Qui_2$ gilt, haben diese beiden
      Zustände jeweils höchstens die Möglichkeit für Input Transitionen oder
      Output und $\tau$ may"=Transitionen, jedoch keine Möglichkeit für Outputs
      oder $\tau$s als must"=Transitionen.\\
      Angenommen der Zustand, der durch die Parallelkomposition aus den
      Zuständen $p_1$ und $p_2$ entsteht, ist nicht ruhig, d.h.\ er hat eine
      ausgehende must"=Transition für einen Output oder ein $\tau$.
      \begin{itemize}
        \item Fall 1 \big($(p_1,p_2)\must[\tau]_{12}$\big): Ein $\tau$ ist eine
          interne Aktion und kann in der Parallelkomposition nicht durch das
          Verbergen von Aktionen bei der Synchronisation entstehen. Ein $\tau$
          in der Parallelkomposition ist also auch nur möglich, wenn dies
          bereits für einen der beiden Zustände als must"=Transition im der
          einzelnen Komponente möglich war für einen der Zustände, aus denen
          $(p_1,p_2)$ zusammensetzt ist. Jedoch verbietet die Voraussetzung,
          dass $p_1$ oder $p_2$ eine ausgehende $\tau$ must"=Transition haben,
          deshalb kann auch $(p_1,p_2)$ keine solche Transition besitzen.
        \item Fall 2 \big($(p_1,p_2)\must[a]_{12}$ mit $a\in O_{12}\backslash
          \Synch(P_1,P_2)$\big): Da es sich bei $a$ um einen Output handelt, der
          nicht in $\Synch (P_1,P_2)$ enthalten ist, kann dieser nicht aus der
          Synchronisation von zwei Aktionen entstanden sein, sondern muss
          bereits für $P_1$ oder $P_2$ als must"=Transition ausführbar gewesen
          sein. Es gilt also \oBdA{} $p_1\must[a]_1$ mit $a\in O_1$. Dies ist
          jedoch aufgrund der Voraussetzung nicht möglich. Somit kann die
          Parallelkomposition diese Transition für $(p_1,p_2)$ ebenfalls nicht
          als must"=Transition enthalten.
        \item Fall 3 \big($(p_1,p_2)\must[a]_{12}$ mit $a\in O\cap\Synch
          (P_1,P_2)$\big): Der Output $a$ ist in diesem Fall durch
          Synchronisation von einem Output mit einem Input entstanden. \OBdA{}
          gilt $a\in O_1\cap I_2$. Für die einzelnen Systeme muss also gelten,
          dass $p_1\must[a]_1$ und $p_2\must[a]_2$. Die Transition für das
          System $P_1$ ist jedoch in der Voraussetzung ausgeschlossen worden.
          Somit ist es nicht möglich, dass $P_{12}$ diese in diesem Fall
          angenommene must"=Transition für den Zustand $(p_1,p_2)$ ausführen
          kann.
      \end{itemize}
      Da alle diese Fälle zu einem Widerspruch mit der Voraussetzung führen
      folgt, dass bereits die Annahme, dass der Zustand $(p_1,p_2)$ nicht ruhig
      ist, falsch war. Es gilt also, dass aus $p_i\in Qui_i$ für $i\in\{1,2\}$
      $(p_1,p_2)\in Qui_{12}$ folgt.
    \item Es gilt $(p_1,p_2)\in Qui_{12}\backslash E_{12}$, somit hat dieser
      Zustand allenfalls die Möglichkeit für must"=Transitionen, die mit Inputs
      beschriftet sind.\\
      Angenommen $p_1\notin Qui _1$, dann ist für $p_1$ entweder eine
      $\tau$"=must"=Transition oder eine Output"=must"=Transition möglich.
      \begin{itemize}
        \item Fall 1 \big($p_1\must[\tau]_1$\big): Da die Transition für $P_1$
          möglich ist, hat auch $P_{12}$ die Möglichkeit für eine
          $\tau$"=must"=Transition. Dies ist jedoch durch die Voraussetzung
          verboten und somit kann dieser Fall nicht eintreten.
        \item Fall 2 \big($p_1\must[a]_1$ mit $a\in O_1\backslash
          \Synch(P_1,P_2)$\big): Da es sich bei $a$ um einen must"=Output
          handelt, der nicht zu synchronisieren ist, wird dieser einfach in die
          Parallelkomposition übernommen. Es müsste alle $(p_1,p_2)
          \must[a]_{12}$ mit $a\in O_{12}$ gelten, was jedoch verboten ist.
          Somit kann die Transition für $P_1$ in diesem Fall nicht möglich
          sein.
        \item Fall 3 \big($p_1\must[a]_1$ mit $a\in O_1\cap\Synch(P_1,P_2)$ und
          $p_2\must[a]_2$\big): In diesem Fall ist die Synchronisation des
          Outputs $a$ von $P_1$ mit dem Input $a$ von $P_2$ möglich, so dass in
          der Parallelkomposition der Output $a$ als must"=Transition für
          $(p_1,p_2)$ entsteht. Diese must"=Transition ist jedoch für $P_{12}$
          nach Voraussetzung nicht erlaubt. Es folgt also auch, dass dieser
          Fall nicht eintreten kann.
        \item Fall 4 \big($p_1\must[a]_1$ mit $a\in O_1\cap\Synch(P_1,P_2)$ und
          $p_2\nmust[a]_2$\big): Da $P_2$ die $a$ Transition nicht als
          must"=Transition enthält, handelt es sich hier um einen neuen
          Kommunikationsfehler. Der neue Kommunikationsfehler kann dadurch
          entstehen, dass die Synchronisation des Outputs $a$ von $P_1$ mit dem
          Input $a$ von $P_2$ an dieser Stelle nicht möglich ist, oder da der
          Input $a$ für $p_2$ nur als may"=Transition vorliegt und somit die
          Gefahr besteht, dass dieser in einer Implementierung nicht vorhanden
          ist. Im zweiten Fall synchronisieren die beiden Transitionen zu einer
          $a$ Output"=may"=Transition, die in $P_{12}$ zulässig wäre. Jedoch
          wird der Zustand $(p_1,p_2)$ in beiden Fällen in die Menge $E_{12}$
          der Parallelkomposition eingefügt (Definition~\ref{ParallelDef}).
          Dies wurde in der Voraussetzung für den Zustand ausgeschlossen und
          dieser Fall ist somit nicht möglich.
      \end{itemize}
      Alle aufgeführten Fälle führen zu einem Widerspruch mit der
      Voraussetzung, somit folgt, dass die Annahme bereits falsch war und
      $p_1\in Qui_1$ gelten muss. Analog kann für $p_2$ argumentiert werden, so
      dass dann auch $p_2\in Qui_2$ folgt.
  \end{enumerate}
\end{proof}

In dem folgenden Satz sind die Punkte 1.\ und 3.\ nur zur Vollständigkeit
aufgeführt. Sie entsprechen Punkt 1.\ und 2.\ aus Satz~\ref{KommFehlerSemSatz}.

\begin{Satz}[Kommunikationsfehler- und Ruhe-Semantik für Parallelkompositionen]
  \label{RuheSemSatz}
  Für zwei komponierbare \MEIO{}s $P_1,P_2$ und ihre Komposition $P_{12}$ gilt:
  \begin{enumerate}
    \item $\ET _{12}=\cont (\prune ((\ET _1\|\EL _2)\cup (\EL _1\|\ET _2)))$,
    \item $\QET _{12}=(\QET _1\|\QET _2)\cup\ET _{12}$,
    \item $\EL _{12}=(\EL _1\|\EL _2)\cup\ET _{12}$.
  \end{enumerate}
\end{Satz}
\begin{proof}
  Es wird nur der 2. Punkt beweisen.\\
  \glqq$\subseteq$\grqq{}:\\
  Hie muss unterschieden werden, ob ein $w\in\StQT _{12}\backslash\ET _{12}$
  oder ein $w\in\ET _{12}$ betrachtet wird. Im zweiten Fall ist das $w$
  offensichtlich in der rechten Seite enthalten. Somit wird ab jetzt ein
  $w\in\StQT _{12}\backslash\ET _{12}$ betrachtet und es wird versuch dessen
  Zugehörigkeit zur rechten Menge zu zeigen. Aufgrund von
  Definition~\ref{RuheTraceDef} weiß man, dass
  $(p_{01},p_{02})\weakmay[w]_{12}(p_1,p_2)$ gilt mit $(p_1,p_2)\in Qui_{12}
  \backslash E_{12}$. Durch Projektion erhält man $p_{01} \weakmay[w_1]_1
  p_1$ und $p_{02}\weakmay[w_2]_2p_2$ mit $w\in w_1\|w_2$. Aus $(p_1,p_2)\in
  Qui_{12}\backslash E_{12}$ kann mit dem zweiten Punkt von
  Lemma~\ref{RuheZustLem} gefolgert werden, dass bereits $q_1\in Qui_1$ und
  $q_2\in Qui_2$ gilt. Somit gilt $w_1\in\StQT _1\subseteq \QET _1$ und $w_2\in
  \StQT _2\subseteq\QET _2$. Daraus folgt dann $w\in \QET _1\|\QET _2$ und
  somit ist $w$ in der rechten Seite der Gleichung enthalten.

  \glqq$\supseteq$\grqq{}:\\
  Es muss wieder danach unterscheiden werden aus welcher Menge das betrachtete
  Element stammt. Falls $w\in\ET _{12}$ gilt, so kann die Zugehörigkeit zur
  linken Seite direkt gefolgert werden. Somit wird für den weiteren Beweis
  dieser Inklusionsrichtung ein Element $s\in\QET _1\|\QET _2$ betrachtet und
  gezeigt, dass es in der linken Menge enthalten ist. Da $\QET _i=\StQT _i\cup
  \ET _i$ gilt, existiert für $w_1$ und $w_2$ mit $w\in w_1\|w_2$
  unterschiedliche Möglichkeiten:
  \begin{itemize}
    \item Fall 1 ($w_1\in\ET _1\lor w_2\in\ET _2$): \OBdA{} gilt $w_1\in\ET
      _1$. Nun kann $w_2\in\StQT _2\subseteq L_2$ oder $w_2\in\ET _2$ gelten
      und somit ist auf jeden Fall $w_2$ in $\EL _2$ enthalten. Daraus kann
      dann mit dem ersten Punkt von Satz~\ref{KommFehlerSemSatz} gefolgert
      werden, dass $w\in\ET _{12}$ gilt und damit ist $w$ in der linken Seite
      der Gleichung enthalten.
    \item Fall 2 ($w_1\in\StQT _1\backslash\ET _1\land w_2\in\StQT _2\backslash
      \ET _2$): Es gilt in diesem Fall $p_{01} \weakmay[w_1]_1 p_1\in Qui_1$
      und $p_{02} \weakmay[w_2]_2 p_2\in Qui_2$. Da $p_1$ und $p_2$ in der
      jeweiligen Ruhe-Menge enthalten sind, ist auch der Zustand, der aus ihnen
      zusammengesetzt ist, in der Parallelkomposition ruhig, wie bereits im
      ersten Punkt von Lemma~\ref{RuheZustLem} gezeigt. Es gilt also für die
      Komposition $(p_{01},p_{02}) \weakmay[w]_{12} (p_1,p_2)\in Qui_{12}$ und
      dadurch ist $w$ in der linken Seite der Gleichung enthalten, da $w\in
      \StQT _{12}\subseteq \QET _{12}$ gilt.
  \end{itemize}
\end{proof}

\begin{Kor}[Ruhe-Präkongruenz]
  Die Relation \QRel{} ist eine Präkongruenz bezüglich $\cdot\|\cdot$.
\end{Kor}

\begin{proof}
  Es muss gezeigt werden: Wenn $P_1\QRel P_2$ gilt, so auch $P_{31}\QRel
  P_{32}$ für jedes komponierbare System $P_3$. D.h.\ es ist zu zeigen, dass
  aus $P_1\ERel P_2$ und $\QET{}_1\subseteq \QET{}_2$ sowohl $P_{31}\ERel
  P_{32}$ als auch $\QET{}_{31}\subseteq \QET{}_{32}$ folgt. Dies ergibt sich,
  wie im Beweis zu Korollar~\ref{KommPraekonKor}, aus der Monotonie von
  $\cdot\|\cdot$ auf Sprachen wie folgt:
  \begin{itemize}
    \item $\begin{aligned}[t]
        P_{31} \overset{\mathrm{Korollar}~\ref{KommPraekonKor}}{
          \overset{\mathrm{und}}{\overset{P_1\ERel P_2}{\ERel}}}
        P_{32},
    \end{aligned}$
    \item $\begin{aligned}[t]
        \QET{}_{31} &\overset{\ref{RuheSemSatz}~2.}{=}
        (\QET{}_3\|\QET{}_1)\cup \ET{}_{31}\\
        &\hspace{-0.6cm}\overset{\ET{}_{31}\subseteq
      \ET{}_{32}}{\overset{\mathrm{und}}{\overset{\QET{}_1\subseteq
      \QET{}_2}{\subseteq}}} (\QET{}_3\|\QET{}_2) \cup \ET{}_{32}\\
        &\overset{\ref{RuheSemSatz}~2.}{=} \QET{}_{32}.
    \end{aligned}$
  \vspace*{-0.7cm}
  \end{itemize}
\end{proof}

Im nächsten Lemma soll eine Verfeinerung bezüglich guter Kommunikation mit
Partner im Sinne von fehler- und ruhe-freier Kommunikation betrachtet werden.

\begin{Lem}[Verfeinerung mit Ruhe-Zuständen]
  Gegeben sind zwei \MEIO{}s $P_1$ und $P_2$ mit der gleichen Signatur. Wenn
  $U\|P_1\QBRel U\|P_2$ für alle Partner $U$ gilt, dann folgt daraus $P_1\QRel
  P_2$.
\end{Lem}

\begin{proof}
  \TODO{zu beweisen}
\end{proof}
