\chapter{Verfeinerungen für Kommunikationsfehler- und Ruhe-Freiheit}


In diesem Kapitel wird die Menge der betrachteten Zustandsmengen von den
Kommunikations"=basierten Fehlern im letzten Kapitel erweitert um
Ruhe-Zustände. Es wird nur noch der Testing-Ansatz des letzten Kapitels
fortgeführt, da dieser sich auf die Parallelkomposition von \EIO{}s stützt und
nicht auf die Definition der Parallelkomposition von \MEIO{}s, die auch anders
gestaltet hätte werden können.\\
Zustände, die keine must-Outputs ohne einen Input ausführen können, werden als
in einer Art Verklemmung angesehen, da sie ohne Zutun von Außen den Zustand
nicht mehr verlassen können, falls ein möglicherweise vorhandener may"=Output
nicht implementiert wird. So ein Zustand hat also keine
must"=Transitions"=Möglichkeiten für einen Output. Falls dieser Zustand die
Möglichkeit für eine interne Aktion via einer must"=Transition hat, darf durch
die $\tau$s niemals ein Zustand erreicht werden, von dem aus ein Output in
Implementierungen sicher gestellt wird. Ein Zustand, der keine Outputs und
$\tau$s via must"=Transitionen ausführen kann, ist also ein Deadlock-Zustand,
in denen das System nichts mehr tun können muss ohne einen Input. Wenn man eine
Erweiterung um $\tau$s zu Zuständen ohne must"=Outputs zulässt, hat man
zusätzlich noch Verklemmungen der Art Livelock, da diese Zustände
möglicherweise beliebig viele interne Aktionen ausführen können, jedoch nie aus
eigener Kraft einen wirklichen Fortschritt in Form eines Outputs bewirken
können müssen. Die Menge der Zustände, die sich in einer Verklemmung
befinden, würde also durch $\left\{p\in P\mid \forall a\in O: p\nweakmust[a]_P
\right\}$ beschrieben werden. Somit wären dies alle Zustände, die keine
Möglichkeit haben ohne einen Input von Außen oder eine implementierte
may"=Output"=Transition je wieder einen Output machen zu können. Falls man
diese Definition verwenden würde, müsste man immer alle Zustände betrachten,
die durch $\tau$s erreichbar sind. Dies würde einige Betrachtungen deutlich
aufwendiger machen und soll deshalb hier nicht behandelt werden. Die Definition
für die betrachteten Verklemmungen, hier Ruhe genannt, beschränkt sich auf
Zustände, die keine Outputs und $\tau$s via must"=Transitionen ausführen
können.

\begin{Def}[Ruhe]
  Ein \emph{Ruhe-Zustand} ist ein Zustand in einem \MEIO{} $P$, der keine
  Outputs und kein $\tau$ zulässt via must"=Transitionen.\\
  Somit ist die Menge der Ruhe-Zustände in einem \MEIO{} $P$ wie folgt formal
  definiert: $Qui(P):=\left\{p\in P\mid \forall\alpha\in (O\cup\{\tau\}):p
  \nmust[\alpha]_P\right\}$.
\end{Def}

Für die Erreichbarkeit wird wie im letzten Kapitel ein optimistischer Anzahl
der lokalen Erreichbarkeit für die Fehler-Zustände verwendet. Ruhe ist kein
unabwendbare \glqq Fehler-Art\grqq{}, sondern kann durch einen Input repariert
werden oder im Fall von vorhandenen may"=Output"=Transitionen oder
may"=$\tau$"=Transitionen, durch eine Implementierung dieser lokalen Aktionen .
Daraus ergibt sich, dass Ruhe im Vergleich zu den Fehlern aus dem letzten
Kapitel als weniger \glqq schlimmer Fehler\grqq{} anzusehen ist. Somit ist ein
Ruhe-Zustand ebenso wie ein Fehler-Zustand erreichbar, sobald er durch Outputs
und $\tau$s erreicht werden kann, jedoch ist nicht jede beliebige Fortsetzung
eines Traces, das durch lokale Aktionen zu einem Ruhe-Zustand führt ein
Ruhe"=Trace.

\begin{Def}[Test und Verfeinerung für Ruhe]
  \label{RuheTestDef}
  Ein \emph{Test} $T$ ist eine Implementierung. Ein \MEIO{} $P$
  \emph{as-erfüllt} einen Ruhe-Test $T$, falls $S\|T$ fehler- und ruhe-frei ist
  für alle $S\in \asimp (P)$. Es wird dann $P \QsatAs T$ geschrieben. Die
  Parallelkomposition $S\|T$ ist \emph{fehler-} und \emph{ruhe-frei}, wenn kein
  Fehler- und kein Ruhe-Zustand lokal erreichbar ist.\\
  Ein \MEIO{} $P$ \emph{Ruhe-verfeinert} $P'$, falls für alle Tests $T$:
  $P'\QsatAs T \Rightarrow P\QsatAs T$.
\end{Def}

Um eine genauere Auseinandersetzung mit den Präkongruenzen zu ermöglichen,
benötigt man wie im letzten Kapitel die Definition von Traces auf der Struktur.
Wie bereits oben erwähnt, ist Ruhe ein reparierbares \glqq Fehlverhalten\grqq{}
im Gegensatz zu Fehlern. Es genügt deshalb für Ruhe die strikten Traces ohne
Kürzung zu betrachten.

\begin{Def}[Ruhe-Traces]
  \label{RuheTraceDef}
  Sei $P$ ein \MEIO{} und definiere:
  \begin{itemize}
    \item \emph{strikte Ruhe-Traces}: $\StQT (P) := \left\{w\in\Sigma ^*\mid
      p_0 \weakmay[w]_P p\in Qui(P)\right\}$.
  \end{itemize}
\end{Def}

\vspace{0.2cm}

\begin{Prop}[Ruhe-Traces und Implementierungen]
  \label{RuheTraceProp}
  Für ein \MEIO{} $P$ gilt für die strikte Ruhe-Traces: $\StQT (P) \subseteq
  \big\{w\in\Sigma ^*\mid \exists P'\in\asimp (P): p'_0 \weakmust[w]_{P'}
  p'\in Qui(P')\big\} = \underset{P'\in\asimp (P)}{\bigcup} \StQT (P')$.
\end{Prop}
\begin{proof}
  Analog zu den Propositionen~\ref{LImpProp} und~\ref{KommTracesProp} ist die
  Inklusion am Besten mit einer as"=Implementierung zu zeigen und der
  entsprechenden as"=Verfeinerungs"=Relation $\mathcal{R}$. Falls der
  Startzustand $p_0$ von $P$ ein Ruhe-Zustand ist, muss man zwei
  as"=Implementierungen betrachten, ansonsten genügt es eine für alle $w$ aus
  $\StQT (P)$ anzugeben, wobei $w$ möglicherweise nicht $\varepsilon$
  entsprechend darf.\\
  Die as"=Implementierung $P'$ für den Fall $p_0\in Qui (P)$ implementiert alle
  must"=Transitionen, keine may"=Transitionen und keine Fehler-Zustände von $P$
  und hat die Identitäts-Relation als starke as"=Verfeinerungs"=Relation
  $\mathcal{R}$. In diesem Fall gilt $\varepsilon\in\StQT (P)$. Für alle
  $a\in\Sigma$ folgt, wenn in $P$ für einen Zustand $p$ $p\nmust[a]_P$, gilt
  auch in $P'$ $p' \nmust[a]_{P'}$ für den Zustand, der mit $p$ in Relation
  steht. Der Startzustand von $P'$ ist mit $\varepsilon$ erreichbar und
  ebenfalls ruhig. Es gilt also $p'_0\in Qui (P')$ und $\varepsilon\in\StQT
  (P')$. Der 2.\ Punkt der Definition~\ref{SimDef} ist für die
  Identitäts-Relation als as"=Verfeinerungs"=Relation $\mathcal{R}$ erfüllt, da
  alle must"=Transitionen aus $P$ entsprechend in $P'$ umgesetzt wurden. Alle
  must"=Transitionen in $P$ müssen zugrundeliegende may"=Transitionen haben,
  somit gilt auch 3.\ von~\ref{SimDef}. Der 1.\ Punkt der Definition ist auch
  erfüllt, da $E_{P'}=\emptyset$ gilt, wenn keine Fehler-Zustände implementiert
  werden.\\
  Für alle $w\neq \varepsilon$ und für $w = \varepsilon$, dass zu einem anderen
  ruhigen Zustand führt wie $p_0$, mit $w\in\StQT (P)$ kann $P'$ als die
  folgende as"=Implementierung gewählt werden:
  \begin{itemize}
    \item $P'= \{q\mid p\in P\} \cup \{q'\mid p\in P\}$,
    \item $p'_0=q_0$,
    \item $I_{P'}=I_P$ und $O_{P'}=O_P$,
    \item $\begin{aligned}[t] \must _{P'}=\may _{P'} &=
        \left\{(q_0,\alpha,q_j)\mid p_0\may[\alpha] p_j\right\}\\
        &\cup \left\{(q_0,\alpha,q'_j)\mid p_0\may[\alpha] p_j\right\}\\
        &\cup \left\{(q_j,\alpha ,q_k)\mid p_j\must[\alpha] p_k\right\}\\
        &\cup \left\{(q'_j,\alpha ,q'_k)\mid p_j\must[\alpha] p_k\right\}\\
        &\cup \left\{(q'_j,\alpha ,q_k)\mid j\neq 0, p_j\may[\alpha] p_k,
        p_j\nmust[\alpha] p_k\right\}\\
        &\cup \left\{(q'_j,\alpha ,q'_k)\mid j\neq 0, p_j\may[\alpha] p_k,
        p_j\nmust[\alpha] p_k\right\},
    \end{aligned}$
    \item $E_{P'}=\emptyset$.
  \end{itemize}
  Als as"=Verfeinerungs"=Relation zwischen $P$ und $P'$ wird die Relation
  $\mathcal{R}=\{(q_j,p_j)\mid p_j\in P\} \cup \{(q'_j,p_j)\mid p_j\in P\}$
  verwendet. Es werden in $P'$ für die ungestrichenen Zustände $q$ nur die
  must"=Transitionen und für die gestrichenen Zustände $q'$ werden die must-
  und may"=Transitionen implementiert. Die must"=Transitionen werden nur zu den
  Zuständen der \glqq gleichen Sorte\grqq{} umgesetzt, wohingegen die
  may"=Transitionen, zu denen es keine entsprechende must"=Transition in $P$
  gibt, von den Zuständen $q'$ zu dem entsprechenden ungestrichenen
  und gestrichenen Zustand implementiert wird. Da die Menge der Fehler-Zustände
  leer ist, gilt~\ref{SimDef}~1.\ für $\mathcal{R}$. Die must"=Transitionen
  werden für die ungestrichenen und gestrichenen Zustände umgesetzt, dies
  erfüllt zusammen mit $\mathcal{R}$ die Definition~\ref{SimDef}~2. Ebenso wird
  der dritte Punkt der Definition~\ref{SimDef} erfüllt, da sowohl die
  gestrichenen wie auch die ungestrichenen Zustände mit den entsprechenden
  Zuständen aus $P$ in der Relation $\mathcal{R}$ stehen. $\mathcal{R}$ ist
  also eine starke alternierende Simulations-Relation auf $P'$ und $P$. Falls
  ein Zustand $p_j$ in $P$ ruhig war, ist es auch der entsprechenden Zustand
  $q_j$ in $P'$, da für $q_j$ alle ausgehenden must"=Transitionen von $p_j$
  implementiert wurden, aber keine einzige may"=Transition, die keine der
  must"=Transitionen entspricht. Wenn also für $p_j$ keine Outputs und kein
  $\tau$ möglich waren via must"=Transitionen, dann ist es dies auch für $q_j$
  nicht. $q_j$ ist in $P'$ mit den selben Traces erreichbar wie $p_j$ in $P$,
  da jeder ungestrichene und gestrichene Zustand in $P'$ die selben eingehenden
  Transitionen hat wie der entsprechende Zustand in $P$. Falls der Trace zu
  $p_j$ may"=Transitionen ohne entsprechende must"=Transitionen enthält, kann
  der Trace in $P'$ ausgeführt werden, in dem von $q_0$ aus der Trace über die
  gestrichenen Zustände genommen wird bis zur letzten Transition, die zu einem
  ungestrichenen Zuständen führt in dem auszuführenden Wort. Ab da hat der
  Trace in $P$ nur must"=Transitionen genommen und kann somit in den
  ungestrichenen Zuständen in $P'$ nachgefolgt werden. Falls der Trace in $P$
  insgesamt nur aus must"=Transitionen bestanden hat, ist direkt von $q_0$ aus
  der Weg über ungestrichene Zustände zu $q_j$ möglich. Es gilt also $\StQT
  (P)\backslash \{\varepsilon\} = \StQT (P')\backslash \{\varepsilon\}$. Falls
  $\varepsilon$ zu einem Ruhe-Zustand $p \neq p_0$ in $P$ geführt hat, gilt
  sogar $\StQT(P) = \StQT (P')$, da ein Trace aus internen Aktionen in $P$ und
  $P'$ zu dem entsprechenden ruhigen Zustand $p$ bzw.\ $q$ führt.
\end{proof}

Für \ET{} und \EL{} gelten die Definitionen aus dem letzten Kapitel. Es wir nur
für Ruhe eine neue Semantik definiert.

\begin{Def}[Ruhe-Semantik]
  \label{RuheSemDef}
  Sei $P$ ein \MEIO{}.
  \begin{itemize}
    \item Die Menge der \emph{fehler-gefluteten Ruhe-Traces} von $P$ ist $\QET
      (P):= \StQT (P)\cup\ET (P)$.
  \end{itemize}
  Für zwei \MEIO{}s $P_1,P_2$ mit der gleichen Signatur wird $P_1\QRel{} P_2$
  geschrieben, wenn $P_1\ERel{} P_2$ und $\QET _1\subseteq \QET _2$ gilt.
\end{Def}

\vspace{0.2cm}

\begin{Prop}[Ruhe-Semanik und Implementierungen]
  \label{RuheSemProp}
  Für die Menge der fehler-gefluteten Ruhe-Traces von $P$ gilt $\QET (P) =
  \underset{P'\in\asimp (P)}{\bigcup} \QET (P')$.
\end{Prop}
\begin{proof}\mbox{}\\
  $\subseteq$:
  \begin{align*}
    \QET (P)&\overset{\ref{RuheSemDef}}{=} \StQT (P) \cup \ET (P)\\
    &\overset{\ref{RuheTraceProp}}{\subseteq} \left(\underset{P'\in\asimp
    (P)}{\bigcup} \StET (P')\right)\cup \ET (P)\\
    &\overset{\ref{KommSemProp}}{=} \left(\underset{P'\in\asimp
    (P)}{\bigcup} \StET (P')\right)\cup \left(\underset{P'\in\asimp
    (P)}{\bigcup} \ET (P')\right)\\
    &= \underset{P'\in\asimp (P)}{\bigcup} \StQT (P') \cup \ET (P')\\
    &\overset{\ref{RuheSemDef}}{=} \underset{P'\in\asimp (P)}{\bigcup} \QET
    (P').\\
  \end{align*}

  $\supseteq$:\\
  Es wird hier für ein $w\in \QET (P')$ einer beliebigen as"=Implementierung
  $P'$ von $P$ gezeigt, dass das Wort $w$ auch in $\QET (P)$ enthalten ist. Es
  kann danach unterschieden werden, ob $w$ aus $\StQT (P')\backslash \ET (P')$
  stammt oder aus $\ET (P')$. Falls $w\in\ET (P')$ gilt, folgt mit
  Proposition~\ref{KommSemProp} bereits, dass $w\in \ET (P) \subseteq \QET (P)$
  gilt. Somit wird für den Rest des Beweises davon ausgegangen, dass $w\in\StQT
  (P')\backslash \ET (P')$ ist. $w$ führt in $P'$ also nur zu einem ruhigen
  Zustand und hat nichts mit Fehler-Zuständen in $P'$ zu tun. Der Trace, der
  durch $w$ in $P'$ beschrieben hat, hat die folgende Form: $\exists w'\in
  \Sigma _{\tau}, \exists \alpha _1, \alpha _2, \dots , \alpha _n, \exists
  p'_1, p'_2, \dots p'_n : \hat{w'} = w \land w' = \alpha _1\alpha _2\dots
  \alpha _n \land p'_0 \may[\alpha _1]_{P'} p'_1 \may[\alpha _2]_{P'} \dots
  p'_{n-1} \may[\alpha _n]_{P'} p'_n \in Qui _{P'}$. Da es eine
  as"=Verfeinerungs"=Relation $\mathcal{R}$ geben muss, die beweist, dass $P'$
  $P$ as"=verfeinert, muss es ein Präfix von $w$ geben, dass auch in $P$
  ausführbar ist. Falls $w$ nicht vollständig ausführbar ist in $P$, muss auf
  dem Weg, auf dem das Präfix von $w$ ausgeführt wird ein Zustand $p_j$ mit $0
  \leq j \leq n$ liegen, der ein Fehler-Zustand ist. Es gilt dann $w\in \ET
  (P)\subseteq \QET (P)$. Falls jedoch $w$ in $P$ ausführbar ist ohne einen
  Fehler-Zustand zu erreichen, gibt es einen analogen Trace zu dem in $P'$, der
  Form: $p_0 \may[\alpha _1]_P p_1 \may[\alpha _2]_P \dots p_{n-1} \may[\alpha
  _n]_P p_n$, wobei $p'_j \mathcal{R} p_j$ für alle $j$ aus $\{0,1,\dots ,n\}$
  gilt. Es wird also durch $w$ ein Zustand $p_n$ erreicht, der mit dem Zustand
  $p'_n$ in der starken as"=Verfeinerungs"=Relation $\mathcal{R}$ stehen.
  $p'_n$ ist ruhig, nach Voraussetzung. Es gilt also für alle $\omega\in O\cup
  \{\tau\}$ $p'_n\nmust[\omega]$. Da $(p'_n,p_n) \in\mathcal{R}$ gilt und beide
  Zustände keine Fehler-Zustände sind, muss auch $p_n\nmust[\omega]$ für alle
  $\omega\in O\cup \{\tau\}$ gelten, da sonst~\ref{SimDef}~2.\ verletzt würde.
  Es gilt also in diesem Fall $w\in\StQT (P)\subseteq\QET (P)$.
\end{proof}

Wie im letzten Kapitel kann aus der vorangegangen Proposition über die
Gleichheit der betrachteten Trace Mengen in der Relation \QRel{} auch eine
Aussage über die lokale Erreichbarkeit \glqq fehlerhafter Zustände\grqq{} in
einer Spezifikation und den zugehörigen Implementierungen getroffen werden.

\begin{Kor}[lokale Ruhe Erreichbarkeit]\mbox{}
  \label{lokaleRuheErrKor}
  \begin{enumerate}[(i)]
    \item Falls in einem \MEIO{} $P$ ein Fehler lokal erreichbar ist, dann
      existiert auch eine as"=Implementierung, in der ein Fehler lokal
      erreichbar ist.
    \item Falls in einem \MEIO{} $P$ einen lokal erreichbaren Ruhe-Zustand
      besitzt, jedoch keinen lokal erreichbaren Fehler, dann existiert auch
      eine as"=Implementierung, in der ein Ruhe-Zustand und kein Fehler lokal
      erreichbar ist.
    \item Falls es eine as"=Implementierung von $P$ gibt, die einen Fehler oder
      Ruhe lokal erreicht, dann ist auch ein Fehler oder Ruhe in $P$ lokal
      erreichbar.
  \end{enumerate}
\end{Kor}
\begin{proof}\mbox{}
  \begin{enumerate}[(i)]
    \item Dieser Punkt folgt direkt aus Proposition~\ref{KommSemProp}.
    \item Da ein Ruhe-Zustand in $P$ lokal erreichbar ist, gilt $w\in \StQT
      _P\backslash \cont (\PrET) _P \subseteq \QET _P$ für $w\in O$. Es muss
      wegen Proposition~\ref{RuheSemProp} mindestens ein $P'\in \asimp (P)$
      geben, für dass $w\in \QET _{P'}$ gilt. Das $w$ kann also in $\ET _{P'}$
      oder in $\StQT _{P'}$ enthalten sein. Da $w$ nur aus lokalen Aktion
      bestehen kann, kann $w$ in $P$ kein Input-kritisch Trace sein und auch
      keine Verlängerung von einem solchen. Es gilt also $w\notin \ET _P$.
      Mit~\ref{KommSemProp} folgt daraus, dass auch in keiner
      as"=Implementierung von $P$ $w$ in der Menge \ET{} enthalten ist. Es gilt
      also $w\in \QET _{P'} \backslash \ET _{P'} \subseteq \StQT _{P'}$. Es ist
      also auch in $P'$ ein Ruhe-Zustand lokal erreichbar.
    \item Sei $P'$ die as"=Implementierung von $P$, in der ein Fehler- oder
      Ruhe-Zustand lokal erreichbar ist. Es gilt dann $w\in \QET _P'$ für $w\in
      O$. Mit Proposition~\ref{RuheSemProp} folgt draus $w\in \QET _P$. Es muss
      also auch in $P$ ein Fehler- oder Ruhe-Zustand lokal erreichbar sein.
  \end{enumerate}
\end{proof}

Für spätere Beweise werden noch Zusammenhänge zwischen Ruhe-Zuständen in den
einzelnen Komponenten und in einer Parallelkomposition dieses Komponenten
benötigt.

\begin{Lem}[Ruhe-Zustände unter Parallelkomposition]\mbox{}
  \label{RuheZustLem}
  \begin{enumerate}
    \item Ein Zustand $(p_1,p_2)$ aus der Parallelkomposition $P_{12}$ ist
      ruhig, wenn es auch die Zustände $p_1$ und $p_2$ in $P_1$ bzw. $P_2$
      sind.
    \item Wenn der Zustand $(p_1,p_2)$ ruhig ist und nicht in $E_{12}$
      enthalten ist, dann sind auch die auf die Teilsysteme projizierten
      Zustände $p_1$ und $p_2$ ruhig.
  \end{enumerate}
\end{Lem}
\begin{proof}\mbox{}
  \begin{enumerate}
    \item Da $p_1\in Qui_1$ und $p_2\in Qui_2$ gilt, haben diese beiden
      Zustände jeweils höchstens die Möglichkeit für Input-Transitionen oder
      Output- und $\tau$-may"=Transitionen, jedoch keine Möglichkeit für Outputs
      oder $\tau$s als must"=Transitionen.\\
      Angenommen der Zustand, der durch die Parallelkomposition aus den
      Zuständen $p_1$ und $p_2$ entsteht, ist nicht ruhig, d.h.\ er hat eine
      ausgehende must"=Transition für einen Output oder ein $\tau$.
      \begin{itemize}
        \item Fall 1 \big($(p_1,p_2)\must[\tau]_{12}$\big): Ein $\tau$ ist eine
          interne Aktion und kann in der Parallelkomposition nicht durch das
          Verbergen von Aktionen bei der Synchronisation entstehen. Ein $\tau$
          in der Parallelkomposition ist also auch nur möglich, wenn dies
          bereits als must"=Transition in einer Komponente möglich war für
          einen der Zustände, aus denen $(p_1,p_2)$ zusammensetzt ist. Jedoch
          verbietet die Voraussetzung, dass $p_1$ oder $p_2$ eine ausgehende
          $\tau$ must"=Transition haben, deshalb kann auch $(p_1,p_2)$ keine
          solche Transition besitzen.
        \item Fall 2 \big($(p_1,p_2)\must[a]_{12}$ mit $a\in O_{12}\backslash
          \Synch(P_1,P_2)$\big): Da es sich bei $a$ um einen Output handelt, der
          nicht in $\Synch (P_1,P_2)$ enthalten ist, kann dieser nicht aus der
          Synchronisation von zwei Aktionen entstanden sein, sondern muss
          bereits für $P_1$ oder $P_2$ als must"=Transition ausführbar gewesen
          sein. Es gilt also \oBdA{} $p_1\must[a]_1$ mit $a\in O_1$. Dies ist
          jedoch aufgrund der Voraussetzung nicht möglich. Somit kann die
          Parallelkomposition diese Transition für $(p_1,p_2)$ ebenfalls nicht
          als must"=Transition enthalten sein.
        \item Fall 3 \big($(p_1,p_2)\must[a]_{12}$ mit $a\in O_{12}\cap\Synch
          (P_1,P_2)$\big): Der Output $a$ ist in diesem Fall durch
          Synchronisation von einem Output mit einem Input entstanden. \OBdA{}
          gilt $a\in O_1\cap I_2$. Für die einzelnen Systeme muss also gelten,
          dass $p_1\must[a]_1$ und $p_2\must[a]_2$. Die Transition für das
          System $P_1$ ist jedoch in der Voraussetzung ausgeschlossen worden.
          Somit ist es nicht möglich, dass $P_{12}$ diese in diesem Fall
          angenommene must"=Transition für den Zustand $(p_1,p_2)$ ausführen
          kann.
      \end{itemize}
      Da alle diese Fälle zu einem Widerspruch mit der Voraussetzung führen
      folgt, dass bereits die Annahme, dass der Zustand $(p_1,p_2)$ nicht ruhig
      ist, falsch war. Es gilt also, dass aus $p_j\in Qui_j$ für $j\in\{1,2\}$
      $(p_1,p_2)\in Qui_{12}$ folgt.
    \item Es gilt $(p_1,p_2)\in Qui_{12}\backslash E_{12}$, somit hat dieser
      Zustand allenfalls die Möglichkeit für must"=Transitionen, die mit Inputs
      beschriftet sind.\\
      Angenommen $p_1\notin Qui _1$, dann ist für $p_1$ entweder eine
      $\tau$"=must"=Transition oder eine Output"=must"=Transition möglich.
      \begin{itemize}
        \item Fall 1 \big($p_1\must[\tau]_1$\big): Da die Transition für $P_1$
          möglich ist, hat auch $P_{12}$ die Möglichkeit für eine
          $\tau$"=must"=Transition. Dies ist jedoch durch die Voraussetzung
          verboten und somit kann dieser Fall nicht eintreten.
        \item Fall 2 \big($p_1\must[a]_1$ mit $a\in O_1\backslash
          \Synch(P_1,P_2)$\big): Da es sich bei $a$ um einen must"=Output
          handelt, der nicht zu synchronisieren ist, wird dieser einfach in die
          Parallelkomposition übernommen. Es müsste also $(p_1,p_2)
          \must[a]_{12}$ mit $a\in O_{12}$ gelten, was jedoch verboten ist.
          Somit kann die Transition für $P_1$ in diesem Fall nicht möglich
          sein.
        \item Fall 3 \big($p_1\must[a]_1$ mit $a\in O_1\cap\Synch(P_1,P_2)$ und
          $p_2\must[a]_2$\big): In diesem Fall ist die Synchronisation des
          Outputs $a$ von $P_1$ mit dem Input $a$ von $P_2$ möglich, so dass in
          der Parallelkomposition der Output $a$ als must"=Transition für
          $(p_1,p_2)$ entsteht. Diese must"=Transition ist jedoch für $P_{12}$
          nach Voraussetzung nicht erlaubt. Es folgt also auch, dass dieser
          Fall nicht eintreten kann.
        \item Fall 4 \big($p_1\must[a]_1$ mit $a\in O_1\cap\Synch(P_1,P_2)$ und
          $p_2\nmust[a]_2$\big): Da $P_2$ die $a$ Transition nicht als
          must"=Transition enthält, handelt es sich hier um einen neuen
          Fehler. Das $a$ kann für $P_2$ kein Output sein, da sonst $P_1$ und
          $P_2$ nicht komponierbar wäre. Der neue Fehler kann dadurch
          entstehen, dass die Synchronisation des Outputs $a$ von $P_1$ mit dem
          Input $a$ von $P_2$ an dieser Stelle nicht möglich ist, oder da der
          Input $a$ für $p_2$ nur als may"=Transition vorliegt und somit die
          Gefahr besteht, dass dieser in einer Implementierung nicht vorhanden
          ist. Im zweiten Fall synchronisieren die beiden Transitionen zu einer
          $a$ Output"=may"=Transition, die in $P_{12}$ zulässig wäre. Jedoch
          wird der Zustand $(p_1,p_2)$ in beiden Fällen in die Menge $E_{12}$
          der Parallelkomposition eingefügt (Definition~\ref{ParallelDef}).
          Dies wurde in der Voraussetzung für den Zustand ausgeschlossen und
          dieser Fall ist somit nicht möglich.
      \end{itemize}
      Alle aufgeführten Fälle führen zu einem Widerspruch mit der
      Voraussetzung, somit folgt, dass die Annahme bereits falsch war und
      $p_1\in Qui_1$ gelten muss. Analog kann für $p_2$ argumentiert werden, so
      dass dann auch $p_2\in Qui_2$ folgt.
  \end{enumerate}
\end{proof}

In dem folgenden Satz sind die Punkte 1.\ und 3.\ nur zur Vollständigkeit
aufgeführt. Sie entsprechen Punkt 1.\ und 2.\ aus Satz~\ref{KommFehlerSemSatz}.

\begin{Satz}[Kommunikationsfehler- und Ruhe-Semantik für Parallelkompositionen]
  \label{RuheSemSatz}
  Für zwei komponierbare \MEIO{}s $P_1,P_2$ und ihre Komposition $P_{12}$ gilt:
  \begin{enumerate}
    \item $\ET _{12}=\cont (\prune ((\ET _1\|\EL _2)\cup (\EL _1\|\ET _2)))$,
    \item $\QET _{12}=(\QET _1\|\QET _2)\cup\ET _{12}$,
    \item $\EL _{12}=(\EL _1\|\EL _2)\cup\ET _{12}$.
  \end{enumerate}
\end{Satz}
\begin{proof}
  Es wird nur der 2. Punkt beweisen.\\
  \glqq$\subseteq$\grqq{}:\\
  Hier muss unterschieden werden, ob ein $w\in\StQT _{12}\backslash\ET _{12}$
  oder ein $w\in\ET _{12}$ betrachtet wird. Im zweiten Fall ist das $w$
  offensichtlich in der rechten Seite enthalten. Somit wird im Folgenden ein
  $w\in\StQT _{12}\backslash\ET _{12}$ betrachtet und es wird versucht dessen
  Zugehörigkeit zur rechten Menge zu zeigen. Aufgrund von
  Definition~\ref{RuheTraceDef} weiß man, dass $(p_{01},p_{02})
  \weakmay[w]_{12} (p_1,p_2)$ gilt mit $(p_1,p_2)\in Qui_{12} \backslash
  E_{12}$. Durch Projektion erhält man $p_{01} \weakmay[w_1]_1 p_1$ und $p_{02}
  \weakmay[w_2]_2p_2$ mit $w\in w_1\|w_2$. Aus $(p_1,p_2)\in Qui_{12}
  \backslash E_{12}$ kann mit dem zweiten Punkt von Lemma~\ref{RuheZustLem}
  gefolgert werden, dass $q_1\in Qui_1$ und $q_2\in Qui_2$ gilt. Somit gilt
  $w_1 \in \StQT _1\subseteq \QET _1$ und $w_2\in \StQT _2\subseteq\QET _2$.
  Daraus folgt $w\in \QET _1\|\QET _2$ und somit ist $w$ in der rechten Seite
  der Gleichung enthalten.

  \glqq$\supseteq$\grqq{}:\\
  Es muss wieder danach unterschieden werden aus welcher Menge das betrachtete
  Element stammt. Falls $w\in\ET _{12}$ gilt, so kann die Zugehörigkeit zur
  linken Seite direkt gefolgert werden. Somit wird für den weiteren Beweis
  dieser Inklusionsrichtung ein Element $w\in\QET _1\|\QET _2$ betrachtet und
  gezeigt, dass es in der linken Menge enthalten ist. Da $\QET _i=\StQT _i\cup
  \ET _i$ gilt, existieren für $w_1$ und $w_2$ mit $w\in w_1\|w_2$
  unterschiedliche Möglichkeiten:
  \begin{itemize}
    \item Fall 1 ($w_1\in\ET _1\lor w_2\in\ET _2$): \OBdA{} gilt $w_1\in\ET
      _1$. Nun kann $w_2\in\StQT _2\subseteq L_2$ oder $w_2\in\ET _2$ gelten
      und somit ist auf jeden Fall $w_2$ in $\EL _2$ enthalten. Daraus kann
      dann mit dem ersten Punkt von Satz~\ref{KommFehlerSemSatz} gefolgert
      werden, dass $w\in\ET _{12}$ gilt und damit ist $w$ in der linken Seite
      der Gleichung enthalten.
    \item Fall 2 ($w_1\in\StQT _1\backslash\ET _1\land w_2\in\StQT _2\backslash
      \ET _2$): Es gilt in diesem Fall $p_{01} \weakmay[w_1]_1 p_1\in Qui_1$
      und $p_{02} \weakmay[w_2]_2 p_2\in Qui_2$. Da $p_1$ und $p_2$ in der
      jeweiligen Ruhe-Menge enthalten sind, ist auch der Zustand, der aus ihnen
      zusammengesetzt ist, in der Parallelkomposition ruhig, wie bereits im
      ersten Punkt von Lemma~\ref{RuheZustLem} gezeigt. Es gilt also für die
      Komposition $(p_{01},p_{02}) \weakmay[w]_{12} (p_1,p_2)\in Qui_{12}$ und
      dadurch ist $w$ in der linken Seite der Gleichung enthalten, da $w\in
      \StQT _{12}\subseteq \QET _{12}$ gilt.
  \end{itemize}
\end{proof}

\begin{Kor}[Ruhe-Präkongruenz]
  \label{RuhePraekonKor}
  Die Relation \QRel{} ist eine Präkongruenz bezüglich $\cdot\|\cdot$.
\end{Kor}
\begin{proof}
  Es muss gezeigt werden: Wenn $P_1\QRel P_2$ gilt, so auch $P_{31}\QRel
  P_{32}$ für jedes komponierbare System $P_3$. D.h.\ es ist zu zeigen, dass
  aus $P_1\ERel P_2$ und $\QET{}_1\subseteq \QET{}_2$ sowohl $P_{31}\ERel
  P_{32}$ als auch $\QET{}_{31}\subseteq \QET{}_{32}$ folgt. Dies ergibt sich,
  wie im Beweis zu Korollar~\ref{KommPraekonKor}, aus der Monotonie von
  $\cdot\|\cdot$ auf Sprachen wie folgt:
  \begin{itemize}
    \item $\begin{aligned}[t]
        P_{31} \overset{\mathrm{Korollar}~\ref{KommPraekonKor}}{
          \overset{\mathrm{und}}{\overset{P_1\ERel P_2}{\ERel}}}
        P_{32},
    \end{aligned}$
    \item $\begin{aligned}[t]
        \QET{}_{31} &\overset{\ref{RuheSemSatz}~2.}{=}
        (\QET{}_3\|\QET{}_1)\cup \ET{}_{31}\\
        &\hspace{-0.6cm}\overset{\ET{}_{31}\subseteq
      \ET{}_{32}}{\overset{\mathrm{und}}{\overset{\QET{}_1\subseteq
      \QET{}_2}{\subseteq}}} (\QET{}_3\|\QET{}_2) \cup \ET{}_{32}\\
        &\overset{\ref{RuheSemSatz}~2.}{=} \QET{}_{32}.
    \end{aligned}$
  \vspace*{-0.7cm}
  \end{itemize}
\end{proof}

Im nächsten Lemma soll eine Verfeinerung bezüglich guter Kommunikation mit
Partnern betrachtet werden. Die gute Kommunikation stützt sich dabei auf die
Definition von Tests und der daraus resultierenden Verfeinerung
in~\ref{RuheTestDef}.

\begin{Lem}[Testing-Verfeinerung mit Ruhe]
  \label{RuheTestVerfeinLem}
  Gegeben sind zwei \MEIO{}s $P_1$ und $P_2$ mit der gleichen Signatur. Wenn
  für alle Tests $T$, die Partner von $P_1$ bzw. $P_2$ sind, $P_2\QsatAs T
  \Rightarrow P_1\QsatAs T$ gilt, dann folgt daraus die Gültigkeit von
  $P_1\QRel P_2$.
\end{Lem}
\begin{proof}
  Da $P_1$ und $P_2$ die gleiche Signatur haben, wird $I:=I_1=I_2$ und
  $O:=O_1=O_2$ definiert. Für jeden Test Partner $T$ gilt $I_T=O$ und
  $O_T=I$.\\
  Um zu zeigen, dass die Relation $P_1\QRel P_2$ gilt, müssen die folgenden
  Punkte nachgewiesen werden:
  \begin{itemize}
    \item $P_1\ERel P_2$,
    \item $\QET _1\subseteq \QET _2$.
  \end{itemize}
  In Lemma~\ref{KommTestVerfeinLem} wurde bereits etwas Ähnliches gezeigt,
  jedoch wurde dort als Voraussetzung $P_2\EsatAs T\Rightarrow P_1\EsatAs T$
  für alle Test Partner $T$ verwendet und hier dieselbe Aussage mit der Test
  Erfüllung für Ruhe. Die hier verwendeten Tests sagen nichts darüber aus,
  welche Art von \glqq fehlerhaftem Zustand\grqq{} enthalten ist. Die Aussage
  des Lemmas~\ref{KommTestVerfeinLem} kann hier also nicht verwendet werden.
  Aus der lokalen Erreichbarkeit eines Fehler-Zustandes in der
  Parallelkomposition einer as"=Implenetierung von von $P_1$ mit $T$ lässt sich
  nur schließen, dass $P_2$ den Test $T$ ebenfalls nicht as"=erfüllt. Dies kann
  aber aufgrund einer as"=Implementierung von $P_2$ sein, die in
  Parallelkomposition mit $T$ einen Fehler- oder Ruhe-Zustand lokal erreicht.
  Analoges gilt auch für die lokale Erreichbarkeit eines Ruhe"=Zustandes in der
  Komposition einer as"=Implementierung von $P_1$ mit einem Test $T$.\\
  Es muss also für den ersten Punkt noch folgendes nachgewiesen werden:
  \begin{itemize}
    \item $\ET _1\subseteq\ET _2$,
    \item $\EL _1\subseteq\EL _2$.
  \end{itemize}
  Es wird nun damit begonnen, den ersten Unterpunkt des ersten Beweispunktes zu
  zeigen, d.h.\ es wird unter der Voraussetzung $P_2\QsatAs T\Rightarrow
  P_1\QsatAs T$ gezeigt, dass $\ET _1\subseteq\ET _2$ gilt. Da beide
  \ET{}-Mengen unter \cont{} abgeschlossen sind, reicht es ein präfix-minimales
  Element $w\in\ET _1$ zu betrachten und zu zeigen, dass dieses $w$ oder eines
  seiner Präfixe in $\ET _2$ enthalten ist. $w$ muss, wegen
  Proposition~\ref{KommSemProp}, in einer as"=Implementierung $P'_1$ von $P_1$
  ebenfalls ein präfix-minimales Element in $\ET _{P'_1}$ sein.
  \begin{itemize}
    \item Fall 1 ($w=\varepsilon$): Es handelt sich um einen lokal erreichbaren
      Fehler-Zustand in $P'_1$. Für $T$ wird ein Transitionssystem verwendet,
      das nur aus dem Startzustand, einer must-Schleife für alle Inputs $x\in
      I_U$ und einer must-Schlinge für $\tau$ besteht. Somit kann $P'_1$ die im
      Prinzip gleichen Fehler-Zustände lokal erreichen wie $P'_1\|T$. Es gibt
      also einen lokal erreichbaren Zustand von $P'_1\|T$, der in $E_{P'_1\|T}$
      enthalten ist. Somit erfüllt $P_1$ nicht alle Tests $T$ und es muss somit
      auch mindestens eine as"=Implementierung $P'_2$ von $P_2$ geben, die den
      Test $T$ ebenfalls nicht erfüllt. Da eine Implementierung den Test $T$
      erfüllt, wenn die Parallelkomposition der Implementierung mit dem Test
      fehler- und ruhe-frei ist, kann die nicht Erfüllung eines Testes sowohl
      an einem Fehler- wie auch einem Ruhe-Zustand liegen. Bei dem lokal
      erreichbaren \glqq fehlerhaften Zustand\grqq{} kann es sich nur um einen
      Fehler handeln, da es in der Komposition mit $T$ keine Ruhe-Zustände
      geben kann. Da $T$ keinen Fehler-Zustand und auch keine fehlenden
      Input-Möglichkeiten enthält, kann der Fehler nur von $P'_2$ geerbt sein.
      Somit muss in $P'_2$ ein Fehler-Zustand lokal erreichbar sein. Es gilt
      also $\varepsilon \in \PrET _{P'_2} \subseteq \ET _{P'_2}$ und mit
      Proposition~\ref{KommSemProp} auch $\varepsilon\in\ET _2$.
    \item Fall 2 ($w=x_1\dots x_n x_{n+1}\in\Sigma ^+$ mit $n\geq 0$ und
      $x_{n+1}\in I$): Es wird der folgende Partner $T$ betrachtet (siehe auch
      Abbildung~\ref{TohneEmitTau}):
      \begin{itemize}
        \item $T=\{p_0,p_1,\dots ,p_{n+1}\}$,
        \item $p_{0T}=p_0$,
        \item $\begin{aligned}[t]
            \may _T = \must _T&=\{(p_j,x_{j+1},p_{j+1})\mid  0\leq j\leq n\}\\
            &\cup\{(p_j,x,p_{n+1})\mid  x\in I_T\backslash\{x_{j+1}\}, 0\leq
            j\leq n\}\\
            &\cup\{(p_{n+1},x,p_{n+1})\mid  x\in I_T\}\\
            &\cup\{(p_j,\tau,p_j)\mid 0\leq j\leq n+1\},
        \end{aligned}$
        \item $E_T=\emptyset$.
      \end{itemize}
      \begin{figure} [h!tbp]
      \begin{center}
        \begin{tikzpicture}[->, >=latex',auto,node distance =3cm, semithick]
          \node (0) {$p_0$};
          \node (1) [right of=0] {$p_1$};
          \node (dots) [right of=1] {$\dots$};
          \node (n) [right of=dots] {$p_n$};
          \node (n1) at ($(1)!0.5!(dots) + (0,-3)$) {$p_{n+1}$};

          \path ($ (0) + (-1,0) $) edge (0)
                (0) edge node {$x_1$} (1)
                    edge [bend right] node [below, sloped] {$x?\neq x_1$} (n1)
                    edge [loop above] node {$\tau$} (0)
                (1) edge node {$x_2$} (dots)
                    edge node [below, sloped] {$x?\neq x_2$} (n1)
                    edge [loop above] node {$\tau$} (1)
                (dots) edge node {$x_n$} (n)
                       edge [dashed] (n1)
                (n) edge node [above, sloped] {$x?\in I_T$} (n1)
                    edge [bend left] node [sloped] {$x_{n+1}$!} (n1)
                    edge [loop above] node {$\tau$} (n)
                (n1) edge [loop below] node {$x?\in I_T, \tau$} (n1);
        \end{tikzpicture}
        \caption{$x?\neq x_j$ steht für alle $x\in I_T\backslash\{x_j\}$}
      \label{TohneEmitTau}
      \end{center}
      \end{figure}
      Die Menge der Ruhe-Zustände des hier betrachteten $T$s ist leer. Da im
      Vergleich zum Transitionssystem in Abbildung~\ref{UohneE} nur die
      $\tau$-Schlingen ergänzt wurden und die Umbennennung der Mengen, ändert
      sich nichts an den Fällen 2a) und 2b) aus dem Beweis der selben Inklusion
      von Lemma~\ref{KommTestVerfeinLem}. Die Begründungen, wieso in den beiden
      Fällen $\varepsilon\in\PrET (P'_1\|T)$ gilt, bleibt also analog zum
      Beweis des ersten Punktes des Lemmas aus dem vorangegangnen Kapitel.
      Durch die must-$\tau$-Schlingen wurde, genau wie im letzten Fall nur
      erreicht, das in einer Parallelkomposition mit $T$ keine Ruhe-Zustände
      möglich sind. Es kann also auch hier aus der lokalen Erreichbarkeit eines
      Fehler in $P'_1\|T$ auf die lokale Erreichbarkeit eines Fehler-Zustandes
      in $P'_2\|T$ für eine as"=Implementierung $P'_2$ von $P_2$ geschlossen
      werden. Die weitere Argumentation verläuft analog zu Fall 2, derselben
      Inklusion im Beweis von Lemma~\ref{KommTestVerfeinLem}. Da $\tau$s nur
      interne Aktionen einer einzelnen Komponente sind, verändert sich auch
      nichts an den Traces über die argumentiert wird. Es können zwar
      möglicherweise $\tau$-Transitionen ausgeführt werden, diese können jedoch
      weder zu einem Fehler führen noch beeinflussen, dass ein anderer Trace
      nicht ausgeführt werden kann.
  \end{itemize}

  Nun wird mit dem zweiten Unterpunkt des ersten Beweispunktes begonnen. Genau
  wie im Beweis zu~\ref{KommTestVerfeinLem} ist hier jedoch aufgrund des
  bereits geführten Beweisteils nur noch $L_1\backslash\ET _1\subseteq\EL _2$
  zu zeigen. Es wird also für ein beliebig gewähltes $w\in L_1\backslash\ET _1$
  gezeigt, dass dieses auch in $\EL _2$ enthalten ist. Aufgrund der
  Propositionen~\ref{LImpProp} und~\ref{KommSemProp} gibt es auch eine
  as"=Implementierung $P'_1$ von $P_1$ für die $w\in L_{P'_1}\backslash \ET
  _{P'_1}$ gilt.
  \begin{itemize}
    \item Fall 1 ($w=\varepsilon$): Ebenso wie in~\ref{KommTestVerfeinLem} gilt
      auch hier, dass $\varepsilon$ immer in $\EL _2$ enthalten ist.
    \item Fall 2 ($w=x_1\dots x_n$ mit $n\geq 1$): Die Konstruktion des
      Partners $T$ weicht wie im letzten Beweisteil nur durch die
      $\tau$-must-Schleifen an den Zuständen des Transitionssystems vom Beweis
      des zweiten Punktes aus Lemma~\ref{KommTestVerfeinLem} ab. Somit ist der
      Partner $T$ dann wie folgt definiert (siehe dazu auch
      Abbildung~\ref{TmitEundTau}):
      \begin{itemize}
        \item $T=\{p_0,p_1,\dots ,p_n,p\}$,
        \item $p_{0T}=p_0$,
        \item $\begin{aligned}[t]
            \may _T = \must _T&=\{(p_j,x_{j+1},p_{j+1})\mid 0\leq j< n\}\\
            &\cup\{(p_j,x,p)\mid x\in I_T\backslash\{x_{j+1}\},0\leq j < n\}\\
            &\cup\{(p_j,\tau ,p_j)\mid 0\leq j\leq n\}\\
            &\cup\{(p,\alpha ,p)\mid \alpha\in I_T\cup \{\tau\}\},
              \end{aligned}$
        \item $E_T=\{p_n\}$.
      \end{itemize}
      \begin{figure} [h!tbp]
      \begin{center}
        \begin{tikzpicture}[->, >=latex',auto,node distance =3cm, semithick]

          \node (0) {$p_0$};
          \node (1) [right of=0] {$p_1$};
          \node (dots) [right of=1] {$\dots$};
          \node (n1) [right of=dots] {$p_{n-1}$};
          \node (n) [right of=n1, rectangle, draw] {$p_n\in E_T$};
          \node (q) at ($(1)!0.5!(dots) + (0,-3)$) {$p$};

          \path ($ (0) + (-1,0) $) edge (0)
                (0) edge node {$x_1$} (1)
                    edge [bend right] node [below, sloped] {$x?\neq x_1$} (q)
                    edge [loop above] node {$\tau$} (0)
                (1) edge node {$x_2$} (dots)
                    edge node [below, sloped] {$x?\neq x_2$} (q)
                    edge [loop above] node {$\tau$} (1)
                (dots) edge node {$x_{n-1}$} (n1)
                       edge [dashed] (q)
                (n1) edge node {$x_n$} (n)
                     edge [bend left] node [below, sloped] {$x?\neq x_n$} (q)
                     edge [loop above] node {$\tau$} (n1)
                (q) edge [loop below] node {$x?\in I_T, \tau$} (q)
                (n) edge [loop above] node {$\tau$} (n);
        \end{tikzpicture}
        \caption{$x?\neq x_j$ steht für alle $x\in I_T\backslash\{x_j\}$, $p_n$
          ist der einzige Fehler-Zustand}
      \label{TmitEundTau}
      \end{center}
      \end{figure}
      Da durch die $\tau$-must-Schlingen an den Zuständen wie oben vermieden
      wird, dass es in einer Komposition mit $T$ und auch in $T$ selbst
      Ruhe-Zustände gibt, verläuft der Rest des Beweises dieses Punktes analog
      zum Beweis der selben Inklusionsrichtung von
      Lemma~\ref{KommTestVerfeinLem}. Und somit gilt für alle Fälle (2a) bis
      2d)), dass $w$ in $\EL _2$ enthalten ist.
  \end{itemize}

  So bleibt nur noch der letzt Beweispunkt zu zeigen, d.h.\ die Inklusion $\QET
  _1\subseteq\QET _2$. Die Inklusion kann jedoch, analog zum Beweis der
  Inklusion der Fehler-gefluteten Sprache, noch weiter eingeschränkt werden. Da
  bereits bekannt ist, dass $\ET _1\subseteq\ET _2$ gilt, muss nur noch $\StQT
  _1\backslash\ET _1\subseteq\QET _2$ gezeigt werden.\\
  Es wird ein $w\in\StQT _1\backslash\ET _1$ gewählt und gezeigt, dass dieses
  auch in $\QET _2$ enthalten ist. Mit den Propositionen~\ref{KommSemProp}
  und~\ref{RuheTraceProp} kann gefolgert werden, dass es auch eine
  as"=Implementierung $P'_1$ von $P_1$ gibt, für die $w\in\StQT _{P'_1}
  \backslash \ET _{P'_1}$ gilt.\\
  Durch die Wahl des $w$s wird vom Startzustand von $P'_1$ durch das Wort $w$
  ein ruhiger Zustand erreichbar. Dies hat nur Auswirkungen auf die
  Parallelkomposition $P'_1\|T$, wenn in $T$ ebenfalls ein Ruhe-Zustand durch
  $w$ erreichbar ist.\\
  Das betrachtete $w$ hat also die Form $w=x_1\dots x_n\in\Sigma ^*$ mit $n\geq
  0$. Es wird der folgende Partner $T$ betrachtet (siehe auch
  Abbildung~\ref{TohneEmitI}):
  \begin{itemize}
    \item $T=\{p_0,p_1,\dots ,p_n, p\}$,
    \item $p_{0T}=p_0$,
    \item $\begin{aligned}[t]
        \may _T = \must _T&=\{(p_j,x_{j+1},p_{j+1})\mid  0\leq j< n\}\\
        &\cup\{(p_j,x,p)\mid  x\in I_T\backslash\{x_{j+1}\}, 0\leq j< n\}\\
        &\cup\{(p_j,\tau,p_j)\mid 0\leq j< n\}\\
        &\cup\{(p_n,x,p)\mid x\in I_T\}\\
        &\cup\{(p,\alpha,p)\mid \alpha\in I_T\cup\{\tau\}\},
    \end{aligned}$
    \item $E_T=\emptyset$.
  \end{itemize}
  \begin{figure} [h!tbp]
  \begin{center}
    \begin{tikzpicture}[->, >=latex',auto,node distance =3cm, semithick]
      \node (0) {$p_0$};
      \node (1) [right of=0] {$p_1$};
      \node (dots) [right of=1] {$\dots$};
      \node (n) [right of=dots, rectangle, dotted, draw] {$p_n\in Qui_T$};
      \node (q) at ($(1)!0.5!(dots) + (0,-3)$) {$p$};

      \path ($ (0) + (-1,0) $) edge (0)
            (0) edge node {$x_1$} (1)
                edge [loop above] node {$\tau$} (0)
                edge [bend right] node [below, sloped] {$x?\neq x_1$} (q)
            (1) edge node {$x_2$} (dots)
                edge [loop above] node {$\tau$} (1)
                edge [below, sloped] node {$x?\neq x_2$} (q)
            (dots) edge node {$x_n$} (n)
                   edge [dashed] (q)
            (n) edge [bend left] node [below,sloped] {$x?\in I_T$} (q)
            (q) edge [loop below] node {$x?\in I_T, \tau$} (q);
    \end{tikzpicture}
    \caption{$x?\neq x_j$ steht für alle $x\in I_T\backslash\{x_j\}$, $p_n$
    ist der einzige Ruhe-Zustand}
  \label{TohneEmitI}
  \end{center}
  \end{figure}
  Falls für das betrachtete $w=\varepsilon$ gilt, reduziert sich der Partner
  $T$ auf den Zustand $p_n=p_0$ und den Zustand $p$. Es ist also in diesem Fall
  der Startzustand gleich dem ruhigen Zustand.\\
  Allgemein ist der Zustand $p_n$ aus $T$ der einzig ruhige Zustand in $T$. Es
  gilt wegen des ersten Punktes von Lemma~\ref{RuheZustLem}, dass auch in der
  Parallelkomposition $P'_1\|T$ ein Ruhe-Zustand mit $w$ erreicht wird. Bei
  allen in $w$ befindlichen Aktionen handelt es sich um synchronisierte
  Aktionen und es gilt $I_T\cap I=\emptyset$. Daraus folgt $w\in O_{P'_1\|T}$
  und $w\in\StQT (P'_1\|T)$. Es kann also in der Parallelkomposition durch $w$
  ein Ruhe-Zustand lokal erreicht werden. Da $w\notin\ET _{P'_1}$ gilt, kann
  auf dem Weg, der mit $w$ im Transitionssystem $P'_1$ zurückgelegt wird, kein
  Fehler lokal erreicht werden. Es kann also weder von $P'_1$ noch von $T$ ein
  Fehler auf diesem Weg geerbt werden und durch den Aufbau von $T$ kann auch
  kein neuer Fehler in der Parallelkomposition der beiden Systeme entstehen. Da
  ein Ruhe-Zustand in $P'_1\|T$ lokal erreichbar ist, muss auch in $P'_2\|T$
  für eine as"=Implementierung $P'_2$ von $P_2$ ein \glqq fehlerhafter
  Zustand\grqq{} lokal erreichbar sein. Es kann zunächst keine Aussage
  getroffen werden, ob das $w$ in $P'_2\|T$ ausführbar ist und ob es sich bei
  dem \glqq fehlerhaften Zustand\grqq{} um Ruhe oder einen Fehler handelt.
  \begin{itemize}
    \item Fall a) ($\varepsilon\in\ET (P'_2\|T)$): Es handelt sich bei dem
      lokal erreichbaren \glqq fehlerhaften Zustand\grqq{} um einen Fehler. Es
      ist somit nicht relevant, ob $w$ ausführbar ist. Der Fehler-Zustand kann
      sowohl von $P'_2$ geerbt sein, wie auch durch fehlende
      Input"=must"=Transitionen als neuer Fehler in der Parallelkomposition
      entstanden sein. Es gilt, dass bereits in $P'_2$ ein Präfix von $w$
      in $\ET _{P'_2}$ enthalten ist, wegen des Beweises des ersten Punktes aus
      Lemma~\ref{KommTestVerfeinLem} und da $T$ nur neue Fehler auf dem Trace
      $w$ zulässt. Die Menge \ET{} ist unter \cont{} abgeschlossen, somit gilt
      $w\in\ET _{P'_1}\subseteq\QET _{P'_2}$. Mit Proposition~\ref{RuheSemProp}
      folgt daraus $w\in\QET _2$.
    \item Fall b) (Ruhe-Zustand lokal erreichbar in $P'_2\|T$ und $\varepsilon
      \notin \ET (P'_2\|T)$): Da in $T$ nur durch $w$ ein ruhiger Zustand
      erreicht werden kann, muss es sich bei dem lokal erreichbaren
      Ruhe-Zustand in $P'_2\|T$ um einen handeln, der mit $w$ erreicht werden
      kann. Mit Lemma~\ref{RuheZustLem} kann somit gefolgert werden, dass auch
      in $P'_2$ ein Ruhe-Zustand mit $w$ erreichbar sein muss. Es gilt $w\in
      \StQT _{P'_2}\subseteq\QET _{P'_2}\subseteq\QET _2$, aufgrund von
      Proposition~\ref{RuheSemProp}.
  \end{itemize}
\end{proof}

\begin{Satz}
  \label{RuheTestVerfSatz}
  Falls $P_1\QRel P_2$ gilt folgt draus auch, dass $P_1$ $P_2$ Ruhe-verfeinert.
\end{Satz}
\begin{proof}
  Nach Definition gilt $w\in\QET (P)$ mit $w\in O(P)^*$ genau dann, wenn in $P$
  ein Ruhe-Zustand oder ein Fehler-Zustand lokal erreichbar ist. $P_1\QRel P_2$
  impliziert, dass $w\in\QET _2$ gilt, wenn $w\in\QET _1$ gilt. Somit ist ein
  Ruhe- oder Fehler-Zustand nur dann in $P_1$ lokal erreichbar, wenn auch ein
  Ruhe- oder Fehler-Zustand in $P_2$ lokal erreichbar ist. Daraus folgt, dass
  es as"=Implementierungen $P'_1$ und $P'_2$ von $P_1$ bzw.\ $P_2$ gibt, die
  analoge Fehler lokal erreichen wegen~\ref{lokaleRuheErrKor}. Es gilt dann
  auch, dass $P'_j\|T$ einen lokal erreichbaren Fehler oder lokal erreichbare
  Ruhe hat, wenn $P'_j$ dies hat, für $j\in\{1,2\}$ und einen Test $T$. Falls
  $P_1$ einen \glqq fehlerhaften Zustand\grqq{} lokal erreicht, dann zeigt sich
  auch in einer as"=Implementierung von $P_1$ dieser durch einen Fehler-Zustand
  oder einen Ruhe-Zustand. In der Parallelkomposition der as"=Implementierung
  mit einem Test $T$ tritt der \glqq fehlerhafte Zustand\grqq{} auf. Falls es
  sich um einen Fehler handelt, dann tritt mit allen Tests $T$ ein Fehler in
  der Parallelkomposition auf. Bei Ruhe hingegen zeigt sich das \glqq
  Fehlverhalten\grqq{} nur, mit Tests $T$, die einen analogen Ruhe-Zustand
  enthalten. Mit der Relation \QRel{} gibt es auch in $P_2$ einen lokal
  erreichbaren \glqq fehlerhaften Zustand\grqq{}, wenn es in $P_1$ einen
  solchen gibt. Es tritt dann auch in einer as"=Implementierung Ruhe oder ein
  Fehler auf, dies zeigt sich dann auch in der Parallelkomposition mit Tests
  $T$. Im Fall von Fehlern, zeigt sich sowohl in der Parallelkomposition einer
  as"=Implementierung von $P_1$ wie auch einer as"=Implementierung von $P_2$
  mit beliebigen Tests $T$, der Fehler, da \ERel{} gilt und somit die
  Argumentationen aus Satz~\ref{KommTestVerfSatz} anwendbar sind. Ruhe tritt
  in $P_2$ nur ohne einen Fehler auf, wenn in $P_1$ auch nur Ruhe und kein
  Fehler auf dem Trace vorhanden war. Ansonsten hätte \ERel{} auch einen Fehler
  in $P_2$ gefordert. In der Parallelkomposition einer as"=Implementierung von
  $P_1$ mit $T$ hat sich die Ruhe bereits gezeigt, also tut sie dies auch in
  der Parallelkomposition einer as"=Implementierung von $P_2$ mit $T$. Es gilt
  also $\neg P_1 \EsatAs T \Rightarrow \neg P_2 \EsatAs T$ für alle Tests $T$.
  Daraus ergibt sich für alle Tests $T$ die Implikation $P_2 \EsatAs T
  \Rightarrow P_1 \EsatAs T$ und somit Ruhe-verfeinert $P_1$ $P_2$.
\end{proof}

Es wurde, wie im letzten Kapitel, eine Kette an Folgerungen gezeigt, die
sich zu einem Ring schließen. Dies ist in Abbildung~\ref{FolgerungsketteQui}
dargestellt.

\begin{figure}[h!tbp]
  \begin{center}
    \begin{tikzpicture}[scale = 3]
      \matrix (m) [matrix of math nodes,row sep=2cm,column sep=4cm]{%
        P_1\QRel P_2 & P_1 \text{ verfeinert } P_2 \\
        \substack{\forall \text{ Test Partner } T:\\P_2\QsatAs T\Rightarrow
        P_1\QsatAs T} &
    \substack{\forall \text{ Tests } T:\\P_2\QsatAs T\Rightarrow P_1\QsatAs T} \\};
        \draw[-implies, double, double distance=1mm]
          (m-1-1) -- node [above] {Satz~\ref{RuheTestVerfSatz}} (m-1-2);
        \draw[-implies, double, double distance=1mm]
          (m-1-2) -- node [right] {Definition~\ref{RuheTestDef}} (m-2-2);
        \draw[-implies, double, double distance=1mm]
          (m-2-1) -- node [left]
          {Lemma~\ref{RuheTestVerfeinLem}} (m-1-1);
        \draw[-implies, double, double distance=1mm]
          (m-2-2) -- node [below]
          {$\substack{T \text{ Test Partner}\\\Downarrow\\ T \text{ Test}}$} (m-2-1);
    \end{tikzpicture}
    \caption{Folgerungskette der Testing-Verfeinerung und Ruhe-Relation}
  \label{FolgerungsketteQui}
  \end{center}
\end{figure}

\begin{Satz}[Zusammenhang der Verfeineruns-Relationenen mit der Ruhe-Relation]
  Für \MEIO{}s $P$ und $Q$ gilt $P \wasRel Q \Rightarrow P \QRel Q \Rightarrow
  P \ERel Q$. Die Implikationen in die andere Richtung gelten jedoch nicht.
\end{Satz}
\begin{proof}\mbox{}\\
  $P \wasRel Q \Rightarrow P \QRel Q$:\\
  Im Beweis des Satzes~\ref{ZusammenhFehlerSatz} wurde bereits bewiesen, dass
  eine schwache as"=Verfeinerungs"=Relation $\mathcal{R}$ zwischen $P$ und $Q$
  die Eigenschaften der Fehler"=Relation \ERel{} erfüllt. Es fehlt für diese
  Implikation also nur noch der Beweis der Inklusion $\QET _P \subseteq \QET
  _Q$. Da bereits $\ET _P\subseteq \ET _Q$ beweisen wurde, reicht es aus zu
  beweißen, dass $\StQT _P \backslash \ET _P\subseteq \QET _Q$ gilt. Für ein
  Wort $w$ aus $\StQT _P \backslash \ET _P$ gilt: $\exists w'\in\Sigma _{\tau}
  ^*, \exists p_1, p_2, \dots , p_n, \exists \alpha _1, \alpha _2, \dots
  ,\alpha _n: \hat{w'} = w \land w' = \alpha _1\alpha _2\dots\alpha _n \land
  p_0 \may[\alpha _1]_P p_1 \may[\alpha _2]_P \dots p_{n-1} \may[\alpha _n]_P
  p_n \in Qui _P$. Aufgrund von~\ref{wSimDef}~4. bzw.~5. gibt es einen analogen
  Trace in $Q$, falls ein $q_j$ für $0\leq j \leq n$ in $E_Q$ angetroffen wird.
  Es gilt $w\in\StET _Q\subset\ET _Q\subset\QET _Q$, falls ein Fehler-Zustand
  in $Q$ auf einem Präfix-Trace von $w$ auftritt. Im folgenden wird davon
  ausgegangen, das $w$ in $Q$ ohne das erreichben eines Fehler"=Zustandes
  ausführbar ist. Es gibt also einen Trace der Form $q_0
  \weakmay[\widehat{\alpha _1}]_Q q_1 \weakmay[\widehat{\alpha _2}]_Q \dots
  q_{n-1} \weakmay[\widehat{\alpha _n}]_Q q_n$ in $Q$ mit $p_j \mathcal{R} q_j$
  für alle $j$ aus $\{0,1,\dots n\}$. $p_n$ ist für $P$ ein Ruhe-Zustand, es
  gilt also für alle $\alpha \in (O\cup \{\tau\})$ $p_n \nmust[\alpha]_P$. Da
  $\mathcal{R}$ eine schwache as"=Verfeinerungs"=Relation ist, muss
  wegen~\ref{wSimDef}~3. auch $q_n \nmust[\alpha]_Q$ gelten für alle $\alpha
  \in (O\cup \{\tau\})$. $q_n$ ist also auch ein ruhiger Zustand. Somit ist $w$
  in $\StQT _Q\subset\QET _Q$ enthalten.

  $P \QRel Q \Rightarrow P \ERel Q$:\\
  Diese Implikation folgt direkt aus der Definition von \QRel{}
  in~\ref{RuheSemDef}. Da $P \QRel Q$ dort definiert wurde als Relation, die
  $P \ERel Q$ un $\QET _P \subset \QET _Q$ erfüllt. Es gilt also $P \ERel Q$.

  $P \wasRel Q \hspace{0.1cm}\not\hspace{-0.1cm}\Leftarrow P \QRel Q$:\\
  Wie im Gegenbeispiel für die analoge Implikation aus der Relation \ERel{}
  beruht der Grund für die nicht Gültigkeit hier auf darauf, dass Simulationen
  strenger sind als Sprach Inklusionen. Jedoch funktioniert hier nicht das
  gleiche Gegenbeispiel, da es zu Problemen mit den Ruhe-Traces führen würde.
  Um diese zu vermeiden, wird wieder die Technik angewendet an alle Zustände
  eine $\tau$-Schleife anzufügen. Das daraus entstehende Gegenbeipiel ist in
  Abbildung~\ref{WasQuiGegenBsp} dargestellt. Die Menge der Inputs $I$ der
  beiden \MEIO{}s ist leer. Es gilt also $\ET _P = \ET _Q = \QET _P = \QET _Q =
  \emptyset$ und $\{\varepsilon\} = L(P) \subset L(Q) = \{\varepsilon , o\}$.\\
  Angenommen es gib eine schache as"=Verfeinerungs"=Relation $\mathcal{R}$
  zwischen $P$ und $Q$. Dann stehen die Startzustände in dieser Relation, es
  gilt also $p_0 \mathcal q_0$. Da es keine Fehler"=Zustände in $P$ gibt,
  ist~\ref{wSimDef}~1.\ erfüllt. Die $\tau$"=must"=Schlingen der
  Startzustände werden bezüglich der Punkt 3. und 5. der
  Definition~\ref{wSimDef} gematched. Der 2. Punkt von~\ref{wSimDef} stellt
  keine Forderungen an die Relation $\mathcal{R}$. Die Transition $q_0
  \must[o]_Q q_1$ fordert durch~\ref{wSimDef}~3. ihre Verfeinerung in $P$. Da
  es jedoch keine mit $o$ beschriftete Transition in $P$ gibt, kann
  $\mathcal{R}$ keine as"=Verfeinerungs"=Relation zwischen $P$ und $Q$ sein.

  \begin{figure}[htbp]
    \begin{center}
      \begin{tikzpicture}[shorten >=1pt,auto,node distance=2.5cm]
        \node [initial,initial text=$Q$:] (q0) at (0,0) {$q_0$};
        \node (q1) [right of=q0] {$q_1$};

        \path[->]
        (q0) edge node{$o!$} (q1)
        (q0) edge[loop above] node{$\tau$} (q0)
        (q1) edge[loop right] node{$\tau$} (q1)
        ;

        \node [initial,initial text=$P$:] (p0) at (7,0) {$p_0$};

        \path[->]
        (p0) edge[loop right] node{$\tau$} (p0)
        ;
      \end{tikzpicture}
      \caption{Gegenbeispiel zu $\wasRel \Leftarrow \QRel$}
      \label{WasQuiGegenBsp}
    \end{center}
  \end{figure}

  $P \QRel Q \hspace{0.1cm}\not\hspace{-0.1cm}\Leftarrow P \ERel Q$:\\
  Die Relation \QRel{} stützt sich auf die Definition der Relation \ERel{}.
  Jedoch erweiter sich die Definition noch um eine weiter Voraussetzung. Es
  muss also in einem entsprechenden Gegenbeispiel die Inklustion $\QET _P
  \subseteq \QET _Q$ verletzt sein. Das Gegenbeispiel ist in
  Abbildung~\ref{QuiEGegenBsp} dargestellt. Es wird $I = \emptyset$
  vorausgesetzt, damit keine Input-kritischen Traces auftretten. Es gilt also
  $\ET _P = \ET _Q = \emptyset$ und $L(P) = L(Q) = \{\varepsilon\}$. Es gilt
  also $P\ERel Q$.\\
  Jedoch gilt für die strickten Ruhe-Traces $\StQT _P = \{\varepsilon\}$ und
  $\StQT _Q = \emptyset$. Es folgt also $\QET _P \not\subseteq \QET _Q$ und
  somit ist die Relation \QRel{} zwischen $P$ und $Q$ auch nicht erfüllt.

  \begin{figure}[htbp]
    \begin{center}
      \begin{tikzpicture}[shorten >=1pt,auto,node distance=2.5cm]
        \node [initial,initial text=$Q$:] (q0) at (0,0) {$q_0$};

        \node [initial,initial text=$P$:] (p0) at (7,0) {$p_0$};

        \path[->]
        (p0) edge[loop right] node{$\tau$} (p0)
        ;

      \end{tikzpicture}
      \caption{Gegenbeispiel zu $\QRel \Leftarrow \ERel$}
      \label{QuiEGegenBsp}
    \end{center}
  \end{figure}
\end{proof}
