\section{Konjunktion}

Um die Konjunktion zwischen zwei \MEIO{}s bilden zu können muss man diese
zuerst normalisieren, da sonst nicht gewolltes Verhalten auftreten kann. Die
Konjunktion soll auf dem Kreuzprodukt der beiden Transitionssysteme beruhen,
also für \MEIO{}s $Q$ und $P$ eine \MEIO{} der Art $Q\times P$ mit
möglicherweise kleinen Abwandlungen. Eine Verfeinerung, die beide
Spezifikationen erfüllt, kann dabei nur Fehler enthalten, die beide
Spezifikationen zulassen. Es sollte also für die Konjunktion $\ET _Q\cap \ET
_P$ gelten. Im Beispiel in Abbildung~\ref{BspKreuzpConj} soll verdeutlicht
werden, wieso es bei \MEIO{}s ohne Normalisierung zu Problemen mit dieser
Forderung kommen kann. Für $P$ un $Q$ ist $i$ ein Fehler"=Trace. Da jedoch die
Output"=Transitionen alleine nicht in das Kreuzprodukt übernommen werden,
enthält $Q\times P$ kleinen Fehler"=Zustand und somit ist auch die Menge $\ET
_{Q\times P}$ leer. Durch eine Normalisierung, in der die lokalen Aktionen
bereits von einem Fehler"=Trace entfernt würden und der letzten Input bereits
zu dem Fehler"=Zustand führen würde, könnte dieses Problem verhindert werden.

\begin{figure}[htbp]
  \begin{center}
    \begin{tikzpicture}[shorten >=1pt,auto,node distance=2.5cm]
      \node [initial,initial text=$Q$:] (q0) at (0,0) {$q_0$};
      \node (q1) [right of=q0] {$q_1$};
      \node (q2) [right of=q1] {$q_2\in E_Q$};

      \path[->]
      (q0) edge node{$i?$} (q1)
      (q1) edge node{$o!$} (q2)
      ;

      \node [initial,initial text=$P$:] (p0) at (9,0) {$p_0$};
      \node (p1) [right of=p0] {$p_1$};
      \node (p2) [right of=p1] {$p_2\in E_P$};

      \path[->]
      (p0) edge node{$i?$} (p1)
      (p1) edge node{$o'!$} (p2)
      ;


      \node [initial,initial text=$Q\times P$:] (qp0) at (5.5,-1.5) {$(q_0,p_0)$};
      \node (qp1) [right of=qp0] {$(q_1,p_1)$};

      \path[->]
      (qp0) edge node{$i?$} (qp1)
      ;
    \end{tikzpicture}
    \caption{Beispiel um die Notwendigkeit der Normalisierung zu verdeutlichen}
    \label{BspKreuzpConj}
  \end{center}
\end{figure}

\begin{Def}[Normal Form]
  Ein \MEIO{} $P$ ist in \emph{Normal Form} (\NF{}), falls die Menge $E_P$ der
  Fehler"=Zustände nur ein einziges Element enthält $E_P=\{e\}$ und es gilt
  zusätzlich die Bedingung $e \must{a} e$ für alle $a\in\Sigma$ und die $p
  \may[\alpha] e$ mit $p\neq e$ impliziert $\alpha\in I$.\\
  Die \emph{Normal Form} von $P$ ist ein \MEIO{} $\NF (P)$, das man aus $P$
  durch die nachfolgenden zwei Schritte erhält:
  \begin{enumerate}[(i)]
    \item Definiere $\overline{E_P} = \{p \mid \exists p'\in E_P, w\in O^*:
      p\weakmay[w]_P p'\}$ und ersetzte $E_P$ durch $\overline{E_P}$.
    \item Falls $p_0\in \overline{E_P}$, ist $NF(P) = (\{p_0\}), I, O, \{p_0
      \must[a] p_0 \mid a\in\Sigma\}, p_0, \{p_0\})$.\\
      Ansonsten, füge einen neuen Zustand $e$ hinzu, der der einzige
      Fehler"=Zustand wird. Wann immer $p \may[\alpha]_P p'\in \overline{E_P}$
      und $p\notin \overline{E_P}$ für ein $\alpha\in \Sigma \cup \{\tau\}$
      gilt (dann gilt zwingendermaßen $\alpha \in I$), entferne alle
      $\alpha$-Transitionen, die von $P$ ausgehen und füge die Transition
      $p\may[\alpha]_{\NF (P)} e$ und falls $p \must[\alpha]_P p'\in
      \overline{E_P}$ galt auch $p \must[\alpha]_{\NF (P)} e$ hinzu. Am Ende
      füge alle $e\must[a]_{\NF (P)} e$ mit $a\in\Sigma$ hinzu.
  \end{enumerate}
\end{Def}

Schritt (i) verändert nichts an den Transitionen und auch nichts daran, ob ein
Zustand einen fehlenden Input hat oder nicht. Die Menge \StET{} wird verändert
jedoch bleibt die Menge \PrET{} gleich, da \StET{} am Ende alle Traces enthält,
die direkt oder durch lokale Aktionen zu einem Fehler"=Zustand des
ursprünglichen $P$ führen. Für die Menge $\MIT\backslash\cont (\PrET)$ ändert
sich nichts, da sich an Zuständen ohne Zusammenhang mit Fehler"=Zuständen
nichts ändert. Das Ergebnis von (i) entspricht also $P$ bezüglich der Relation
\ERel{}. Im ersten Fall von Schritt (ii) gilt $\ET _P =\Sigma ^* = \ET (\NF
(P))$. Im zweiten Fall ändert der Schritt (ii) nichts an Zuständen, die
fehlende Inputs haben, solange diese nicht in der Menge $\overline{E_P}$
enthalten sind, da meine must"=Transition zum Zustand $e$ nur hinzu gefügt
wird, wenn der entsprechende Input davor bereits eine ausgehende
must"=Transition des Zustands $p$ war. Die Menge \MIT{} kann sich jedoch
verkleinern, falls der Trace in $\cont (\PrET)$ enthalten war. Die Menge \ET
bleibt also im Schritt (ii) gleich. Die Menge \EL wird mit \ET geflutet und
somit gilt für \EL auch $\EL (P) = \EL (\NF (P))$. Somit ist $\NF (P)$
äquivalent zu $P$ bezüglich der Relation \ERel{}. Es gilt also die folgende
Proposition.

\begin{Prop}[Normal Form]
  Jedes \MEIO{} $P$ ist äquivalent zu $\NF (P)$ bezüglich \ERel{}.
\end{Prop}

Somit können wird davon ausgehen, dass ein \MEIO{} in Normal Form ist, falls
dies notwendig ist.
