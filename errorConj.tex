\section{Konjunktion}

Neben der Parallelkomposition ist die Konjunktion einer der wichtigsten
Operatoren, die man auf Transitionssysteme anwenden kann, die wie
Spezifikationen und Verfeinerungen behandelt werden. Dieser Operator erlaubt es
unterschiedliche Verhaltensweisen eines Systems separat zu definieren und
daraus durch Konjunktion dann eine vollständige Spezifikation den gesamten
Systems zu erhalten. Die Konjunktion sollte also die größte Spezifikation sein,
die alle gegebenen einzel Komponenten Spezifikationen verfeinert, d.h.\ sie
sollte die größte untere Schranke der Verfeinerungs Präkongruenz
charakterisieren.\\
Um die Konjunktion zwischen zwei \MEIO{}s bilden zu können muss man diese
zuerst normalisieren, da sonst nicht gewolltes Verhalten auftreten kann. Die
Konjunktion soll auf dem Kreuzprodukt der beiden Transitionssysteme beruhen,
also für \MEIO{}s $Q$ und $P$ eine \MEIO{} der Art $Q\times P$ mit
möglicherweise kleinen Abwandlungen. Eine Verfeinerung, die beide
Spezifikationen erfüllt, kann dabei nur Fehler enthalten, die beide
Spezifikationen zulassen. Es sollte also für die Konjunktion $\ET _Q\cap \ET
_P$ gelten. Im Beispiel in Abbildung~\ref{BspKreuzpConj} soll verdeutlicht
werden, wieso es bei \MEIO{}s ohne Normalisierung zu Problemen mit dieser
Forderung kommen kann. Für $P$ un $Q$ ist $i$ ein Fehler"=Trace. Da jedoch die
Output"=Transitionen alleine nicht in das Kreuzprodukt übernommen werden,
enthält $Q\times P$ kleinen Fehler"=Zustand und somit ist auch die Menge $\ET
_{Q\times P}$ leer. Durch eine Normalisierung, in der die lokalen Aktionen
bereits von einem Fehler"=Trace entfernt würden und der letzten Input bereits
zu dem Fehler"=Zustand führen würde, könnte dieses Problem verhindert werden.

\begin{figure}[htbp]
  \begin{center}
    \begin{tikzpicture}[shorten >=1pt,auto,node distance=2.5cm]
      \node [initial,initial text=$Q$:] (q0) at (0,0) {$q_0$};
      \node (q1) [right of=q0] {$q_1$};
      \node (q2) [right of=q1] {$q_2\in E_Q$};

      \path[->]
      (q0) edge node{$i?$} (q1)
      (q1) edge node{$o!$} (q2)
      ;

      \node [initial,initial text=$P$:] (p0) at (9,0) {$p_0$};
      \node (p1) [right of=p0] {$p_1$};
      \node (p2) [right of=p1] {$p_2\in E_P$};

      \path[->]
      (p0) edge node{$i?$} (p1)
      (p1) edge node{$o'!$} (p2)
      ;

      \node [initial,initial text=$Q\times P$:] (qp0) at (5.5,-1.5) {$(q_0,p_0)$};
      \node (qp1) [right of=qp0] {$(q_1,p_1)$};

      \path[->]
      (qp0) edge node{$i?$} (qp1)
      ;
    \end{tikzpicture}
    \caption{Beispiel um die Notwendigkeit der Normalisierung zu verdeutlichen}
    \label{BspKreuzpConj}
  \end{center}
\end{figure}

\begin{Def}[Normal Form]
  \label{normalFormDef}
  Ein \MEIO{} $P$ ist in \emph{Normal Form} (\NF{}), falls die Menge $E_P$ der
  Fehler"=Zustände nur ein einziges Element enthält $E_P=\{e\}$ und es gilt
  zusätzlich die Bedingung $e \must{a} e$ für alle $a\in\Sigma$ und die $p
  \may[\alpha] e$ mit $p\neq e$ impliziert $\alpha\in I$.\\
  Die \emph{Normal Form} von $P$ ist ein \MEIO{} $\NF (P)$, das man aus $P$
  durch die nachfolgenden zwei Schritte erhält:
  \begin{enumerate}[(i)]
    \item Definiere $\overline{E_P} = \{p \mid \exists p'\in E_P, w\in O^*:
      p\weakmay[w]_P p'\}$ und ersetzte $E_P$ durch $\overline{E_P}$.
    \item Falls $p_0\in \overline{E_P}$, ist $NF(P) = (\{p_0\}), I, O, \{p_0
      \must[a] p_0 \mid a\in\Sigma\}, p_0, \{p_0\})$.\\
      Ansonsten, füge einen neuen Zustand $e$ hinzu, der der einzige
      Fehler"=Zustand wird. Wann immer $p \may[\alpha]_P p'\in \overline{E_P}$
      und $p\notin \overline{E_P}$ für ein $\alpha\in \Sigma \cup \{\tau\}$
      gilt (dann gilt zwingendermaßen $\alpha \in I$), entferne alle
      $\alpha$-Transitionen, die von $P$ ausgehen und füge die Transition
      $p\may[\alpha]_{\NF (P)} e$ und falls $p \must[\alpha]_P p'\in
      \overline{E_P}$ galt auch $p \must[\alpha]_{\NF (P)} e$ hinzu. Am Ende
      füge alle $e\must[a]_{\NF (P)} e$ mit $a\in\Sigma$ hinzu.
  \end{enumerate}
\end{Def}

Schritt (i) verändert nichts an den Transitionen und auch nichts daran, ob ein
Zustand einen fehlenden Input hat oder nicht. Die Menge \StET{} wird verändert
jedoch bleibt die Menge \PrET{} gleich, da \StET{} am Ende alle Traces enthält,
die direkt oder durch lokale Aktionen zu einem Fehler"=Zustand des
ursprünglichen $P$ führen. Für die Menge $\MIT\backslash\cont (\PrET)$ ändert
sich nichts, da sich an Zuständen ohne Zusammenhang mit Fehler"=Zuständen
nichts ändert. Das Ergebnis von (i) entspricht also $P$ bezüglich der Relation
\ERel{}. Im ersten Fall von Schritt (ii) gilt $\ET _P =\Sigma ^* = \ET (\NF
(P))$. Im zweiten Fall ändert der Schritt (ii) nichts an Zuständen, die
fehlende Inputs haben, solange diese nicht in der Menge $\overline{E_P}$
enthalten sind, da meine must"=Transition zum Zustand $e$ nur hinzu gefügt
wird, wenn der entsprechende Input davor bereits eine ausgehende
must"=Transition des Zustands $p$ war. Die Menge \MIT{} kann sich jedoch
verkleinern, falls der Trace in $\cont (\PrET)$ enthalten war. Die Menge \ET
bleibt also im Schritt (ii) gleich. Die Menge \EL wird mit \ET geflutet und
somit gilt für \EL auch $\EL (P) = \EL (\NF (P))$. Somit ist $\NF (P)$
äquivalent zu $P$ bezüglich der Relation \ERel{}. Es gilt also die folgende
Proposition.

\begin{Prop}[Normal Form]
  Jedes \MEIO{} $P$ ist äquivalent zu $\NF (P)$ bezüglich \ERel{}.
\end{Prop}

Somit können wird davon ausgehen, dass ein \MEIO{} in Normal Form ist, falls
dies notwendig ist.\\
Die Transformation in die Normal Form aus der Definition~\ref{normalFormDef}
ist effizient für endliche \MEIO{}s: Für Schritt (i)  kann eine Tiefensuche
(DFS) in linearer Zeit von den Zuständen in $E$ über umgekehrte lokalen
Transitionen. Für der erste Fall von Schritt (ii) benötigt nur konstante Zeit,
da nur der Startzustand überprüft werden muss. Für den zweiten Fall kann erneut
eine DFS ausgeführt werden vom Startzustand $p_0$ ohne die Transitionen zu
benutzen, die gelöscht werden sollen. Alle Zustände, die dadurch nicht besucht
werden sind nicht erreichbar in $\NF (P)$ und können entfernt werden. $\NF (P)$
kann also möglicherweise deutlich kleiner sein wie $P$. Es ergibt sich das
folgende Lemma.

\begin{Lem}
  Falls ein \MEIO{} in Normal Form (mit einem Fehler"=Zustand $e$) ist, $w\in
  \ET (P)$ gilt genau dann, wenn $w$ sich aus einem Ablauf ergibt, der $e$
  erreicht. Ein $w\in\EL (P)$ ergibt sich genau dann, wenn $w$ ein Ablauf von
  $P$ ist. Somit gilt $\EL (P) = L(P)$.
\end{Lem}

Es wurden durch die Normalisierung die Probleme mit den Traces in $\cont
(\PrET)$ gelöst, und man erhält dann Transitionssysteme, die die Eigenschaften
der \MIA{}s aus z.B.~\cite{Vogler2016MIA3} erfüllen. Jedoch Kan die Konjunktion
hier trotzdem nicht auf die gleiche Art gebildet werden, wie dort. Das Problem,
das noch nicht gelöst ist, beruht auf den Input"=kritischen Traces.
In~\cite{Vogler2016MIA3} kann ein Zustand in der Konjunktion, der für eines der
Transitionssysteme einen must"=Input ($P$) hat und für den anderen keine
entsprechende may"=Transition ($Q$) der Zustand als inkonsistent angesehen
werden und aus dem Konjunktion"=Transitionssystem entfernt werden. Dies würde
jedoch hier zu einem Problem mit den Traces führen. Es soll für die Komposition
$\EL _Q \cap \EL _P$ und $\ET _P \cap \ET _Q$ gelten. Der Trace in $Q$ mit
dem entsprechenden nicht vorhanden Input ist ein Input"=kritischer Trace. Er
ist also in der Menge $\ET _Q\subseteq \EL _Q$ enthalten. Da es in $P$ ein
must"=Transition gibt ist der entsprechende Trace in der Sprache von $P$
enthalten, also auch in der Menge $\EL _P$. Es muss also noch den geforderten
Eigenschaften der Konjunktion auch gelten, dass der Trace in der
Fehler"=gefluteten Sprache vor Konjunktion enthalten ist. Durch das entfernen
des Zustandes wie in~\cite{Vogler2016MIA3} wird in vielen Fällen der Trace
nicht in der Fehler"=gefluteten Sprache enthalten sein (siehe dazu auch das
Beispiel in Abbildung~\ref{BspMITConj}).

\begin{figure}[htbp]
  \begin{center}
    \begin{tikzpicture}[shorten >=1pt,auto,node distance=2.5cm]
      \node [initial,initial text=$Q$:] (q0) at (0,0) {$q_0$};
      \node (q1) [right of=q0] {$q_1$};
      \node (q2) [right of=q1] {$q_2$};

      \path[->]
      (q0) edge[loop above] node{$i?$} (q0)
      (q0) edge[dashed] node{$o!$} (q1)
      (q1) edge node{$i?$} (q2)
      (q2) edge[loop above] node{$i?$} (q2)
      ;

      \node [initial,initial text=$P$:] (p0) at (9,0) {$p_0$};
      \node (p1) [right of=p0] {$p_1$};

      \path[->]
      (p0) edge[loop above] node{$i?$} (p0)
      (p0) edge[dashed] node{$o!$} (p1)
      ;

      \node [initial,initial text=$Q\land P$:] (qp0) at (6.5,-1.5) {$(q_0,p_0)$};

      \path[->]
      (qp0) edge[loop above] node{$i?$} (qp0)
      ;
    \end{tikzpicture}
    \caption{Beispiel für \MIT{}-Problem, $\land$ bezeichnet die Konjunktion
    aus~\cite{Vogler2016MIA3}}
    \label{BspMITConj}
  \end{center}
\end{figure}

Man kann diese Problem lösen, in dem man die Zustände aus dem in Konjunktion
entstehenden Transitionssystemen nicht streicht sondern beibehält und die
Transition zu einem Zustands"=Tupel aus dem Zustand, der durch die
must"=Transition erreicht wird und dem Fehler"=Zustand des anderen
Transitionssystems führt.\\
Bis auf diesem Unterschied orientiert sich die Definition der Konjunktion und
des Konjunktiven Produkts an den Definition aus~\cite{Vogler2016MIA3}. Es wird
also zuerst eine Konjunktives Produkt gebildet, aus dem dann in der zweiten
Definition die Zustandspaaren entfernte werden, die nicht Vereinbar sind mit den
Spezifikationen.

\begin{Def}[Konjunktives Produkt]
  Für zwei \MEIO{}s $P_1 = (P_1,I,O,\must _1,\may _1,p_{01},\{e_1\})$ und $P_2
  = (P_2,I,O,\must _2,\may _2,p_{02},\{e_2\})$ in Normal Form mit der selben
  Signatur ist das \emph{Konjunktive Produkt} definiert als $P_1\& P_2 =
  (P_1\times P_2, I, O, \must, \may ,(p_{01},p_{02}),\{(e_1,e_2)\})$ mit den
  folgenden Regeln für die Transitionen:
  \begin{itemize}
    \item[(OMust)] $(p_1,p_2) \must[\omega] (p'_1,p'_2)$, falls
      $p_1 \must[\omega]_1 p'_1$ und $p_2 \weakmay[\hat{\omega}]_2 p'_2$ oder
      $p_1 \weakmay[\hat{\omega}]_1 p'_1$ und $p_2 \must[\omega]_2 p'_2$,
    \item[(IMust1)] $(p_1,p_2) \must[i] (p'_1,p'_2)$, falls $p_1 \must[i]_1
      p'_1$ und $p_2 \may[i]_2 \weakmay[\varepsilon]_2 p'_2$ oder $p_1
      \may[i]_1 \weakmay[\varepsilon]_1 p'_1$ und $p_2 \must[i]_2 p'_2$,
    \item[(IMust2)] $(p_1,p_2) \must[i] (p'_1,e_2)$, falls $p_1 \must[i]_1
      p'_1$ und $p_2 \nmay[i]_2$,
    \item[(IMust3)] $(p_1,p_2) \must[i] (e_1,p'_2)$, falls $p_1 \nmay[i]_1$ und
      $p_2 \must[i]_2 p'_2$,
    \item[(EMust1)] $(p_1,e_2) \must[\alpha] (p'_1,e_2)$, falls $p_1
      \must[\alpha]_1 p'_1$,
    \item[(EMust2)] $(e_1,p_2) \must[\alpha] (e_1,p'_2)$, falls $p_2
      \must[\alpha]_2 p'_2$,
    \item[(May1)] $(p_1,p_2) \may[\tau] (p'_1,p_2)$, falls $p_1
      \weakmay[\tau]_1 p'_1$,
    \item[(May2)] $(p_1,p_2) \may[\tau] (p_1,p'_2)$, falls $p_2
      \weakmay[\tau]_2 p'_2$,
    \item[(OMay)] $(p_1,p_2) \may[\omega] (p'_1,p'_2)$, falls $p_1
      \weakmay[\omega]_1 p'_1$ und $p_2 \weakmay[\omega]_2 p'_2$,
    \item[(IMay1)] $(p_1,p_2) \may[i] (p'_1,p'_2)$, falls $p_1 \may[i]_1
      \weakmay[\varepsilon]_1 p'_1$ und $p_2 \may[i]_2 \weakmay[\varepsilon]_2
      p'_2$,
    \item[(IMay2)] $(p_1,p_2) \may[i] (p'_1,e_2)$, falls $p_1 \may[i]_1 p'_1$
      und $p_2 \nmay[i]_2$,
    \item[(IMay3)] $(p_1,p_2) \may[i] (e_1,p'_2)$, falls $p_1 \nmay[i]_1$ und
      $p_2 \may[i]_2 p'_2$,
    \item[(EMay1)] $(p_1,e_2) \may[\alpha] (p'_1,e_2)$, falls $p_1
      \may[\alpha]_1 p'_1$,
    \item[(EMay2)] $(e_1,p_2) \may[\alpha] (e_1,p'_2)$, falls $p_2
      \may[\alpha]_2 p'_2$.
  \end{itemize}
\end{Def}

Aus schwachen Transitionen der einzelnen \MEIO{}s entstehen im Konjunktiven
Produkt einzelnen Transitionen (z.B.\ (OMust), (IMust), (May)).
$\tau$"=Transitionen können via der Regel (OMay) synchronisiert werden. In den
Regel (EMust) und (EMay) wirkt die Fehler"=Zustände als neutrales Element. In
der Parallelkomposition hingeben hatten die Fehler"=Zustände absorbierende
Wirkung.

\begin{Def}[Konjunktion]
  Gegeben ein Konjunktives Produkt $P_1\& P_2$, die Menge $F\subseteq P_1\times
  P_2$ der (logisch) \emph{inkonsistenten Zustände} ist definiert als die
  kleinste Menge, die die folgende Regeln erfüllt für $p_1\neq e_1$ und $p_2
  \neq e_2$:
  \begin{itemize}
    \item[(F1)] aus $p_1\must[o]_1$ und $p_2\nweakmay[o]_2$ folgt $(p_1,p_2)\in
      F$,
    \item[(F2)] aus $p_1\nweakmay[o]_1$ und $p_2\must[o]_2$ folgt $(p_1,p_2)\in
      F$,
    \item[(F3)] aus $(p_1,p_2) \must[\alpha] (p'_1,p'_2)$ und $(p'_1,p'_2) \in
      F$ folgt $(p_1,p_2)\in F$.
      \\ \TODO{überlegen ob Regel sinnvoll ist (eher nicht, verändert die
      Sprache)}
  \end{itemize}
  Die Konjunktion $P_1\land P_2$ erhält man dort das entfernen der Zustände
  $(p_1,p_2)\in F$ aus $P_1\& P_2$. Dies entfernt ebenfalls alle may- und
  must"=Transitionen, die zu dem Zustand führen oder diesen verlassen. Es wird
  $p_1\land p_2$ für Zustände $(p_1,p_2)$ in $P_1\land P_2$ geschrieben, alle
  diese Zustände sind definiert und konsistent durch die Konstruktion. Falls
  $(p_{01},p_{02})\in F$, dann existiert die Konjunktion von $P_1$ und $P_2$
  nicht.
\end{Def}
