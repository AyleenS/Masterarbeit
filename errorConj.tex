\section{Konjunktion}

Neben der Parallelkomposition ist die Konjunktion einer der wichtigsten
Operatoren, die man auf Transitionssysteme anwenden kann, die wie hier als
Spezifikationen und Verfeinerungen aufgefasst werden. Dieser Operator erlaubt
es unterschiedliche Verhaltensweisen eines Systems separat zu definieren und
durch Konjunktion daraus dann eine vollständige Spezifikation des Gesamtsystems
zu erhalten. Die Konjunktion sollte also die gröbste Spezifikation sein,
die alle gegebenen Spezifikationen, der einzel Komponenten, verfeinert, d.h.\
sie sollte die größte untere Schranke der Verfeinerungs Präkongruenz
charakterisieren.\\
Um die Konjunktion zwischen zwei \MEIO{}s bilden zu können muss man diese
zuerst normalisieren, da sonst nicht gewolltes Verhalten auftreten kann. Die
Konjunktion soll auf dem Kreuzprodukt der beiden Transitionssysteme beruhen,
also für \MEIO{}s $Q$ und $P$ eine \MEIO{} der Art $Q\times P$ mit
möglicherweise kleinen Abwandlungen sein. Eine Verfeinerung, die beide
Spezifikationen erfüllt, kann dabei nur Fehler enthalten, die beide
Spezifikationen zulassen. Es sollte also für die Konjunktion $\ET _Q\cap \ET
_P$ und insbesondere $\cont (\PrET _Q)\cap \cont (\PrET _P)$ gelten. Im
Beispiel in Abbildung~\ref{BspKreuzpConj} soll verdeutlicht werden, wieso es
bei \MEIO{}s ohne Normalisierung zu Problemen mit dieser Forderung kommen kann.
Für $P$ un $Q$ ist $i$ ein gekürzter Fehler"=Trace. Da jedoch die
Output"=Transitionen $o$ bzw. $o'$ nicht in das Kreuzprodukt übernommen werden,
enthält $Q\times P$ kleinen Fehler"=Zustand und somit ist auch die Menge $\ET
_{Q\times P}$ leer. Durch eine Normalisierung, in der die lokalen Aktionen
bereits von einem strikten Fehler"=Trace entfernt würden und der letzten Input
bereits zu dem Fehler"=Zustand führen würde, könnte dieses Problem verhindert
werden. Es wären dann also bereits die Zustände die durch die Traces in \PrET{}
erreicht werden Fehler"=Zustände.

\begin{figure}[htbp]
  \begin{center}
    \begin{tikzpicture}[shorten >=1pt,auto,node distance=2.5cm]
      \node [initial,initial text=$Q$:] (q0) at (0,0) {$q_0$};
      \node (q1) [right of=q0] {$q_1$};
      \node (q2) [right of=q1] {$q_2\in E_Q$};

      \path[->]
      (q0) edge node{$i?$} (q1)
      (q1) edge[loop above] node{$i?$} (q1)
      (q1) edge node{$o!$} (q2)
      (q2) edge[loop above] node{$i?$} (q2)
      ;

      \node [initial,initial text=$P$:] (p0) at (9,0) {$p_0$};
      \node (p1) [right of=p0] {$p_1$};
      \node (p2) [right of=p1] {$p_2\in E_P$};

      \path[->]
      (p0) edge node{$i?$} (p1)
      (p1) edge[loop above] node{$i?$} (p1)
      (p1) edge node{$o'!$} (p2)
      (p2) edge[loop above] node{$i?$} (p2)
      ;

      \node [initial,initial text=$Q\times P$:] (qp0) at (5.5,-2) {$(q_0,p_0)$};
      \node (qp1) [right of=qp0] {$(q_1,p_1)$};

      \path[->]
      (qp0) edge node{$i?$} (qp1)
      (qp1) edge[loop above] node{$i?$} (qp1)
      ;
    \end{tikzpicture}
    \caption{Beispiel um die Notwendigkeit der Normalisierung zu verdeutlichen}
    \label{BspKreuzpConj}
  \end{center}
\end{figure}

\begin{Def}[Normal Form]
  \label{normalFormDef}
  Ein \MEIO{} $P$ ist in \emph{Normal Form} (\NF{}), falls die Menge $E_P$ der
  Fehler"=Zustände nur ein einziges Element enthält $E_P=\{e\}$ und es sollen
  zusätzlich die folgenden Bedingungen gelten:
  \begin{itemize}
    \item $e \must[a] e$ für alle $a\in\Sigma$,
    \item falls $p \may[\alpha] e$ mit $p\neq e$ gilt, soll auch $\alpha$ ein
      Element der Menge der Input"=Aktionen $I$ sein.
  \end{itemize}
  Die \emph{Normal Form} von $P$ ist ein \MEIO{} $\NF (P)$, das man aus $P$
  durch die nachfolgenden zwei Schritte erhält:
  \begin{enumerate}[(i)]
    \item Definiere $\overline{E_P} = \{p \mid \exists p'\in E_P, w\in O^*:
      p\weakmay[w]_P p'\}$ und ersetzte $E_P$ durch $\overline{E_P}$.
    \item Falls $p_0\in \overline{E_P}$, ist $NF(P) = (\{p_0\}, I, O, \{p_0
      \must[a] p_0 \mid a\in\Sigma\}, p_0, \{p_0\})$.\\
      Ansonsten, füge einen neuen Zustand $e$ hinzu, der der einzige
      Fehler"=Zustand wird. Wann immer $p \may[\alpha]_P p'\in \overline{E_P}$
      und $p\notin \overline{E_P}$ für ein $\alpha\in \Sigma \cup \{\tau\}$
      gilt (dann gilt zwingendermaßen $\alpha \in I$), entferne alle
      $\alpha$-Transitionen, die von $P$ ausgehen und füge die Transition
      $p\may[\alpha]_{\NF (P)} e$ und falls $p \must[\alpha]_P p'\in
      \overline{E_P}$ galt auch $p \must[\alpha]_{\NF (P)} e$ hinzu. Am Ende
      füge alle $e\must[a]_{\NF (P)} e$ mit $a\in\Sigma$ hinzu.
  \end{enumerate}
\end{Def}

Schritt (i) verändert nichts an den Transitionen und auch nichts daran, ob ein
Zustand einen nicht sichergestellten Input hat oder nicht. Die Menge \StET{}
wird verändert jedoch bleibt die Menge \PrET{} gleich, da \StET{} am Ende alle
Traces enthält, die direkt oder durch lokale Aktionen zu einem Fehler"=Zustand
des ursprünglichen \MEIO{}s $P$ führen. Für die Menge $\MIT\backslash\cont
(\PrET)$ ändert sich nichts, da sich an Zuständen ohne einen Zusammenhang mit
Fehler"=Zuständen nichts ändert. Das Ergebnis von (i) entspricht also dem
\MEIO{} $P$ bezüglich der Relation \ERel{}. Im ersten Fall von Schritt (ii)
gilt $\ET _P =\Sigma ^* = \ET (\NF (P))$. Im zweiten Fall ändert der Schritt
(ii) nichts an Zuständen, die nicht sichergestellte Inputs haben, solange diese
nicht in der Menge $\overline{E_P}$ enthalten sind, da eine must"=Transition
die zum Zustand $e$ führt nur hinzu gefügt wird, wenn der entsprechende Input
davor bereits eine ausgehende must"=Transition des Zustands $p$ war. Die Menge
\MIT{} verkleinert sich jedoch, falls der Schnitt mit $\cont (\PrET)$ nicht
leer war. Die Menge der Fehler"=Traces \ET{} bleibt somit im Schritt (ii)
gleich. Die Sprache des \MEIO{}s $P$ wird durch die Konstruktion in (ii)
zunächst um die Traces in $\cont (\PrET)$ verkleinert. Jedoch durch die
must"=Schlinge für alle sichtbaren Aktionen am Fehler"=Zustand, enthält die
Sprache von $\NF (P)$ em Ende sogar alle Traces aus $\cont (\PrET)$ zusätzlich
zu den Wörtern, die bereits in $L (P)$ enthalten waren. Die Menge \EL{} wird
mit \ET{} geflutet und es gilt deshalb für \EL{} auch $\EL (P) = \EL (\NF
(P))$. Somit ist $\NF (P)$ äquivalent zu $P$ bezüglich der Relation \ERel{}. Es
gilt also die folgende Proposition.

\begin{Prop}[Normal Form]
  \label{normalFormProp}
  Jedes \MEIO{} $P$ ist äquivalent zu seiner Normal Form $\NF (P)$ bezüglich
  der Relation \ERel{}.
\end{Prop}

Somit können wird davon ausgehen, dass ein \MEIO{} in Normal Form ist, falls
dies notwendig oder hilfreich ist.\\
Die Transformation in die Normal Form aus der Definition~\ref{normalFormDef}
ist effizient für endliche \MEIO{}s: Für Schritt (i)  kann eine Tiefensuche
(DFS) in linearer Zeit von den Zuständen in $E$ über umgekehrte mit lokalen
Aktionen beschriftete Transitionen ausgeführt werden. Der erste Fall von
Schritt (ii) benötigt nur konstante Zeit, da ausschließlich der Startzustand
überprüft werden muss. Für den zweiten Fall kann erneut eine DFS ausgeführt
werden. Dies DFS soll vom Startzustand $p_0$ ablaufen ohne dabei die
Transitionen zu benutzen, die gelöscht werden sollen. Alle Zustände, die
dadurch nicht besucht werden sind nicht erreichbar in $\NF (P)$ und können
entfernt werden. $\NF (P)$ ist also möglicherweise deutlich kleiner wie $P$.\\
Es wurden durch die Normalisierung der \MEIO{}s das Problem mit den Traces in
$\cont (\PrET)$ gelöst, und man erhält dann Transitionssysteme, die die
Eigenschaften der \MIA{}s aus z.B.~\cite{Vogler2016MIA3} erfüllen. Jedoch kann
die Konjunktion hier trotzdem nicht auf die gleiche Art gebildet werden, wie dort.
Es gibt noch ein weiteres ungelöstes Problem. Es beruht auf den
Input"=kritischen Traces. In~\cite{Vogler2016MIA3} kann ein Zustand in der
Konjunktion, der für eines der Transitionssysteme $P$ einen must"=Input hat und
für des andere Transitionssystem $Q$ keine entsprechende may"=Transition
besitzt, als inkonsistent angesehen werden und aus der Konjunktion entfernt
werden. Dies würde jedoch hier zu einem Problem mit den Traces führen. Es soll
für die Komposition $\ET _{Q\land P} = \ET _Q \cap \ET _P$ und $\EL _{Q\land P}
= \EL _Q \cap \EL _P$ gelten. Der Trace in $Q$ mit einem nicht sichergestellten
Input ist ein Input"=kritischer Trace. Er ist also in der Menge $\ET
_Q\subseteq \EL _Q$ enthalten. Da es in $P$ ein must"=Transition gibt ist der
entsprechende Trace in der Sprache von $P$ enthalten, also auch in der Menge
$\EL _P$. Es muss also nach den geforderten Eigenschaften der Konjunktion auch
gelten, dass der Trace in der Fehler"=gefluteten Sprache der Konjunktion
enthalten ist. Durch das entfernen des Zustandes wie in~\cite{Vogler2016MIA3}
würde der Trace in den meisten Fällen kein Element der Fehler"=gefluteten
Sprache sein (siehe dazu auch das Beispiel in Abbildung~\ref{BspMITConj}). Im
nachfolgenden Beispiel würde die Komposition von $Q$ und $P$
für~\cite{Vogler2016MIA3} nicht existieren, da der Zustand $(q_0,p_0)$ als
(logisch) inkonsistent angesehen wird.

\begin{figure}[htbp]
  \begin{center}
    \begin{tikzpicture}[shorten >=1pt,auto,node distance=2.5cm]
      \node [initial,initial text=$Q$:] (q0) at (0,0) {$q_0$};
      \node (q1) [right of=q0] {$q_1$};
      \node (q2) [right of=q1] {$q_2$};

      \path[->]
      (q0) edge[loop above] node{$i?$} (q0)
      (q0) edge[dashed] node{$o!$} (q1)
      (q1) edge node{$i?$} (q2)
      (q2) edge[loop above] node{$i?$} (q2)
      ;

      \node [initial,initial text=$P$:] (p0) at (9,0) {$p_0$};
      \node (p1) [right of=p0] {$p_1$};

      \path[->]
      (p0) edge[loop above] node{$i?$} (p0)
      (p0) edge[dashed] node{$o!$} (p1)
      ;

      \node [initial,initial text=$Q\land P$:] (qp0) at (5,-2.5)
      {$\underset{F}{\underset{\text{\rotatebox[origin=c]{-90}{$\in$}}}{(q_0,p_0)}}$};
      \node (qp1) [right of=qp0]
      {$\underset{F}{\underset{\text{\rotatebox[origin=c]{-90}{$\in$}}}{(q_1,p_1)}}$};

      \path[->]
      (qp0) edge[loop above] node{$i?$} (qp0)
      (qp0) edge[dashed] node{$o!$} (qp1)
      ;
    \end{tikzpicture}
    \caption{Beispiel für \MIT{}-Problem, $\land$ bezeichnet die Konjunktion
    aus~\cite{Vogler2016MIA3}}
    \label{BspMITConj}
  \end{center}
\end{figure}

Man kann diese Problem lösen, in dem man die Zustände aus dem in Konjunktion
entstehenden Transitionssystemen nicht streicht sondern beibehält und die
Transition zu einem Zustands"=Tupel aus dem Zustand, der durch die
must"=Transition in $P$ erreicht wird und dem Fehler"=Zustand von $Q$ führt.
Falls in $Q$ der Input als may"=Transition ausführbar ist. Zählt der Trace in
$Q$ bereits als Input"=kritisch und es kommt nicht mehr darauf an, ob die
Verlängerungen ausführbar sind. Die Verlängerungen sind alle in der Menge $\ET
_Q$ enthalten. Die Ausführbarkeit von Aktionen nach der Input"=Transition in
der Komposition hängt also nur von $P$ ab. Die Transitionen der Komposition
sollten also nach dem Input direkt von $P$ geerbt werden, dieses Verhalten
erhält man, in dem man den may"=Input von $Q$ auf den Zustand $e_Q$ umbiegt.\\
Es ergibt sich das folgende Lemma.

\begin{Lem}
  \label{AblaufNFLem}
  Sei $P$ ein \MEIO{} in Normal Form (mit dem Fehler"=Zustand $e$). $w\in
  \cont (\PrET (P)) $ gilt genau dann, wenn $w$ sich aus einem Ablauf ergibt,
  der $e$ erreicht. $va = w\in \MIT \backslash \cont(\PrET (P))$ mit $v\in
  \Sigma ^*$ und $a\in I$ gilt genau dann, wenn $v$ ein Ablauf von $P$ ist, der
  zu einem Zustand führt, für den $a$ keine ausgehende must"=Transition ist.
\end{Lem}

Auch die Behandlung von Outputs in Fällen, in denen das eine \MEIO{} eine
must"=Transition und das andere keine Möglichkeit für den Output besitzt, führt
zu Problemen mit den Traces, die möglich sein sollten, da sie im Schnitt der
\EL{}-Mengen enthalten sind. Es wird in dieser Arbeit also verzichtet (logisch)
inkonsistente Zustände wie z.B.\ in~\cite{Vogler2016MIA3} zu definiere und
diese aus den Transitionssystemen zu entfernen. Man kann sich von der
Korrektheit dieses Vorgehens recht leicht dadurch überzeugen, dass für alle
\MEIO{}s $\varepsilon$ in der Menge \EL{} enthalten ist, also muss auch eine
Konjunktion zweier \MEIO{}s das leere Wort immer ausführen können. Es gibt also
immer eine Konjunktion. Mit inkonsistenten Zuständen kann es jedoch ,wie im
Beispiel in Abbildung~\ref{BspMITConj} gesehen, dazu kommen, dass der
Startzustand inkonsistent ist und somit die Konjunktion nicht existieren würde.
Die Definition der Konjunktion orientiert sich somit eher an der des
konjunktiven Produkts aus~\cite{Vogler2016MIA3}, wie an der Definition der
Konjunktion aus~\cite{Vogler2016MIA3}. Jedoch mussten für die
Input"=Transitionen einige Veränderungen vorgenommen werden.

\begin{Def}[Konjunktion]
  \label{ConjDef}
  Für zwei \MEIO{}s $P_1 = (P_1,I,O,\must _1,\may _1,p_{01},\{e_1\})$ und $P_2
  = (P_2,I,O,\must _2,\may _2,p_{02},\{e_2\})$ in Normal Form mit der selben
  Signatur ist die \emph{Konjunktion} definiert als $P_1\land P_2 =
  (P_1\times P_2, I, O, \must, \may ,(p_{01},p_{02}),\{(e_1,e_2)\})$ mit den
  folgenden Regeln für die Transitionen:
  \begin{itemize}
    \item[(OMust)] $(p_1,p_2) \must[\omega] (p'_1,p'_2)$, falls
      $p_1 \must[\omega]_1 p'_1$ und $p_2 \weakmay[\hat{\omega}]_2 p'_2$ oder
      $p_1 \weakmay[\hat{\omega}]_1 p'_1$ und $p_2 \must[\omega]_2 p'_2$,
    \item[(IMust1)] $(p_1,p_2) \must[i] (p'_1,p'_2)$, falls $p_1 \must[i]_1
      p'_1$ und $p_2 \must[i]_2 \weakmay[\varepsilon]_2 p'_2$ oder $p_1
      \must[i]_1 \weakmay[\varepsilon]_1 p'_1$ und $p_2 \must[i]_2 p'_2$,
    \item[(IMust2)] $(p_1,p_2) \must[i] (p'_1,e_2)$, falls $p_1 \must[i]_1
      p'_1$ und $p_2 \nmust[i]_2$,
    \item[(IMust3)] $(p_1,p_2) \must[i] (e_1,p'_2)$, falls $p_1 \nmust[i]_1$
      und $p_2 \must[i]_2 p'_2$,
    \item[(EMust1)] $(p_1,e_2) \must[\alpha] (p'_1,e_2)$, falls $p_1
      \must[\alpha]_1 p'_1$,
    \item[(EMust2)] $(e_1,p_2) \must[\alpha] (e_1,p'_2)$, falls $p_2
      \must[\alpha]_2 p'_2$,
    \item[(May1)] $(p_1,p_2) \may[\tau] (p'_1,p_2)$, falls $p_1
      \weakmay[\tau]_1 p'_1$,
    \item[(May2)] $(p_1,p_2) \may[\tau] (p_1,p'_2)$, falls $p_2
      \weakmay[\tau]_2 p'_2$,
    \item[(OMay)] $(p_1,p_2) \may[\omega] (p'_1,p'_2)$, falls $p_1
      \weakmay[\omega]_1 p'_1$ und $p_2 \weakmay[\omega]_2 p'_2$,
    \item[(IMay1)] $(p_1,p_2) \may[i] (p'_1,p'_2)$, falls $p_1 \must[i]_1
      \weakmay[\varepsilon]_1 p'_1$ und $p_2 \must[i]_2 \weakmay[\varepsilon]_2
      p'_2$,
    \item[(IMay2)] $(p_1,p_2) \may[i] (p'_1,e_2)$, falls $p_1 \may[i]_1 p'_1$
      und $p_2 \nmust[i]_2$,
    \item[(IMay3)] $(p_1,p_2) \may[i] (e_1,p'_2)$, falls $p_1 \nmust[i]_1$ und
      $p_2 \may[i]_2 p'_2$,
    \item[(EMay1)] $(p_1,e_2) \may[\alpha] (p'_1,e_2)$, falls $p_1
      \may[\alpha]_1 p'_1$,
    \item[(EMay2)] $(e_1,p_2) \may[\alpha] (e_1,p'_2)$, falls $p_2
      \may[\alpha]_2 p'_2$.
  \end{itemize}
\end{Def}

Aus schwachen Transitionen der einzelnen \MEIO{}s entstehen im der Konjunktion
einzelnen Transitionen (z.B.\ (OMust), (IMust1), (May)). $\tau$"=Transitionen
können via der Regel (OMay) synchronisiert werden. In den Regel (EMust) und
(EMay) wirken die Fehler"=Zustände als neutrales Element. In der
Parallelkomposition hingeben hatten die Fehler"=Zustände absorbierende
Wirkung.\\
$P_1$ und $P_2$ müssen für die Definition~\ref{ConjDef} in Normal Form sein.
Der daraus resultierende \MEIO{} $P_1\land P_2$ ist ebenfalls wieder in Normal
Form. Es gibt nur den Zustand $(e_1,e_2)$ als Fehler"=Zustand. Da für $e_1$ und
$e_2$ bereits alle Aktionen via must"=Schleife ausführbar sind, hat auch der
Zustand $(e_1,e_2)$ die Möglichkeit alle Aktionen via must"=Transition zu sich
selbst auszuführen. Die Transitionen, die in $P_1$ und $P_2$ zu den
Fehler"=Zustand führen sind bereits alle Inputs. Es kann also in der
Konjunktion durch das gemeinsame Ausführen der Transitionen nur Inputs zum
Zustand $(e_1,e_2)$ führen. Es kann jedoch auch durch Input"=kritische Traces
der Komponenten möglich sein, den Zustand $(e_1,e_2)$ zu erreicht. Hierfür
kommen aber nur die Regel (IMust2), (IMust3), (IMay2) und (IMay3) in Frage.
Diese befassen sich jedoch nur mit Input"=Transitionen. Es kann also durch
diese Regeln auch keine zu $(e_1,e_2)$ führende Transition entstehen, die mit
einer lokalen Aktion beschriftet wäre.\\
Auf Grundlage dieser Definitionen kann nun gezeigt werden, dass es sich bei der
Konjunktion $\land$ um die gewünschte größte untere Schranke bezüglich der
Relation \ERel{} handelt.

\begin{Satz}[Konjunktion]
  Für drei \MEIO{}s $S$, $P_1$ und $P_2$ mit der gleichen Signatur gilt
  $S\ERel P_1\land P_2$ genau dann, wenn $S\ERel P_1$ und $S\ERel P_2$ erfüllt
  ist.
\end{Satz}

\begin{proof}
  Die Definition von $\land$ setzt normalisierte \MEIO{}s voraus. Deshalb ist
  es sinnvoll auch hier von \MEIO{}s $P_1$ und $P_2$ in Normal Form auszugehen.
  Dies ist möglich durch die Aussage der Proposition~\ref{normalFormProp}.

  \glqq $\Leftarrow$\grqq{}:\\
  Um diese Richtung zu beweisen muss gezeigt werden, dass $\ET _S \subseteq \ET
  _{P_1\land P_2}$ und $\EL _S \subseteq \EL _{P_1\land P_2}$ gilt, wenn $\ET
  _S \subseteq \ET _1$, $\EL _S \subseteq \EL _1$, $\ET _S \subseteq \ET _2$
  und $\EL _S \subseteq \EL _2$ vorausgesetzt wird.

  Als erstes wird versucht die Teilmengenbeziehung $\ET _S \subseteq \ET
  _{P_1\land P_2}$ zu beweisen. Dafür wird ein beliebiges $w$ aus der Menge
  $\ET _S$ gewählt. Das $w$ muss ebenfalls in $\ET _1\cap \ET _2$ enthalten. In
  Lemma~\ref{AblaufNFLem} wurde ein Unterschied zwischen Elementen der Menge
  $\cont (\PrET (P))$ und $\cont (\MIT (P))$ deutlich. Dies ist auch der Grund
  für die nachfolgende Fallunterscheidung.
  \begin{itemize}
    \item Fall 1 ($w\in(\cont (\PrET _1)\cap \cont(\PrET _2))$): Es führt also
      nach Lemma~\ref{AblaufNFLem} sowohl in $P_1$ wie auch in $P_2$ der Ablauf
      von $w$ zum Fehler"=Zustand $e_1$ bzw.\ $e_2$. Da in beiden
      Transitionssystemen der Ablauf von $w$ möglich ist, enthält auch
      $P_1\land P_2$ die dafür notwendigen Transitionen. Das $w$ führt in der
      Konjunktion zum Zustand $(e_1,e_2)$ und ist wegen~\ref{AblaufNFLem} in
      der Menge $\ET _{P_1\land P_2}$ enthalten.
    \item Fall 2 ($w\in(\cont (\MIT _1)\cap \cont (\MIT _2))$): Es existieren
      Präfixe $w_1$ und $w_2$ von $w$ für die $w_1\in\MIT _1$ und $w_2\in\MIT
      _2$ gilt.
      \begin{itemize}
        \item Fall 2a) ($w_1=w_2$): In beiden $P_j$ ist das $v$, für das es
          einen Input $a$ gibt, so dass $w_1 = w_2 = va$ gilt, Element der
          Sprache $L_j$. Das $v$ muss wegen~\ref{AblaufNFLem} in den \MEIO{}s
          $P_j$ zu jeweils einem Zustand $p_j$ führen, in dem das $a$ nicht
          sichergestellt ist. In der Komposition muss es einen Ablauf von $v$
          geben, der zu dem Zustand $(p_1,p_2)$ führt. Da für keinen der beiden
          Zustände $p_j$ der Input $a$ eine ausgehende must"=Transition ist,
          kann es mit Definition~\ref{ConjDef} auch keine mit $a$ beschriftete
          ausgehende must"=Transition für $(p_1,p_2)$ geben. Es gilt also mit
          Lemma~\ref{AblaufNFLem} $w_1=w_2\in\MIT _{P_1\land P_2}$ und somit
          $w\in\ET _{P_1\land P_2}$.
        \item Fall 2b) ($w_1\neq w_2$): \OBdA{} ist $w_1$ kürzer wie $w_2$. Es
          gibt Wörter $v_1$, $v_2$ und Inputs $a_1$, $a_2$, so dass $w_1 =
          v_1a_1$ und $w_2 = v_2a_2$ gilt. $v_1$ ist ein echtes Präfix von
          $v_2$. Es ist also $v_1$ in beiden \MEIO{}s $P_j$ ausführbar. In
          $P_1$ wird dadurch ein Zustand $p_1$ erreicht für den
          $p_1\anmust[a_1]$ gilt. Für den Zustand $p_2$, den $P_2$ durch $v_1$
          erreicht gilt jedoch $p_2\must[a_1]$. In der Konjunktion kommt also
          die Regel (IMust3) der Definition~\ref{ConjDef} zur Anwendung. Die
          restlichen Transitionen von $v_2$ nach der Ausführung von $w_1$ erbt
          die Konjunktion also direkt von $P_2$. In $P_1\land P_2$ führt der
          Ablauf von $v_2$ zu einem Zustand der Form $(e_1,p'_2)$. Für $p'_2$
          ist $a_2$ keine ausgehende must"=Transition. Da für $e_1$ alle
          Aktionen eine must"=Transition zu $e_1$ besitzen kann erneut die
          Regel (IMust3) der Definition~\ref{ConjDef} angewendet werden. Durch
          $w_2$ wird in $P_1\land P_2$ der Zustand $(e_1,e_2)$ erreicht. Es
          gilt durch Lemma~\ref{AblaufNFLem} $w_2\in\PrET _{P_1\land P_2}$ und
          daraus folgt dann auch $w\in\ET _{P_1\land P_2}$.
      \end{itemize}
    \item Fall 3 ($w\in (\cont (\PrET _1)\cap \cont (\MIT _2))$ oder $w\in
      (\cont (\MIT _1)\cap \cont (\PrET _2))$): Die beiden hier Vereinigten
      Fälle sind symmetrisch, somit kann wegen der Kommunativität der
      Konjunktioin \oBdA{} davon ausgegangen werden, dass $w\in (\cont (\PrET
      _1)\cap \cont (\MIT _2))$ gilt. In $P_1$ führt der Ablauf $w$ zum
      Fehler"=Zustand $e_1$. Es gibt ein Präfix $w'$ von $w$, dass in $\MIT _2$
      enthalten ist. In $P_2$ ist das Präfix $v$ von $w'$ ausführbar, für das
      es einen Input $a$ gibt, so dass $w'=va$ gilt. $va$ ist entweder in $\MIT
      _1$ enthalten und es kann Fall 2 angewendet werden, oder der Input $a$
      ist in $P_1$ nach der Ausführung von $v$ durch eine must"=Transition
      möglich. Falls in $P_1$ die $a$-must"=Transtion enthalten ist, folgt mit
      der Definition~\ref{ConjDef}, dass in der Konjunktion durch $w'$ ein
      Zustand erreicht werden muss, dessen zweites Element des Tupels $e_2$
      ist. Somit wird in $P_1\land P_2$ durch $w$ der Zustand $(e_1,e_2)$
      erreicht. Mit Lemma~\ref{AblaufNFLem} folgt daraus $w\in\ET _{P_1\land
      P_2}$.
  \end{itemize}

  Ein beliebiges $w$ der fehler"=gefluteten Sprache des \MEIO{}s $S$ muss
  in $P_1$ und $P_2$ ebenfalls ein Wort aus der jeweiligen Menge $\EL _j$ sein.
  Da die fehler"=geflutete Sprache die Vereinigung der Sprache $L$ und der
  Fehler"=Traces \ET{} ist, können unterschiedliche Fälle für das Wort $w$
  eintreten.
  \begin{itemize}
    \item Fall I ($w\in\ET _1\cap \ET _2$): In diesem Fall gilt aufgrund des
      Beweises der vorangegangen Inklusion $w\in\ET _{P_1\land P_2}\subseteq
      \EL _{P_1\land P_2}$.
    \item Fall II ($w\in\ET _1\cap L_2\backslash \ET _2$ oder $w\in
      L_1\backslash \ET _1\cap \ET _2$): Dieses beiden Fälle sind symmetrisch
      und da die Konjunktion kommunaitv ist kann \oBdA{} davon ausgegangen
      werden, dass $w\in\ET _1\cap L_2\backslash \ET _2$ gilt. Es gibt also in
      $P_2$ einen Ablauf des Wortes $w$.
      \begin{itemize}
        \item Fall IIa) ($w\in\cont (\PrET _1)$): Es gibt wegen
          Lemma~\ref{AblaufNFLem} auch in $P_1$ einen Ablauf, der mit $w$
          beschriftet ist. In diesem Fall gilt also $w\in L_{P_1\land P_2}
          \subseteq \EL _{P_1\land P_2}$.
        \item Fall IIb) ($w\in\cont (\MIT _1)\backslash\cont (\PrET _1)$): $w$
          ist in $P_1$ nicht ausführbar. Es gibt ein echtes Präfix $v$ von $w$,
          dass in $P_1$ zu einem Zustand führt, der einen Input $a$ nicht
          sicherstellt, so dass $va$ ebenfalls ein Präfix von $w$ ist. Da
          $w\notin \ET _2$ gilt, kann $va$ in $P_2$ kein Input"=kritischer
          Trace sein. Es muss also nach dem Ablauf $v$ in $P_2$ ein Zustand
          erreicht werden, der für den Input $a$ einen ausgehende
          must"=Transition besitzt. Es folgt mit Definition~\ref{ConjDef}, dass
          in der Konjunktion durch $va$ ein Zustand erreicht wird, der als
          erstes Element des Tupels den Zustand $e_1$ enthält. Die noch
          fehlenden Transitionen des Wortes $w$ nach der Ausführung von $va$
          werden also direkt von $P_2$ geerbt. Somit gibt es in $P_1\land P_2$
          einen mit $w$ beschrifteten Ablauf und es gilt $w\in L_{P_1\land P_2}
          \subseteq \EL _{P_1\land P_2}$.
      \end{itemize}
    \item Fall III ($w\in L_1\backslash\ET _1 \cap L_2\backslash\ET _2$): Das
      Wort $w$ ist in beiden \MEIO{}s $P_j$ ausführbar. Die dafür verwendeten
      Transitionen werden durch die Definition~\ref{ConjDef} in die Konjunktion
      übernommen. Es gibt also auch in $P_1\land P_2$ einen Ablauf, der mit $w$
      beschriftet ist. Somit gilt $w\in L_{P_1\land P_2}\subseteq \EL
      _{P_1\land P_2}$.
  \end{itemize}

  \glqq $\Rightarrow$\grqq{}:\\
  Für diese Richtung des Beweises können die Inklusionen $\ET _S\subseteq \ET
  _{P_1\land P_2}$ und $\EL _S\subseteq \EL _{P_1\land P_2}$ vorausgesetzt
  werden um $\ET _S\subseteq \ET _1$, $\EL _S\subseteq \EL _1$, $\ET
  _S\subseteq \ET _2$ und $\EL _S\subseteq \EL _2$ zu zeigen. Da die
  Konjunktion aufgrund ihrer Definition in~\ref{ConjDef} kommutativ ist, reicht
  es die Inklusionen für einen der beiden \MEIO{}s zu zeigen.

  Als erstes soll die Inklusion $\ET _S\subseteq \ET _1$ betrachtet werden.
  Durch die Voraussetzungen gilt für ein beliebiges $w$ aus $\ET _S$ auch die
  Zugehörigkeit zur Menge $\ET _{P_1\land P_2}$. Es wird also durch $w$
  entweder der Fehler"=Zustand $(e_1,e_2)$ in der Konjunktion von $P_1$ und
  $P_2$ erreicht oder durch ein Präfix $v$ von $w$ ein Zustand, in dem der
  Input $a$, der in $w$ auf das $v$ folgen würde, nicht sicher gestellt ist.
  \begin{itemize}
    \item Fall 1 ($w\in\cont (\PrET _{P_1\land P_2})$): Der Ablauf von $w$
      führt zum Zustand $(e_1,e_2)$. Es muss also durch die Konjunktion von
      $P_1$ und $P_2$ der Zustand $e_1$ durch das Wort $w$ in der ersten
      Komponente der Zustands"=Tupel erreicht worden sein. Dies ist möglich
      durch eine nicht sichergestellte Input"=Transition innerhalb des $w$s in
      $P_1$ oder dadurch, dass $w$ ein Ablauf in $P_1$ ist, der zum
      Fehler"=Zustand $e_1$ führt. Das Wort $w$ ist in beiden Fällen in der
      Menge $\ET _1$ enthalten.
    \item Fall 2 ($w\in\cont (\MIT _{P_1\land P_2})$): Das Präfix $v$ von $w$
      ist in $P_1\land P_2$ ausführbar zu einem Zustand $(p_1,p_2)$, für den
      $(p_1,p_2)\nmust[a]_{P_1\land P_2}$ gilt. Es darf also für die Zustände
      $p_1$ und $p_2$ ebenfalls keine ausgehenden $a$-must"=Transition geben,
      da sonst~\ref{ConjDef} auch eine must"=Transition in der Konjunktion der
      beiden \MEIO{}s fordern würde. $va$ ist also für $P_1$ ein
      Input"=kritischer Trace mit $va\in\MIT _1\subseteq \ET _1$. Da die Menge
      \ET{} unter Fortsetzung abgeschlossen ist, gilt auch $w\in\ET _1$.
  \end{itemize}

  Um die Inklusion $\EL _S\subseteq\EL _1$ zu beweisen, wird ein beliebiges $w$
  aus der Menge $\EL _S$ gewählt. Nach Voraussetzung ist das $w$ auch in der
  fehler"=gefluteten Sprache der Konjunktion $P_1\land P_2$ enthalten. Falls
  es in $P_1\land P_2$ keinen Ablauf für $w$ gibt, gilt $w\in\cont (\MIT
  _{P_1\land P_2})$, dieser Fall führt zu $w\in\ET _1\subseteq\EL _1$, wie
  bereits in Fall 2 der letzten Inklusion gezeigt wurde. Es kann also davon
  ausgegangen werden, dass $w$ in der Konjunktion ausführbar ist. Der Fall $w$
  führt in $P_1\land P_2$ zum Fehler"=Zustand wurde auch bereits von der
  letzten Inklustion in Fall 1 behandelt und führt zu $w\in\ET _1\subseteq\EL
  _1$. Man kann sich für diesen Beweisteil deshalb sogar auf Abläufe von $w$
  einschränken, die zu einem Zustand in $P_1\land P_2$ führen, der nicht der
  Fehler"=Zustand ist. Jedoch ist es trotzdem möglich, dass durch $w$ ein
  Zustand erreicht wird, in dessen Tupel einer der Zustände der Fehler"=Zustand
  der jeweiligen Komponente ist.
  \begin{itemize}
    \item Fall I ($(p_{01},p_{02})\weakmay[w]_{P_1\land P_2} (p_1,p_2)$ mit
      $p_1\neq e_1$): Die Konjunktion kann für das Wort $w$ nur Transitionen
      verwendet haben, die in $P_1$ mindestens via schwacher may"=Transitionen
      möglich waren. Es muss also auch in $P_1$ einen Ablauf des Wortes $w$
      geben, der ebenfalls zum Zustand $p_1$ führt. Es gilt also $w\in L_1
      \subseteq \EL _1$.
    \item Fall II ($(p_{01},p_{02})\weakmay[w]_{P_1\land P_2} (e_1,p_2)$): Da
      das Zustands"=Tupel $(e_1,p_2)$ den Fehler"=Zustand von $P_1$ in der
      ersten Komponente enthält, gibt es zwei Möglichkeiten für $P_1$ wie
      dieser Zustand in der Konjunktion erreicht wurde.
      \begin{itemize}
        \item Fall IIa) ($w$ ausführbar in $P_1$): $P_1$ kann durch $w$ den
          Zustand $e_1$ erreichen. Es gilt also $w\in L_1\cap \ET _1\subseteq
          \EL _1$.
        \item Fall IIb) ($w$ nicht vollständig ausführbar in $P_1$): Es gibt
          ein Präfix $w'$ von $w$, das man in das Wort $v$ und den Input $a$
          aufteilen kann, so dass $w'=va$ gilt und durch $v$ in $P_1$ ein
          Zustand erreicht wird, für den der Input $a$ nicht sichergestellt
          ist. In der Konjunktion kam die Transitions"=Regel (IMust3) oder
          (IMay3) der Definition~\ref{ConjDef} zum Einsatz um einen Zustand zu
          erreichen, der $e_1$ als erste Komponente enthält. Es gilt in diesem
          Fall also $va\in\MIT _1\subseteq\ET _1$. Wegen der Abgeschlossenheit
          der Menge \ET{} unter Fortsetzungen gilt $w\in \ET _1
      \subseteq \EL _1$.
      \end{itemize}
  \end{itemize}
\end{proof}
