\section{Zusammenhänge}

\begin{Satz}[Zusammenhang der Verfeinerungs-Relationen mit der Fehler-Relation]
  \label{ZusammenhFehlerSatz}
  Für \MEIO{}s $P$ und $Q$ gilt $P\wasRel Q\Rightarrow P\ERel Q$. Die
  Implikation in die andere Richtung gilt jedoch nicht.
\end{Satz}
\begin{proof}\mbox{}\\
  $P\wasRel Q\Rightarrow P\ERel Q$:\\
  Um diese Implikation zu beweisen wird gezeigt, dass eine beliebige
  schwache as"=Verfeinerungs"=Relation $\mathcal{R}$ zwischen $P$ und $Q$ auch
  die Eigenschaften der Relation $\ERel$ erfüllt. Da $\mathcal{R}$ eine
  schwache as"=Verfeinerungs"=Relation zwischen $P$ und $Q$ ist, muss $p_0
  \mathcal{R} q_0$ gelten. Es sind die folgenden Punkte nachzuweisen:
  \begin{itemize}
    \item $\ET _P \subseteq \ET _Q$,
    \item $\EL _P \subseteq \EL _Q$.
  \end{itemize}
  Für den ersten Punkt wird ein beliebiges $w$ aus $\ET _P$ betrachtet und
  gezeigt, dass dieses auch in $\ET _Q$ enthalten ist. Es kann davon
  ausgegangen werden, dass $w$ präfix-minimal ist, da beide \ET{}-Mengen unter
  \cont{} abgeschlossen sind. $w$ kann ein Element aus $\PrET (P)$ sein oder
  ein Element aus $\MIT (P)$.
  \begin{itemize}
    \item Fall 1 ($w\in\PrET (P)$): Es existiert ein $v\in O_P$, sodass das
      Wort $wv$ in $P$ einen Fehler-Zustand erreicht. Es gibt eine
      Transitionsfolge wie in Lemma~\ref{AblaefeSchwVerfSpezLem}, so dass $wv =
      (\alpha _1\alpha _2\dots \alpha _n)|_{\Sigma}$ gilt und der dadurch
      erreichte Zustand $p_n$ ein Zustand aus der Menge $E_P$ ist. Falls eines
      des $q_j$ für $0 \leq j < n$ in $E_Q$ enthalten ist, folgt mit dem Beweis
      von Lemma~\ref{AblaefeSchwVerfSpezLem}, dass ein Präfix von $wv$ in
      $\StET (Q)$ enthalten ist. Mit $w=\prune (wv)$ und dem Abschluss von
      \ET{} unter \cont{} gilt dann $w\in\ET _Q$. Ansonsten gibt es in $Q$
      einen Trace für das Wort $wv$, dass einen Zustand $q_n$ erreicht, für den
      $p_n \mathcal{R} q_n$ gilt. Mit~\ref{wSimDef}.1 folgt, dass $q_n\in E_Q$
      gelten muss und somit auch $w\in\ET _Q$ mit der Begründung von oben.
    \item Fall 2 ($w\in\MIT (P)$): $w$ ist in $P$ ein Input-kritischer Trace.
      Es existiert also eine Aufteilung von $w$ in $va$ mit $v\in \Sigma ^*$
      und $a\in I$, wobei $v$ in $P$ zu einem Zustand ausführbar ist, der den
      Input $a$ nicht sicherstellt. Es gibt also ein Ablauf des Wortes $v$ wie
      in Lemma~\ref{AblaefeSchwVerfSpezLem}, der zu einem Zustand $p_n$ führt,
      der keine ausgehende must"=Transition für $a$ besitzt. Falls ein $q_j$
      für $0 \leq j \leq n$ in $E_Q$ enthalten ist, folgt mit dem Beweis des
      Lemmas~\ref{AblaefeSchwVerfSpezLem} und dem Abschluss der Menge \ET{}
      unter \cont{} $w\in\ET _Q$. Ansonsten wird durch $v$ in $Q$ ein Zustand
      $q_n$ erreicht, für den $p_n\mathcal{R} q_n$ gilt. Da $a$ für $p_n$ in
      $P$ keine ausgehende must"=Transition sein kann, gilt mit~\ref{wSimDef}.2
      auch $q_n \nmust[a]$. $w$ ist in $\MIT _Q \subseteq \ET _Q$ enthalten.
  \end{itemize}
  Für den zweiten Punkt kann man sich auf die Inklusion $\EL _P\backslash\ET _P
  \subseteq \EL _Q$ einschränken, da der erste Punkt bereits vorausgesetzt
  werden kann. $\EL _P\backslash\ET _P$ ist eine Teilmenge der Sprache $L _P$.
  Somit ist ein $w$ aus $\EL _P\backslash\ET _P$ in $P$ ausführbar. Es gibt
  also einen Trace für $w$ in $P$ der Form, wie sie in
  Lemma~\ref{AblaefeSchwVerfSpezLem} vorausgesetzt wird. Falls ein $q_j$ für $0
  \leq j \leq n$ in $E_Q$ enthalten ist, gilt $w\in\ET _Q \subseteq \EL _Q$. Es
  wird also im Folgenden davon ausgegangen, dass kein $q_j$ ein Fehler-Zustand
  ist. Es gilt dann $q_0 \weakmay[\widehat{\alpha _1}]_Q q_1
  \weakmay[\widehat{\alpha _2}]_Q \dots q_{n-1} \weakmay[\widehat{\alpha _n}]_Q
  q_n$ in $Q$ mit $p_j \mathcal{R} q_j$ für $0 \leq j \leq n$ und somit $w\in L
  _Q\subseteq \EL _Q$.

  $P\wasRel Q\hspace{0.1cm}\not\hspace{-0.1cm}\Leftarrow P\ERel Q$:\\
  Die nicht Gültigkeit dieser Implikation beruht darauf, dass Simulationen
  strenger sind als Sprach Inklusionen. Das Gegenbeispiel hier ist also so
  aufgebaut, dass $\ET (P) =\ET (Q) = \emptyset$ und $L(P) \subseteq L(Q)$
  gilt, jedoch keine schwache as"=Verfeinerungs"=Relation zwischen $P$ und $Q$
  existieren kann. $Q$ und $P$ sind in der Abbildung~\ref{WasEGegenBsp}
  dargestellt. Damit $\ET (P) =\ET (Q) = \emptyset$ gilt, dürfen keine der
  Zustände Fehler-Zustände sein und es muss gefordert werden, dass die Menge
  $I$ der Inputs für die \MEIO{}s leer ist, ansonsten würde es Input-kritische
  Traces gegen. $P$ kann keine Aktionen ausführen und $Q$ nur die Output Aktion
  $o$ somit gilt für die Sprachen $\{\varepsilon\} = L(P) \subset L(Q) =
  \{\varepsilon , o\}$.\\
  Angenommen es gibt eine schwache as"=Verfeinerungs"=Relation $\mathcal{R}$
  zwischen $P$ und $Q$. Dafür muss $(p_0,q_0)\in \mathcal{R}$ gelten.Da die
  Transition $q_0 \must[o]_Q q_1$ in $Q$ vorhanden ist, wird die Verfeinerung
  dieser in $P$ gefordert es müsste auch eine must"=Output"=Transition in $P$
  geben. Es gibt jedoch keine must"=Transition in $P$, dies stellt einen
  Widerspruch zur Annahme dar und es folgt, dass es keine schwache
  as"=Verfeinerungs"=Relation zwischen $P$ und $Q$ geben kann.

  \begin{figure}[htbp]
    \begin{center}
      \begin{tikzpicture}[shorten >=1pt,auto,node distance=2.5cm]
        \node [initial,initial text=$Q$:] (q0) at (0,0) {$q_0$};
        \node (q1) [right of=q0] {$q_1$};

        \path[->]
        (q0) edge node{$o!$} (q1)
        ;

        \node [initial,initial text=$P$:] (p0) at (7,0) {$p_0$};
      \end{tikzpicture}
      \caption{Gegenbeispiel zu $\wasRel \Leftarrow \ERel$ mit $I_P = I_Q =
      \emptyset$}
      \label{WasEGegenBsp}
    \end{center}
  \end{figure}
\end{proof}

In dieser Arbeit werden im Gegensatz zu~\cite{Vogler2016MIA3} die Fehler bei
einer Parallelkomposition beibehalten bzw.\ aus dieser entstehend angesehen. Es
werden dabei alle Transitionen, die nicht durch fehlende
Synchronisations"=Möglichkeiten wegfallen, übernommen. In~\cite{Vogler2016MIA3}
hingegen wird der Ansatz verfolgt alle Fehler zu entfernen und durch einen
universal Zustand zu ersetzten, der nur eingehende may"=Input"=Transitionen
zulässt. Auf diese Normierung wurde hier mit Absicht verzichtete um sehen zu
können, dass die Fehler einen Ursprung haben, den man später auch noch einsehen
kann. Jedoch gibt es trotzdem einen Zusammenhang zwischen diesen beiden
Ansätzen.\\
Die Unterscheidung, die die Basisrelation \EBRel{} hier herbei führt, würde
dort der Unterscheidung zwischen Transitionssystemen, die den universal
Zustand $e$ als Startzustand haben und denen, die einen Startzustand ungleich
$e$ besitzen, entsprechen. Falls hier in einer as"=Implementierung von $P$ ein
Fehler lokal erreichbar ist, dann muss auch in $P$ ein Fehler-Zustand lokal
erreichbar sein, wegen~\ref{lokalFehlerErrKor}~(i). Dies entspricht
$\varepsilon \in \PrET (P)$. Das Abschneiden den lokalen Aktionen wird hier nur
in der Trace Menge praktiziert, in~\cite{Vogler2016MIA3} jedoch direkt auf den
Transitionssystemen. Im Fall der lokalen Fehler-Erreichbarkeit bleibt also
nur noch der Zustand $e$ als universal Zustand übrig, für den jedes Verhalten
zulässig ist. Falls man hier dieses Pruning mit beliebigem Verhalten nachmachen
wollen würde, müsste man an den Zustand $e$ noch eine may-Schleife für alle
Aktionen aus $I$ und $O$ hinzufügen.\\
Da auch der Testing-Ansatz die lokale Fehler-Erreichbarkeit verwendet,
existiert der Zusammenhang auch dort für die Parallelkomposition mit dem
entsprechenden Test.\\
Falls es keine Input-kritischen Traces gibt, entspricht \ERel{} der Relation
$\sqsubseteq$ aus~\cite{Vogler2016MIA3}, falls auf die \MEIO{}s das
entsprechende Abschneiden der Fehler-Traces angewendet würde und diese dann
durch einen universal Zustand mit may"=Schleife für alle Inputs und Outputs
ersetzt würden. Gibt es in $Q$ jedoch einen Input-kritische Traces, würde $P
\ERel Q$ zulassen, dass $P$ einen gekürzten Fehler-Trace anstatt dessen
besitzt. Diese Art der Verfeinerung lässt $\sqsubseteq$ nicht zu. Jedoch sind
die Input-kritischen Traces auch dort die potentiellen Auslöser für neue
Fehler-Zustände in einer Parallelkomposition. Diese Problem basiert auf dem
Problem, dass \ERel{} keine schwache as"=Verfeinerungs"=Relation sein muss, die
in \wasRel{} enthalten ist.
