\section{Zusammenhänge}
\label{errorZusammenh}

\begin{Satz}[Zusammenhang der Verfeinerungs-Relationen mit der Fehler-Relation]
  \label{ZusammenhFehlerSatz}
  Für \MEIO{}s $P$ und $Q$ gilt $P\wasRel Q\Rightarrow P\ERel Q$. Die
  Implikation in die andere Richtung gilt jedoch nicht.
\end{Satz}
\begin{proof}\mbox{}\\
  $P\wasRel Q\Rightarrow P\ERel Q$:\\
  Um diese Implikation zu beweisen wird gezeigt, dass eine beliebige
  schwache as"=Verfeinerungs"=Relation $\mathcal{R}$ zwischen $P$ und $Q$ auch
  die Eigenschaften der Relation $\ERel$ erfüllt. Da $\mathcal{R}$ eine
  schwache as"=Verfeinerungs"=Relation zwischen $P$ und $Q$ ist, muss $p_0
  \mathcal{R} q_0$ gelten. Es sind die folgenden Punkte nachzuweisen:
  \begin{itemize}
    \item $\ET _P \subseteq \ET _Q$,
    \item $\EL _P \subseteq \EL _Q$.
  \end{itemize}
  Für den ersten Punkt wird ein beliebiges $w$ aus $\ET _P$ betrachtet und
  gezeigt, dass dieses auch in $\ET _Q$ enthalten ist. Es kann davon
  ausgegangen werden, dass $w$ präfix-minimal ist, da beide \ET{}-Mengen unter
  \cont{} abgeschlossen sind. $w$ kann ein Element aus $\PrET (P)$ sein oder
  ein Element aus $\MIT (P)$.
  \begin{itemize}
    \item Fall 1 ($w\in\PrET (P)$): Es existiert ein $v\in O_P^*$, so dass das
      Wort $wv$ in $P$ einen Fehler-Zustand erreicht. Es gibt eine
      Transitionsfolge wie in Lemma~\ref{AblaefeSchwVerfSpezLem}, so dass $wv =
      (\alpha _1\alpha _2\dots \alpha _n)|_{\Sigma}$ gilt und der dadurch
      erreichte Zustand $p_n$ ein Zustand aus der Menge $E_P$ ist. Falls ein
      Ablauf zu einem $q_k\in E_Q$ existiert wie in
      Lemma~\ref{AblaefeSchwVerfSpezLem}, folgt, dass ein Präfix von $wv$ in $\StET
      (Q)$ enthalten ist. Mit $w=\prune (wv)$ und dem Abschluss von \ET{} unter
      \cont{} gilt dann $w\in\ET _Q$. Ansonsten gibt es in $Q$ einen Trace für
      das Wort $wv$, der einen Zustand $q_n$ erreicht, für den $p_n \mathcal{R}
      q_n$ gilt. Mit~\ref{wSimDef}.1 folgt, dass $q_n\in E_Q$ gelten muss und
      somit auch $w\in\ET _Q$ mit der Begründung von oben.
    \item Fall 2 ($w\in\MIT (P)$): $w$ ist in $P$ ein Input-kritischer Trace.
      Es existiert also eine Aufteilung von $w$ in $va$ mit $v\in \Sigma ^*$
      und $a\in I$, wobei $v$ in $P$ zu einem Zustand ausführbar ist, der den
      Input $a$ nicht sicherstellt. Es gibt also einen Ablauf des Wortes $v$ in
      $P$ wie in Lemma~\ref{AblaefeSchwVerfSpezLem}, der zu einem Zustand $p_n$
      führt, der keine ausgehende must"=Transition für $a$ besitzt. Falls es
      einen Ablauf wie in~\ref{AblaefeSchwVerfSpezLem} zu einem $q_k$ in $E_Q$
      gibt, folgt mit dem Lemma~\ref{AblaefeSchwVerfSpezLem} und dem Abschluss
      der Menge \ET{} unter \cont{} $w\in\ET _Q$. Ansonsten wird durch $v$ in
      $Q$ ein Zustand $q_n$ erreicht, für den $p_n\mathcal{R} q_n$ gilt. Da $a$
      für $p_n$ in $P$ keine ausgehende must"=Transition sein kann, gilt
      mit~\ref{wSimDef}.2 auch $q_n \nmust[a]$. Das $w$ ist in $\MIT _Q
      \subseteq \ET _Q$ enthalten.
  \end{itemize}
  Für den zweiten Punkt kann man sich auf die Inklusion $\EL _P\backslash\ET _P
  \subseteq \EL _Q$ einschränken, da der erste Punkt bereits vorausgesetzt
  werden kann. Es soll ein beliebiges Wort $w$ aus der Menge $\EL
  _P\backslash\ET _P$ betrachtet werden. $\EL _P\backslash\ET _P$ ist eine
  Teilmenge der Sprache $L _P$. Es gibt also einen Ablauf für $w$ in $P$ der
  Form, wie sie in Lemma~\ref{AblaefeSchwVerfSpezLem} vorausgesetzt wird. Falls
  ein $q_k$ aus~\ref{AblaefeSchwVerfSpezLem} für $0 \leq k \leq n$ in $E_Q$
  enthalten ist, gilt $w\in\ET _Q \subseteq \EL _Q$. Es wird also im Folgenden
  davon ausgegangen, dass kein $q_j$ ein Fehler-Zustand ist. Es gilt dann $q_0
  \weakmay[\widehat{\alpha _1}]_Q q_1 \weakmay[\widehat{\alpha _2}]_Q \dots
  q_{n-1} \weakmay[\widehat{\alpha _n}]_Q q_n$ in $Q$ mit $p_n \mathcal{R} q_n$
  und somit $w\in L _Q\subseteq \EL _Q$.

  $P\wasRel Q\hspace{0.1cm}\not\hspace{-0.1cm}\Leftarrow P\ERel Q$:\\
  Die nicht Gültigkeit dieser Implikation beruht darauf, dass Simulationen
  strenger sind als Sprach"=Inklusionen. Das Gegenbeispiel hier ist also so
  aufgebaut, dass $\ET (P) =\ET (Q) = \emptyset$ und $L(P) \subseteq L(Q)$
  gilt, jedoch keine schwache as"=Verfeinerungs"=Relation zwischen $P$ und $Q$
  existieren kann. $P$ und $Q$ sind in der Abbildung~\ref{WasEGegenBsp}
  dargestellt. Damit $\ET (P) =\ET (Q) = \emptyset$ gilt, dürfen keine der
  Zustände Fehler-Zustände sein und es muss gefordert werden, dass die Menge
  $I$ der Inputs für die \MEIO{}s leer ist, ansonsten würde es Input-kritische
  Traces geben. $P$ kann keine Aktionen ausführen und $Q$ nur die
  Output"=Aktion $o$, somit gilt für die Sprachen $\{\varepsilon\} = L(P)
  \subset L(Q) = \{\varepsilon , o\}$.\\
  Angenommen es gibt eine schwache as"=Verfeinerungs"=Relation $\mathcal{R}$
  zwischen $P$ und $Q$. Dafür muss $(p_0,q_0)\in \mathcal{R}$ gelten. Da die
  Transition $q_0 \must[o]_Q q_1$ in $Q$ vorhanden ist, wird die Verfeinerung
  dieser in $P$ gefordert, d.h.\ es müsste auch eine must"=Output"=Transition
  in $P$ geben. Es gibt jedoch keine must"=Transition in $P$, dies stellt einen
  Widerspruch zur Annahme dar und es folgt, dass es keine schwache
  as"=Verfeinerungs"=Relation zwischen $P$ und $Q$ geben kann.

  \begin{figure}[htbp]
    \begin{center}
      \begin{tikzpicture}[->, >=latex', auto,node distance=2.5cm, semithick]
        \node [initial,initial text=$P$:] (p0) at (0,0) {$p_0$};

        \node [initial,initial text=$Q$:] (q0) at (6,0) {$q_0$};
        \node (q1) [right of=q0] {$q_1$};

        \path
        (q0) edge node{$o!$} (q1)
        ;
      \end{tikzpicture}
      \caption{Gegenbeispiel zu $\wasRel \Leftarrow \ERel$ mit $I_P = I_Q =
      \emptyset$}
      \label{WasEGegenBsp}
    \end{center}
  \end{figure}
\end{proof}

In dieser Arbeit werden im Gegensatz zu~\cite{Vogler2016MIA3} die Fehler bei
einer Parallelkomposition beibehalten bzw.\ aus dieser entstehend angesehen. Es
werden dabei alle Transitionen, die nicht durch fehlende
Synchronisations"=Möglichkeiten wegfallen, übernommen. In~\cite{Vogler2016MIA3}
hingegen wird der Ansatz verfolgt alle Fehler zu entfernen und durch einen
Universal"=Zustand zu ersetzten, der ausschließlich eingehende
may"=Input"=Transitionen zulässt. Dafür werden die lokalen Aktionen, die zu
einem Fehler führen abgeschritten. Außer dem universalen Zustand gibt es in
einem \MIA{} wie in~\cite{Vogler2016MIA3} keine weiteren Fehler"=Zustände. Auf
diese Normierung wurde hier mit Absicht verzichtete, um sehen zu können, dass
die Fehler einen Ursprung haben, den man später auch noch einsehen kann. Jedoch
gibt es trotzdem einen Zusammenhang zwischen diesen beiden Ansätzen.\\
Die Unterscheidung, die die Basisrelation \EBRel{} hier herbei führt, würde
dort der Unterscheidung zwischen Transitionssystemen, die den
Universal"=Zustand $e$ als Startzustand haben und denen, die einen Startzustand
ungleich $e$ besitzen, entsprechen. Falls hier in einer as"=Implementierung von
$P$ ein Fehler lokal erreichbar ist, dann muss auch in $P$ ein Fehler-Zustand
lokal erreichbar sein, wegen~\ref{lokalFehlerErrKor}~(i). Dies entspricht
$\varepsilon \in \PrET (P)$. Das Abschneiden der lokalen Aktionen wird hier nur
in der Trace"=Menge praktiziert, in~\cite{Vogler2016MIA3} jedoch direkt auf den
Transitionssystemen. Im Fall der lokalen Fehler-Erreichbarkeit bleibt dort also
nur noch der Zustand $e$ als Universal!=Zustand übrig, für den jedes Verhalten
als Verfeinerung zulässig ist, der jedoch keine ausgehenden Transitionen
besitzt. Für die Fehler"=Zustände ist in dem hier verwendeten Ansatz auf
Trace-Ebene bzw.\ bezüglich der Verfeinerungs-Relationen beliebiges Verhalten
möglich. In den Transitionssystemen muss es analog zu $e$ die Transitionen für
das beliebige Verhalten im Allgemeinen nicht geben. Jedoch sind ausgehende
Transitionen von Zuständen aus $E$ hier dennoch zulässig.\\
Da auch der Testing-Ansatz die lokale Fehler-Erreichbarkeit verwendet,
existiert der Zusammenhang auch dort für die Parallelkomposition mit dem
entsprechenden Test.

Jeder \MEIO{} kann normiert werden, so dass er nur noch einen Fehler"=Zustand
besitzt. Im Folgenden sollen nicht mehr alle \MIA{}s wie
in~\cite{Vogler2016MIA3} mit den \MEIO{}s verglichen werden, da die \MIA{}s
disjunktive must"=Transitionen besitzen. Jedoch sind Modale Transitionssysteme,
die syntaktisch konsistent sind und einen Universal"=Zustand haben, der die
gleichen Voraussetzungen wie für \MIA{}s erfüllt, ebenfalls \MIA{}s, nur statt
der disjunktiven must"=Transitionen, haben deren must"=Transitionen auch nur
einen Zustand als Ziel und keine Menge. Im Folgenden sollen \MIA{}s Modale
Transitionssysteme sein, die diese Voraussetzungen erfüllen. Dann gibt es zu
jedem \MEIO{} einen äquivalenten \MIA{}.\\
Die Relation $\sqsubseteq$ aus~\cite{Vogler2016MIA3} wurde als Grundlage für
die Definition der schwachen Simulationen in~\ref{wSimDef} verwendet. Für
normalisierte \MEIO{}s entspricht sie also der Relation \wasRel{} und es gelten
die bereits nachgewiesenen Zusammenhänge zu den anderen Relationen.

\begin{Def}[Normalform]
  \label{NFDef}
  Ein \MEIO{} $P$ ist in \emph{Normalform} (\NF{}), falls die Menge $E_P$ der
  Fehler"=Zustände nur ein einziges Element enthält $E_P=\{e\}$ und für jede
  Transition, für die $p \may[\alpha] e$ mit $p\neq e$ gilt, $\alpha$ ein
  Element der Menge der Input"=Aktionen $I$ ist. Der Zustand $e$ soll keine
  ausgehenden Transitionen besitzen.

  Die \emph{Normalform} von $P$ ist ein \MEIO{} $\NF (P)$, das man aus $P$
  durch die nachfolgenden zwei Schritte erhält:
  \begin{enumerate}[(i)]
    \item Definiere $\overline{E_P} = \left\{p \mid \exists p'\in E_P, w\in
      O^*: p\weakmay[w]_P p'\right\}$ und ersetzte $E_P$ durch
      $\overline{E_P}$. Der dadurch entstehenden \MEIO{} heißt $\overline{P}$.
    \item Falls $p_0\in \overline{E_P}$, ist $NF(P) = \left(\{p_0\}, I, O,
      \emptyset , \emptyset , p_0, \{p_0\}\right)$.\\
      Ansonsten, füge einen neuen Zustand $e$ hinzu, der der einzige
      Fehler"=Zustand wird. Wann immer $p \may[\alpha]_P p'\in \overline{E_P}$
      und $p\notin \overline{E_P}$ für ein $\alpha\in \Sigma \cup \{\tau\}$
      gilt (dann gilt zwingendermaßen $\alpha \in I$), entferne alle
      $\alpha$-Transitionen, die von $P$ ausgehen und füge die Transition
      $p\may[\alpha]_{\NF (P)} e$ hinzu. Entferne die Zustände aus der Menge
      $\overline{E_P}$ des Transitionssystems.
  \end{enumerate}
\end{Def}

Schritt (i) der Normalform Konstruktion verändert nichts an den Transitionen
und auch nichts daran, ob ein Zustand einen nicht sichergestellten Input hat
oder nicht. Die Menge \StET{} wird verändert jedoch bleibt die Menge \PrET{}
gleich, da \StET{} nach dem Schritt (i) alle Traces enthält, die direkt oder
durch lokale Aktionen zu einem Fehler"=Zustand des ursprünglichen \MEIO{}s
führen. Für die Menge $\MIT (P)\backslash\cont (\PrET (P))$ ändert sich nichts
durch die Anwendung des ersten Schrittes auf $P$, da sich an den Zuständen ohne
Zusammenhang zu Fehler"=Zuständen nichts ändert. $\overline{P}$ ist also
äquivalent zu $P$ bezüglich der Relation \ERel{}.\\
Im ersten Fall von Schritt (ii) gilt $\ET (P) = \Sigma ^* = \ET (\NF (P))$. Im
zweiten Fall von Schritt (ii) werden Zustände, die nicht in $\overline{E_P}$
enthalten sind und die keine Transitionen zu Zuständen in der Menge
$\overline{E_P}$ besitzen, unverändert aus $\overline{P}$ in $\NF (P)$
übernommen. Transitionen zwischen solchen Zuständen werden ebenfalls ohne
Veränderungen in das Transitionssystem $\NF (P)$ übernommen. Ein Ablauf zu
einem Traces aus der Menge $\MIT (P)\backslash\cont (\PrET (P))$ beinhaltet nur
solche unveränderten Zustände und Transitionen. Somit ist jeder Trace aus $\MIT
(P)\backslash\cont (\PrET (P))$ auch in $\NF (P)$ ein Input-kritischer Trace
und alle Traces $\MIT (\NF (P))\backslash\cont (\PrET (\NF (P)))$ sind auch in
$P$ Input-kritische Traces, die nicht in $\cont (\PrET (P))$ enthalten sind. In
\ERel{} ist nur die Menge \ET{} relevant, deshalb müssen Traces aus $\MIT
(P)\cap \PrET (P)$ in $\NF (P)$ keine Input-kritischen Traces sein, damit $P$
und $\NF (P)$ äquivalent sein können bezüglich \ERel{}. Da jeweils die erste
Transition, die zu einem Zustand in der Menge $\overline{E _P}$ führt zum
Zustand $e$ umgebogen wird, sind die präfix-minimalen Traces, die in $\PrET
(P)$ enthalten sind auch alle in $\PrET (\NF (P))$ enthalten und jedes Element
aus $\PrET (\NF (P))$ ist ein präfix-minimales Element aus $\PrET (P)$. Durch
das Umbiegen der präfix-minimalen Elemente aus $\PrET (P)$ auf den Zustand $e$,
bei dem der letzte Input nur als may"=Transition ungesetzt wird und $e$ keine
ausgehenden Transitionen besitzt, entstehen neue Input-kritische Traces. Jedoch
sind diese neuen Input-kritischen Traces gleichzeitig in $\MIT (\NF (P))$ und
in $\cont (\PrET (\NF (P)))$ enthalten. Die Menge der Fehler"=Traces \ET{}
bleibt somit im Schritt (ii) gleich. Die Sprache des \MEIO{}s $P$ wird durch
die Konstruktion in (ii) um die Traces in $\cont (\PrET (P))$, die nicht
präfix-minimal sind, verkleinert. Da diese Traces jedoch in der Menge \ET{}
enthalten sind, mit denen die Sprache $L$ in \EL{} geflutet wird, spielt diese
bezüglich der Relation \ERel{} keine Rolle. Es gilt also auch $\EL (P) = \EL
(\NF (P))$. Somit ist $\NF (P)$ äquivalent zu $P$ bezüglich der Relation \ERel{}.

\begin{Prop}[Normalform]
  Jedes \MEIO{} $P$ ist äquivalent zu seiner Normalform $\NF (P)$ bezüglich
  der Relation \ERel{}.
\end{Prop}

Durch die Konstruktion in Definition~\ref{NFDef} kann man zu jedem \MEIO{} $P$
einen bezüglich der Relation \ERel{} äquivalenten \MEIO{} in Normalform
konstruieren. Um zu zeigen, dass es auch immer einen bezüglich \ERel{}
äquivalenten \MIA{} ohne disjunktive must"=Transitionen gibt, reicht es nun
aus zu beweisen, dass ein \MEIO{} in Normalform in einem \MIA{} umgewandelt
werden kann.

\begin{Satz}[Existenz äquvialenter \MIA{}s]
  Für jeden \MEIO{} $P$ gibt es einen äquivalenten \MIA{} $P'$ bezüglich der
  Relation \ERel{}.
\end{Satz}
\begin{proof}
  Man kann \oBdA{} davon ausgehen, dass $P$ ein \MEIO{} in Normalform ist. $P$
  besitzt also genau einen Fehler"=Zustand. Dies ist der universale Zustand des
  \MIA{}s. Alle eingehenden Transitionen in den Zustand $e$ sind
  may"=Input"=Transitionen und alle must"=Transitionen haben zugrundeliegende
  may"=Transitionen. Die Menge der Zustände von $P'$ entspricht $P$. Die
  Startzustände von $P$ und $P'$ sind gleich, genau so wie die Mengen der
  Input- und Output-Aktionen. Für die Relation $\may _{P'}$ gilt $\may _{P'} =
  \may _{P}$ und analog für die must"=Transition. Das daraus resultierende
  Transitionssystem $P'$ ist ein \MIA{} nach der Definition
  in~\cite{Vogler2016MIA3} bis auf die disjunktiven must"=Transitionen, die
  hier durch must"=Transitionen ohne Disjunktionen ersetzt wurden.
\end{proof}

Die \MIA{}s aus~\cite{Vogler2016MIA3} dürfen disjunktive must"=Transitionen
besitzen. Falls man jedoch die hier gemachte Einschränkung vornimmt, ist jeder
\MIA{} auch ein \MEIO{}.
