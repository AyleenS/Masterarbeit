\section{Zusammenhänge}

\begin{Satz}[Zusammenhang der Verfeinerungs-Relationen mit der Fehler-Relation]
  \label{ZusammenhFehlerSatz}
  Für \MEIO{}s $P$ und $Q$ gilt $P\asRel Q\Rightarrow P\wasRel Q\Rightarrow
  P\ERel Q$. Die Implikationen in die andere Richtung gelten jedoch nicht.
\end{Satz}
\begin{proof}\mbox{}\\
  $P\asRel Q\Rightarrow P\wasRel Q$:\\
  Um diese Implikation zu zeigen, muss man nachweisen, dass jede starke
  as"=Verfeinerungs"=Relation auch die Definition~\ref{wSimDef} der schwachen
  as"=Verfeinerungs"=Relation erfüllt. In beiden Simulations-Definitionen
  (\ref{SimDef} und~\ref{wSimDef}) müssen die Punkte für alle $(p,q) \in
  \mathcal{R}$ mit $q\notin E_Q$ gelten. Sei $\mathcal{R}$ nun eine
  as"=Verfeinerungs"=Relation. Es gilt also mit~\ref{SimDef}~1.\, dass $p$ kein
  Fehler-Zustand von $P$ ist. Somit ist auch 1.\ von~\ref{wSimDef} erfüllt. Für
  alle $\alpha\in\Sigma _{\tau}$ impliziert~\ref{SimDef}~2. $q \must[\alpha]_Q
  q'$ $p\must[\alpha]_P p'$ für ein $p'$ mit $p'\mathcal{R} q'$. Da $\Sigma
  _{\tau} = I \cup O \cup \{\tau\}$ gilt, sind dadurch 2.\ und 3.\ der
  Definition~\ref{wSimDef} erfüllt. Die schwache $\varepsilon$-Transition aus
  2.\ führt keine echten Transitionen aus, sondern bleibt beim Zustand $p'$.
  Die schwache $\hat{\omega}$-Transition aus 3.\ entspricht in $\mathcal{R}$
  nur einer einzigen Transition für $\omega$. Die Punkte 4.\ und 5.\ aus
  Definition~\ref{wSimDef} werden durch~\ref{SimDef}~3.\ erfüllt. Es gilt für
  $\mathcal{R}$ $p\may[\alpha]_P p'$ impliziert $q \may[\alpha]_Q q'$ für ein
  $q'$ mit $p'\mathcal{R}q'$. Die in~\ref{wSimDef} geforderten schwachen
  may"=Transitionen werden hier jeweils stark durch eine einzige Transition
  umgesetzt. $\mathcal{R}$ ist also auch eine schwache
  as"=Verfeinerungs"=Relation.

  $P\wasRel Q\Rightarrow P\ERel Q$:\\
  Um diese Implikation zu beweisen wird gezeigt, dass eine beliebige
  schwache as"=Verfeinerungs"=Relation $\mathcal{R}$ zwischen $P$ und $Q$ auch
  die Eigenschaften der Relation $\ERel$ erfüllt. Da $\mathcal{R}$ eine
  schwache as"=Verfeinerungs"=Relation zwischen $P$ und $Q$ ist, muss $p_0
  \mathcal{R} q_0$ gelten. Es sind die folgenden Punkte nachzuweisen:
  \begin{itemize}
    \item $\ET _P \subseteq \ET _Q$,
    \item $\EL _P \subseteq \EL _Q$.
  \end{itemize}
  Für den ersten Punkt wird ein beliebiges $w$ aus $\ET _P$ betrachtet und
  gezeigt, dass dieses auch in $\ET _Q$ enthalten ist. Es kann davon
  ausgegangen werden, dass $w$ präfix-minimal ist, da beide \ET{}-Mengen unter
  \cont{} abgeschlossen sind. $w$ kann ein Element aus $\PrET (P)$ sein oder
  ein Element aus $\MIT (P)$.
  \begin{itemize}
    \item Fall 1 ($w\in\PrET (P)$): Es existiert ein $v\in O_P$, sodass das
      Wort $wv$ in $P$ einen Fehler-Zustand erreicht. Für die entsprechende
      Transitionsfolge existieren Zustände $p_1, p_2,\dots , p_n$ in $P$ und
      Aktionen $\alpha _1, \alpha _2,\dots , \alpha _n$, die zusammengesetzt
      ein Wort $w'=\alpha _1 \alpha _2 \dots \alpha _n$ bilden, dass ohne die
      internen Aktionen $wv$ entspricht. Die entsprechende Transitionsfolge in
      $P$ ist dann $p_0 \may[\alpha _1]_P p_1 \may[\alpha _2]_P \dots p_{n-1}
      \may[\alpha _n]_P p_n \in E_P$. Abhängig davon, ob $\alpha _j$ ein Input
      oder eine lokale Aktion ist, kann mit~\ref{wSimDef}~4.\ bzw.~5.\
      argumentiert werden, dass das jeweilige $\alpha _j$ auch in $Q$ schwach
      ausführbar ist, solange der entsprechende Zustand $q_{j-1}$ kein
      Fehler-Zustand ist für $j\in \{1,2,\dots n\}$. Falls ein $q_j$ für $j <
      n$ in $E_Q$ enthalten ist, wurde bis dort ein Präfix von $wv$ ausgeführt.
      Dieses Präfix ist in $\StET _Q$ enthalten. Es gilt also mit $w=\prune
      (wv)$ und dem Abschluss von \ET{} unter \cont{} $w\in\ET _Q$. Ansonsten
      gibt es in $Q$ einen Trace $q_0 \weakmay[\widehat{\alpha _1}]_Q q_1
      \weakmay[\widehat{\alpha _2}]_Q \dots q_{n-1} \weakmay[\widehat{\alpha
      _n}]_Q q_n$, wobei $p_j \mathcal{R} q_j$ für alle $0 \leq j \leq n$ gilt.
      Mit~\ref{wSimDef}~1.\ folgt, dass $q_n\in E_Q$ gelten muss und somit auch
      $w\in\ET _Q$ mit der Begründung von oben.
    \item Fall 2 ($w\in\MIT (P)$): $w$ ist in $P$ ein Input-kritischer Trace.
      Es existiert also eine Aufteilung von $w$ in $va$ mit $v\in \Sigma ^*$
      und $a\in I$. Der Trace in $P$ kann wie folgt dargestellt werden:
      $\exists v'\in \Sigma _{\tau}^*, \exists p_1, p_2, \dots , p_n, \exists
      \alpha _1, \alpha _2, \dots , \alpha _n: \hat{v'}=v \land v'=\alpha _1
      \alpha _2 \dots \alpha _n \land p_0 \may[\alpha _1]_P p_1 \may[\alpha
      _2]_P \dots p_{n-1} \may[\alpha _n]_P p_n \nmust[a]_P$. Basierend
      auf~\ref{wSimDef}~4.\ bzw.~5.\ kann daraus gefolgert werden, dass die
      $\alpha _j$ auch schwach ausführbar sind in $Q$, falls kein $q_{j-1}$
      angetroffen wird, dass in $E_Q$ enthalten ist für $j\in \{1,2,\dots n\}$.
      Falls eines der $q_j$ mit $0 \leq j \leq n$ ein Fehler-Zustand ist, dann
      ist $w$ in $\ET _Q$ enthalten, da ein Präfix von $w$ ein strikter
      Fehler-Trace in $Q$ ist. Ansonsten gilt $q_0 \weakmay[\widehat{\alpha
      _1}]_Q q_1 \weakmay[\widehat{\alpha _2}]_Q \dots q_{n-1}
      \weakmay[\widehat{\alpha _n}]_Q q_n$ mit $p_j \mathcal{R} q_j$ für alle
      $j\in \{0,1, \dots , n\}$. Da $a$ für $p_n$ in $P$ keine ausgehende
      must"=Transition sein kann, gilt mit~\ref{wSimDef}~2.\ auch $q_n
      \nmust[a]$. $w$ ist in $\MIT _Q \subseteq \ET _Q$ enthalten.
  \end{itemize}
  Für den zweiten Punkt kann man sich auf die Inklusion $\EL _P\backslash\ET _P
  \subseteq \EL _Q$ einschränken, da der erste Punkt bereits vorausgesetzt
  werden kann. $\EL _P\backslash\ET _P$ ist eine Teilmenge der Sprache $L _P$.
  Somit ist ein $w$ in $P$ ausführbar. Es gibt also einen Trace $p_0
  \may[\alpha _1]_P p_1 \may[\alpha _2]_P \dots p_{n-1} \may[\alpha _n]_P p_n$
  in $P$, wobei $w$ der Aktionsfolge $\alpha _1\alpha _2 \dots \alpha _n$
  entspricht bis auf die internen Aktionen. Erneut können
  hier~\ref{wSimDef}~4.\ und~5.\ dazu verwendet werden einen analogen Trace in
  $Q$ zu finden. Falls ein $q_j$ für $0 \leq j \leq n$ in $E_Q$ enthalten ist,
  gilt $w\in\ET _Q \subseteq \EL _Q$. Es wird also im Folgenden davon
  ausgegangen, dass kein $q_j$ ein Fehler-Zustand ist. Es gilt dann $q_0
  \weakmay[\widehat{\alpha _1}]_Q q_1 \weakmay[\widehat{\alpha _2}]_Q \dots
  q_{n-1} \weakmay[\widehat{\alpha _n}]_Q q_n$ in $Q$ mit $p_j \mathcal{R} q_j$
  für $0 \leq j \leq n$ und somit $w\in L _Q\subseteq \EL _Q$.

  $P\asRel Q\hspace{0.1cm}\not\hspace{-0.1cm}\Leftarrow P\wasRel Q$:\\
  Im Abbildung~\ref{asWasGegenBsp} wird ein Gegenbeispiel dargestellt mit einem
  \MEIO{} $Q$ und einer schwachen as"=Verfeinerung $P$ von $Q$, die jedoch
  keine starke as"=Verfeinerung von $Q$ ist. Die schwache
  as"=Verfeinerungs"=Relationen $\mathcal{R}$ zwischen $P$ und $Q$ enthält die
  Tupel $(p_0,q_0)$ und $(p_1,q_{12})$. Damit $\mathcal{R}$ eine schwache
  Simulations"=Relation zwischen $P$ und $Q$ sein kann müssen die Startzustände
  in Relation stehen. Dies ist durch $(p_0,q_0)\in \mathcal{R}$ erfüllt.
  Es sind keine Fehler-Zustände in $Q$ und $P$ enthalten, somit ist 1.\ der
  Definition~\ref{wSimDef} bereits für beide Zustands-Tupel erfüllt. Für das
  Tupel $(p_1,q_{12})$ sind auch 2.-5.\ von~\ref{wSimDef} erfüllt, da weder
  $p_1$ noch $q_{12}$ ausgehende Transitionen besitzen. Für $(p_0,q_0)\in
  \mathcal{R}$ gibt es keine ausgehende must"=Transitionen. Also ist 2.\ und
  3.\ von~\ref{wSimDef} bereits erfüllt. Falls $\alpha$ ein Input ist,
  fordert~\ref{wSimDef}~4., dass die Transition $p_0 \may[\alpha]_P p_1$ in $Q$
  schwach ausführbar ist in der Form $q_0 \may[\alpha]_Q
  \weakmay[\varepsilon]_Q q$. Ein entsprechendes $q$ ist in diesem Fall
  $q_{12}$ und es gilt $p_1 \mathcal{R} q_{12}$. Falls $\alpha$ eine lokale
  Aktion ist, lautet die Forderung $q_0 \weakmay[\hat{\alpha}]_Q q$ für $Q$ und
  $q_{12}$ ist wieder der passende Zustand für $q$, der mit $p_1$ in Relation
  stehen. $\mathcal{R}$ ist also eine schwache as"=Verfeinerungs"=Relation
  zwischen $P$ und $Q$.\\
  Angenommen es gibt auch eine starke as"=Verfeinerungs"=Relation
  $\mathcal{R}'$ zwischen $P$ und $Q$, dann muss $p_0 \mathcal{R}' q_0$
  gelten. Mit~\ref{SimDef}~3.\ wird gefordert, dass die Transition $p_0
  \may[\alpha]_P p_1$ durch eine Transition der Form $q_0 \may[\alpha]_Q q$
  in $Q$ gematched werden muss. Für den Zustand $q$ kommt dieses mal nur
  $q_{11}$ in Frage. Es muss also $(p_1,q_{11})\in \mathcal{R}$ gelten. Der
  zweite Punkt der Definition~\ref{SimDef} fordert, dass die
  $\tau$-must"=Transition aus $Q$ auch in $P$ auftauchen muss. Es müsste also
  ein $p$ geben, für dass $p_1 \must[\tau]_P p$ gilt und das Tupel $(p,q_{12})$
  müsste in $\mathcal{R}$ enthalten sein. Da es keine solche Transition gibt,
  tritt ein Widerspruch zur Annahme auf. Es kann also keine starke
  as"=Verfeinerungs"=Relation zwischen $P$ und $Q$ geben.

  \begin{figure}[htbp]
    \begin{center}
      \begin{tikzpicture}[shorten >=1pt,auto,node distance=2.5cm]
        \node [initial,initial text=$Q$:] (q0) at (0,0) {$q_0$};
        \node (q11) [right of=q0] {$q_{11}$};
        \node (q12) [right of=q11] {$q_{12}$};

        \path[->]
        (q0) edge[dashed] node{$\alpha$} (q11)
        (q11) edge node{$\tau$} (q12)
        ;

        \node [initial,initial text=$P$:] (p0) at (10,0) {$p_0$};
        \node (p1) [right of=p0] {$p_1$};

        \path[->]
        (p0) edge[dashed] node{$\alpha$} (p1)
        ;
      \end{tikzpicture}
      \caption{Gegenbeispiel zu $\asRel \Leftarrow \wasRel$}
      \label{asWasGegenBsp}
    \end{center}
  \end{figure}

  $P\wasRel Q\hspace{0.1cm}\not\hspace{-0.1cm}\Leftarrow P\ERel Q$:\\
  Die nicht Gültigkeit dieser Implikation beruht darauf, dass Simulationen
  strenger sind als Sprach Inklusionen. Das Gegenbeispiel hier ist also so
  aufgebaut, dass $\ET (P) =\ET (Q) = \emptyset$ und $L(P) \subseteq L(Q)$
  gilt, jedoch keine schwache as"=Verfeinerungs"=Relation zwischen $P$ und $Q$
  existieren kann. $Q$ und $P$ sind in der Abbildung~\ref{WasEGegenBsp}
  dargestellt. Damit $\ET (P) =\ET (Q) = \emptyset$ gilt, dürfen keine der
  Zustände Fehler-Zustände sein und es muss gefordert werden, dass die Menge
  $I$ der Inputs für die \MEIO{}s leer ist, ansonsten würde es Input-kritische
  Traces gegen. $P$ kann keine Aktionen ausführen und $Q$ nur die Output Aktion
  $o$ somit gilt für die Sprachen $\{\varepsilon\} = L(P) \subset L(Q) =
  \{\varepsilon , o\}$.\\
  Angenommen es gibt eine schwache as"=Verfeinerungs"=Relation $\mathcal{R}$
  zwischen $P$ und $Q$. Dafür muss $(p_0,q_0)\in \mathcal{R}$ gelten. Da es
  keine Fehler-Zustände in $P$ gibt, ist~\ref{wSimDef}~1.\ erfüllt. Die
  Punkte 2., 4.\ und 5.\ der Definition~\ref{wSimDef} stellen keine
  Forderungen an die Relation $\mathcal{R}$. Da jedoch die Transition $q_0
  \must[o]_Q q_1$ in $Q$ vorhanden ist, wird die Verfeinerung dieser in $P$
  gefordert es müsste also auch eine must"=Output"=Transition in $P$ geben. Da
  diese nicht vorhanden ist, stellt dies einen Widerspruch zur Annahme dar und
  es folgt, dass es keine schwache as"=Verfeinerungs"=Relation zwischen $P$ und
  $Q$ geben kann.

  \begin{figure}[htbp]
    \begin{center}
      \begin{tikzpicture}[shorten >=1pt,auto,node distance=2.5cm]
        \node [initial,initial text=$Q$:] (q0) at (0,0) {$q_0$};
        \node (q1) [right of=q0] {$q_1$};

        \path[->]
        (q0) edge node{$o!$} (q1)
        ;

        \node [initial,initial text=$P$:] (p0) at (7,0) {$p_0$};
      \end{tikzpicture}
      \caption{Gegenbeispiel zu $\wasRel \Leftarrow \ERel$ mit $I_P = I_Q =
      \emptyset$}
      \label{WasEGegenBsp}
    \end{center}
  \end{figure}
\end{proof}

In dieser Arbeit werden im Gegensatz zu~\cite{Vogler2016MIA3} die Fehler bei
einer Parallelkomposition beibehalten bzw.\ aus dieser entstehend angesehen. Es
werden dabei alle Transitionen, die nicht durch fehlende
Synchronisations"=Möglichkeiten wegfallen, übernommen. In~\cite{Vogler2016MIA3}
hingegen wird der Ansatz verfolgt alle Fehler zu entfernen und durch einen
universal Zustand zu ersetzten, der nur eingehende may"=Input"=Transitionen
zulässt. Auf diese Normierung wurde hier mit Absicht verzichtete um sehen zu
können, dass die Fehler einen Ursprung haben, den man später auch noch einsehen
kann. Jedoch gibt es trotzdem einen Zusammenhang zwischen diesen beiden
Ansätzen.\\
Die Unterscheidung, die die Basisrelation \EBRel{} hier herbei führt, würde
dort der Unterscheidung zwischen Transitionssystemen, die den universal
Zustand $e$ als Startzustand haben und denen, die einen Startzustand ungleich
$e$ besitzen, entsprechen. Falls hier in einer as"=Implementierung von $P$ ein
Fehler lokal erreichbar ist, dann muss auch in $P$ ein Fehler-Zustand lokal
erreichbar sein, wegen~\ref{lokalFehlerErrKor}~(i). Dies entspricht
$\varepsilon \in \PrET (P)$. Das Abschneiden den lokalen Aktionen wird hier nur
in der Trace Menge praktiziert, in~\cite{Vogler2016MIA3} jedoch direkt auf den
Transitionssystemen. Im Fall der lokalen Fehler-Erreichbarkeit bleibt also
nur noch der Zustand $e$ als universal Zustand übrig, für den jedes Verhalten
zulässig ist. Falls man hier dieses Pruning mit beliebigem Verhalten nachmachen
wollen würde, müsste man an den Zustand $e$ noch eine may-Schleife für alle
Aktionen aus $I$ und $O$ hinzufügen.\\
Da auch der Testing-Ansatz die lokale Fehler-Erreichbarkeit verwendet,
existiert der Zusammenhang auch dort für die Parallelkomposition mit dem
entsprechenden Test.\\
Falls es keine Input-kritischen Traces gibt, entspricht \ERel{} der Relation
$\sqsubseteq$ aus~\cite{Vogler2016MIA3}, falls auf die \MEIO{}s das
entsprechende Abschneiden der Fehler-Traces angewendet würde und diese dann
durch einen universal Zustand mit may"=Schleife für alle Inputs und Outputs
ersetzt würden. Gibt es in $Q$ jedoch einen Input-kritische Traces, würde $P
\ERel Q$ zulassen, dass $P$ einen gekürzten Fehler-Trace anstatt dessen
besitzt. Diese Art der Verfeinerung lässt $\sqsubseteq$ nicht zu. Jedoch sind
die Input-kritischen Traces auch dort die potentiellen Auslöser für neue
Fehler-Zustände in einer Parallelkomposition. Diese Problem basiert auf dem
Problem, dass \ERel{} keine schwache as"=Verfeinerungs"=Relation sein muss, die
in \wasRel{} enthalten ist.
