\chapter{Verfeinerungen für Kommunikationsfehler-, Ruhe- und Divergenz-Freiheit}

In diesem Kapitel soll die Menge der betrachteten Zustände noch einmal
erweitert werden. Somit werden dann Fehler-, Ruhe- und Divergenz-Zustände
betrachtet. Wie im letzten Kapitel wird auch hier nur der Testing-Ansatz
betrachtet.

\begin{Def}[Divergenz]
  Ein \emph{Divergenz-Zustand} ist ein Zustand in einem \MEIO{} $P$, der eine
  unendliche Folge an $\tau$s ausführen kann via may-Transitionen.\\
  Die Menge $Div(P)$ besteht aus all diesen divergenten Zuständen des \MEIO{}s
  $P$.
\end{Def}

Die unendliche Folge an $\tau$s kann durch eine Schleife an einem durch $\tau$s
erreichbaren Zustand ausführbar sein oder durch einen Weg, der mit $\tau$s
ausführbar ist, mit dem unendliche viele Zustände durchlaufen werden. Es ist
jedoch zu beachten, dass ein Zustand, von dem aus unendlich viele Zustände
durch $\tau$s  erreichbar sind, nicht divergent sein muss. Es ist auch möglich,
dass dieser Zustand eine unendliche Verzweigung hat und somit keine unendlichen
Folgen an $\tau$s ausführen kann.\\
Als Erreichbarkeitsbegriff wird wieder die lokale Erreichbarkeit verwendet und
somit eine optimistische Betrachtungsweise. Da das Divergieren eines Systems
nicht mehr verhindert werden kann, sobald ein divergenter Zustand lokal
erreicht werden kann, ist Divergenz als ähnlich \glqq schlimm\grqq{} zu
bewerten wie ein Fehler-Zustand.

\begin{Def}[Test und Verfeinerung für Divergenz]
  \label{DivTestDef}
  Ein \emph{Test} $T$ ist eine Implementierung. Ein \MEIO{} $P$
  \emph{as-erfüllt} einen Divergenz-Test $T$, falls $S\|T$ fehler-, ruhe- und
  divergenz-frei ist für alle $S\in \asimp (P)$. Es wird dann $P\DsatAs T$
  geschrieben. Die Parallelkomposition $S\|T$ ist \emph{fehler-}, \emph{ruhe-}
  und \emph{divergenz-frei}, wenn kein Fehler-, Ruhe- oder Divergenz-Zustand
  lokal erreichbar ist.\\
  Ein \MEIO{} $P$ \emph{Divergenz-verfeinert} $P'$, falls für alle Tests $T$:
  $P'\DsatAs T \Rightarrow P\DsatAs T$.
\end{Def}

Da nun die grundlegenden Definitionen für Divergenz festgehalten sind, kann man
sich einen Begriff für die Traces zu divergenten Zuständen bilden. Da oben
bereits festgestellt wurde, dass Divergenz als ähnlich \glqq schlimmes
Fehlverhalten\grqq{} anzusehen ist wie Fehler und dass das Divergieren eines
Systems nicht mehr verhinderbar ist, sobald ein divergenter Zustand lokal
erreichbar ist, kommt für die Divergenz-Traces wieder die \prune{}-Funktion zum
Einsatz. Ein System, das unendliche viele $\tau$s ausführen kann, ist von außen
nicht von so einem System zu unterscheiden, das einen Fehler-Zustand erreicht.
Somit wird in den Trace-Mengen auch nicht zwischen Fehler"=Traces und
Divergenz-Traces explizit unterschieden. Dadurch genügt es nicht mehr nur mit
den Fehler"=Traces die Sprache zu fluten, sondern es muss sowohl mit den
Fehler"=Traces wie auch den Divergenz"=Traces geflutet werden. Ebenso werden
die strikten Ruhe"=Traces mit diesen beiden Trace-Mengen geflutet.

\begin{Def}[Divergenz-Traces]
  Sei $P$ ein \MEIO{} und definiere:
  \begin{itemize}
    \item \emph{strikte Divergenz-Traces}: $\StDT (P) := \left\{w\in\Sigma
      ^*\mid p_0\weakmay[w]_P p\in Div(P)\right\}$,
    \item \emph{gekürzte Divergenz-Traces}: $\PrDT (P) := \left\{\prune (w)\mid
      w\in\StDT (P)\right\}$.
  \end{itemize}
\end{Def}

Analog zu den Propositionen~\ref{KommTracesProp} und~\ref{RuheTraceProp} gibt es
hier auch eine Proposition, die die Divergenz-Traces eines \MEIO{}s mit den
Divergenz-Traces seiner as"=Implementierungen verbindet. Die Begründung
verläuft analog zu den Propositionen der vorangegangenen Kapitel.

\begin{Prop}[Divergenz-Traces und Implementierungen]
  \label{DivTraceProp}
  Sei $P$ ein \MEIO{}.
  \begin{enumerate}
    \item Für die strikten Divergenz-Traces gilt: $\StDT (P) \subseteq
      \big\{w\in\Sigma ^*\mid \exists P' \in \asimp (P): p'_0\weakmust[w]_{P'}
      p'\in Div(P')\big\} = \underset{P'\in\asimp (P)}{\bigcup} \StDT (P')$.
    \item Für die gekürzten Divergenz-Traces von $P$ gilt der Zusammenhang:
      $\PrDT (P) \subseteq \{\prune (w) \mid \exists P'\in\asimp (P): w\in\StDT
      (P')\} = \underset{P'\in\asimp (P)}{\bigcup} \PrDT (P')$.
  \end{enumerate}
\end{Prop}
\begin{proof}\mbox{}
  \begin{enumerate}
    \item Um diese Inklusion beweisen zu können wird wieder eine
      as"=Implementierung $P'$ von $P$ und eine passende
      as"=Verfeinerungs"=Relation $\mathcal{R}$ angegeben, so dass alle
      strikten Divergenz-Traces von $P$ auch in $P'$ enthalten sind. In diesem
      Fall funktioniert der Ansatz, alle Traces aus $P$ in $P'$ zu
      implementieren und keine Fehler-Zustände zu übernehmen. Die Definition
      von $P'$ lautet also:
      \begin{itemize}
        \item $P'=P$,
        \item $p'_0=p_0$,
        \item $I_{P'}=I_P$ und $O_{P'}=O_P$,
        \item $\must _{P'} =\may _{P'} = \may _P$,
        \item $E_{P'}=\emptyset$.
      \end{itemize}
      Die passende as"=Verfeinerungs"=Relation $\mathcal{R}$ ist die
      Identität-Relation. Wie bereits im Beweis zu Proposition~\ref{LImpProp}
      begründet erfüllt $\mathcal{R}$ alle Punkte der Definition~\ref{SimDef}
      um eine as"=Verfeinerungs"=Relation zu sein. Es wird ein $w$ aus $\StDT
      (P)$ betrachtet. Es gibt also einen Trace in $P$ auf dem das Wort $w$
      ausgeführt wird und der einen divergenten Zustand erreicht. Es gilt also
      $\exists w' \in \Sigma _{\tau}^*, \exists \alpha _1, \alpha _2, \dots ,
      \alpha _n, \exists p_1, p_2, \dots , p_n: \hat{w'} = w \land w' = \alpha
      _1\alpha _2\dots\alpha _n \land p_0 \may[\alpha _1]_P p_1 \may[\alpha
      _2]_P \dots p_{n-1} \may[\alpha _n]_P p_n \in Div _P$. Die
      Identitäts"=Relation $\mathcal{R}$ setzt die Zustände des Traces mit den
      analogen Zuständen aus $P'$ in Relation. Zusätzlich mit der
      Implementierung aller Transitionen aus $P$ in $P'$ ergibt sich der selbe
      Trace in $P'$. Es gilt also $p'_0 \may[\alpha _1]_{P'} p'_1 \may[\alpha
      _2]_{P'} \dots p'_{n-1} \may[\alpha _n]_{P'} p'_n$ mit $(p'_j,p_j) \in
      \mathcal{R}$ für $0\leq j \leq n$. Da $p_n$ in $Div _P$ enthalten ist,
      gibt es von diesem Zustand in $P$ aus die Möglichkeit eine unendliche
      Folge an $\tau$s auszuführen. Die Ausführbarkeit muss sich dabei auf mit
      $\tau$ beschriftete may"=Transitionen in $P$ stützen. Da diese
      Transitionen in $P'$ alle übernommen wurden, ist auch für $p'_n$ eine
      unendliche Folge an $\tau$s ausführbar. Es gilt also $p'_n\in Div _{P'}$
      und somit $w\in \StDT (P')$. Insgesamt folgt also für diese $P'$ $\StDT
      (P) = \StDT (P')$.
    \item Dieser Punkt entspricht 1.\ bis auf die Anwendung der
      \prune{}-Funktion auf beiden Seiten des Inklusions-Symbols. Da $\prune{}$
      monoton ist, folgt dieser Punkt direkt auf dem letzten.
  \end{enumerate}
\end{proof}

Da die Ruhe"=Traces mit den Fehler- und Divergenz"=Traces geflutet werden
sollen, kann die Ruhe"=Semantik nicht aus dem letzten Kapitel übernommen
werden. Auch die geflutete Sprache aus dem Fehler-Kapitel kann nicht
beibehalten werden. Nur die Fehler"=Traces \ET{} können ohne Veränderung auch
in diesem Kapitel verwendet werden. Jedoch werden diese Traces im weiteren
Verlauf nur innerhalb der größeren Trace-Menge \EDT{} relevant sein.

\begin{Def}[Kommunikationsfehler-, Ruhe- und Divergenz-Semantik]
  \label{DivSemDef}
  Sei $P$ ein \MEIO{}.
  \begin{itemize}
    \item Die Menge der \emph{Divergenz-Traces} von $P$ ist $\DT (P) := \cont
      (\PrDT (P))$.
    \item Die Menge der \emph{Fehler-Divergenz-Traces} von $P$ ist $\EDT (P) :=
      \ET (P)\cup\DT (P)$.
    \item Die Menge der \emph{Fehler-divergenz-gefluteten
      Ruhe-Traces} von $P$ ist $\QDT (P) := \StQT (P)\cup\EDT (P)$.
    \item Die Menge der \emph{Fehler-divergenz-gefluteten
      Sprache} von $P$ ist $\EDL (P) := L(P)\cup\EDT (P)$.
  \end{itemize}
  Für zwei \MEIO{}s $P_1,P_2$ mit der gleichen Signatur schreibt man $P_1\DRel
  P_2$, wenn $\EDT _1\subseteq \EDT _2, \QDT _1\subseteq \QDT _2$ und $\EDL
  _1\subseteq \EDL _2$ gilt.
\end{Def}

\vspace{0.2cm}

\begin{Prop}[Kommunikationsfehler-, Ruhe-, Divergenz-Semantik und
  Implementierungen]
  \label{DivSemProp}
  Sie $P$ ein \MEIO{}.
  \begin{enumerate}
    \item Für die Menge der Divergenz-Traces von $P$ gilt $\DT (P) \subseteq
      \underset{P'\in\asimp (P)}{\bigcup} \DT (P')$.
    \item Für die Menge der Fehler-Divergenz-Traces von $P$ gilt die folgende
      Gleichheit $\EDT (P) = \underset{P'\in\asimp (P)}{\bigcup} \EDT (P')$.
    \item Für die Menge der Fehler-divergenz-gefluteten Ruhe-Traces von $P$
      gilt $\QDT (P) = \underset{P'\in\asimp (P)}{\bigcup} \QDT (P')$.
    \item Für die Menge der Fehler-divergenz-gefluteten Sprache von $P$ gilt
      $\EDL (P) = \underset{P'\in\asimp (P)}{\bigcup} \EDL (P')$.
  \end{enumerate}
\end{Prop}
\begin{proof}\mbox{}\\
  1.:\\
  Es gilt bereits $\PrDT (P) \subseteq \underset{P'\in\asimp (P)}{\bigcup}
  \PrDT (P')$, wegen 2.\ der Proposition~\ref{DivTraceProp}. Aus der Monotonie
  von \cont{} folgt also auch die hier geforderte Inklusion.

  2. \glqq$\subseteq$\grqq{}:
  \begin{align*}
    \EDT (P)&\overset{\ref{DivSemDef}}{=} \ET (P)\cup \DT (P)\\
    &\hspace{-0.2cm}\overset{\ref{KommSemProp}~1.}{=} \left(\underset{P'\in
    \asimp (P)}{\bigcup} \ET (P')\right)\cup \DT (P)\\
    &\overset{1.}{\subseteq} \left(\underset{P'\in
    \asimp (P)}{\bigcup} \ET (P')\right)\cup \left(\underset{P'\in \asimp
    (P)}{\bigcup} \DT (P')\right)\\
    &= \underset{P'\in \asimp (P)}{\bigcup} \ET (P') \cup \DT (P')\\
    &\overset{\ref{DivSemDef}}{=} \underset{P'\in\asimp (P)}{\bigcup} \EDT
    (P').\\
  \end{align*}

  2. \glqq$\supseteq$\grqq{}:\\
  Für ein präfix-minimales $w$ aus der Menge $\EDT (P')$ einer
  as"=Implementierung $P'$ von $P$ wird für diese Inklusion gezeigt, dass auch
  $w\in\EDT (P)$ gilt. Es genügt ein präfix-minimales Element, da die
  \EDT{}-Mengen unter \cont{} abgeschlossen sind. Falls das $w$ in $\ET (P')$
  enthalten ist, folgt $w\in\ET (P) \subseteq\EDT (P)$ aufgrund des ersten
  Punktes der Proposition~\ref{KommSemProp}. Es ist also nur noch der Fall zu
  betrachten, in dem $w\in\DT (P') \backslash \ET (P')$ gilt. Es gibt also
  einen Trace der Form $p'_0 \may[\alpha _1]_{P'} p'_1 \may[\alpha _2]_{P'}
  \dots p'_{n-1} \may[\alpha _n]_{P'} p'_n\in Div _{P'}$ in $P'$, wobei $\alpha
  _1\alpha _2\dots\alpha _n$ $wv$ bis auf interne Aktionen entspricht mit $v\in
  O^*$. Keiner der Zustände $p'_j$ mit $0\leq j \leq n$ ist in $E_{P'}$
  enthalten. Da $P'$ eine as"=Implementierung von $P$ ist, muss es eine
  as"=Verfeinerungs"=Relationen $\mathcal{R}$ zwischen $P'$ und $P$ geben mit
  $(p'_0,p_0) \in\mathcal{R}$. Solange in $P$ kein Fehler-Zustand erreicht
  wird, fordert~\ref{SimDef}~3., dass der Trace aus $P'$ in $P$ ebenfalls
  möglich ist. Falls ein Fehler-Zustand in $P$ erreicht wird, ist ein Präfix
  von $wv$ in $\StET (P)$ enthalten und mit $w =\prune (wv)$ gilt somit $w\in
  \ET (P) \subseteq \EDT(P)$. Es gibt den Trace $p_0 \may[\alpha _1]_P p_1
  \may[\alpha _2]_P \dots p_{n-1} \may[\alpha _n]_P p_n$ mit $p'_j\mathcal{R}
  p_j$, falls keines der $p_j$'s in $E_P$ enthalten ist für $j\in\{0,1,\dots
  ,n\}$. Für $p'_n$ in ein Transitionsfolge aus unendlichen vielen $\tau$s
  ausführbar. Es gibt also entweder eine Schleife, die via $\tau$s immer wieder
  durchlaufen werden kann, oder einen Weg mit unendlichen vielen Zuständen, die
  durch $\tau$ Transitionen verbunden sind. Falls innerhalb der Schleife oder
  auf dem Weg ein Zustand in $E_{P'}$ enthalten ist, fordert $\mathcal{R}$
  mit~\ref{SimDef}~1., dass der erste solche Zustand in Relation mit einem
  Zustand aus $E_P$ steht und somit $w \in \ET _P \subseteq \EDT _P$ gilt.
  Falls alle Zustände, die auf dem der unendlichen $\tau$ Folge von $p'_n$ in
  $P'$ aus liegen nicht in $E_{P'}$ enthalten sind, fordert~\ref{SimDef}~3.\
  auch deren Ausführbarkeit von $p_n$ aus in $P$, bis entweder ein Zustand in
  $E_P$ erreicht wird, oder ebenfalls eine unendliche $\tau$ Folge in $P$ von
  $p_n$ aus ausführbar ist. Im ersten Fall gilt wieder $w \in \ET _P \subseteq
  \EDT _P$. Ansonsten ist $p_n\in Div _P$ erfüllt und somit $w\in\StDT (P)$.
  Mit $w=\prune (wv)$ folgt $w\in\PrDT (P) \subseteq \EDT (P)$.

  3. \glqq$\subseteq$\grqq{}:
  \begin{align*}
    \QDT (P)&\overset{\ref{DivSemDef}}{=} \StQT (P)\cup \EDT (P)\\
    &\overset{\ref{RuheTraceProp}}{\subseteq} \left(\underset{P'\in \asimp
    (P)}{\bigcup} \StQT (P')\right)\cup \EDT (P)\\
    &\overset{2.}{=} \left(\underset{P'\in \asimp (P)}{\bigcup} \StQT
    (P')\right)\cup \left(\underset{P'\in \asimp (P)}{\bigcup} \EDT
    (P')\right)\\
    &= \underset{P'\in \asimp (P)}{\bigcup} \StQT (P') \cup \EDT (P')\\
    &\overset{\ref{DivSemDef}}{=} \underset{P'\in\asimp (P)}{\bigcup} \QDT
    (P').\\
  \end{align*}

  3. \glqq$\supseteq$\grqq{}:\\
  Dieser Beweis verläuft analog zu dem Beweis von \glqq$\supseteq$\grqq{} der
  Proposition~\ref{RuheSemProp}, man muss nur die \ET{}-Mengen durch
  \EDT{}-Mengen ersetzten und für den Fall $w\in\EDT (P')$ folgt $w\in\EDT (P)$
  wegen des zweiten Punktes dieser Proposition und nicht wegen
  Proposition~\ref{KommSemProp}.

  4. \glqq$\subseteq$\grqq{}:
  \begin{align*}
    \EDL (P)&\overset{\ref{DivSemDef}}{=} L (P)\cup \EDT (P)\\
    &\overset{\ref{LImpProp}}{\subseteq} \left(\underset{P'\in \asimp
    (P)}{\bigcup} L (P')\right) \cup \EDT (P)\\
    &\overset{2.}{=} \left(\underset{P'\in \asimp (P)}{\bigcup} L (P')\right)
    \cup \left(\underset{P'\in \asimp (P)}{\bigcup} \EDT (P')\right)\\
    &= \underset{P'\in \asimp (P)}{\bigcup} L (P') \cup \EDT (P')\\
    &\overset{\ref{DivSemDef}}{=} \underset{P'\in\asimp (P)}{\bigcup} \EDL
    (P').\\
  \end{align*}

  4. \glqq$\supseteq$\grqq{}:\\
  Für den Beweis dieser Inklusion kann man auf den Beweis
  von~\ref{KommSemProp}~2.~\glqq$\supseteq$\grqq{} zurück greifen. Es müssen
  wie bei 3.\ nur die \ET{}-Mengen durch \EDT{}-Mengen ersetzt werden und die
  Einschränkung auf die geflutete Sprache ohne die Menge \EDT{} ist möglich
  wegen 2.\ der aktuellen Proposition.
\end{proof}

Aus der so eben bewiesenen Proposition über die Gleichheit der betrachteten
Traces, lässt sich wie in den letzten beiden Kapiteln eine Aussage über die
lokale Erreichbarkeit der \glqq fehlerhaften Zustände\grqq{} in einer
Spezifikation und den zugehörigen Implementierungen treffen.

\begin{Kor}[lokale Divergenz Erreichbarkeit]\mbox{}
  \label{lokalDivErrProp}
  \begin{enumerate}[(i)]
    \item Falls in einem \MEIO{} $P$ ein Fehler lokal erreichbar ist, dann
      existiert auch eine as"=Implementierung, in der ein Fehler lokal
      erreichbar ist.
    \item Falls in einem \MEIO{} $P$ Divergenz lokal erreichbar ist, dann
      existiert auch eine as"=Implementierung, in der Divergenz lokal
      erreichbar ist.
    \item Falls ein \MEIO{} $P$ einen lokal erreichbaren Ruhe-Zustand besitzt,
      dann existiert auch eine as"=Implementierung, in der ein Ruhe-Zustand
      lokal erreichbar ist.
    \item Falls es eine as"=Implementierung von $P$ gibt, die einen Fehler,
      Ruhe oder Divergenz lokal erreicht, dann ist auch ein Fehler, Ruhe oder
      Divergenz in $P$ lokal erreichbar.
  \end{enumerate}
\end{Kor}
\begin{proof}\mbox{}
  \begin{enumerate}[(i)]
    \item Dieser Punkt folgt wie in~\ref{lokaleRuheErrKor} direkt aus
      Korollar~\ref{lokalFehlerErrKor}~(i).
    \item Ein divergenter Zustand ist in $P$ lokal erreichbar, wenn
      $\varepsilon \in \DT _P$ gilt. Mit~\ref{DivSemProp}~1. folgt daraus, dass
      es auch mindestens eine as"=Implementierung $P'$ aus $\asimp (P)$ geben
      muss, für die $\varepsilon$ in $\DT (P')$ enthalten ist. Da \DT{} die
      Menge der vorgesetzten um lokale Aktionen gekürzten strikten
      Divergenz-Traces ist, muss es lokale Aktionen in $P'$ geben, die zu einem
      divergenten Zustand führen. Es ist also auch in $P'$ Divergenz lokal
      erreichbar.
    \item Dieser Punkt folgt direkt aus Korollar~\ref{lokaleRuheErrKor}~(ii).
    \item In $P'\in\asimp (P)$ sei ein Fehler-, Ruhe- oder Divergenz-Zustand
      lokal erreichbar. Es gilt dann also $w\in\QDT _{P'}$ für $w\in O^*$. Mit
      Proposition~\ref{DivSemProp}~3.\ gilt auch $w\in\QDT _P$ Die Menge \QDT{}
      setzt sich aus den Mengen \ET{}, \StQT{} und \DT{} zusammen. Es muss also
      in $P$ ein Fehler-, Ruhe- oder Divergenz-Zustand lokal erreichbar sein,
      da ein $w$, bestehend nur aus lokalen Aktionen, in $\QDT _P$ enthalten
      ist.
  \end{enumerate}
\end{proof}

\DRel{} ist somit keine Einschränkung von \ERel{} so wie \QRel{}. Es können
Systeme mit einem Fehler nicht von Systemen mit Divergenz
unterschieden werden. Da die Divergenz-Test zwischen diesen Fehler-Arten auch
keine Unterscheidung machen, muss eine sinnvolle Relation dies Eigenschaft auch
übernehmen, so wie \DRel{} dies tut.

\begin{Satz}[Kommunikationsfehler-, Ruhe- und Divergenz-Semantik für
  Parallelkompositionen]
  \label{DivSemSatz}
  Für zwei komponierbare \MEIO{}s $P_1,P_2$ und ihre Komposition $P_{12}$ gilt:
  \begin{enumerate}
    \item $\EDT _{12} =\cont (\prune ((\EDT _1\|\EDL _2)\cup (\EDL _1\|\EDT
      _2)))$,
    \item $\QDT _{12} =(\QDT _1\|\QDT _2)\cup \EDT _{12}$,
    \item $\EDL _{12} =(\EDL _1\|\EDL _2)\cup \EDT _{12}$.
  \end{enumerate}
\end{Satz}
\begin{proof}\mbox{}\\
  1. \glqq$\subseteq$\grqq{}:\\
  Da beide Seiten der Gleichung unter \cont{} abgeschlossen sind, genügt es ein
  präfix-minimales Element $w$ zu betrachten. Es muss hier unterschieden
  werden, ob $w\in\ET _{12}$ oder $w\in\DT _{12}\backslash\ET _{12}$ betrachtet
  wird. Im ersten Fall ist das $w$ in der rechten Seite der Gleichung enthalten
  wegen des Beweises des ersten Punktes von Satz~\ref{KommFehlerSemSatz} und da
  $\ET (P)\subseteq \EDT (P)$ und $\EL (P)\subseteq \EDL (P)$ gilt. Deshalb
  wird im weiteren Verlauf dieses Beweises davon ausgegangen, dass $w\in\DT
  _{12}\backslash\ET _{12}$ gilt und es wird versuch zu zeigen, dass dieses $w$
  ebenfalls in der rechten Seite enthalten ist. Da das betrachtete $w$
  präfix-minimal ist, gilt $w\in\PrDT _{12}\backslash\ET _{12}$. Aus der
  Definition~\ref{DivSemDef} weiß man, dass ein $v\in O^*_{12}$ existiert,
  sodass $(p_{01},p_{02})\weakmay[w]_{12} (p_1,p_2)\weakmay[v]_{12}
  (p'_1,p'_2)$ gilt mit $(p'_1,p'_2)\in Div _{12}$. Durch die Projektion auf die
  Transitionssysteme $P_1$ und $P_2$ erhält man $p_{01}\weakmay[w_1]_1 p_1
  \weakmay[v_1]_1 p'_1$ und $p_{02}\weakmay[w_2]_2 p_2\weakmay[v_2]_2 p'_2$ mit
  $w\in w_1\|w_2$ und $v\in v_1\|v_2$. Aus $(p'_1,p'_2)\in Div _{12}$ folgt,
  dass \oBdA{} $p'_1\in Div _1$ gilt, d.h.\ $w_1v_2\in\StDT _1\subseteq\EDT
  _1$. Da $p_{02}\weakmay[w_2v_2]_2$ gilt, erhält man $w_2v_2\in\EDL _2$. Somit
  gilt insgesamt $wv\in\EDT _1\|\EDL _2$ und da $v\in O^*_{12}$, ist $w$ in der
  rechten Seite der Gleichung enthalten und es folgt insgesamt $\prune (wv) =
  \prune (w)$.

  1. \glqq$\supseteq$\grqq{}:\\
  Es wird ebenso wie oben nur ein präfix-minimales $x$ betrachtet wegen des
  Abschlusses beider Seiten der Gleichung unter \cont{}. Es wird also für ein
  beliebiges $x\in\prune ((\EDT _1\|$ $\EDL _2)\cup (\EDL _1\|\EDT _2))$ gezeigt,
  dass dieses oder eines seiner Präfixe auch in $\EDT _{12}$ enthalten ist. Da
  das $x$ aus der \prune{}-Funktion entstanden ist, lässt sich ein $y$ aus
  $O^*_{12}$ finden, sodass $xy\in (\EDT _1\|\EDL _2)\cup (\EDL _1\|\EDT _2)$.
  Es wird nun noch vorausgesetzt, dass \oBdA{} $xy\in \EDT _1\|\EDL _2$ gilt,
  d.h.\ es existiert $w_1\in\EDT _1$ und $w_2\in\EDL _2$ mit $xy\in
  w_1\|w_2$.\\
  \TODO{erzwungenen Zeilenumbruch kontrollieren}\\
  Die folgende Argumentation läuft analog zu der im Beweis der Inklusion $\ET
  _{12} \supseteq \cont($ $\prune ((\ET _1\|\EL _2)\cup (\EL _1\|\ET _2)))$ aus
  Satz~\ref{KommFehlerSemSatz}. Es muss dazu nur jeweils an den Stellen, an
  denen $\PrET (P)\cup\MIT (P)$ steht auch noch eine Vereinigung mit $\PrDT
  (P)$ vorgenommen werden. Für Fall I und II aus dem Beweis der oben genannten
  Inklusion von Satz~\ref{KommFehlerSemSatz} ist jeweils kein weiterer
  Unterfall für $v'_2$ notwendig da, wenn $v'_2$ nicht ausführbar ist, bereits
  ein Fehler-Zustand in der Parallelkomposition
  entsteht. Somit ist egal, ob auch noch Divergenz vorlag. Falls $v'_2$
  ausführbar, ist nicht relevant, ob eine Divergenz-Möglichkeit bestanden hat,
  da diese nicht an der Ausführbarkeit ändert. Am Ende ist ein zusätzlicher
  Fall für $v_1\in\PrDT _1$ zu ergänzen:
  \TODO{erzwungenen Zeilenumbruch kontrollieren}
  \begin{itemize}
    \item Fall III ($v_1\in\PrDT _1$): Es existiert ein $u_1$ aus $O^*_1$,
      sodass $p_{01}\weakmay[v_1]_1 p_1\weakmay[u_1]_1 p'_1$ mit $p'_1\in Div
      _1$ gilt. Da es hier keine disjunkten Inputmengen gibt kann das $a$, auf das $v_1$
      im Fall $v_1\neq\varepsilon$ endet, ebenfalls der letzte Buchstabe von $v_2$
      sein. Im Fall von $v_2\in\MIT _2$ kann somit $a=b$ gelten und damit wäre
      $v_2=v'_2$. Dieser Fall verläuft jedoch analog zu Fall Ic) aus dem Beweis
     der oben genannten Inklusion von Satz~\ref{KommFehlerSemSatz} und wird
      somit hier nicht weiter betrachtet. Deshalb gilt für alle im folgenden
      betrachteten Fälle $p_{02}\weakmay[v'_2]_2 p_2$ mit $(p_{01},p_{02})
      \weakmay[v']_2$.
      \begin{itemize}
        \item Fall IIIa) \big($u_2\in (O_1\cap I_2)^*,c\in (O_1\cap I_2)$,
          sodass $u_2c$ ein Präfix von $u_1|_{I_2}$ mit $p_2\weakmust[u_2]_2
          p'_2 \nmust[c]_2$\big): Für ein Präfix von $u'_1c$ von $u_1$ mit
          $(u'_1c)|_{I_2}=u_2c$ weiß man, dass $p_1\weakmay[u'_1]_1 p''_1
          \may[c]_1$. Somit gilt $u'_1\in u'_1\|u_2$ und $(p_1,p_2)
          \weakmay[u'_1]_{12} (p''_1,p'_2)\in E_{12}$, da für $P_2$ der
          entsprechende Input fehlt, der mit dem Output von $c$ von $P_1$ zu
          koppeln wäre. Es handelt sich also um einen neuen
          Fehler. Es wird $v:=\prune (v'u'_1) \in\PrET _{12}$
          gewählt, dies ist ein Präfix von $v'$, da $u_1\in O^*_1$.
        \item Fall IIIb) \big($p_2\weakmust[u_2]_2 p'_2$ mit $u_2=u_1|_{I_2}$
          \big): Somit ist $u_1\in u_1\|u_2$ und $(p_1,p_2)\weakmay[u_1]_{12}
          (p'_1,p'_2)\in Div _{12}$, da $p_1\in Div _1$. $P_{12}$ hat also die
          Divergenz von $P_1$ geerbt. Es wird nun $v:=\prune (v'u_1)\in\PrDT
          _{12}$ gewählt, das wiederum ein Präfix von $v'$ ist.
      \end{itemize}
  \end{itemize}

  2. \glqq$\subseteq$\grqq{}:\\
  Diese Inklusionsrichtung kann analog zum Beweis derselben Inklusionsrichtung
  des zweiten Punktes von Satz~\ref{RuheSemSatz} gezeigt werden. Es muss dabei
  nur in der Argumentation die Menge $\ET _{12}$ durch die Menge $\EDT _{12}$
  und die Mengen $\QET (P)$ durch die Mengen $\QDT (P)$ für die entsprechenden
  Transitionssysteme $P$ ersetzt werden. Dadurch kann ebenso gefolgert werden,
  dass im Fall $w\in\StQT _{12}\backslash\EDT _{12}$ der erreichte Zustand
  $(p_1,p_2)$ kein Fehler sein kann, da $\ET _{12}\subseteq\EDT
  _{12}$ gilt und somit lässt sich auch hier der zweite Punkt von
  Lemma~\ref{RuheZustLem} anwenden.

  2. \glqq$\supseteq$\grqq{}:\\
  Es muss wieder danach unterschieden werden, aus welcher Menge das betrachtete
  Element stammt. Falls $w$ ein Element von $\EDT _{12}$ ist, folgt die
  Zugehörigkeit zur linken Seite der Gleichung direkt. Somit wird für den
  weiteren Verlauf dieses Beweises davon ausgegangen, dass $w\in\QDT _1\|\QDT
  _2$ gilt. Für dieses $w$ soll dann gezeigt werden, dass es auch in $\QDT
  _{12}$ enthalten ist. Da $\QDT _i=\StQT _i\cup\EDT _i$ gilt, existierten für
  $w_1$ und $w_2$ mit $w\in w_1\|w_2$ unterschiedliche Möglichkeiten:
  \begin{itemize}
    \item Fall 1 ($w_1\in\EDT _1\lor w_2\in\EDT _2$): \OBdA{} gilt $w_1\in\EDT
      _1$. Es kann nun $w_2\in\StQT _2\subseteq L_2$ gelten oder $w_2\in\EDT
      _2\subseteq \EDL _2$ und somit gilt auf jeden Fall $w_2\in\EDL _2$.
      Daraus kann mit dem ersten Punkt dieses Satzes gefolgert werden, dass
      $w\in\EDT _{12}$ gilt und somit $w$ in der linken Seite der Gleichung
      enthalten ist.
    \item Fall 2 ($w_1\in\StQT _1\backslash\EDT _1\land w_2\in\StQT
      _2\backslash\EDT _2$): Dieser Fall läuft analog zu Fall 2 derselben
      Inklusionsrichtung des Beweises von Satz~\ref{RuheSemSatz}. Hierfür muss
      die Menge $\QET _{12}$ durch $\QDT _{12}$ ersetzt werden.
  \end{itemize}

  3.:\\
  Durch die Definition~\ref{DivSemDef} ist klar, dass $L_i\subseteq\EDL _i$ und
  $\EDT _i\subseteq\EDL _i$ gilt. Die Argumentation wird von der rechten Seite
  der Gleichung aus begonnen:
  \begin{align*}
    (\EDL _1\| \EDL _2)\cup \EDT _{12}
    &\overset{\ref{DivSemDef}}{=}\left(\left(L _1\cup \EDT _1\right)\|\left(L
    _2\cup \EDT _2\right)\right)\cup \EDT _{12}\\
    &=(L _1\|L _2) \cup \underset{\overset{1.}{\subseteq} \EDT
    _{12}}{\underset{\subseteq (\EDL _1\|\EDT _2)}{\underbrace{(L _1\|\EDT
    _2)}}} \cup \underset{\overset{1.}{\subseteq} \EDT
    _{12}}{\underset{\subseteq (\EDT _1\|\EDL _2)}{\underbrace{(\EDT _1\|L
    _2)}}}\\
    &\quad\quad\cup \underset{\overset{1.}{\subseteq} \EDT
    _{12}}{\underset{\subseteq (\EDL _1\|\EDT _2)}{\underbrace{(\EDT _1\|\EDT
    _2)}}} \cup \EDT _{12}\\
    &=(L _1\|L _2) \cup \EDT _{12}\\
    &\overset{\ref{LParallelProp}}{=}L _{12}\cup \EDT _{12}\\
    &\overset{\ref{DivSemDef}}{=}\EDL _{12}.
  \end{align*}
\end{proof}

Analog wie in den beiden vorangegangenen Kapitel, ergibt sich aus diesem Satz
als direkte Folgerung, dass es sich bei der Relation \DRel{} um eine
Präkongruenz handelt.

\begin{Kor}[Divergenz-Präkongruenz]
  Die Relation \DRel{} ist eine Präkongruenz bezüglich $\cdot\|\cdot$.
\end{Kor}
\begin{proof}
  Um zu zeigen, dass es sich bei \DRel{} um eine Präkongruenz handelt, muss
  nachgewiesen werden, dass aus $P_1\DRel P_2$ auch $P_{31}\DRel P_{32}$ für
  jedes komponierbare System $P_3$ folgt. D.h.\ es ist zu zeigen, dass aus
  $\EDT _1\subseteq\EDT _2, \QDT _1\subseteq\QDT _2$ und $\EDL _1\subseteq\EDL
  _2$, sowohl $\EDT _{31} \subseteq \EDT _{32}, \QDT _{31} \subseteq \QDT
  _{32}$ als auch $\EDL _{31}\subseteq\EDL _{32}$ folgt. Dies ergibt sich, wie
  in den Beweisen zu den Korollaren~\ref{KommPraekonKor}
  und~\ref{RuhePraekonKor}, aus der Monotonie von \cont{}, \prune{} und
  $\cdot\|\cdot$ auf Sprachen wie folgt:
  \begin{itemize}
    \item $\begin{aligned}[t]
        \EDT{}_{31} &\overset{\ref{DivSemSatz}~1.}{=}
        \cont{}\left(\prune{}\left(\left(\EDT{}_3\|\EDL{}_1\right) \cup
        \left(\EDL{}_3\|\EDT{}_1\right)\right)\right)\\
        &\hspace{-0.6cm}\overset{\EDT{}_1\subseteq
      \EDT{}_2}{\overset{\mathrm{und}}{\overset{\EDL{}_1\subseteq
    \EDL{}_2}{\subseteq}}}
    \cont{}\left(\prune{}\left(\left(\EDT{}_3\|\EDL{}_2\right) \cup
        \left(\EDL{}_3\|\EDT{}_2\right)\right)\right)\\
      &\overset{\ref{DivSemSatz}~1.}{=} \EDT{}_{32},
    \end{aligned}$
    \item $\begin{aligned}[t]
        \QDT{}_{31} &\overset{\ref{DivSemSatz}~2.}{=} (\QDT{}_3\|\QDT{}_1)
        \cup \EDT{}_{31}\\
        &\hspace{-0.8cm}\overset{\EDT{}_{31}\subseteq
      \EDT{}_{32},}{\overset{\mathrm{und}}{\overset{\QDT{}_1\subseteq
      \QDT{}_2}{\subseteq}}} (\QDT{}_3\|\QDT{}_2) \cup \EDT{}_{32}\\
      &\overset{\ref{DivSemSatz}~2.}{=} \QDT{}_{32}.
    \end{aligned}$
    \item $\begin{aligned}[t]
        \EDL{}_{31} &\overset{\ref{DivSemSatz}~3.}{=} (\EDL{}_3\|\EDL{}_1)
        \cup \EDT{}_{31}\\
        &\hspace{-0.8cm}\overset{\EDT{}_{31}\subseteq
      \EDT{}_{32},}{\overset{\mathrm{und}}{\overset{\EDL{}_1\subseteq
      \EDL{}_2}{\subseteq}}} (\EDL{}_3\|\EDL{}_2) \cup \EDT{}_{32}\\
      &\overset{\ref{DivSemSatz}~3.}{=} \EDL{}_{32}.
    \end{aligned}$
  \end{itemize}
\end{proof}

Im nächsten Lemma soll eine Verfeinerung bezüglich guter Kommunikation
betrachtet werden. Die Vorgaben für gute Kommunikation gibt hierbei die
Definition der Tests und die daraus resultierende Verfeinerung
in~\ref{DivTestDef} vor. Es muss in diesem Lemma eine Veränderung zu den
analogen Lemmata aus den vorangegangenen Kapitel vorgenommen werden. Die
Einschränkung der Tests $T$ auf Partner, kann nicht mehr beibehalten werden, da
die Strategie zur Vermeidung von Ruhe im Beweis aus dem letzten Kapitel hier zu
Divergenz führen würde. Somit werden für die Ruhe-Vermeidung in diesem Kapitel
Aktionen außerhalb der Menge \Synch{} benötigt, die nicht die interne Aktionen
$\tau$ sind. Jedoch müssen trotzdem nicht alle Tests $T$ betrachtet werden. Es
kann eine Einschränkung gemacht werden, sodass $T$ fast ein Partner ist. Zur
Vereinfachung von umständlichen Formulierungen im Folgenden wird hierfür nun
ein neuer Begriff definiert. Der jedoch auch bereits so in
z.B.~\cite{Schinko2016BA} für \EIO{}s verwendet und definiert wurde.

\begin{Def}[{\boldmath$\omega$}-Partner]
  Ein \MEIO{} $P_1$ ist ein $\omega$-Partner von einem \MEIO{} $P_2$, wenn
  $I_1=O_2$ und $O_1=I_2\cup\{\omega\}$ mit $\omega\notin I_2\cup O_2$ gilt.
\end{Def}

Ein $\omega$-Partner $P_1$ von $P_2$ unterscheidet sich von einem Partner von
$P_2$ nur um den Output $\omega$, der nicht in der Menge $\Synch (P_1,P_2)$
enthalten ist.

\begin{Lem}[Testing-Verfeinerung mit Divergenz]
  \label{DivTestVerfeinLem}
  Gegeben sind zwei \MEIO{}s $P_1$ und $P_2$ mit der gleichen Signatur. Wenn
  für alle Tests $T$, die $\omega$-Partner von $P_1$ bzw. $P_2$ sind, $P_2
  \DsatAs T\Rightarrow P_1 \DsatAs T$ gilt, dann folgt daraus die Gültigkeit
  von $P_1\DRel P_2$.
\end{Lem}
\begin{proof}
  Da $P_1$ und $P_2$ die gleiche Signatur haben, definiert man $I:=I_1=I_2$ und
  $O:=O_1=O_2$. Für jeden $\omega$-Partner Test $T$ gilt $I_T=O$ und
  $O_T=I\cup\{\omega\}$ mit $\omega\notin I\cup O$.\\
  Um zu zeigen, dass die Relation $P_1\DRel P_2$ gilt, müssen die folgenden
  Punkte nachgewiesen werden:
  \begin{itemize}
    \item $\EDT _1\subseteq \EDT _2$,
    \item $\QDT _1\subseteq \QDT _2$,
    \item $\EDL _1\subseteq \EDL _2$.
  \end{itemize}
  In den Lemmata~\ref{KommTestVerfeinLem} und~\ref{RuheTestVerfeinLem} wurde
  bereits etwas Ähnliches gezeigt. Jedoch kann daraus aufgrund der
  unterschiedlichen Implikationen, die vorausgesetzt werden, nichts über dieses
  Lemma und dessen Gültigkeit ausgesagt werden. Es kann in diesem Lemma, ebenso
  wie im Lemma~\ref{RuheTestVerfeinLem}, aus der lokalen Erreichbarkeit eines
  Fehlers in einer Parallelkomposition einer as"=Implementierung von $P_1$ mit
  einem Test $T$ und der Implikation $P_2 \DsatAs T\Rightarrow P_1 \DsatAs T$
  nur geschlossen werden, dass es in einer Parallelkomposition einer
  as"=Implementierung von $P_2$ mit $T$ auch einen lokal erreichbaren \glqq
  fehlerhaften Zustand\grqq{} geben muss, jedoch kann die \glqq
  Fehlerhaftigkeit\grqq{} hier ein Fehler, Ruhe oder Divergenz sein. Analog
  verhält es sich, wenn in der Parallelkomposition einer as"=Implementierung
  von $P_1$ mit einem Test $T$ ein Divergenz-Zustand oder Ruhe-Zustand lokal
  erreichbar ist. Es kann nur geschlossen werden, dass $P_2$ den Test $T$ nicht
  erfüllen darf. Die nicht Erfüllung kann jedoch auf einem beliebigen \glqq
  Fehlverhalten\grqq{} des \MEIO{}s basieren.

  Als Erstes wird der erste Beweispunkt gezeigt, also die Inklusion $\EDT
  _1\subseteq\EDT _2$.\\
  Es wird für ein präfix-minimales $w$ aus $\EDT _1$ gezeigt, dass dieses $w$
  oder eines seiner Präfixe in $\EDT _2$ enthalten ist. Diese Möglichkeit
  bietet sich, da beide Mengen unter \cont{} abgeschlossen sind. Wegen
  Proposition~\ref{DivSemProp}~2.\ ist $w$ ein präfix-minimales Elemente der
  Menge $\EDT _{P'_1}$ einer as"=Implementierung $P'_1$ von $P_1$.
  \begin{itemize}
    \item Fall 1 ($w=\varepsilon$): Es handelt sich um einen lokal erreichbaren
      Fehler oder um lokale erreichbare Divergenz in $P'_1$. Für $T$ wird ein
      Transitionssysteme verwendet, das nur aus dem Startzustand und einer
      must"=Schleife für alle Inputs $x\in I_T$ und einer must"=Schleife für
      $\omega$ besteht. Somit kann $P'_1$ im Prinzip die gleichen Fehler- und
      Divergenz-Zustände wie $P'_1\|T$ lokal erreichen. $P_1$ erfüllt den Test
      $T$ also nicht. $P_2$ darf $T$ somit auch nicht erfüllen. Es muss eine
      as"=Implementierung $P'_2$ von $P_2$ existieren, für die $P'_2\|T$ einen
      \glqq fehlerhaften Zustand\grqq{} lokal erreicht. Durch die Struktur von
      $T$ ist in einer Parallelkomposition mit $T$ kein Ruhe-Zustand möglich.
      Der \glqq fehlerhafte Zustand\grqq{}, der in $P'_2\|T$ lokal erreichbar
      ist, muss also ein Fehler- oder Divergenz-Zustand sein. Da von $T$ kein
      Fehler und keine Divergenz geerbt werden kann und durch die Inputschleife
      auch kein neuer Fehler entstehen kann, muss der \glqq fehlerhafte
      Zustand\grqq{} von $P'_2$ geerbt sein. Somit muss in $P'_2$ ein Fehler-
      oder Divergenz-Zustand lokal erreichbar sein. Da $\EDT (P) =\ET
      (P)\cup\DT (P)$ gilt, folgt $w\in \EDT _{P'_2}$ und
      mit~\ref{DivSemProp}~2. $w\in \EDT _2$.
    \item Fall 2 ($w=x_1\dots x_n x_{n+1}\in\Sigma ^+$ mit $n\geq 0$ und
      $x_{n+1}\in I$): Es wird der folgende $\omega$-Partner $T$ betrachtet
      (siehe auch Abbildung~\ref{TohneEmitO}):
      \begin{itemize}
        \item $T=\{p_0,p_1,\dots ,p_{n+1}\}$,
        \item $p_{0T}=p_0$,
        \item $\begin{aligned}[t]
            \may _T = \must _T&=\{(p_j,x_{j+1},p_{j+1})\mid  0\leq j\leq n\}\\
            &\cup\{(p_j,x,p_{n+1})\mid  x\in I_T\backslash\{x_{j+1}\}, 0\leq
            j\leq n\}\\
            &\cup\{(p_{n+1},x,p_{n+1})\mid  x\in I_T\}\\
            &\cup\{(p_j,\omega ,p_{n+1})\mid 0\leq j\leq n+1\},
        \end{aligned}$
        \item $E_T=\emptyset$.
      \end{itemize}
      \begin{figure} [h!tbp]
      \begin{center}
        \begin{tikzpicture}[->, >=latex',auto,node distance =3cm, semithick]
          \node (0) {$p_0$};
          \node (1) [right of=0] {$p_1$};
          \node (dots) [right of=1] {$\dots$};
          \node (n) [right of=dots] {$p_n$};
          \node (n1) at ($(1)!0.5!(dots) + (0,-3)$) {$p_{n+1}$};

          \path ($ (0) + (-1,0) $) edge (0)
                (0) edge node {$x_1$} (1)
                    edge [bend right] node [below, sloped] {$x?\neq x_1, \omega
                    !$} (n1)
                (1) edge node {$x_2$} (dots)
                    edge node [below, sloped] {$x?\neq x_2, \omega !$} (n1)
                (dots) edge node {$x_n$} (n)
                       edge [dashed] (n1)
                (n) edge node [above, sloped] {$x?\in I_T, \omega !$} (n1)
                    edge [bend left] node [sloped] {$x_{n+1}$!} (n1)
                (n1) edge [loop below] node {$x?\in I_T, \omega !$} (n1);
        \end{tikzpicture}
        \caption{$x?\neq x_i$ steht für alle $x\in I_T\backslash\{x_i\}$}
      \label{TohneEmitO}
      \end{center}
      \end{figure}
      Die Mengen der Divergenz- und Ruhe-Zustände des hier betrachteten $T$s
      sind leer. Da im Vergleich zum Transitionssystem in
      Abbildung~\ref{UohneE} nur die $\omega$-Transitionen zu $p_{n+1}$ ergänzt
      und die Mengen unbenannt wurden, ändert sich nichts an dem Fall 2a) im
      ersten Punkt des Beweises von Lemma~\ref{KommTestVerfeinLem}. Im Fall 2b)
      muss die Menge $O^*$ durch $(O\cup\{\omega\})^*$ ersetzt werden. Die
      Begründungen, wieso in den beiden Fällen $\varepsilon\in\PrET (P'_1\|T)$
      für ein $P'_1\in\asimp (P_1)$ gilt, bleibt also analog zum Beweis von
      Lemma~\ref{KommTestVerfeinLem}. Da nun aber auch Divergenz betrachtet
      wird, muss ein weiterer Fall ergänzt werden:
      \begin{itemize}
        \item Fall 2c) ($w\in\PrDT _{P'_1}$): In $P'_1\|T$ erhält man
          $(p_{01},p_0) \weakmay[w] (p'',p_{n+1})\weakmay[u] (p',p_{n+1})$ für
          $u\in (O\cup \{\omega\})^*$ und $p'\in Div_1$. Daraus folgt
          $(p',p_{n+1})\in Div_{P'_1\|T}$ und somit $wu\in\StDT (P'_1\|F)$. Da
          alle Aktionen aus $w$ synchronisiert werden und $I_T\cap
          I_1=\emptyset$ gilt $x_1,\dots , x_n, x_{n+1}\in O_{P'_1\|T}$. Da
          zusätzlich $u$ in $(O\cup \{\omega\}) ^*$ enthalten ist, folgt $u\in
          O^*_{P'_1\|T}$. Somit ergibt sich $\varepsilon\in\PrDT (P'_1\|T)$.
      \end{itemize}
      Da $\varepsilon$ in $\PrET (P'_1\|T) \cup \PrDT (P'_1\|T)$ enthalten ist,
      ist ein Fehler oder Divergenz lokal erreichbar in $P'_1\|T$. Mit der
      Implikation $P_2 \DsatAs T\Rightarrow P_1 \DsatAs T$ kann geschlossen
      werden, dass in der Parallelkomposition einer as"=Implementierung $P'_2$
      von $P_2$ mit dem Test $T$ ein \glqq fehlerhafter Zustand\grqq{} lokal
      erreichbar sein muss. Durch die $\omega$-Transitionen an den Zuständen
      von $T$ kann es in Komposition mit $T$ keine Ruhe-Zustände geben. Die
      \glqq Fehlerhaftigkeit\grqq{} muss also ein Fehler oder Divergenz sein.
      \begin{itemize}
        \item Fall 2i) ($\varepsilon\in\ET (P'_1\|T)$ wegen neuem Fehler): Da
          jeder Zustand von $T$ alle Inputs $x\in I_T=O$ zulässt, muss ein
          lokal erreichbarer Fehler-Zustand in diesem Fall der Form sein, dass
          ein Output $a\in O_T\backslash\{\omega\}$ von $T$ möglich ist, der
          nicht mit einem passenden Input aus $P'_2$ synchronisiert werden
          kann. Durch die Konstruktion von $T$ ist in $p_{n+1}$ kein Output
          außer $\omega$ möglich. Ein neuer Fehler muss also die Form
          $(p',p_j)$ haben mit $j\leq n$, $p'\nmust[x_{i+1}]_{P'_2}$ und
          $x_{i+1}\in O_T\backslash\{\omega\}$. Durch Projektion erhält man
          dann $p_{02} \lweakmay[x_1\dots x_i]_{P'_2} p'
          \nmust[x_{i+1}]_{P'_2}$ und damit gilt $x_1\dots x_{i+1}\in\MIT
          _{P'_2}\subseteq \ET _{P'_2}$. Somit ist ein Präfix von $w$ in $\EDT
          _{P'_2}$ enthalten. Wegen des Abschlusses unter \cont{} und wegen
          Proposition~\ref{DivSemProp}~2.\ gilt $w\in\EDT _2$.
        \item Fall 2ii) ($\varepsilon\in\ET (P'_2\|T)$ wegen geerbtem Fehler):
          $T$ hat $x_1\dots x_iu$ ausgeführt mit $u\in (O\cup\{\omega\})^*$ und
          ebenso hat $P'_2$ den Weg $x_1\dots x_iu|_{\Sigma _2}$ ausgeführt.
          Durch dies hat $P'_2$ einen Zustand aus $E_{P'_2}$ erreicht, da von
          $T$ kein Fehler geerbt werden kann. Es gilt dann $\prune (x_1\dots
          x_iu| _{\Sigma _2})=\prune (x_1\dots x_i)\in\PrET _{P'_2}\subseteq
          \ET _{P'_2}$. Da $x_1\dots x_i$ ein Präfix von $w$ ist, führt in
          diesem Fall eine Verlängerung um lokale Aktionen von einem Präfix von
          $w$ zu einem Fehler-Zustand. Da \ET{} der Menge aller Verlängerungen
          von gekürzten Fehler-Traces entspricht, ist $x_1\dots x_i$ in $\EDT
          _{P'_2}$ enthalten und mit~\ref{DivSemProp}~2.\ ist ein Präfix von
          $w$ in $\EDT _2$ enthalten.
        \item Fall 2iii) ($\varepsilon\in\DT (P'_2\|T)\backslash\ET
          (P'_2\|T)$): Da $T$ nicht unendlich viele Zustände hat und auch keine
          $\tau$-Schleifen besitzt, kann das Divergenzverhalten nur von $P'_2$
          geerbt sein. $T$ hat $x_1\dots x_iu$ ausgeführt mit $u\in (O \cup
          \{\omega\})^*$ und ebenso hat $P'_2$ den Weg $x_1\dots x_iu|_{\Sigma
          _2}$ ausgeführt. Durch dies hat $P'_2$ einen Zustand aus $Div_{P'_2}$
          erreicht. Es gilt dann $\prune (x_1\dots x_iu|_{\Sigma _2}) = \prune
          (x_1\dots x_i)\in\PrDT _{P'_2}\subseteq\DT _{P'_2}$, da $u|_{\Sigma
          _2}$ in $O^*$ enthalten ist. Da $x_1\dots x_i$ ein Präfix von $w$
          ist, führt in diesem Fall eine Verlängerung um lokale Aktionen von
          einem Präfix von $w$ zu einem divergenten Zustand. Da \DT{} die Menge
          aller Verlängerungen von gekürzten Divergenz-Traces ist und $\DT
          _{P'_2}\subseteq \EDT _{P'_2}$ gilt, ist in diesem Fall das Präfix
          $x_1\dots x_i$ von $w$ in $\EDT _{P'_2}$ enthalten.
          Mit~\ref{DivSemProp}~2.\ folgt daraus, dass ein Präfix von $w$ in
          $\EDT _2$ enthalten ist.
      \end{itemize}
  \end{itemize}

  Als nächstes wird nun der zweite Beweispunkt gezeigt, d.h.\ die Inklusion
  $\QDT _1\subseteq\QDT _2$. Diese Inklusion kann jedoch noch, analog zum
  Beweis der Inklusion der Fehler-gefluteten Sprache aus dem Fehler-Kapitel,
  weiter eingeschränkt werden. Da bereits bekannt ist, das $\EDT _1 \subseteq
  \EDT _2$ gilt, muss nur noch $\StQT _1\backslash \EDT _1\subseteq\QDT _2$
  gezeigt werden.\\
  Es wird ein $w\in\StQT _1\backslash\EDT _1$ gewählt und gezeigt, dass dieses
  auch in $\QDT _2$ enthalten ist.\\
  \TODO{zu beweisen}

  \TODO{Beweis für dritten Punkt}
\end{proof}

\begin{Satz}
  \label{DivTestVerfSatz}
  Falls $P_1 \DRel P_2$ gilt folgt daraus auch, dass $P_1$ $P_2$
  Divergenz-verfeinert.
\end{Satz}
\begin{proof}
  \TODO{zu beweisen}
\end{proof}

Wie in den letzten beiden Kapitel, wurde eine Folgerungskette gezeigt, die sich
zu einem Ring schließt. Dies ist in Abbildung~\ref{FolgerungsketteDiv}
dargestellt.

\begin{figure}[h!tbp]
  \begin{center}
    \begin{tikzpicture}[scale = 3]
      \matrix (m) [matrix of math nodes,row sep=2cm,column sep=4cm]{%
        P_1\DRel P_2 & P_1 \text{ verfeinert } P_2 \\
        \substack{\forall \text{ Test } \omega\text{-Partner } T:\\P_2\DsatAs
        T\Rightarrow P_1\DsatAs T} &
    \substack{\forall \text{ Tests } T:\\P_2\DsatAs T\Rightarrow P_1\DsatAs T} \\};
        \draw[-implies, double, double distance=1mm]
          (m-1-1) -- node [above] {Satz~\ref{DivTestVerfSatz}} (m-1-2);
        \draw[-implies, double, double distance=1mm]
          (m-1-2) -- node [right] {Definition~\ref{DivTestDef}} (m-2-2);
        \draw[-implies, double, double distance=1mm]
          (m-2-1) -- node [left]
          {Lemma~\ref{DivTestVerfeinLem}} (m-1-1);
        \draw[-implies, double, double distance=1mm]
          (m-2-2) -- node [below]
          {$\substack{T \text{ Test } \omega\text{-Partner}\\\Downarrow\\ T \text{ Test}}$} (m-2-1);
    \end{tikzpicture}
    \caption{Folgerungskette der Testing-Verfeinerung und Divergenz-Relation}
  \label{FolgerungsketteDiv}
  \end{center}
\end{figure}


\TODO{Satz für den Zusammenhang der Verfeinerungs-Relationen erweitern:
$\asRel \Rightarrow \DRel \overset{?}{\Rightarrow} \QRel$ und $\wasRel
\not\Rightarrow \DRel$}
