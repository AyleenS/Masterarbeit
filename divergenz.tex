\chapter{Verfeinerungen für Kommunikationsfehler-, Ruhe- und Divergenz-Freiheit}

In diesem Kapitel soll die Menge der betrachteten Zustände noch einmal
erweitert werden. Somit werden dann Fehler-, Ruhe- und Divergenz-Zustände
betrachtet.

\begin{Def}[Divergenz]
  Ein \emph{Divergenz-Zustand} ist ein Zustand in einem \MEIO{} $P$, der eine
  unendliche Folge an $\tau$s ausführen kann via may-Transitionen.\\
  Die Menge $Div(P)$ besteht aus all diesen divergenten Zuständen des \MEIO{}s
  $P$.
\end{Def}

\begin{Prop}[Divergenz und Implementierung]
  \label{DivProp}
  Für ein \MEIO{} $P$ gilt $Div (P) = \left\{p'\in P'\mid P'\in\asimp (P)
  \land P' \text{ kann von } p' \text{ aus eine unendliche Folge von } \tau
  \text{s ausführen} \right\}$.
\end{Prop}
\begin{proof}
  Falls von einem $p'$ in einem $P'\in\asimp (P)$ eine unendliche $\tau$-Folge
  ausführbar ist, dann ist dies via must"=Transitionen möglich. Diese
  Transitionen müssen jedoch aus möglichen may"=Transitionen in $P$ folgen.
  Deshalb war auch der Zustand, der zu $p'$ in Simulations-Relation nach
  Definition~\ref{SimDef} steht, auch bereits divergent und in $Div (P)$
  enthalten. Für jedes $p$ aus $Div (P)$ gibt es eine unendliche Folge von
  $\tau$s, die via may"=Transitionen möglich ist in $P$. Also gibt es auch in
  $\asimp (P)$ mindestens eine Implementierung, die alle diese beteiligen
  may"=Transitionen als must"=Transitionen implementiert und einen zu $p$
  analogen Zustand als divergent enthält und somit in die Menge $\left\{p'\in
  P'\mid P'\in\asimp (P) \land P' \text{ kann von } p' \text{ aus eine
  unendliche Folge von } \tau \text{s ausführen} \right\}$ einfügt.
\end{proof}

Die unendliche Folge an $\tau$s kann durch eine Schleife an einem durch $\tau$s
erreichbaren Zustand ausführbar sein oder durch einen Weg, der mit $\tau$s
ausführbar ist, mit dem unendliche viele Zustände durchlaufen werden. Es ist
jedoch zu beachten, dass ein Zustand, von dem aus unendlich viele Zustände
durch $\tau$s  erreichbar sind, nicht divergent sein muss. Es ist auch möglich,
dass dieser Zustand eine unendliche Verzweigung hat und somit keine unendlichen
Folgen an $\tau$s ausführen kann.\\
Als Erreichbarkeitsbegriff wird wieder die lokale Erreichbarkeit verwendet. Da
das Divergieren eines Systems nicht mehr verhindert werden kann, sobald ein
divergenter Zustand lokal erreicht werden kann, ist Divergenz als ähnlich \glqq
schlimm\grqq{} zu bewerten wie ein Fehler-Zustand des Types
Kommunikationsfehler.

\begin{Def}[fehler-, ruhe- und divergenz-freie Kommunikation]
  Zwei \MEIO{}s $P_1$ und $P_2$ kommunizieren \emph{fehler-, ruhe- und
  divergenz-frei}, wenn keine as"=Implementierung ihrer Parallelkomposition
  $P_{12}$ einen Fehler-, Ruhe- oder Divergenz-Zustand lokal erreichen kann.
\end{Def}

\begin{Def}[Divergenz-Verfeinerungs-Basisrelation]
  Für \MEIO{}s $P_1$ und $P_2$ mit der gleichen Signatur wird $P_1\DBRel{} P_2$
  geschrieben, wenn ein Fehler-, Ruhe- oder Divergenz-Zustand in einer
  as"=Implementierung von $P_1$ nur dann lokal erreichbar ist, wenn es auch
  eine as"=Implementierung von $P_2$ gibt, in der ein solcher lokal erreichbar
  ist. Die \emph{Basisrelation} stellt eine \emph{Verfeinerung} bezüglich
  \emph{Fehler}, \emph{Ruhe} und \emph{Divergenz} dar.\\
  \DCRel{} bezeichnet die \emph{vollständig abstrakte Präkongruenz} von
  \DBRel{} bezüglich $\cdot\|\cdot$.
\end{Def}

Da nun die grundlegenden Definitionen für Divergenz festgehalten sind, kann man
sich einen Begriff für die Traces zu divergenten Zuständen bilden. Da oben
bereits festgestellt wurde, dass Divergenz als ähnlich „schlimmer Fehler“
anzusehen ist wie Kommunikationsfehler und dass das Divergieren eines Systems
nicht mehr verhinderbar ist, sobald ein divergenter Zustand lokal erreichbar
ist, kommt für die Divergenz-Traces wieder die \prune{}-Funktion zum Einsatz. Ein
System, das unendliche viele $\tau$s ausführen kann, ist von außen nicht von so
einem System zu unterscheiden, das einen Fehler-Zustand der Art
Kommunikationsfehler erreicht. Somit wird in den Trace-Mengen auch nicht
zwischen Kommunikationsfehler"=Traces und Divergenz-Traces explizit
unterschieden. Dadurch genügt es nicht mehr nur mit den
Kommunikationsfehler"=Traces die Sprache zu fluten, sondern es muss sowohl mit den
Kommunikationsfehler"=Traces wie auch den Divergenz"=Traces geflutet werden.
Ebenso werden die strikten Ruhe"=Traces mit diesen beiden Trace-Mengen
geflutet.

\begin{Def}[Divergenz-Traces]
  Sei $P$ ein \MEIO{} und definiere:
  \begin{itemize}
    \item \emph{strikte Divergenz-Traces}: $\StDT (P) := \left\{w\in\Sigma
      ^*\mid p_0\weakmay[w]_P p\in Div(P)\right\}$,
    \item \emph{gekürzte Divergenz-Traces}: $\PrDT (P) := \bigcup\left\{\prune
      (w)\mid w\in\StDT (P)\right\}$.
  \end{itemize}
\end{Def}

Analog zu den Propositionen~\ref{KommTracesProp} und~\ref{QuiTraceProp} gibt es
hier auch eine Proposition, die die Divergenz-Traces eines \MEIO{}s mit den
Divergenz-Traces seiner as"=Implementierungen verbindet. Die Begründung würde
analog wie die der beiden Propositionen der vorangegangenen Kapitel laufen, in
Kombination mit den Argumenten des Beweises zur Proposition~\ref{DivProp} in
diesem Kapitel.

\begin{Prop}[Divergenz-Traces und Implementierung]
  Für ein \MEIO{} $P$ gilt $\StDT (P) = \left\{w\in\Sigma ^*\mid \exists P' \in
  \asimp (P): p'_0\weakmust[w]_{P'} p'\in Div(P')\right\}$.
\end{Prop}

Da die Ruhe"=Traces mit den Kommunikationsfehler- und Divergenz"=Traces
geflutet werden sollen, kann die Ruhe"=Semantik nicht aus dem letzten Kapitel
übernommen werden auch die geflutete Sprache aus dem
Kommunikationsfehler-Kapitel kann nicht übernommen werden. Nur die
Kommunikationsfehler"=Traces \ET{} können ohne Veränderung auch in diesem
Kapitel verwendet werden. Jedoch werden diese Traces im weiteren Verlauf nur
innerhalb der größeren Trace-Menge \EDT{} relevant sein.

\begin{Def}[Kommunikationsfehler-, Ruhe- und Divergenz-Semantik]
  \label{DivSemDef}
  Sei $P$ ein \MEIO{}.
  \begin{itemize}
    \item Die Menge der \emph{Divergenz-Traces} von $P$ ist $\DT (P) := \cont
      (\PrDT (P))$.
    \item Die Menge der \emph{Fehler-Divergenz-Traces} von $P$ ist $\EDT (P) :=
      \ET (P)\cup\DT (P)$.
    \item Die Menge der \emph{Kommunikationsfehler-divergenz-gefluteten
      Ruhe-Traces} von $P$ ist $\QDT (P) := \StQT (P)\cup\EDT (P)$.
    \item Die Menge der \emph{Kommunikationsfehler-divergenz-gefluteten
      Sprache} von $P$ ist $\EDL (P) := L(P)\cup\EDT (P)$.
  \end{itemize}
  Für zwei \MEIO{}s $P_1,P_2$ mit der gleichen Signatur schreibt man $P_1\DRel
  P_2$, wenn $\EDT _1\subseteq \EDT _2, \QDT _1\subseteq \QDT _2$ und $\EDL
  _1\subseteq \EDL _2$ gilt.
\end{Def}

\DRel{} ist somit keine Einschränkung von \ERel{} so wie \QRel{}. Es können
Systeme mit einem Kommunikationsfehler nicht von Systemen mit Divergenz
unterschieden werden. Da die Basisrelation zwischen diesen Fehler-Arten auch
keine Unterscheidung kennt, muss eine sinnvolle Relation dies Eigenschaft auch
übernehmen, so wie \DRel{} dies tut.

\begin{Satz}[Kommunikationsfehler-, Ruhe- und Divergenz-Semantik für
  Parallelkompositionen]
  \label{DivSemSatz}
  Für zwei komponierbare \MEIO{}s $P_1,P_2$ und ihre Komposition $P_{12}$ gilt:
  \begin{enumerate}
    \item $\EDT _{12} =\cont (\prune ((\EDT _1\|\EDL _2)\cup (\EDL _1\|\EDT
      _2)))$,
    \item $\QDT _{12} =(\QDT _1\|\QDT _2)\cup \EDT _{12}$,
    \item $\EDL _{12} =(\EDL _1\|\EDL _2)\cup \EDT _{12}$.
  \end{enumerate}
\end{Satz}
\begin{proof}\mbox{}\\
  1. \glqq$\subseteq$\grqq{}:\\
  Da beide Seiten der Gleichung unter \cont{} abgeschlossen sind, genügt es ein
  präfix-minimales Element $w$ zu betrachten. Es muss hier unterschieden
  werden, ob $w\in\ET _{12}$ oder $w\in\DT _{12}\backslash\ET _{12}$ betrachtet
  wird. Im ersten Fall ist das $w$ in der rechten Seite der Gleichung enthalten
  wegen des Beweises des ersten Punktes von Satz~\ref{KommFehlerSemSatz} und da
  $\ET (P)\subseteq \EDT (P)$ und $\EL (P)\subseteq \EDL (P)$ gilt. Deshalb
  wird im weiteren Verlauf dieses Beweises davon ausgegangen, dass $w\in\DT
  _{12}\backslash\ET _{12}$ gilt und es wird versuch zu zeigen, dass dieses $w$
  ebenfalls in der rechten Seite enthalten ist. Da das betrachtete $w$
  präfix-minimal ist, gilt $w\in\PrDT _{12}\backslash\ET _{12}$. Aus der
  Definition~\ref{DivSemDef} weiß man, dass ein $v\in O^*_{12}$ existiert,
  sodass $(p_{01},p_{02})\weakmay[w]_{12} (p_1,p_2)\weakmay[v]_{12}
  (p'_1,p'_2)$ gilt mit $(p'_1,p'_2)\in Div _{12}$. Durch die Projektion auf die
  Transitionssysteme $P_1$ und $P_2$ erhält man $p_{01}\weakmay[w_1]_1 p_1
  \weakmay[v_1]_1 p'_1$ und $p_{02}\weakmay[w_2]_2 p_2\weakmay[v_2]_2 p'_2$ mit
  $w\in w_1\|w_2$ und $v\in v_1\|v_2$. Aus $(p'_1,p'_2)\in Div _{12}$ folgt,
  dass \oBdA{} $p'_1\in Div _1$ gilt, d.h.\ $w_1v_2\in\StDT _1\subseteq\EDT
  _1$. Da $p_{02}\weakmay[w_2v_2]_2$ gilt, erhält man $w_2v_2\in\EDL _2$. Somit
  gilt insgesamt $wv\in\EDT _1\|\EDL _2$ und da $v\in O^*_{12}$, ist $w$ in der
  rechten Seite der Gleichung enthalten und es folgt insgesamt $\prune (wv) =
  \prune (w)$.

  1. \glqq$\supseteq$\grqq{}:\\
  Es wird ebenso wie oben nur ein präfix-minimales $x$ betrachtet wegen des
  Abschlusses beider Seiten der Gleichung unter \cont{}. Es wird also für ein
  beliebiges $x\in\prune ((\EDT _1\|$ $\EDL _2)\cup (\EDL _1\|\EDT _2))$ gezeigt,
  dass dieses oder eines seiner Präfixe auch in $\EDT _{12}$ enthalten ist. Da
  das $x$ aus der \prune{}-Funktion entstanden ist, lässt sich ein $y$ aus
  $O^*_{12}$ finden, sodass $xy\in (\EDT _1\|\EDL _2)\cup (\EDL _1\|\EDT _2)$.
  Es wird nun noch vorausgesetzt, dass \oBdA{} $xy\in \EDT _1\|\EDL _2$ gilt,
  d.h.\ es existiert $w_1\in\EDT _1$ und $w_2\in\EDL _2$ mit $xy\in
  w_1\|w_2$.\\
  \TODO{erzwungenen Zeilenumbruch kontrollieren}\\
  Die folgende Argumentation läuft analog zu der im Beweis der Inklusion $\ET
  _{12} \supseteq \cont($ $\prune ((\ET _1\|\EL _2)\cup (\EL _1\|\ET _2)))$ aus
  Satz~\ref{KommFehlerSemSatz}. Es muss dazu nur jeweils an den Stellen, an
  denen $\PrET (P)\cup\MIT (P)$ steht auch noch eine Vereinigung mit $\PrDT
  (P)$ vorgenommen werden. Für Fall I und II aus dem Beweis der oben genannten
  Inklusion von Satz~\ref{KommFehlerSemSatz} ist jeweils kein weiterer
  Unterfall für $v'_2$ notwendig da, wenn $v'_2$ nicht ausführbar ist, bereits
  ein Fehler-Zustand der Art Kommunikationsfehler in der Parallelkomposition
  entsteht. Somit ist egal, ob auch noch Divergenz vorlag. Falls $v'_2$
  ausführbar, ist nicht relevant, ob eine Divergenz-Möglichkeit bestanden hat,
  da diese nicht an der Ausführbarkeit ändert. Am Ende ist ein zusätzlicher
  Fall für $v_1\in\PrDT _1$ zu ergänzen:
  \TODO{erzwungenen Zeilenumbruch kontrollieren}
  \begin{itemize}
    \item Fall III ($v_1\in\PrDT _1$): Es existiert ein $u_1$ aus $O^*_1$,
      sodass $p_{01}\weakmay[v_1]_1 p_1\weakmay[u_1]_1 p'_1$ mit $p'_1\in Div
      _1$ gilt. Da es hier keine disjunkten Inputmengen gibt kann das $a$, auf das $v_1$
      im Fall $v_1\neq\varepsilon$ endet, ebenfalls der letzte Buchstabe von $v_2$
      sein. Im Fall von $v_2\in\MIT _2$ kann somit $a=b$ gelten und damit wäre
      $v_2=v'_2$. Dieser Fall verläuft jedoch analog zu Fall Ic) aus dem Beweis
     der oben genannten Inklusion von Satz~\ref{KommFehlerSemSatz} und wird
      somit hier nicht weiter betrachtet. Deshalb gilt für alle im folgenden
      betrachteten Fälle $p_{02}\weakmay[v'_2]_2 p_2$ mit $(p_{01},p_{02})
      \weakmay[v']_2$.
      \begin{itemize}
        \item Fall IIIa) \big($u_2\in (O_1\cap I_2)^*,c\in (O_1\cap I_2)$,
          sodass $u_2c$ ein Präfix von $u_1|_{I_2}$ mit $p_2\weakmust[u_2]_2
          p'_2 \nmust[c]_2$\big): Für ein Präfix von $u'_1c$ von $u_1$ mit
          $(u'_1c)|_{I_2}=u_2c$ weiß man, dass $p_1\weakmay[u'_1]_1 p''_1
          \may[c]_1$. Somit gilt $u'_1\in u'_1\|u_2$ und $(p_1,p_2)
          \weakmay[u'_1]_{12} (p''_1,p'_2)\in E_{12}$, da für $P_2$ der
          entsprechende Input fehlt, der mit dem Output von $c$ von $P_1$ zu
          koppeln wäre. Es handelt sich also um einen neuen
          Kommunikationsfehler. Es wird $v:=\prune (v'u'_1) \in\PrET _{12}$
          gewählt, dies ist ein Präfix von $v'$, da $u_1\in O^*_1$.
        \item Fall IIIb) \big($p_2\weakmust[u_2]_2 p'_2$ mit $u_2=u_1|_{I_2}$
          \big): Somit ist $u_1\in u_1\|u_2$ und $(p_1,p_2)\weakmay[u_1]_{12}
          (p'_1,p'_2)\in Div _{12}$, da $p_1\in Div _1$. $P_{12}$ hat also die
          Divergenz von $P_1$ geerbt. Es wird nun $v:=\prune (v'u_1)\in\PrDT
          _{12}$ gewählt, das wiederum ein Präfix von $v'$ ist.
      \end{itemize}
  \end{itemize}

  2. \glqq$\subseteq$\grqq{}:\\
  Diese Inklusionsrichtung kann analog zum Beweis derselben Inklusionsrichtung
  des zweiten Punktes von Satz~\ref{RuheSemSatz} gezeigt werden. Es muss dabei
  nur in der Argumentation die Menge $\ET _{12}$ durch die Menge $\EDT _{12}$
  und die Mengen $\QET (P)$ durch die Mengen $\QDT (P)$ für die entsprechenden
  Transitionssysteme $P$ ersetzt werden. Dadurch kann ebenso gefolgert werden,
  dass im Fall $w\in\StQT _{12}\backslash\EDT _{12}$ der erreichte Zustand
  $(p_1,p_2)$ kein Kommunikationsfehler sein kann, da $\ET _{12}\subseteq\EDT
  _{12}$ gilt und somit lässt sich auch hier der zweite Punkt von
  Lemma~\ref{RuheZustLem} anwenden.

  2. \glqq$\supseteq$\grqq{}:\\
  Es muss wieder danach unterschieden werden, aus welcher Menge das betrachtete
  Element stammt. Falls $w$ ein Element von $\EDT _{12}$ ist, folgt die
  Zugehörigkeit zur linken Seite der Gleichung direkt. Somit wird für den
  weiteren Verlauf dieses Beweises davon ausgegangen, dass $w\in\QDT _1\|\QDT
  _2$ gilt. Für dieses $w$ soll dann gezeigt werden, dass es auch in $\QDT
  _{12}$ enthalten ist. Da $\QDT _i=\StQT _i\cup\EDT _i$ gilt, existierten für
  $w_1$ und $w_2$ mit $w\in w_1\|w_2$ unterschiedliche Möglichkeiten:
  \begin{itemize}
    \item Fall 1 ($w_1\in\EDT _1\lor w_2\in\EDT _2$): \OBdA{} gilt $w_1\in\EDT
      _1$. Es kann nun $w_2\in\StQT _2\subseteq L_2$ gelten oder $w_2\in\EDT
      _2\subseteq \EDL _2$ und somit gilt auf jeden Fall $w_2\in\EDL _2$.
      Daraus kann mit dem ersten Punkt dieses Satzes gefolgert werden, dass
      $w\in\EDT _{12}$ gilt und somit $w$ in der linken Seite der Gleichung
      enthalten ist.
    \item Fall 2 ($w_1\in\StQT _1\backslash\EDT _1\land w_2\in\StQT
      _2\backslash\EDT _2$): Dieser Fall läuft analog zu Fall 2 derselben
      Inklusionsrichtung des Beweises von Satz~\ref{RuheSemSatz}. Hierfür muss
      die Menge $\QET _{12}$ durch $\QDT _{12}$ ersetzt werden.
  \end{itemize}

  3.:\\
  Durch die Definition~\ref{DivSemDef} ist klar, dass $L_i\subseteq\EDL _i$ und
  $\EDT _i\subseteq\EDL _i$ gilt. Die Argumentation wird von der rechten Seite
  der Gleichung aus begonnen:
  \begin{align*}
    (\EDL _1\| \EDL _2)\cup \EDT _{12}
    &\overset{\ref{DivSemDef}}{=}\left(\left(L _1\cup \EDT _1\right)\|\left(L
    _2\cup \EDT _2\right)\right)\cup \EDT _{12}\\
    &=(L _1\|L _2) \cup \underset{\overset{1.}{\subseteq} \EDT
    _{12}}{\underset{\subseteq (\EDL _1\|\EDT _2)}{\underbrace{(L _1\|\EDT
    _2)}}} \cup \underset{\overset{1.}{\subseteq} \EDT
    _{12}}{\underset{\subseteq (\EDT _1\|\EDL _2)}{\underbrace{(\EDT _1\|L
    _2)}}}\\
    &\quad\quad\cup \underset{\overset{1.}{\subseteq} \EDT
    _{12}}{\underset{\subseteq (\EDL _1\|\EDT _2)}{\underbrace{(\EDT _1\|\EDT
    _2)}}} \cup \EDT _{12}\\
    &=(L _1\|L _2) \cup \EDT _{12}\\
    &\overset{\ref{LParallelProp}}{=}L _{12}\cup \EDT _{12}\\
    &\overset{\ref{DivSemDef}}{=}\EDL _{12}.
  \end{align*}
\end{proof}

\begin{Kor}[Divergenz-Präkongruenz]
  Die Relation \DRel{} ist eine Präkongruenz bezüglich $\cdot\|\cdot$.
\end{Kor}
\begin{proof}
  Um zu zeigen, dass es sich bei \DRel{} um eine Präkongruenz handelt, muss
  nachgewiesen werden, dass aus $P_1\DRel P_2$ auch $P_{31}\DRel P_{32}$ für
  jedes komponierbare System $P_3$ folgt. D.h.\ es ist zu zeigen, dass aus
  $\EDT _1\subseteq\EDT _2, \QDT _1\subseteq\QDT _2$ und $\EDL _1\subseteq\EDL
  _2$, sowohl $\EDT _{31} \subseteq \EDT _{32}, \QDT _{31} \subseteq \QDT
  _{32}$ als auch $\EDL _{31}\subseteq\EDL _{32}$ folgt. Dies ergibt sich, wie
  in den Beweisen zu den Korollaren~\ref{KommPraekonKor}
  und~\ref{RuhePraekonKor}, aus der Monotonie von \cont{}, \prune{} und
  $\cdot\|\cdot$ auf Sprachen wie folgt:
  \begin{itemize}
    \item $\begin{aligned}[t]
        \EDT{}_{31} &\overset{\ref{DivSemSatz}~1.}{=}
        \cont{}\left(\prune{}\left(\left(\EDT{}_3\|\EDL{}_1\right) \cup
        \left(\EDL{}_3\|\EDT{}_1\right)\right)\right)\\
        &\hspace{-0.6cm}\overset{\EDT{}_1\subseteq
      \EDT{}_2}{\overset{\mathrm{und}}{\overset{\EDL{}_1\subseteq
    \EDL{}_2}{\subseteq}}}
    \cont{}\left(\prune{}\left(\left(\EDT{}_3\|\EDL{}_2\right) \cup
        \left(\EDL{}_3\|\EDT{}_2\right)\right)\right)\\
      &\overset{\ref{DivSemSatz}~1.}{=} \EDT{}_{32},
    \end{aligned}$
    \item $\begin{aligned}[t]
        \QDT{}_{31} &\overset{\ref{DivSemSatz}~2.}{=} (\QDT{}_3\|\QDT{}_1)
        \cup \EDT{}_{31}\\
        &\hspace{-0.8cm}\overset{\EDT{}_{31}\subseteq
      \EDT{}_{32},}{\overset{\mathrm{und}}{\overset{\QDT{}_1\subseteq
      \QDT{}_2}{\subseteq}}} (\QDT{}_3\|\QDT{}_2) \cup \EDT{}_{32}\\
      &\overset{\ref{DivSemSatz}~2.}{=} \QDT{}_{32}.
    \end{aligned}$
    \item $\begin{aligned}[t]
        \EDL{}_{31} &\overset{\ref{DivSemSatz}~3.}{=} (\EDL{}_3\|\EDL{}_1)
        \cup \EDT{}_{31}\\
        &\hspace{-0.8cm}\overset{\EDT{}_{31}\subseteq
      \EDT{}_{32},}{\overset{\mathrm{und}}{\overset{\EDL{}_1\subseteq
      \EDL{}_2}{\subseteq}}} (\EDL{}_3\|\EDL{}_2) \cup \EDT{}_{32}\\
      &\overset{\ref{DivSemSatz}~3.}{=} \EDL{}_{32}.
    \end{aligned}$
  \end{itemize}
\end{proof}

\begin{Def}[$\omega$-Partner]
  Ein \MEIO{} $P_1$ ist ein $\omega$-Partner von einem \MEIO{} $P_2$, wenn
  $I_1=O_2$ und $O_1=I_2\cup\{\omega\}$ mit $\omega\notin I_2\cup O_2$ gilt.
\end{Def}

\begin{Lem}[Verfeinerung mit Divergenz-Zuständen]
  Gegeben sind zwei \MEIO{}s $P_1$ und $P_2$ mit der gleichen Signatur. Wenn
  $U\|P_1\DBRel U\|P_2$ für alle $\omega$-Partner $U$ gilt, dann folgt daraus
  $P_1\DRel P_2$.
\end{Lem}
\begin{proof}
  \textbf{TODO zu beweisen}
\end{proof}
