\chapter{Verfeinerungen für Kommunikationsfehler-, Ruhe- und Divergenz-Freiheit}

In diesem Kapitel soll die Menge der betrachteten Zustände noch einmal
erweitert werden. Somit werden dann Fehler-, Ruhe- und Divergenz-Zustände
betrachtet.

\begin{Def}[Divergenz]
  Ein \emph{Divergenz-Zustand} ist ein Zustand in einem \MEIO{} $P$, der eine
  unendliche Folge an $\tau$s ausführen kann via may-Transitionen.\\
  Die Menge $Div(P)$ besteht aus all diesen divergenten Zuständen des \MEIO{}s
  $P$.
\end{Def}

\begin{Prop}[Divergenz und Implementierung]
  \label{DivProp}
  Für ein \MEIO{} $P$ gilt $Div (P) = \left\{p'\in P'\mid P'\in\asimp (P)
  \land P' \text{ kann von } p' \text{ aus eine unendliche Folge von } \tau
  \text{s ausführen} \right\}$.
\end{Prop}

\begin{proof}
  Falls von einem $p'$ in einem $P'\in\asimp (P)$ eine unendliche $\tau$ Folge
  ausführbar ist, dann ist dies via must"=Transitionen möglich. Diese
  Transitionen müssen jedoch aus möglichen may"=Transitionen in $P$ folgen.
  Deshalb war auch der Zustand, der zu $p'$ in Simulations-Relation nach
  Definition~\ref{SimDef} steht, auch bereits divergent und in $Div (P)$
  enthalten. Für jedes $p$ aus $Div (P)$ gibt es eine unendliche Folge von
  $\tau$s, die via may"=Transitionen möglich ist in $P$. Also gibt es auch in
  $\asimp (P)$ mindestens eine Implementierung, die alle diese beteiligen
  may"=Transitionen als must"=Transitionen implementiert und einen zu $p$
  analogen Zustand als divergent enthält und somit in die Menge $\left\{p'\in
  P'\mid P'\in\asimp (P) \land P' \text{ kann von } p' \text{ aus eine
  unendliche Folge von } \tau \text{s ausführen} \right\}$ einfügt.
\end{proof}

Die unendliche Folge an $\tau$s kann durch eine Schleife an einem durch $\tau$s
erreichbaren Zustand ausführbar sein oder durch einen Weg, der mit $\tau$s
ausführbar ist, mit dem unendliche viele Zustände durchlaufen werden. Es ist
jedoch zu beachten, dass ein Zustand, von dem aus unendlich viele Zustände
durch $\tau$s  erreichbar sind, nicht divergent sein muss. Es ist auch möglich,
dass dieser Zustand eine unendliche Verzweigung hat und somit keine unendlichen
Folgen an $\tau$s ausführen kann.\\
Als Erreichbarkeitsbegriff wird wieder die lokale Erreichbarkeit verwendet. Da
das Divergieren eines Systems nicht mehr verhindert werden kann, sobald ein
divergenter Zustand lokal erreicht werden kann, ist Divergenz als ähnlich \glqq
schlimm\grqq{} zu bewerten wie ein Fehler-Zustand des Types
Kommunikationsfehler.

\begin{Def}[fehler-, ruhe- und divergenz-freie Kommunikation]
  Zwei \MEIO{}s $P_1$ und $P_2$ kommunizieren \emph{fehler-, ruhe- und
  divergenz-frei}, wenn keine as"=Implementierung ihrer Parallelkomposition
  $P_{12}$ einen Fehler-, Ruhe- oder Divergenz-Zustand lokal erreichen kann.
\end{Def}

\begin{Def}[Divergenz-Verfeinerungs-Basisrelation]
  Für \MEIO{}s $P_1$ und $P_2$ mit der gleichen Signatur wird $P_1\DBRel{} P_2$
  geschrieben, wenn ein Fehler-, Ruhe- oder Divergenz-Zustand in einer
  as"=Implementierung von $P_1$ nur dann lokal erreichbar ist, wenn es auch
  eine as"=Implementierung von $P_2$ gibt, in der ein solcher lokal erreichbar
  ist. Die \emph{Basisrelation} stell eine \emph{Verfeinerung} bezüglich
  \emph{Fehler}, \emph{Ruhe} und \emph{Divergenz} dar.\\
  \DCRel{} bezeichnet die \emph{vollständig abstrakte Präkongruenz} von
  \DBRel{} bezüglich $\cdot\|\cdot$.
\end{Def}

Da nun die grundlegenden Definitionen für Divergenz festgehalten sind, kann man
sich einen Begriff für die Traces zu divergenten Zuständen bilden. Da oben
bereits festgestellt wurde, dass Divergenz als ähnlich „schlimmer Fehler“
anzusehen ist wie Kommunikationsfehler und dass das Divergieren eines Systems
nicht mehr verhinderbar ist, sobald ein divergenter Zustand lokal erreichbar
ist, kommt für die Divergenztraces wieder die prune-Funktion zu Einsatz. Ein
System, das unendliche viele $\tau$s ausführen kann, ist von außen nicht von so
einem System zu unterscheiden, das einen Fehler-Zustand der Art
Kommunikationsfehler erreicht. Somit wird in den Trace-Mengen auch nicht
zwischen Kommunikationsfehler"=Traces und Divergenz-Traces explizit
unterschieden. Dadurch genügt es nicht mehr nur mit den
Kommunikationsfehler"=Traces die Sprache fluten, sondern es muss sowohl mit den
Kommunikationsfehler"=Traces wie auch den Divergenz"=Traces geflutet werden.
Ebenso werden die strikten Ruhe"=Traces mit diesen beiden Trace-Mengen
geflutet.

\begin{Def}[Divergenz-Traces]
  Sei $P$ ein \MEIO{} und definiere:
  \begin{itemize}
    \item \emph{strikte Divergenz-Traces}: $\StDT (P) := \left\{w\in\Sigma
      ^*\mid p_0\weakmay[w]_P p\in Div(P)\right\}$,
    \item \emph{gekürzte Divergenz-Traces}: $\PrDT (P) := \bigcup\left\{\prune
      (w)\mid w\in\StDT (P)\right\}$.
  \end{itemize}
\end{Def}

Analog zu den Propositionen~\ref{KommTracesProp} und~\ref{QuiTraceProp} gibt es
hier auch eine Proposition, die die Divergenz-Traces eines \MEIO{}s mit den
Divergenz-Traces seiner as"=Implementierungen verbindet. Die Begründung würde
analog wie zu den beiden Propositionen der vorrangige nahten Kapitel laufen, in
Kombination mit den Argumenten des Beweises zur Proposition~\ref{DivProp} in
diesem Kapitel.

\begin{Prop}[Divergenz-Traces und Implementierung]
  Für ein \MEIO{} $P$ gilt $\StDT (P) = \left\{w\in\Sigma ^*\mid \exists P' \in
  \asimp (P): p'_0\weakmust[w]_{P'} p'\in Div(P')\right\}$.
\end{Prop}

Da die Ruhe"=Traces mit den Kommunikationsfehler- und Divergenz"=Traces
geflutet werden sollen, kann die Ruhe"=Semantik nicht aus dem letzten Kapitel
übernommen werden auch die geflutete Sprache aus dem
Kommunikationsfehler-Kapitel kann nicht übernommen werden. Nur die
Kommunikationsfehler"=Traces \ET{} können ohne Veränderung auch in diesem
Kapitel verwendet werden. Jedoch werden diese Traces im weiteren Verlauf nur
innerhalb der größeren Trace-Menge \EDT{} relevant sein.

\begin{Def}[Kommunikationsfehler-, Ruhe- und Divergenz-Semantik]
  Sei $P$ ein \MEIO{}.
  \begin{itemize}
    \item Die Menge der \emph{Divergenz-Traces} von $P$ ist $\DT (P) := \cont
      (\PrDT (P))$.
    \item Die Menge der \emph{Fehler-Divergenz-Traces} von $P$ ist $\EDT (P) :=
      \ET (P)\cup\DT (P)$.
    \item Die Menge der \emph{fehler-divergenz-gefluteten Ruhe-Traces} von $P$
      ist $\QDT (P) := \StQT (P)\cup\EDT (P)$.
    \item Die Menge der \emph{fehler-divergenz-gefluteten Sprache} von $P$ ist
      $\EDL (P) := L(P)\cup\EDT (P)$.
  \end{itemize}
  Für zwei \MEIO{}s $P_1,P_2$ mit der gleichen Signatur schreibt man $P_1\DRel
  P_2$, wenn $\EDT _1\subseteq \EDT _2, \QDT _1\subseteq \QDT _2$ und $\EDL
  _1\subseteq \EDL _2$ gilt.
\end{Def}

\DRel{} ist somit keine Einschränkung von \ERel{} so wie \QRel{}. Es können
Systeme mit einem Kommunikationsfehler nicht von Systemen mit Divergenz
unterschieden werden. Da die Basisrelation zwischen diesen Fehler-Arten auch
keine Unterscheidung kennt, muss eine sinnvolle Relation dies Eigenschaft auch
übernehmen, so wie \DRel{} dies tut.

\begin{Satz}[Kommunikationsfehler-, Ruhe- und Divergenz-Semantik für
  Parallelkompositionen]
  Für zwei komponierbare \MEIO{}s $P_1,P_2$ und ihre Komposition $P_{12}$ gilt:
  \begin{enumerate}
    \item $\EDT _{12} =\cont (\prune ((\EDT _1\|\EDL _2)\cup (\EDL _1\|\EDT
      _2)))$,
    \item $\QDT _{12} =(\QDT _1\|\QDT _2)\cup \EDT _{12}$,
    \item $\EDL _{12} =(\EDL _1\|\EDL _2)\cup \EDT _{12}$.
  \end{enumerate}
\end{Satz}

\begin{proof}
  \TODO{zu beweisen}
\end{proof}
